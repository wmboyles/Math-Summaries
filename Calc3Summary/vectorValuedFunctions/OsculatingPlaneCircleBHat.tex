\subsection{Osculating Plane/Circle \& B-Hat $\left(\hat{B}\right)$}
\noindent
The osculating plane in the plane containing $\vec{r}(t)$, $\hat{T}$ and $\hat{N}$. It is only defined when $\hat{N}\neq 0$. This means strait lines do not have an osculating plane.\\

\noindent
The osculating circle lives in the osculating plane, is centered at $\vec{r}(t)+\frac{\hat{N}(t)}{\kappa(t)}$, and has radius $\frac{1}{\kappa(t)}$. The tangent line at $\vec{r}(t)$ is also tangent to the osculating circle because both points have the same curvature.

[INSERT IMAGES]

\noindent
The vector that is normal to the plane is $\hat{B}(t)=\hat{T}\times\hat{N}$, which is called the binormal vector because it is perpindicular to both $\hat{T}$ and $\hat{N}$. Together, $\hat{T}$, $\hat{N}$, and $\hat{B}$ form the Frenet Serret Frame, also called the TNB frame.

[INSERT IMAGES]

\noindent
We can write the equation for the osculating plane as $\hat{B}(t)\cdot\left(\langle x,y,z\rangle - \vec{r}(t)\right) =0$.