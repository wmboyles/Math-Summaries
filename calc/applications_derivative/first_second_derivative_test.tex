\section{First and Second Derivative Tests}
\subsection{First Derivative Test}
As we saw in previous sections, we can use the first derivative to find critical values, which allow us to find extrema.
\begin{theorem}[First Derivative Test]
	Let $f(x)$ be a continuous function.
	At a critical point $c$,
	\begin{enumerate}
		\item If $f^\prime$ changes sign from positive to negative ($f^\prime(x) > 0$ for $x < c$ and $f^\prime(x) < 0$ for $x > c$), then $f$ has a local maximum at $c$.
		\item If $f^\prime$ changes sign from negative to positive ($f^\prime(x) < 0$ for $x < c$ and $f^\prime(x) > 0$ for $x > c$), then $f$ has a local minimum at $c$.
		\item If $f^\prime$ does not change sign at $c$, then $f$ does not have a local extrema at $c$.
		\item At a left endpoint $a$, if ($f^\prime < 0$ / $f^\prime > 0$), then $f$ has a local (maximum / minimum) at $a$.
		\item At a right endpoint $b$, if ($f^\prime < 0$ / $f^\prime > 0$), then $f$ has a local (minimum / maximum) at $b$.
	\end{enumerate}
\end{theorem}

\begin{example}
	Find the local extrema of $f(x) = x^3 - 12x - 5$.
	Identify any absolute extrema.
\end{example}
Taking the derivative,
\begin{equation*}
	f^\prime(x) = 3x^2 - 12 = 3(x+2)(x-2).
\end{equation*}
\indent
So, the critical values are $x=-2$ and $x=2$.
$f^\prime$ is negative between these values and positive outside of them, so $x=-2$ is a local maximum, while $x=2$ is a local minimum.
\begin{equation*}
	f(-2) = 11 \text{ and } f(2) = -21.
\end{equation*}
\indent
As $x$ approaches $\infty$ (the right endpoint), $f$ also approaches $\infty$, so there is no absolute maximum.
Similarly, as $x$ approaches $-\infty$ (the left endpoint), $f$ also approaches $-\infty$, so there is no absolute minimum.

\subsection{Second Derivative Test}
The second derivative tells us how the derivative is changing.
Whether the derivative is increasing or decreasing describes the concavity.
\begin{definition}
	On some open interval $I$, the graph of a twice differentiable function $f(x)$ is
	\begin{enumerate}
		\item Concave up if $f^{\prime\prime} > 0$ on $I$.
		\item Concave down if $f^{\prime\prime} < 0$ on $I$.
	\end{enumerate}
\end{definition}

\noindent
Concave up portions of a graph tend to look like valleys, while concave down portions tend to look like hills.
Unlike the first derivative, we can tell if a function is increasing or decreasing using concavity.
We can however say if the function is increasing or decreasing more or less rapidly.

\begin{figure}[H]
	\label{mvt}
	\centering
	\includegraphics[width = 0.5\textwidth]{./applications_derivative/concavity.png}
	\caption{\hyperref{https://tutorial.math.lamar.edu/classes/calci/shapeofgraphptii.aspx}{}{}{Paul's Online Notes - The Shape of a Graph, Part II}}
\end{figure}

\noindent
The points where a function changes concavity are called "inflection points".
At these points, the function is increasing or decreasing most rapidly, depending on the sign of the first derivative.

\noindent
Rather than looking at the sign of $f^\prime$ around critical values, we can look at the concavity at the critical point.
\begin{theorem}[Second Derivative Test]
	Let $c$ be a critical value of $f$.
	\begin{enumerate}
		\item If $f^{\prime\prime}(c) < 0$, then $c$ is a local maximum.
		\item If $f^{\prime\prime}(c) > 0$, then $c$ is a local minimum.
		\item If $f^{\prime\prime}(c) = 0$, then the test is inclusive.
	\end{enumerate}
\end{theorem}
\noindent
In other words, if we're at a critical point that's turning into a valley, then the critical point must be the top of a hill.
If we're at a critical point that's turning into a hill, then the critical point must be the bottom of a valley.
This test is particularly useful because you only need to know $f^{\prime\prime}$ at $c$ rather than an entire interval\footnote{It also extends to higher dimensions much better than the first derivative test.}.

\begin{example}
	Use the second derivative test to find the local extreme values of $f(x) = x^3 - 12x - 5$.
\end{example}
Finding critical values,
\begin{align*}
	f^\prime(x) &= 3x^2 - 12 = 3(x+2)(x-2). \\
	f^\prime(x) &= 0\text{ at } x=-2, x=2.
\end{align*}
\indent
Taking the second derivative,
\begin{equation*}
	f^{\prime\prime}(x) = 6x.
\end{equation*}
\indent
Evaluating the second derivative at the critical values,
\begin{equation*}
	f^{\prime\prime}(-2) = -12 \text{ and } f^{\prime\prime}(2) = 12.
\end{equation*}
\indent
So, $x=-2$ is a local maximum and $x=2$ is a local minimum.