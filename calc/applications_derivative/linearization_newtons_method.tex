\section{Linearization \& Newton's Method}

\subsection{Linearization}
As we've seen, tangent lines intersect their function at most once: at the point of tangency.
However, we know that differentiable functions are locally linear, so we'd expect the tangent line to be a decent approximation of the function near the point of tangency.

\begin{definition}
	If $f$ is differentiable at $a$, then the approximating function
	\begin{equation*}
		L(x) = f^\prime(a)(x-a) + f(a)
	\end{equation*}
	is the linearization of $f$ at $a$.
\end{definition}

\begin{example}
	Find the linearization of $f(x) = \ln{(x+1)}$ at $x=0$.
	How accurate is this approximation at $x=0.1$?
\end{example}
Following the definition,
\begin{align*}
	f(0) = \ln{(0+1)} = 0 \\
	f^\prime(x) &= \frac{1}{x+1} \\
	f^\prime(0) &= \frac{1}{0+1} = 1 \\
	L(x) &= 1(x-0) + 0 = x.
\end{align*}
\indent
Calculating the error at $x=0.1$,
\begin{align*} 
	L(0.1) &= 0.1 \\
	f(0.1) &\approx .0953 \\
	\text{\% error} &= \frac{\abs{L(0.1)-f(0.1)}}{L(0.1)}100\text{\%} \approx 4.7\text{\%}.
\end{align*}
\indent
So, we can see the linear approximation is pretty good.

\noindent
We call the difference in $x$ between the point of tangency and the point we're trying to approximate the differential.
\begin{definition}
	Let $y=f(x)$ be a differentiable function.
	The differential $\mathrm{d}x$ is an independent variable.
	The differential $\mathrm{d}y$ is $\mathrm{d}y = f^\prime(x)\mathrm{d}x$.
\end{definition}
\noindent
$\mathrm{d}y$ is the approximated change in $y$ expected by the linearization for some given change in $x$, $\mathrm{d}x$.

\begin{example}
	Find $\mathrm{d}y$ for $y=\frac{2x}{1+x^2}$, $x=-2$, and $\mathrm{d}x = 0.1$.
\end{example}
\begin{align*}
	y^\prime &= \frac{-4x^2}{\left(1+x^2\right)^2} + \frac{2}{1+x^2} \\
	y^\prime(-2) &= -6/25 \\
	\mathrm{d}y &= (-6/25)(0.1) = -0.024.
\end{align*}

\subsection{Newton's Method}
We can use the fact that the tangent line approximates the function to find the zeroes of functions.
Starting with an initial guess $x_0$ for the $x$ value of the zero, we look at the the tangent line at $x_0$ and find where it intersects the $x-axis$.
\begin{align*}
	L_0(x) &= f^\prime(x_0)(x - x_0) + f(x_0) \\
	0 &= f^\prime(x_0)(x - x_0) + f(x_0) \\
	-f^\prime(x_0)(x - x_0) &= f(x_0) \\
	x - x_0 &= -\frac{f(x_0)}{f^\prime(x_0)} \\
	x &= x_0 - \frac{f(x_0)}{f^\prime(x_0)}.
\end{align*}
This $x$ value serves as our next guess for the zero.
We repeat this process, until we find the zero or are satisfied with our error\footnote{For most well-behaved functions, Newton's Method can get within a small margin of error or a zero relatively quickly. There is also a generalized, sometimes faster version of Newton's Method that approximates the function with higher-order polynomials than just lines.}.
This yields a recursive formula
\begin{equation*}
	x_{n+1} = x_n - \frac{f(x_n)}{f^\prime(x_n)}.
\end{equation*}