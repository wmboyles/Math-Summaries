\section{Definition of the Derivative}
\begin{definition}
	The derivative of a function $f(x)$, notated $f^\prime(x)$, is the slope of $f$ at any point along $f$.
	This is also the slope of the tangent line to $f$ at this point, which is also the instantaneous rate of change of $f$ at this point.
	\begin{equation}
		f^\prime(x) = \lim_{h \to 0}{\frac{f(x+h)-f(x)}{h}} \text{ (assuming the limit exists).}
	\end{equation}
\end{definition}

\begin{example}
	Find the derivative of $f(x)=x^2+4$.
\end{example}
\begin{answer}
	Applying the definition,
	\begin{align*}
		f^\prime(x) &= \lim_{h\to 0}{\frac{f(x+h)-f(x)}{h}} \\
		&= \lim_{h \to 0}{\frac{(x+h)^2+4 - x^2 - 4}{h}} \\
		&= \lim_{h \to 0}{\frac{x^2 + 2xh + h^2 + 4 - x^2 - 4}{h}} \\
		&= \lim_{h \to 0}{\frac{2xh + h^2}{h}} \\
		&= \lim_{h\to 0}{2x + h} \\
		&= 2x.\footnotemark
	\end{align*}
\end{answer}
\footnotetext{Note that the constant term 4 didn't contribute anything to the outcome. It was canceled immediately when subtracting $f(x)$.}


There are a couple different notations that all mean the derivative of $y = f(x)$.
You should be familiar with all of them.
\begin{table}[H]
\begin{center}
\begin{tabular}{ l l l }
	$\begin{aligned}y^\prime\end{aligned}$ & & $\begin{aligned}\dd{y}{x}\end{aligned}$ \\
	& & \\
	$\begin{aligned}\dd{f}{x}\end{aligned}$ & & $\begin{aligned}\dd{}{x}f(x)\end{aligned}$
\end{tabular}
\end{center}
\end{table}

\subsection{Derivative at a Point}
The formula given in the definition is useful because it gives a function that can give the derivative at any point, but it may not be useful or feasible to use this formula.
Instead, we can use a formula to just give us the derivative at one point.
\begin{equation}
	f^\prime(a) = \lim_{x \to a}{\frac{f(x)-f(a)}{x-a}}.
\end{equation}

\begin{example}
	Find the derivative of $f(x) = \frac{1}{x}$ at $x=2$.
\end{example}
\begin{answer}
	Applying the formula for the derivative at a point,
	\begin{align*}
		f^\prime(2) &= \lim_{x \to 2}{\frac{f(x)-f(2)}{x-2}} \\
		&= \lim_{x \to 2}{\frac{1/x - 1/2}{x-2}} \\
		& = \lim_{x \to 2}{\frac{-(x-2)}{2x(x-2)}} \\
		&= \lim_{x \to 2}{\frac{-1}{2x}} \\
		&= \frac{-1}{4}.
	\end{align*}
\end{answer}

\subsection{Left \& Right Hand Derivatives}
The normal derivative is defined in terms of a two-sided limit, meaning that the left and right hand limits are equal.
However, we can also calculate left and right hand derivatives at every point along the function's domain, which may be useful if the left and right hand derivatives are not equal or the point in question is on the boundary of the domain.

\begin{example}
	Find the left and high hand derivatives of the following function at $x=1$.
	Say if the derivative at this point exists and why.
	\begin{equation*}
		f(x) = \begin{cases}
			x^2 + x & x \leq 1 \\
			x+1 & x > 1
		\end{cases}.
	\end{equation*}
\end{example}
\begin{answer}
	Evaluating the left hand limit,
	\begin{align*}
		f^\prime(1^-) &= \lim_{x \to 1^-}{\frac{f(x)-f(1)}{x-1}} \\
		&= \lim_{x \to 1^-}{\frac{x^2 + x - 2}{x-1}} \\
		&= \lim_{x \to 1^-}{\frac{(x-1)(x+2)}{x-1}} \\
		&= \lim_{x \to 1^-}{x+2} \\
		&= 3.
	\end{align*}
	
	Evaluating the right hand limit,
	\begin{align*}
		f^\prime(1^+) &= \lim_{x \to 1^+}{\frac{f(x)-f(1)}{x-1}} \\
		&= \lim_{x \to 1^+}{\frac{x + 1 - 2}{x-1}} \\
		&= \lim_{x \to 1^+}{\frac{x-1}{x-1}} \\
		&= 1.
	\end{align*}
	
	Since $f^\prime(1^-) \neq f^\prime(1^+)$, the derivative does not exist at $x=1$.
\end{answer}