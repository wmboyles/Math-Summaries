\subsection{Higher Order Derivatives}
\begin{definition}
	Let $n$ be a positive integer. The $n^{th}$ derivative of $f$ is
	\begin{equation*}
		f^{(n)}(x) = \dd{}{x}{f^{(n-1)}}(x),
	\end{equation*}
	where $f^{(n-1)}(x)$ is the ${n-1}^{th}$ derivative of $f$.
	Using equivalent notation,
	\begin{equation*}
		\dd{{}^n}{x^n}f = \dd{{}^{n-1}}{x^{n-1}}f.
	\end{equation*}
\end{definition}

\begin{example}
	Find the third derivative of $f(x) = x^3 + x^2$.
\end{example}
\begin{answer}
	Taking the derivative once using the power rule and sum and difference rule,
	\begin{equation*}
		f^\prime(x) = 3x^2 + 2x.
	\end{equation*}
	
	Taking the derivative a second time using the power, constant multiple rules, and sum and difference rules,
	\begin{equation*}
		f^{\prime\prime}(x) = 6x + 2.
	\end{equation*}
	
	Taking the derivative a final time using the power, constant multiple, constant, and sum and difference rules,
	\begin{equation*}
		f^{\prime\prime\prime}(x) = 6.
	\end{equation*}
\end{answer}