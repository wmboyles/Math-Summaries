\section{Physical Interpretations of the Derivative}
As we've seen, the idea derivative is fundamentally about continuous change.
This idea makes the derivative very useful for describing physical situations.

\begin{example}
		Find the rate of change of the area of a circle with respect to its radius in meters.
		Find the rate of change of the volume of a sphere with respect to its radius in meters.
		What are these quantities (with appropriate units) when $r=5\text{m}$?
\end{example}
Starting with the area of a circle,
\begin{align*}
	A &= \pi r^2
	\dd{A}{r} = 2\pi r.
\end{align*}
\indent
You might recognize this as the formula for the circumference of a circle.\\
\indent
Starting with the volume of a sphere,
\begin{align*}
	V &= \frac{4}{3}\pi r^3 \\
	\dd{V}{r} &= 4\pi r^2.
\end{align*}
\indent
You might recognize this as the formula for the surface area of a sphere.\\
\indent
When $r=5\text{m}$,
\begin{align*}
	\dd{A}{r}\biggr\rvert_{r=5\text{m}} &= 2\pi\left(5\text{m}\right) = 10\pi\text{m}. \\
	\dd{V}{r}\biggr\rvert_{r=5\text{m}} &= 4\pi\left(5\text{m}\right)^2 = 100\pi\text{m}^2.
\end{align*}

\subsection{Displacement, Velocity, and Acceleration}
If we have an object whose position is determined by a single variable, like time $t$, then we can model the position as a function.
\begin{equation*}
	s = f(t).
\end{equation*}
The displacement over some interval of length $\Delta t$, would be
\begin{equation*}
	\Delta s = f(t + \Delta t) - f(t).
\end{equation*}
The average velocity over this interval would be
\begin{equation*}
	\bar{v} = \frac{f(t + \Delta t) - f(t)}{\Delta t}.
\end{equation*}
As $\Delta t$ approaches 0, we see that $\bar{v}$ is exactly the definition of the derivative of $f$ with respect to $t$: the instantaneous velocity.
\begin{equation*}
	v(t) = \dd{s}{t} = \lim_{\Delta t \to 0}{\frac{f(t + \Delta t) - f(t)}{\Delta t}}.
\end{equation*}
We can go through the same steps to derive that instantaneous acceleration is the derivative of instantaneous velocity with respect to $t$.
\begin{equation*}
	a(t) = \dd{v}{t} = \dd{{}^2s}{t^2} = \lim_{\Delta t \to 0}{\frac{v(t + \Delta t) - v(t)}{\Delta t}}.
\end{equation*}
Although in everyday language we might use the terms speed and velocity interchangeably, speed is defined as the absolute value of velocity, meaning it is a scalar quantity while velocity is a vector quantity.
\begin{equation*}
	\text{Speed} = \abs{v(t)} = \biggr\lvert \dd{s}{t} \biggr\rvert.
\end{equation*}

\noindent
Modeling the motion of free-falling bodies was one of the earliest motivations for discovering calculus.
Through experiments and applications of physics theory, we know that the height of a falling body when dropped from initial height $h_0$ meters is modeled by
\begin{equation*}
	s(t) = h_0 - \frac{1}{2}gt^2,
\end{equation*}
where $t$ is the time in seconds since the object was released and $g = 9.81m/s^2$ is the acceleration due to gravity near Earth.

\begin{example}
	A ball is dropped from an initial height of 100 meters.
	How long does it take for the ball to hit the ground?
	What speed is the ball traveling when it hits the ground?
\end{example}
In this case, $h_0 = 100\text{m}$.
So,
\begin{equation*}
	s(t) = 100\text{m} - \frac{1}{2}gt^2.
\end{equation*}
We want to solve for $t$ where $s(t)=0$.
\begin{equation*}
	t = \sqrt{\frac{200\text{m}}{g}} \approx 4.515\text{s}.
\end{equation*}
We take take the derivative of $s$ with respect to $t$ and then take the absolute value to get the speed.
\begin{align*}
	v(t) &= \dd{}{t}s(t) = -gt. \\
	\text{Speed}(t) &= \abs{v(t)} = gt.
\end{align*}
Plugging in the time we got when the ball hits the ground,
\begin{equation*}
	\text{Speed}_{\text{Ground}} \approx g(4.515\text{s}) \approx 44.294\text{m/s}.
\end{equation*}