\section{Radius of Convergence}
We've established some tests to determine if an infinite series is or isn't convergent, but like we saw with geometric and Taylor series, it's possible that a series only converges over some interval.
We'd like to be able to find this interval so we know if we can rightly use Taylor series to approximate a function.

\subsection{Convergence Theorem for Power Series}
\begin{theorem}[Convergence Theorem for Power Series]
	There are three possibilities for any power series of the form
	\begin{equation*}
		\sum_{i=0}^{\infty}{c_i(x-a)^i}
	\end{equation*}
	with respect to convergence.
	\begin{enumerate}
		\item The power series converges, but only on some finite interval centered at $x=a$.
			That is, there is a positive real number $r$ such that the series converges for $\abs{x-a}<r$ and diverges otherwise.
			The series may or may not converge at the endpoints.
		\item The power series converges for all real numbers.
		\item The power series converge only at $x=a$ and diverges elsewhere (i.e $r=0$).
	\end{enumerate}
	We call this value $r$ the radius of convergence.
\end{theorem}

Generally, we start by applying the ratio test to determine where the series converges absolutely.
If we find the series converges for all real numbers of just at $x=a$, we are done.
Otherwise the series converges for some finite interval and we need to apply a different test to determine convergence at endpoints.

\begin{example}
	Determine the radius of convergence for the following series:
	\begin{equation*}
		\sum_{i=0}^{\infty}{\frac{ix^i}{10^i}}.
	\end{equation*}
\end{example}
\begin{answer}
	Since this series isn't quite geometric enough for an nth root test but does have all poitive terms, it seems most suited for a ratio test.
	\begin{equation*}
		\lim_{n\to\infty}{\frac{\frac{(n+1)x^{n+1}}{10^{n+1}}}{\frac{nx^n}{10^n}}} = \lim_{n\to\infty}{\frac{(n+1)x^{n+1}}{10nx^n}} = \lim_{n\to\infty}{\frac{(n+1)x}{10n}} = \frac{x}{10}.
	\end{equation*}
	
	$x$ can be positive or negative.
	Recall that with a ratio test, the series converges if the limit is less than 1.
	\begin{align*}
		\abs{\frac{x}{10}} &< 1 \\
		\abs{x} &< 10.
	\end{align*}
	
	So, the radius of convergence is 10.
	Note that this analysis doesn't tell us for sure of the series converges for $x=\pm 10$ (it happens to not converge in this case).
\end{answer}

\begin{example}
	Determine the radius of convergence for the following series:
	\begin{equation*}
		\sum_{i=0}^{\infty}{i!x^i}.
	\end{equation*}
\end{example}
\begin{answer}
	We can again use a ratio test.
	\begin{equation*}
		\lim_{n\to\infty}{\frac{(n+1)!x^{n+1}}{n!x^n}} = \lim_{n\to\infty}{(n+1)x} = \begin{cases} \infty & x\neq 0 \\ 0 & x = 0 \end{cases}.
	\end{equation*}
	
	We see that the only time the limit is less than 1 is when $x=0$.
	So, the radius of convergence is 0.
\end{answer}

\begin{example}
	Determine the radius of convergence for the following series\footnote{You might recognize this as the Maclaurin series for $e^x$.}:
	\begin{equation*}
		\sum_{i=0}^{\infty}{\frac{x^i}{i!}}.
	\end{equation*}
\end{example}
\begin{answer}
	Applying the ratio test,
	\begin{equation*}
		\lim_{n\to\infty}{\frac{\frac{x^{n+1}}{(n+1)!}}{\frac{x^n}{n!}}} = \lim_{n\to\infty}{\frac{x}{n+1}} = 0.
	\end{equation*}
	
	Since the limit is less than 1 everywhere, the series converges for all real $x$.
\end{answer}

\subsection{Convergence at Endpoints}
If a series only converges over some finite interval, our method of applying the ratio test is inconclusive.
Instead, we plug in the endpoint value for $x$ into our series and apply a different test, which usually depends on the specific series we're working with.
The most common tests to use are the direct comparison test, the limit comparison test, the integral test, and the alternating series test.

\begin{example}
	Given that the following series converges by the ratio test for $\abs{x}<10$, determine convergence at the endpoints.
	\begin{equation*}
		\sum_{i=0}^{\infty}{\frac{ix^i}{10^i}}.
	\end{equation*}
\end{example}
\begin{answer}
	Plugging in $x=\pm10$,
	\begin{equation*}
		\sum_{i=0}^{\infty}{\frac{i(\pm10)^i}{10^i}} = \sum_{i=0}^{\infty}{(\pm 1)^i i}.
	\end{equation*}
	
	This series diverges by the nth term test for both $x=10$ and $x=-10$.
	So the radius of convergence is $\abs{x} < 10$.
\end{answer}

\begin{example}
	For what values of $x$ does the following series converge:
	\begin{equation*}
		\sum_{i=1}^{\infty}{(-1)^{i+1}\frac{x^{2i}}{2i}}.
	\end{equation*}
\end{example}
\begin{answer}
	Applying the ratio test to the absolute series,
	\begin{equation*}
		\lim_{n\to\infty}{\frac{\frac{x^{2n+2}}{2n+2}}{\frac{x^{2n}}{2n}}} = \lim_{n\to\infty}{\frac{2nx^2}{2n+2}} = x^2.
	\end{equation*}

	So, by the ratio test, the series converges absolutely when $\abs{x}<1$.
	The series is the same at $x=1$ and $x=-1$.
	\begin{equation*}
		\sum_{i=0}^{\infty}{\frac{(-1)^{i+1}}{2i}}.
	\end{equation*}

	Since the series, disregarding the $(-1)^{i+1}$ has all terms positive and decreasing and passes the nth term test, it converges by the alternating series test.
	Therefore the interval of convergence is $[-1,1]$.
\end{answer}

It's important to remember that although a series might converge over some interval it might do so very slowly, especially at endpoints.
That is why when estimating a function value using a Taylor series we still need to estimate the error.