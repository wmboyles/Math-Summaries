\documentclass[oneside, 12pt]{book}

% ------------------------------------------------------------------------------
% Setup for table of contents
\setcounter{tocdepth}{3} 	% TOC should label down to subsubsections
\setcounter{secnumdepth}{2}	% TOC should not number further than a subsection number
% ------------------------------------------------------------------------------


% ------------------------------------------------------------------------------
% General Setup
\usepackage[english]{babel}
\usepackage{amsfonts, amsmath, amsthm, amssymb}		% Formatting symbols, theorems, lemmas, definitions, and examples

\usepackage{float}							% For making sure tables and figures stay in place
\usepackage{fullpage}						% Create ~1" margins
\usepackage[bottom,flushmargin]{footmisc}	% Put footnotes at bottom of page; don't intent footnotes
\usepackage[titletoc]{appendix}

\setcounter{chapter}{-1}				% Start with chapter 0

\usepackage{enumitem}					% For custom labels on enumerations
\usepackage{hyperref}					% For inserting links
\usepackage{graphicx} 					% For inserting images

% Tikz stuff
\usepackage{tikz}
\usetikzlibrary{patterns}
\usetikzlibrary{calc,patterns,decorations.pathmorphing,decorations.markings}
% ------------------------------------------------------------------------------

% ------------------------------------------------------------------------------
% Theorems, lemmas, definitions and examples should not be numbered
\newtheorem*{theorem}{Theorem}
\newtheorem*{corollary}{Corollary}
\newtheorem*{definition}{Definition}
\newtheorem*{lemma}{Lemma}
\newtheorem*{example}{Example}
% ------------------------------------------------------------------------------

% ------------------------------------------------------------------------------
% Shortcuts
\newcommand{\dd}[2]{\frac{\mathrm{d} #1}{\mathrm{d} #2}}							% d[] / d[]
\renewcommand{\d}[1]{\mathrm{d} #1}
\renewcommand{\qedsymbol}{$\blacksquare$}
\newcommand{\abs}[1]{\lvert #1 \rvert}												% absolute value

\DeclareMathOperator{\arcsec}{arcsec}												% arc-secant
\DeclareMathOperator{\arccot}{arccot}												% arc-cotangent
\DeclareMathOperator{\arccsc}{arccsc}												% arc-cosecant

\newcommand{\R}{\mathbb{R}}															% Real numbers
% ------------------------------------------------------------------------------

\begin{document}
	% Title page setup
	\title{Single Variable Calculus: A Summary}
	\author{William Boyles}
	\date{}
	
	\frontmatter
		\maketitle
		\tableofcontents
		
	\mainmatter
		\chapter{Background \& Review}
\noindent
Everything mentioned in this chapter should already be familiar to you from other math classes. These topics span three main areas: algebra/pre-calculus, single variable calculus, and matrices. These topics will be used either implicitly or with only a passing reference.\\

\noindent
If you are unfamiliar with anything mentioned, you can use many of the great online resources, like Khan Academy, to familiarize yourself before moving forward.

\section{Algebra and Pre-Calculus}
\noindent

\subsection{Complex Numbers}
\noindent
$i$ is called the imaginary unit. It's defined by $i^2 = -1$. It and $-i$ are the solutions to the equation $x^2+1=0$.\\

\noindent
Complex numbers ($\mathbb{C}$) have the form $z = \alpha + \beta i$, where $\alpha$ and $\beta$ are real numbers. The $\alpha$ part of $z$ is called the real part, so $\Re(z) = \alpha$. The $\beta$ part of $z$ is called the imaginary part, so $\Im(z) = \beta i$.\\

\noindent
Often, complex numbers are visualized as points or vectors in a 2D plane, called the complex plane, where $\alpha$ is the x-component, and $\beta$ is the y-component. Thinking of complex numbers like points helps us define the magnitude of complex numbers and compare them. Since a point $(x,y)$ has a distance $\sqrt{x^2+y^2}$ from the origin, we can say the magnitude of $z$, $\lvert z \rvert$ is $\sqrt{\alpha^2 + \beta^2}$. Thinking of complex numbers like vectors helps us understand adding two complex numbers, since you just add the components like vectors.\\

\noindent
A common operation on complex numbers is the complex conjugate. The complex conjugate of $z = \alpha + \beta i$ is $\overline{z} = \alpha - \beta i$. $z$ and $\overline{z}$ are called a conjugate pair.\\

\noindent
Conjugate pairs have the following properties.
\\Let $z$, $w \in \mathbb{C}$.
\begin{enumerate}[label=]
	\item $\overline{z \pm w} = \overline{z} \pm \overline{w}$
	\item $\overline{zw}=\overline{z}\overline{w}$
	\item $\overline{z}=z \Leftrightarrow z \in \mathbb{R}$
	\item $z\overline{z} = \lvert z \rvert^2 = \lvert \overline{z} \rvert^2$
	\item $\overline{\overline{z}} = z$
	\item $\overline{z}^n = \overline{z^n}$
	\item $z^{-1} = \frac{\overline{z}}{\lvert z \rvert^2}$
\end{enumerate} 			% Complex Numbers
\subsection{Factoring Polynomials}
\noindent
We want to break up a polynomial like $f(x) = a_0 + a_1x^1 + \ldots a_nx^n$ into linear factors so that $f(x) = c(x-b_1)\cdot \ldots \cdot(x - b_n)$. This form makes it simple to see that the roots of $f$, solutions to $f(x) = 0$, are $x = b_1 \ldots b_n$.\\

\noindent
For quadratics, $f(x) = ax^2 + bx + c$, there exists a simple formula that will give us both roots, the quadratic formula.
\begin{equation*}
	x = \frac{-b \pm \sqrt{b^2-4ac}}{2a}
\end{equation*}

\noindent
We can see that when $b^2 - 4ac < 0$, like for $f(x) = x^2 + 5x + 1$, we will get complex roots $\alpha \pm \beta i$. For any polynomial, these roots come in pairs, so if $\alpha + \beta i$ is a root, then so is $\alpha - \beta i$. This means that every conjugate pair $\alpha \pm \beta i$ has a quadratic equation with those roots. Sometimes we will not factor quadratics with complex roots into linear terms.\\

\noindent
Although there do exist explicit formulas for finding roots for cubic (degree 3) and quartic (degree 4) equations, they are too long and not useful enough to memorize. When working by hand, we instead use other tricks to find roots.\\

\noindent
There are a few useful tricks that can help. If the polynomial doesn't have a constant term, then 0 is a root. If all the coefficients sum to 0, then 1 is a root. For certain polynomials with an even number of terms, like all cubics of the form $ax^3 + bx^2 + cax + cb$ we can factor out a term from the first two and last two terms to get $x^2(ax+b)+c(ax+b) = (ax+b)(x^2+c)$. For other polynomials, we might just try guessing and checking values. However, we need a more efficient way that works in general.\\

\noindent
Since we are looking to find linear factors $f(x) = (x-b_1)\cdot \ldots \cdot(x-b_n)$, we can see that the constant term in the polynomial is the product of the roots $b_1 \ldots b_n$. In fact, since the coefficients of polynomials are completely determined by the roots and the leading coefficient, all the coefficients are sums and products of roots. You might remember when factoring quadratics that the coefficient of $x$ term is the sum of the two roots. These rules are called Vieta's formulas.\\

\noindent
So, if we have the constant term, we can check all of its integer factors to see if any are roots. For each root, we can divide, using a technique like synthetic division, to continue finding the rest of the roots. This method is especially useful on tests because the roots tend to be integers.

%\begin{example}
%Factor the polynomial $x^5 + x^4 -2x^3 + 4x^2 -24x$.	
%\end{example}
%\noindent
%We can immediately see that there is no constant term, so $x=0$ is a root. Now we need to work on factoring $x^4 + x^3 -2x^2 + 4x - 24$.\\
%The factors of -24 are: -24, -12, -8, -6, -4, -3, -2, -1, 1, 2, 3, 4, 6, 8, 12, and 24. Starting from roots close to 0 and working outwards, we find that $x=2$ is a root. So, we synthetic divide like so
%\begin{table}[H]
%	\centering
%	\begin{tabular}{llllll}
%		$x=2 \mid$ & 1            & 1 & -2 & 4  & -24 \\
%		& $\downarrow$ & 2 & 6  & 8  & 24  \\ \hline
%		& 1            & 3 & 4  & 12 & $\mid 0$  
%	\end{tabular}
%\end{table}
%\noindent
%to see that now we need to work on factoring $x^3+3x^2+4x+12$.
%$x^3+3x^2+4x+12 = x^2(x+3)+4(x+3) = (x+3)(x^2+4)$, so $x=-3$ is a root, and we need to work on factoring $x^2+4$.
%$x^2+4$ has two complex roots $\pm 2i$, so we'll leave it as a quadratic.
%\begin{equation*}	
%	x^5 + x^4 -2x^3 + 4x^2 -24x = x(x-2)(x-3)(x^2+4)
%\end{equation*} 		% Factoring Polynomials
\subsection{Trig Functions \& The Unit Circle}
\noindent
Imagine aa circle of radius 1 centered at the origin that we'll call the unit circle. The x and y coordinates of a point on the unit circle are completely determined by the angle $\theta$ in radians between the x-axis and a line from the origin to the point.\\

\noindent
The function $\cos{\theta}$ tells us x-coordinate of the point, while $\sin{\theta}$ tells us the y-coordinate of the point. The function $\tan{\theta} = \frac{\sin{\theta}}{\cos{\theta}}$ tells us the slope of the line from the origin to the point. Most of the trig functions have geometric interpretations as shown below. The most used ones are $\sin$, $\cos$, $\tan=\frac{\sin}{\cos}$, $\cot = \frac{\cos}{\sin}$, $\csc=\frac{1}{\sin}$, and $\sec=\frac{1}{\cos}$.

\begin{figure}[H]
	\label{unitCircle}
	\centering
	\includegraphics[width = 0.75\textwidth]{./backgroundReview/algebraPreCalc/unitCircle2.png}
	\caption{\hyperref{https://en.wikipedia.org/wiki/Unit_circle}{}{}{Wikipedia - Unit circle}}
\end{figure}

\noindent
We can also think about the inverses of these trig functions. These are either notated with a -1 exponent on the function, or the prefix arc in front of the function name. Many of these functions are only defined on a part of the domain $\left[0, 2\pi\right]$. Below is a table of the inverse trig functions and their domains.

\begin{table}[H]
	\centering
	\begin{tabular}{l|l}
		Function  & Domain                                                 \\ \hline
		$\arcsin$ & $\left[-1, 1\right]$           						   \\
		$\arccos$ & $\left[-1, 1\right]$                                   \\
		$\arctan$ & $\left(-\infty, \infty\right)$                         \\
		$\arccot$ & $\left(-\infty, \infty\right)$                         \\
		$\arccsc$ & $\left(-\infty, -1\right] \cup \left[1, \infty\right)$ \\
		$\arcsec$ & $\left(-\infty, -1\right] \cup \left[1, \infty\right)$
	\end{tabular}
\end{table}
 	% Trig Functions / Unit Circle
\subsection{Trig Identities}
\noindent
As we could see in Figure \ref{unitCircle}, $\sin$ and $\cos$ form a right triangle with hypotenuse 1. So, using the Pythagorean Theorem
\begin{equation*}
	\sin^2{\theta} + \cos^2{\theta} = 1
\end{equation*}
By dividing by $\sin^2$ or $\cos^2$, we can also get
\begin{equation*}
	1 + \cot^2{\theta} = \csc^2{\theta} \text{ and } \tan^2{\theta} + 1 = \sec^2{\theta}
\end{equation*}
Together, these 3 identities are called the Pythagorean Identities.\\

\noindent
We can also relate functions and co-functions.
\begin{equation*}
	\text{xxx}(\theta) = \text{coxxx}\left(\frac{\pi}{2} - \theta\right)
\end{equation*}

Some of the most useful and used identities are the sum and difference identities.
\begin{equation*}
	\sin{\left(\alpha \pm \beta\right)} = \sin{\alpha}\cos{\beta} \pm \cos{\alpha}\sin{\beta}
\end{equation*} \begin{equation*}
	\cos{\left(\alpha \pm \beta\right)} = \cos{\alpha}\cos{\beta} \mp \sin{\alpha}\sin{\beta}
\end{equation*} \begin{equation*}
	\tan{\left(\alpha \pm \beta\right)} = \frac{\tan{\alpha} \pm \tan{\beta}}{1 \mp \tan{\alpha}\tan{\beta}}
\end{equation*} \begin{equation*}
	\sin{\alpha} \pm \sin{\beta} = 2\sin{\left(\frac{\alpha \pm \beta}{2}\right)}\cos{\left(\frac{\alpha \mp \beta}{2}\right)}
\end{equation*} \begin{equation*}
	\cos{\alpha} + \cos{\beta} = 2\cos{\left(\frac{\alpha + \beta}{2}\right)}\cos{\left(\frac{\alpha - \beta}{2}\right)}
\end{equation*} \begin{equation*}
	\cos{\alpha} - \cos{\beta} = -2\sin{\left(\frac{\alpha + \beta}{2}\right)}\sin{\left(\frac{\alpha - \beta}{2}\right)}
\end{equation*} 			% Trig Identites
\subsection{Exponentials \& Logarithms}
\begin{definition}
	e is the base of the natural logarithm. It's definied by the limit
	\begin{equation*}
		e = \lim\limits_{n\rightarrow\infty}{\left(1+\frac{1}{n}\right)^n}.
	\end{equation*}
\end{definition}


$\exp{x} = e^x$ and $\ln{x}$ are inverse functions of each other such that
\begin{equation*}
	e^{\ln{x}} = x \text{, } \ln{e^x} = x.
\end{equation*}


Just like other exponents, the normal rules for adding, subtracting, and multiplying powers apply.
\begin{equation*}
	e^xe^y = e^{x+y}\text{, }\frac{e^x}{e^y}=e^{x-y}\text{, and }\left(e^x\right)^k=e^{xk}.
\end{equation*}


Similar rules apply for logarithms.
\begin{equation*}
	\ln{x}+\ln{y} = \ln{xy}\text{, }\ln{x}-\ln{y} = \ln{\left(\frac{x}{y}\right)}\text{, and }\ln{\left(a^b\right)}=b\ln{a}.
\end{equation*}


You can also change a logarithm of any base to a natural logarithm.
\begin{equation*}
	\log_{b}{a} = \frac{\ln{a}}{\ln{b}}.
\end{equation*}


$e$ is also unique in that it is the only real number $a$ satisfying the equation
\begin{equation*}
	\frac{\mathrm{d}}{\mathrm{d}x}a^x = a^x.
\end{equation*}
meaning $e^x$ is its own derivative.\footnote{Don't worry if you don't know what a derivative is yet. It's one of the first topics we'll cover in calculus.} 		% Exponential and logarithms
\subsection{Partial Fractions}
\noindent
If we have a function of two polynomials $f(x) = \frac{P(x)}{Q(x)}$, it's often easier to break this quotient into a sum of parts where the denominator is a linear or quadratic factor and the numerator is always a smaller degree than the denominator.

\begin{example}
	\begin{equation*}
		\frac{2x-1}{x^3-6x^2+11x-6} = \frac{1/2}{x-1}+\frac{-3}{x-2}+\frac{5/2}{x-3}.
	\end{equation*}
\end{example}

\noindent
One natural way to find these small denominators comes from the linear factors of the denominator where we keep quadratics with complex roots.
This way, when making a common denominator, we get back the original big denominator.
However, there are a few special cases we have to take care of.

\input{./backgroundReview/algebraPreCalc/linearFactors.tex}
\input{./backgroundReview/algebraPreCalc/repeatedLinearFactors.tex}
\input{./backgroundReview/algebraPreCalc/quadraticFactors.tex}
\input{./backgroundReview/algebraPreCalc/repeatedQuadraticFactors.tex}
\input{./backgroundReview/algebraPreCalc/improperFractions.tex}
 			% Partial Fractions % Algebra and Pre-Calc
\section{Single Variable Calculus}
\noindent

% Derivatives and Integrals
% u-subtitution
% Integration by parts % Single Variable Calculus
\chapter{Matrices}

We'll introduce matrices, how they can represent linear maps and systems of linear equations, and useful operations we can perform on them.

\section{Definition}

\begin{definition}
	An $m \times n$ matrix is an array of objects (usually field elements) arranged in $m$ rows and $n$ columns.
\end{definition}

Matrices are usually written inside square brackets.
We tend to use uppercase letters like $M$ to represent matrices as variables.
The notation $M_{a,b}$ represents the element in row $a$ and column $b$.

\begin{example}
	Matrix $M$ is $3 \times 4$.
	\begin{equation*}
		M = \begin{bmatrix}
			1 & 4  & 0 \\
			8 & -1 & -2 \\
			3 & 7  & 4
		\end{bmatrix}
	\end{equation*}
	We see that $M_{2,2} = -1$ and $M_{3,1} = 3$.
\end{example}


\section{Basic Operations}

\subsection{Vector Space Operations}
Two matrices are considered equal if they have the same number of rows and columns and all entries are equal.
Scalar multiplication works by multiplying each element by the scalar.
Addition works element by element.

\begin{example}
	\begin{align*}
		A &= \begin{bmatrix}
			1 & 2 \\
			-1 & 3
		\end{bmatrix} \text{, } B = \begin{bmatrix}
			4 & 0 \\
			-2 & 5
		\end{bmatrix}. \\
		A + B &= \begin{bmatrix}
			1 + 4 & 2 + 0 \\
			-1 + -2 & 3 + 5
		\end{bmatrix} = \begin{bmatrix}
			5 & 2 \\
			-3 & 8
		\end{bmatrix} \\
		3A &= \begin{bmatrix}
			3\cdot1 & 3\cdot 2 \\
			3\cdot -1 & 3\cdot 3
		\end{bmatrix} = \begin{bmatrix}
			3 & 6 \\
			-3 & 9
		\end{bmatrix}.
	\end{align*}
\end{example}

Notice addition is commutative, associative, has an additive identity (the all 0's matrix), and has an additive inverse (scalar multiply by -1).
Further, scalar multiplication has a multiplicative identity (1), is distributive over both addition and field multiplication.
Thus, the set of all matrices of the same size form a vector space.
We denote the vector space of all $m \times n$ matrices with real entries as $\mathcal{M}_{m \times n}$.

\subsection{Multiplication}
We can also define an operation for multiplying two matrices of compatible sizes that outputs another matrix.

\begin{definition}
	Let $A$ be an $m \times n$ matrix, and let $B$ be and $n \times k$ matrix.
	Then $C = AB$ is an $m \times k$ matrix where
	\begin{equation*}
		C_{i,j} = \sum_{d=1}^{n}{A_{i,d} B_{d,j}}.
	\end{equation*}
\end{definition}
If you're familiar with the concept of dot products, then $C_{i,j}$ is the dot product of the $i$th row of $A$ with the $j$th column of $B$.

\begin{example}
	\begin{align*}
		A &= \begin{bmatrix}
			1 & 2 & 3 \\
			0 & -1 & 2
		\end{bmatrix} \text{, } B = \begin{bmatrix}
			1 & -1 & 0 & 2 \\
			2 & 0 & 3 & -1 \\
			0 & 1 & 1 & 5 
		\end{bmatrix}. \\
		AB &= \begin{bmatrix}
			5 & 2 & 9 & 15 \\
			-2 & 2 & -1 & 11
		\end{bmatrix}.
	\end{align*}
\end{example}

Similar to scalar multiplication, there exists a multiplicative identity matrix.
However, this matrix only behaves like an identity when the matrix it's being multiplied is $n \times n$ (i.e. a square matrix).

\begin{definition}
	The $n \times n$ matrix $I_n$ is called the identity matrix and is defined by
	\begin{equation*}
		I_{i,j} = \begin{cases}
			1 & i=j \\
			0 & \text{otherwise}
		\end{cases}.
	\end{equation*}
\end{definition}

\begin{theorem}
	Let $A$ be an $n \times n$ matrix.
	Then $AI_n = I_n A = A$.
\end{theorem}
\begin{proof}
	Let $C = AI_n$.
	Notice,
	\begin{align*}
		C_{i,j} &= \sum_{d=1}^{n}{A_{i,d}(I_{n})_{d,j}} \\
			&= \sum_{d=1}^{n}{A_{i,d} \begin{cases}
					1 & d=j \\
					0 & \text{otherwise}
			\end{cases}} \\
			&= \sum_{d=1}^{n}{\begin{cases}
				A_{i,j} & d=j \\
				0 & \text{otherwise}
			\end{cases}} \\
			&= A_{i,j}.
	\end{align*}
	This same line of reasoning works to also show that $I_nA = A$.
	Since all entries of $A$ and $C$ are equal, $C = AI_n = A$, as desired.
\end{proof}

Also similar to scalar multiplication, matrix multiplication is associative and distributive.
\begin{theorem}
	Let $A$, $B$, and $C$ be matrices.
	Let $k$ be a scalar.
	Then the following properties hold (assuming the matrices have the correct dimensions):
	\begin{itemize}
		\item \textbf{Associative}: $A(BC) = (AB)C$.
		\item \textbf{Distributive Over Matrix Multiplication}: $k(AB) = (kA)B = A(kB)$.
		\item \textbf{Left Distributive Over Addition}: $A(B + C) = AB + AC$.
		\item \textbf{Right Distributive Over Addition}: $(A + B)C = AC + BC$.
	\end{itemize}
\end{theorem}

Unlike scalar multiplication, matrix multiplication is not commutative.
For one, if $AB$ is defined, $BA$ won't also be defined unless $A$ and $B$ are both square matrices
Even if this is the case, $AB \neq BA$ in general.

\begin{theorem}
	An $n \times n$ matrix $A$ commutes only and all matrices in the vector space $\linspan(\{I_n, A\})$.
\end{theorem}
\section{As Linear Maps}

\input{./matrices/matricesAsLinearMapsDefinition.tex}
\input{./matrices/matricesAsLinearMapsProperties.tex}



\section{As Systems of Linear Equations}

\input{./matrices/matricesAsLinearMapsInvertability.tex}
\input{./matrices/matricesAsLinearMapsColumnSpaceNullSpace.tex}
\subsection{Determinants}
\noindent
The determinant of a matrix is a signed number that tells by how much the transformation represented by a matrix scales volumes in a space. The number is negative if the space was "flipped" during a transformation. The number is negative if the dimension of the output space is less than that of the input space.\\

\noindent
The determinant is only defined for square matrices. It's easiest to understand the definition of a determinant recursively.
\begin{equation*}
	\det{\left[ a \right]} = \lvert a \rvert = a
\end{equation*}
\begin{equation*}
	\det{\left[
		\begin{array}{cc}
			a & b \\
			c & d
		\end{array}
	\right]} = \begin{array}{|cc|}
		a & b \\
		c & d
	\end{array} = ad - bc
\end{equation*}
We can define $a_{ij}$ as the entry in the ith row and jth column of matrix $A$ and $A_{ij}$ as the adjudicate matrix, which is the matrix $A$ if row $i$ and column $j$ were removed. This allows us to write a general formula for the determinant.
\begin{definition}
	\begin{equation*}
		\det{A} = \sum_{j=1}^{n}{\left(-1\right)^{i+j}a_{ij}A_{ij}} \text{ (for fixed i)} = \sum_{i=1}^{n}{\left(-1\right)^{i+j}a_{ij}A_{ij}} \text{ (for fixed j)}
	\end{equation*}
\end{definition}
\noindent
This formula allows us to use any row or column to calculate the determinant, which is especially useful if a certain row contains lots of 0's.\\

\noindent
Below are some properties of the determinant for some $n \times n$ matrix $A$ and scalar $\lambda$.
\begin{enumerate}[label=]
	\item \begin{equation*}
		\det{I_n} = 1
	\end{equation*}
	\item \begin{equation*}
		\det{(A^T)} = \det{A}
	\end{equation*}
	\item If $A$ is invertable,
		\begin{equation*}
			\det{(A^{-1})} = \frac{1}{\det{A}}
		\end{equation*}
	\item \begin{equation*}
		\det{(\lambda A)} = \lambda^n\det{A}
	\end{equation*}
	\item \begin{equation*}
		\det{(AB)} = \det{A}\det{B}
	\end{equation*}
	\item If $A$ is triangular,
		\begin{equation*}
			\det{A} = \prod_{i=1}^{n}{a_{ii}}
		\end{equation*}
\end{enumerate}

\ifodd\includeBackgroundReviewExamples\input{./backgroundReview/matrices/determinants_example}\fi % Matrix operations
		\chapter{Limits \& Continuity}
\noindent
Limits are a way of describing what happens to a function $f(x)$ as $x$ gets arbitrarily close to a value from some direction (positive or negative).
This allows us not only to deal with "holes" in some functions but describe some of the building blocks of calculus, namely the derivative.

\section{Limit Definition}
\begin{definition}
	Let $f : D \subseteq \R \to \R$.
	Let $c \in R$ be a limit point (ie $c \in D$ or $c$ is on the boundary of $D$).
	$f$ has a limit $L$ as $x$ approaches $c$ if for any given positive real number $\epsilon$, there is a positive real number $\delta$ such that for all $x \in D$,
	\begin{equation}
		0 < \abs{x-c} < \delta \implies \abs{f(x) - L} < \epsilon.
	\end{equation}
	We write this as
	\begin{equation*}
		\lim_{x \to c}{f(x)} = L.	
	\end{equation*}
\end{definition}

\begin{figure}[H]
	\label{epsilon_delta}
	\centering
	\includegraphics[width = 0.5\textwidth]{./limits_continuity/limit_epsilon_delta.png}
	\caption{\hyperref{https://en.wikipedia.org/wiki/(\%CE\%B5,\_\%CE\%B4)-definition\_of\_limit}{}{}{Wikipedia - $(\epsilon, \delta)\text{-definition of limit}$}}
\end{figure}
\noindent
Visually, what this means is that for any "error bound" of $y$ values $\epsilon$, I can give you a corresponding error bound of $x$ values $\delta$ such that all values of $f(z)$ for $z \in (c -\delta, c+ \delta)$ bound are between $L - \epsilon$ and $L + \epsilon$.

\noindent
We don't use this definition of the limit very often because it's a bit cumbersome.
However, it's important to know that when we use the limit, this is the formal definition making things work.

\begin{example}
	Use the $(\epsilon, \delta)$ definition of the limit to show that
	\begin{equation*}
		\lim_{x\to 0}{x\sin{\frac{1}{x}}} = 0.
	\end{equation*}
\end{example}
Letting $\epsilon > 0$, we need to find corresponding $\delta > 0$ that satisfies the definition for $L = 0$.
Knowing that $\sin$ is bounded between -1 and 1,
\begin{equation*}
	\abs{x\sin{\frac{1}{x}} - 0} = \abs{x\sin{\frac{1}{x}}} = \abs{x}\abs{\sin{\frac{1}{x}}} \leq \abs{x}.
\end{equation*}
\indent
Letting $\delta = \epsilon$, if $0 < \abs{x - 0} < delta$, then $\abs{x\sin{\frac{1}{x}} - 0} \leq \abs{x} < \epsilon$, as required by the definition.
\section{Limit Properties}
Limit have many nice properties all allow us to make useful simplifications when evaluating a limit.
Let
\begin{equation*}
	\lim_{x \to c}{f(x)} = L \text{ and } \lim_{x \to c}{g(x)} = M.
\end{equation*}
\begin{align*}
	\textbf{Sum and Difference Rule: }& \lim_{x\to c}{\left(f(x) \pm g(x)\right)} = L \pm M \\
	\textbf{Product Rule: }& \lim_{x\to c}{\left(f(x)g(x) \right)} = LM \\
	\textbf{Constant Multiple Rule: }& \lim_{x \to c}{k\cdot f(x)} = k \lim_{x \to c}{f(x)} = kL \\
	\textbf{Quotient Rule: }& \lim_{x \to c}{\frac{f(x)}{g(x)}} = \frac{\lim_{x \to x}{f(x)}}{\lim_{x \to c}{g(x)}} = \frac{L}{M} \text{, if} M \neq 0 \\
	\textbf{Power Rule: }& \text{If } n \neq  \in \R \text{, } \lim_{x\to c}{\left(f(x)\right)^n} = \left(\lim_{x \to c}{f(x)}\right)^n = L^n
\end{align*}

\subsection{``Substitution Rule''}
Although it may seem obvious from our idea that limits describe behavior at a point that if $f(x)$ is defined at $x=c$, then $\lim_{x\to c}{f(x)} = f(c)$.
However, this is \textit{not} always the case.
Remember that our definition of a limit required these $\epsilon$ and $\delta$ neighborhoods around the limit point.
If $f(x)$ is defined at $x=c$, but $(c, f(c))$ is not a point in these neighborhoods for any $\epsilon > 0$, then the limit will not evaluate to $f(c)$.

\begin{example}
	Find the limit of $f(x)$ as $x$ approaches $2$ for the following function.
	\begin{equation*}
		f(x) = \begin{cases}
			x^2 & x \neq 2 \\
			0 & x = 2
		\end{cases}.
	\end{equation*}
\end{example}
We can clearly see that $f(2) =  0$, but for $\epsilon = 0.1$ for example, there is no $\delta$ that can satisfy our definition, as points like $(2 - \delta, 4 - 2\delta + \delta^2)$ would outside the neighborhood around $(2,0)$.
In fact, the correct limit value is $4$, the same as if $f(x) = x^2$ for all $x$.
There are some more nuances we'll need to describe before we can say when it's OK to substitute to evaluate a limit.
\section{Left \& Right Hand Limits}
Our definition of the limit requires that the function get arbitrarily close to the limit value when approaching from both the left and right hand sides.
However, we can evaluate limits by specifying that we only approach from one side.
We usually notate this with a superscript $+$ or $-$ next to the $x$ limit value.
So,
\begin{equation*}
	\lim_{x \to 0^+}{f(x)}
\end{equation*}
would mean ``the limit of $f(x)$ as $x$ approaches $0$ from the right'', while
\begin{equation*}
	\lim_{x \to 0^-}{f(x)}
\end{equation*}
would mean ``the limit of $f(x)$ as $x$ approaches $0$ from the left.''

Our definition of the limit from both sides requires the left and right sides to be the same.
If they are different, the the limit does not exist.
\begin{align*}
	\lim_{x \to c^+}{f(x)} = \lim_{x \to c^-}{f(x)} &\implies \lim_{x \to c^+}{f(x)} = \lim_{x \to c^-}{f(x)} = \lim_{x \to c}{f(x)} \\
	\lim_{x \to c^+}{f(x)} \neq \lim_{x \to c^-}{f(x)} &\implies \lim_{x \to c}{f(x)} \text{ does not exist (DNE)}.
\end{align*}
\section{Sandwich Theorem}
We can use the Sandwich Theorem to indirectly find limits by ``sandwiching'' the function in question between two functions we do know the limit of.
If these two sandwiching functions go to the same value in the limit, then so to must the function in question.
\begin{theorem}[The Sandwich Theorem]
	If $g(x) \leq f(x) \leq h(x)$ and $\lim_{x \to c}{g(x)} = \lim_{x\to c}{h(x)} = L$, then $\lim_{x \to c}{f(x)} = L$.
\end{theorem}

\begin{example}
	Evaluate the following limit
	\begin{equation*}
		\lim_{\theta \to 0}{\frac{\sin{\theta}}{\theta}}
	\end{equation*}
\end{example}
\begin{answer}
	We'll need to use some geometric ideas to solve this limit.
	Consider the following on a unit circle.
	
	\begin{figure}[H]
		\label{sin_limit_proof}
		\centering
		\includegraphics[width = 0.5\textwidth]{./limits_continuity/sin_limit_proof.png}
		\caption{\hyperref{}{}{}{Triangle with internal angle $\theta$ inside a unit circle.}}
	\end{figure}
	
	We can see that the area of the swept arc is between the two triangle with base of length 1 and heights of $\sin{\theta}$ and $\tan{\theta}$.
	So, we can write the following inequality.
	\begin{align*}
		\frac{1}{2}\sin{\theta} &\leq \frac{1}{2}\theta \leq \frac{1}{2}\frac{\sin{\theta}}{\cos{\theta}} \\
		\sin{\theta} \leq \theta &\leq \frac{\sin{\theta}}{\cos{\theta}}
	\end{align*}
	
	Taking the reciprocal of each part and multiplying by $\sin{\theta}$,
	\begin{equation*}
		1 \geq \frac{\sin{\theta}}{\theta} \geq \cos{\theta}.
	\end{equation*}
	
	Taking the limit of as $\theta$ approaches 0 of each term,
	\begin{align*}
		1 \geq & \lim_{\theta \to 0}{\frac{\sin{\theta}}{\theta}} \geq \lim_{\theta\to 0}{\cos{\theta}} \\
		1 \geq & \lim_{\theta \to 0}{\frac{\sin{\theta}}{\theta}} \geq 1.
	\end{align*}
	
	So, by the Sandwich Theorem,
	\begin{equation*}
		\lim_{\theta \to 0}{\frac{\sin{\theta}}{\theta}} = 1.
	\end{equation*}
\end{answer}
\section{Infinite Limits}
Although our limit definition works for finite values of $c$, it's also useful to think about what happens as $c$ goes to $\pm\infty$.
We'll need to add to our limit definition to incorporate infinite values, since it doesn't make sense to talk about neighborhoods at infinity.
\begin{definition}
	Let $f$ be a real-valued function defined on some subset $D \subseteq \R$ that contains arbitrarily large values.
	\begin{equation*}
		\lim_{x \to \infty}{f(x)} = L
	\end{equation*}
	if for every real $\epsilon > 0$, there is a real number $N > 0$ such that for all $x \in D$,
	\begin{equation}
		x > N \implies \abs{f(x) - L} < \epsilon.
	\end{equation}
\end{definition}

All the same properties that we described for finite limits, like the Sum and Difference Rule, still hold for infinite limits.

\subsection{End Behavior Model}
When x is numerically large, we can often model the behavior of a complicated function with a simplier one that behaves roughly the same for numerically large input values and is the same in the limit.
There are a few rules that these follow.
\begin{enumerate}
	\item For a polynomial, the end-behavior is highest-degree term.
	\item For a rational function, like a ratio of polynomials, the end behavior is the ratio of the highest degree terms.
	\item For more complicated functions, we may need to use some reasoning about the graph of the function and limit properties to determine end-behavior.
\end{enumerate}

\subsection{Horizontal Asymptotes}
Horizontal Asymptotes are a special type of end-behavior model.
\begin{definition}
	The line $y=b$ is a horizontal asymptote of $y = f(x)$ if $\lim_{x\to \infty}{f(x)} = b$ or $\lim_{x \to -\infty}{f(x)} = b$.
\end{definition}

We can determine horizontal asymptotes for rational functions (usually quotient of polynomials).
There are a few cases to consider
\begin{enumerate}
	\item If the numerator is a higher degree than the denominator, there is no horizontal asymptote, so we'll need a different method to calculate what happens at $\pm\infty$.
	\item If the denominator is a higher degree than the numerator, then there is a horizontal asymptote at $y = 0$.
	\item If the numerator and denominator have the same degree, there is a horizontal asymptote at $y = k$ where k is the ratio of the highest degree terms.
\end{enumerate}

\begin{example}
	Find the following limits, if they exist.\\
	\begin{table}[H]
	\begin{center}
	\begin{tabular}{ l l l}
		1. $\begin{aligned}[t]
			\lim_{x \to \infty}{\frac{x^3 - 6x + 1}{x^2 + 2x - 3}}
		\end{aligned}$ & 
		2. $\begin{aligned}[t]
			\lim_{x\to -\infty}{\frac{x-9}{2x-x^2}}
		\end{aligned}$ &
		3. $\begin{aligned}[t]
			\lim_{x\to \infty}{\frac{6x^2-4x^5+7x-1}{12x^5-3x^2+2}}
		\end{aligned}$ \\
		\hspace{1pt} & \hspace{1pt}\\
		4. $\begin{aligned}[t]
			\lim_{x\to \infty}{\frac{3x+1}{\abs{x}+2}}
		\end{aligned}$ &
		5. $\begin{aligned}[t]
			\lim_{x \to \infty}{x + e^{-x}}
		\end{aligned}$ &
		6. $\begin{aligned}[t]
			\lim_{x \to -\infty}{x + e^{-x}}
		\end{aligned}$
	\end{tabular}
	\end{center}
	\end{table}
\end{example}
\begin{answer}
	\begin{enumerate}
		\item Since the numerator degree is bigger than the denominator degree, we'll need to use the end behavior model.
			The end behavior model tells us that the numerator term dominates and has positive values, so the limit evaluates to $\infty$.
		\item Since the denominator has higher degree than the numerator, there is a HA at $y=0$, so the limit evaluates to $0$.
		\item Since the numerator and denominator have the same degree, the limit is the ratio of the highest-degree coefficients, $\frac{-1}{3}$.
		\item The numerator and denominator have the same degree. For $x > 0$, $\abs{x}+2 = x+2$, so the limit is the ratio of highest-degree coefficients, $3$.
		\item Looking at the two terms, we can see that as $x$ gets large, $e^{-x}$ gets very small, contributing less and less to the overall value.
			So, we can say that this function as a right end behavior model of $x$, so the limit is $\infty$.
		\item Looking at the two terms, we that that as $x$ gets very large and negative, $e^{-x}$ changes much faster than $x$.
			That is, $e^{-x}$ contributes more and more to the overall value of the function compared to $x$.
			So, we can say that this function has a left end behavior model of $e^{-x}$, so the limit is $\infty$.
	\end{enumerate}
\end{answer}
\section{Continuity Definition}
When we were looking at limits, we noticed that we can't always substitute to find the limit, even if the function is defined there.
In the example given to show that substitution and the limit can give different results, we saw a special type of function that seemed to have a "hole" at the point we were interested in finding the limit of.
This function is said to be discontinuous at this point, and in this section we'll define when a function is or isn't continuous at a point based on this idea of the limit and substitution giving different values.

\begin{definition}
	Let $f(x)$ be a real-valued function defined over $D \subseteq \R$.
	$f(x)$ is continuous at some point $x = c$ if all of the following hold.
	\begin{enumerate}
		\item $\lim_{x \to c}{f(x)}$ exists
		\item $f(c)$ is defined
		\item $\lim_{x \to c}{f(x)} = f(c)$ (substitution works)
	\end{enumerate}
	Otherwise, $f(x)$ is discontinuous at $c$.\footnote{Note that it's not necessary for $c \in D$.}
\end{definition}


We say that a function is continuous on an interval if it's continuous on every point in that interval.\footnote{If the interval is closed on one or both sides, we check continuity on the open interval. Then, we check the closed endpoints by looking at the limit from only one side.}

\begin{example}
	Find the points of continuity and discontinuity of the following functions
	\begin{table}[H]
	\begin{center}
	\begin{tabular}{ l l }
		1. $\begin{aligned}
			f(x) = \frac{1}{x^2+1}
		\end{aligned}$ &
		2. $\begin{aligned}
			g(x) = e^{1/x}
		\end{aligned}$
	\end{tabular}
	\end{center}
	\end{table}
\end{example}
\begin{answer}
	\begin{enumerate}
		\item There are no points where $f(x)$ or its limit are undefined.
			Further, there are no points where $f$ and its limit at that point are different.
			So, $f$ is continuous on $(-\infty, \infty)$ and discontinuous on $\emptyset$.
		\item Since $1/x$ is undefined at $x = 0$, $g(x)$ is also undefined at $x=0$.
			At every other point, $g$ and its limit are defined and are equal.
			So, $g$ is continuous on $(\infty, 0) \cup (0, \infty)$ and discontinuous on $[0]$.
	\end{enumerate}
\end{answer}
\section{Discontinuity Types}
There are four major types of discontinuity.
\begin{enumerate}[label=]
	\item \textbf{Removable: } If $f$ is discontinuous at $c$ but we can remove the discontinuity by setting $f$ equal to its limit at $c$, then $f$ has a removable discontinuity at $c$.
	\item \textbf{Jump: } If $f$ is discontinuous at $c$, and both of the one-sided limits exist but are different, then $f$ has a jump discontinuity at $c$.
	\item \textbf{Infinite: } If $f$ has a vertical asymptote at $c$, meaning one or both sides go to $\pm\infty$, then $f$ has an infinite discontinuity at $c$.
	\item \textbf{Oscillating: } If $f$ oscillates without limit at $c$, then $f$ has an oscillating discontinuity at $c$. An example of such a function would be $\sin{\frac{1}{x}}$ at $x=0$.
\end{enumerate}


It might seem strange that $\sin{\frac{1}{x}}$ has an oscillating discontinuity at $x=0$ because we were able to find the limit as $x$ approaches of 0 of $x\sin{\frac{1}{x}}$, a very similar function.
However, remembering how we applied the Sandwich Theorem to find this limit, we see that the $x$ term bounds the amplitude of the oscillations, allowing the limit to be $0$.

\begin{example}
	For the following function state the following: its domain, any discontinuities and their types, what values should redefine the function to remove any removable discontinuities (give the extended function).
	\begin{equation*}
		f(x) = \frac{x^3-7x-6}{x^2-9}
	\end{equation*}
\end{example}
\begin{answer}
	Polynomials are continuous on their entire domain of all real numbers.
	So, rational functions like $f$ can only be discontinuous when the denominator is equal to $0$.
	This happens in two places: $x=3$ and $x=-3$.
	We'll check the limits from each side at each of these points to determine the type of discontinuity.
	For $x=3$,
	\begin{equation*}
		\lim_{x\to 3^+}{f(x)} = \lim_{x\to 3^-}{f(x)} = \lim_{x\to 3}{f(x)} = \lim_{x\to 3}{\frac{(x+2)(x+1)(x-3)}{(x+3)(x-3)}} = \lim_{x\to 3}{\frac{(x+2)(x+1)}{(x+3)}} = \frac{20}{6} = \frac{10}{3}.
	\end{equation*}
	
	So, $f$ has a removable discontinuity at $x=3$ because the left and right limits are the same.
	For $x=-3$,
	\begin{equation*}
		\lim_{x\to -3^+}{f(x)} = -\infty \text{ and } \lim_{x\to -3^+}{f(x)} = \infty.
	\end{equation*}
	
	So, $f$ has an infinite discontinuity at $x=-3$ because both of the left and right limits go to $\pm\infty$.
	The value we got from the limits at $x=3$ gives us the value we need to redefine $f$ as to remove the discontinuity.
	The extended function is therefore
	\begin{equation*}
		f_{e}(x) = \begin{cases}
			f(x) & x \neq 3 \\
			\frac{10}{3} & x = 3
		\end{cases}
	\end{equation*}
\end{answer}
\section{Continuity Properties}
These properties should look very similar to the properties of limits.
Let $f$ and $g$ be continuous functions at $c$.
\begin{align*}
	\textbf{Sum and Difference Rule: }& f \pm g \text{ is continuous at } c. \\
	\textbf{Product Rule: }& f \cdot g \text{ is continuous at } c. \\
	\textbf{Constant Multiple Rule: }& kf \text{ is continous at } c \text{ for all real } k. \\
	\textbf{Quotient Rule: }& \frac{f}{g} \text{ is continuous at } c \text{ as long as the value of the extended function of } \\
		g \text{ at } c \text{ is not } 0. \\
	\textbf{Composition Rule: }& f \circ g \text{ is continuous at } c \text{ if } f \text{ is continuous at } g(c). \\
	\textbf{Absolute Value Rule: }& \abs{f} \text{ is continuous at } c.
\end{align*}

The following types of functions are continuous on their domains
\begin{itemize}
	\item Polynomials
	\item Rational functions, except where the denominator is 0
	\item Trigonometric functions where defined
\end{itemize}

\begin{example}
	Show that the following function is continuous.
	\begin{equation*}
		f(x) = \tan{\left(\frac{x^2}{x^2+4}\right)}.
	\end{equation*}
\end{example}
We can write $f$ as the composition of $\tan{x}$ and $\frac{x^2}{x^2+4}$.
$\tan{x}$ is continuous on its domain because it is a trigonometric function.
The only points not in its domain are $(2n+1)\frac{\pi}{2}$, where $n$ is an integer.
$\frac{x^2}{x^2+4}$ is a rational function, but it's denominator is never $0$, so it is continuous over all real numbers.
Now, we just need to check that all points in the range of $\frac{x^2}{x^2+4}$ are in the domain of $\tan{x}$.
The range of $\frac{x^2}{x^2+4}$ is $[0,1)$.
None of the points in this interval are not in the domain of $\tan{x}$, so the composition is continuous over all real numbers.
\section{Intermediate Value Theorem}
\begin{theorem}[Intermediate Value Theorem (IVT)]
	If $f$ is continuous on the closed interval $[a,b]$, then for all $c \in [f(a), f(b)]$, there exists $x \in [a,b]$ such that $f(x) = c$.
\end{theorem}

That is, if $f$ is continuous on $[a,b]$, then $f$ must take on every value between $f(a)$ and $f(b)$.
This encapsulates the idea that if a function is continuous on some interval, then it is "connected" on that interval.

\begin{example}
	Use the IVT to show that $e^{-x} = x$ has at least one solution.
\end{example}
\begin{answer}
	Let $f(x) = e^{-x} - x$.
	We are looking for $x$ where $f(x) = 0$.
	$f(0) = 1$ and $f(1) = \frac{1}{e} - 1$.
	Since $f$ is continuous on the closed interval $[0,1]$, it must take on every value between $1$ and $\frac{1}{e} - 1$.
	Since $1$ is positive and $\frac{1}{e} - 1$ is negative, 0 is between these two values.
	Thus, by the IVT, there must exist a solution between $x=0$ and $x=1$.
\end{answer}
		\subsubsection{Derivatives}
The derivative of a function $y = f(x)$, notated $f^\prime(x)$, gives the slope of the tangent line to $f$ at $x$. It's defined as
\begin{equation*}
f^\prime(x) = \lim\limits_{h \to 0}{\frac{f(x+h) - f(x)}{h}}
\end{equation*}

\noindent
Below are some properties of the derivative. Let $f$ and $g$ be functions of $x$ and $p$ a scalar.
\begin{enumerate}[label=]
	\item Linearity
	\begin{equation*}
	\left( pf \pm g \right)^\prime = pf^\prime \pm g^\prime
	\end{equation*}
	\item Product Rule
	\begin{equation*}
	\left( fg \right)^\prime = f^\prime g + f g^\prime
	\end{equation*}
	\item Quotient Rule
	\begin{equation*}
	\left( \frac{f}{g} \right)^\prime = \frac{f^\prime g - f g^\prime}{g^2}
	\end{equation*}
	\item Chain Rule
	\begin{equation*}
	\left( f \circ g \right)^\prime = \left( f^\prime \circ g \right) \cdot g^\prime
	\end{equation*}
	\item Power Rule
	\begin{equation*}
	\dd{x} p^x = px^{p-1} \text{, } p \neq 0
	\end{equation*}
	\item Exponent Rule
	\begin{equation*}
	\dd{x} p^x = p^x \ln{p} \text{, } p > 0
	\end{equation*}
	\item The Power Rule and Exponent Rule are two cases of the same rule
	\begin{equation*}
	\dd{x} f^g = gf^{g-1}f^\prime + f^g\ln{f}g^\prime
	\end{equation*}
\end{enumerate}
Using the definition of the derivative and these rules, we can find the derivatives to some common functions
\begin{enumerate}[label=]
	\item \begin{equation*}
	\dd{x} p = 0
	\end{equation*}
	\item \begin{equation*}
	\dd{x} e^x = e^x
	\end{equation*}
	\item \begin{equation*}
	\dd{x} \ln{x} = \frac{1}{x}
	\end{equation*}
	\item \begin{equation*}
	\dd{x} \sin{x} = \cos{x}
	\end{equation*}
	\item \begin{equation*}
	\dd{x} \cos{x} = -\sin{x}
	\end{equation*}
	\item \begin{equation*}
	\dd{x} \tan{x} = \sec^2{x}
	\end{equation*}
\end{enumerate}
		\chapter{Applications of the Derivative}

\section{Extreme Values}
Extreme values of a function are the minimum and maximum values it attains on an interval.
The absolute extreme values would the the extreme values accross the function's entire domain.

\begin{example}
	Find the extreme values of $x^2$ over the following intervals.
	\begin{enumerate}
		\item $(-\infty, \infty)$
		\item $[0,2]$
		\item $(0,2]$
		\item $(0,2)$
	\end{enumerate}
\end{example}
\begin{enumerate}
	\item For the max, there is no maximum value because we can keep increasing $x$ to get a larger output.
		The min is 0 when $x=0$.
	\item The max is 4 when $x=2$.
		The min is 0 when $x=0$.
	\item The max is 4 when $x=2$.
		Although the function approaches 0 in the limit as $x$ approaches 0, there is no min because 0 itself is not a value $x^2$ can take on the interval.
	\item Although the function approaches 4 as $x$ approaches 2, there is no max because 4 itself is not a value $x^2$ can take on the interval.
		Like in the previous question, there is no min.
\end{enumerate}
We see that a function can fail to have a max or min value, but this cannot happen on a finite, closed interval.

\begin{theorem}[Extreme Value Theorem]
	If $f$ is continuous on some finite, closed interval $[a,b]$, then $f$ must have both a minimum and maximum value on the interval.
\end{theorem}

\noindent
We can find these extreme values by following these steps.
\begin{enumerate}
	\item Find any relative/local minima and maxima.
	\item Find the function values for these local minima and maxima.
	\item Find the function values at the endpoints of the interval, $a$ and $b$.
	\item The smallest such function value will be the absolute minima, while the largest such function value will be the absolute maxima.
\end{enumerate}

\begin{theorem}
	If a function $f$ has a local extrema at a point $c$ interior to its domain and $f^\prime$ exists at $c$, then
	\begin{equation*}
		f^\prime(c) = 0.
	\end{equation*}
\end{theorem}
\noindent
So, the only candidate values we need to check are the endpoints and where the derivative is 0.
Such points are called critical points.

\begin{example}
	Find the absolute extrema of $y = x^3 + x^2 - 8x + 5$ on the interval $[-3,2]$.
\end{example}
\begin{enumerate}
	\item We'll take the derivative to find all the critical points.
			$y^\prime = 3x^2 + 2x - 8$, which is 0 when $x=-2$ and $x=4/3$.
	\item $(-2)^3 + (-2)^2 - 8(-2) + 5 = 17$ and $(4/3)^3 + (4/3)^2 - 8(4/3) + 5 = -41/27$.
	\item $(-3)^3 + (-3)^2 - 8(-3) + 5 = 11$ and $(2)^3 + (2)^2 - 8(2) + 5 = 1$.
	\item Of these values, $(-2,17)$ is the absolute maxima and $(4/3, -41/27)$ is the absolute minima.
\end{enumerate}
\section{Mean Value Theorem for Derivatives}
\begin{theorem}[Mean Value Theorem for Derivatives]
	If $f$ is continuous on the interval $[a,b]$ and differentiable on the interval $(a,b)$, then there exists at least one point in $(a,b)$ such that
	\begin{equation*}
		f^\prime(c) = \frac{f(b)-f(a)}{b-a}.
	\end{equation*}
\end{theorem}

That is, there's at least one point where the instantaneous rate of change and average rate of change are equal.
Another way of visualizing this is that there's at least one point where the tangent and secant lines are parallel.

\begin{figure}[H]
	\label{mvt_derivatives}
	\centering
	\includegraphics[width = 0.5\textwidth]{./applications_derivative/mvt.png}
	\caption{\hyperref{https://en.wikipedia.org/wiki/Mean\_value\_theorem}{}{}{Wikipedia - Mean Value Theorem}}
\end{figure}

\begin{example}
	A trucker drives 150 miles of a route in 2 hours.
	The speed limit along the route is 65 miles per hour.
	Show that at at least one point, the trucker must have been speeding.
\end{example}
\begin{answer}
	We can model the trucker's position along the route $s$ as a function of time $t$ where $s(0)=0$ and $s(2)=150$.
	We know that velocity is the derivative of position, so $v(t) = s^\prime(t)$.
	It's reasonable to assume that $s$ is differentiable on the interval $[0,2]$.
	So, by the Mean Value Theorem, there must exist a point $c$ where
	\begin{equation*}
		v(t) = \frac{s(2)-s(0)}{2-0} = \frac{150-0}{2} = 75\text{mph}.
	\end{equation*}
	At this point, the trucker was exceeding the speed limit of 65 miles per hour.
\end{answer}

\begin{definition}
	Let $f$ be defined on an interval $I$.
	Let $a$ and $b$ be any two different points in $I$.
	\begin{align*}
		\text{$f$ increases on $I$ if } a < b &\implies f(a) < f(b). \\
		\text{$f$ decreases on $I$ if } a < b &\implies f(a) > f(b).
	\end{align*}
\end{definition}

\begin{corollary}
	Let $f$ be continuous of $[a,b]$ and differentiable on $(a,b)$.
	\begin{align*}
		\text{If $f^\prime > 0$ at every point on $(a,b)$ then $f$ increases on $[a,b]$}. \\
		\text{If $f^\prime < 0$ at every point on $(a,b)$ then $f$ decreases on $[a,b]$}.
	\end{align*}
\end{corollary}

This should make sense given our theorem about local extrema.
If $f$ could still increase/decrease while its derivative was negative/positive, then we couldn't be sure that $f$ is at a local maxima/minima when $f^\prime=0$.

\begin{corollary}
	If $f^\prime(x) = 0$ at all points in an interval $I$, then there is some constant $C$ such that $f(x) = C$ for all points in $I$.
\end{corollary}

This follows from the Mean Value Theorem.
Since $f^\prime = 0$, the numerator in the Mean Value Theorem, $f(b) - f(a)$, must also be 0, meaning $f(b) = f(a) = C$.

\begin{corollary}
	If $f^\prime(x) = g^\prime(x)$ at every point in some interval $I$, then there is come constant $C$ such that $f(x) = g(x) + C$.
\end{corollary}

That is, functions with the same derivative differ by a constant.
This should make sense given our constant and sum and difference derivative rules.
If we let $h^\prime(x) = f^\prime(x) - g^\prime(x) = 0$ and apply the previous corollary, we get $C$.

\begin{definition}
	A function $F(x)$ is the antiderivative	of $f(x)$ if $F^\prime(x) = f(x)$ for all points in the domain of $f$.
\end{definition}

As we saw in the previous corollary, a function will have infinitely many antiderivatives that differ by a constant.
\section{First and Second Derivative Tests}
\subsection{First Derivative Test}
As we saw in previous sections, we can use the first derivative to find critical values, which allow us to find extrema.
\begin{theorem}[First Derivative Test]
	Let $f(x)$ be a continuous function.
	At a critical point $c$,
	\begin{enumerate}
		\item If $f^\prime$ changes sign from positive to negative ($f^\prime(x) > 0$ for $x < c$ and $f^\prime(x) < 0$ for $x > c$), then $f$ has a local maximum at $c$.
		\item If $f^\prime$ changes sign from negative to positive ($f^\prime(x) < 0$ for $x < c$ and $f^\prime(x) > 0$ for $x > c$), then $f$ has a local minimum at $c$.
		\item If $f^\prime$ does not change sign at $c$, then $f$ does not have a local extrema at $c$.
		\item At a left endpoint $a$, if ($f^\prime < 0$ / $f^\prime > 0$), then $f$ has a local (maximum / minimum) at $a$.
		\item At a right endpoint $b$, if ($f^\prime < 0$ / $f^\prime > 0$), then $f$ has a local (minimum / maximum) at $b$.
	\end{enumerate}
\end{theorem}

\begin{example}
	Find the local extrema of $f(x) = x^3 - 12x - 5$.
	Identify any absolute extrema.
\end{example}
Taking the derivative,
\begin{equation*}
	f^\prime(x) = 3x^2 - 12 = 3(x+2)(x-2).
\end{equation*}
\indent
So, the critical values are $x=-2$ and $x=2$.
$f^\prime$ is negative between these values and positive outside of them, so $x=-2$ is a local maximum, while $x=2$ is a local minimum.
\begin{equation*}
	f(-2) = 11 \text{ and } f(2) = -21.
\end{equation*}
\indent
As $x$ approaches $\infty$ (the right endpoint), $f$ also approaches $\infty$, so there is no absolute maximum.
Similarly, as $x$ approaches $-\infty$ (the left endpoint), $f$ also approaches $-\infty$, so there is no absolute minimum.

\subsection{Second Derivative Test}
The second derivative tells us how the derivative is changing.
Whether the derivative is increasing or decreasing describes the concavity.
\begin{definition}
	On some open interval $I$, the graph of a twice differentiable function $f(x)$ is
	\begin{enumerate}
		\item Concave up if $f^{\prime\prime} > 0$ on $I$.
		\item Concave down if $f^{\prime\prime} < 0$ on $I$.
	\end{enumerate}
\end{definition}

\noindent
Concave up portions of a graph tend to look like valleys, while concave down portions tend to look like hills.
Unlike the first derivative, we can tell if a function is increasing or decreasing using concavity.
We can however say if the function is increasing or decreasing more or less rapidly.

\begin{figure}[H]
	\label{mvt}
	\centering
	\includegraphics[width = 0.5\textwidth]{./applications_derivative/concavity.png}
	\caption{\hyperref{https://tutorial.math.lamar.edu/classes/calci/shapeofgraphptii.aspx}{}{}{Paul's Online Notes - The Shape of a Graph, Part II}}
\end{figure}

\noindent
The points where a function changes concavity are called "inflection points".
At these points, the function is increasing or decreasing most rapidly, depending on the sign of the first derivative.

\noindent
Rather than looking at the sign of $f^\prime$ around critical values, we can look at the concavity at the critical point.
\begin{theorem}[Second Derivative Test]
	Let $c$ be a critical value of $f$.
	\begin{enumerate}
		\item If $f^{\prime\prime}(c) < 0$, then $c$ is a local maximum.
		\item If $f^{\prime\prime}(c) > 0$, then $c$ is a local minimum.
		\item If $f^{\prime\prime}(c) = 0$, then the test is inclusive.
	\end{enumerate}
\end{theorem}
\noindent
In other words, if we're at a critical point that's turning into a valley, then the critical point must be the top of a hill.
If we're at a critical point that's turning into a hill, then the critical point must be the bottom of a valley.
This test is particularly useful because you only need to know $f^{\prime\prime}$ at $c$ rather than an entire interval\footnote{It also extends to higher dimensions much better than the first derivative test.}.

\begin{example}
	Use the second derivative test to find the local extreme values of $f(x) = x^3 - 12x - 5$.
\end{example}
Finding critical values,
\begin{align*}
	f^\prime(x) &= 3x^2 - 12 = 3(x+2)(x-2). \\
	f^\prime(x) &= 0\text{ at } x=-2, x=2.
\end{align*}
\indent
Taking the second derivative,
\begin{equation*}
	f^{\prime\prime}(x) = 6x.
\end{equation*}
\indent
Evaluating the second derivative at the critical values,
\begin{equation*}
	f^{\prime\prime}(-2) = -12 \text{ and } f^{\prime\prime}(2) = 12.
\end{equation*}
\indent
So, $x=-2$ is a local maximum and $x=2$ is a local minimum.
\section{Modeling \& Optimization}
Now that we know some ways to find extreme values, we can apply them to answer optimization problems.
The general steps needed to solve an optimization problem are:
\begin{enumerate}
	\item Write an equation that represents what you're trying to maximize/minimize. This is called your primary equation.
	\item Use additional information to eliminate excess variables.
	\item Find extreme values.
	\item Select the extreme values that fit the problem's constraints. Make sure your answer is what the problem is asking for.
\end{enumerate}

\begin{example}
	A farmer has 1000 linear feet of fence and wants to create a rectangular pasture.
	The pasture borders a river, which doesn't need a fence.
	What is the maximum area he can enclose?
\end{example}
\begin{answer}
	\begin{enumerate}
		\item Since the pasture is rectangular, we know the two lengths of fence perpendicular to the river will be the same length, which we'll call $x$.
			We'll say the remaining side parallel to the river has length $y$.
			So, the area enclosed is
			\begin{equation*}
				A(x,y) = xy.
			\end{equation*}
		\item There is no reason for the farmer not to use all 1000 linear feet of fence, so we'd expect the sum of the lengths of the 3 fenced sides to be equal to 1000 feet: $2x + y = 1000$, or $y = 1000 - 2x$.
			We can then substitute back into our primary equation to get it in terms of just $x$.
			\begin{equation*}
				A(y) = x(1000-2x) = 1000x - 2x^2.
			\end{equation*}
		\item We'll do a first derivative test to find extreme values.
		\begin{align*}
			A^\prime(y) &= 1000 - 4x \\
			A^\prime(y) &= 0 \text{ at } x = 250 \\
			A^\prime(200) &= 200 > 0 \\
			A^\prime(300) &= -200 < 0.
		\end{align*}
		\item So, $x=250 \implies y = 500$ is a local maximum.
			This corresponds to an area of $A = 250\cdot 500 = 125000\text{ft}^2$.
	\end{enumerate}
\end{answer}

\begin{example}
	What is the maximum area of a rectangle that has two vertices on the $x-axis$ and two vertices on the portion of the graph $y=8-x^2$ where $y > 0$?
\end{example}
\begin{answer}
	\begin{enumerate}
		\item We can define any such trapeziod (which include all such rectangles) by the $x$ values of the vertices on the $x$ axis.
			We'll call these two values $x_1$ and $x_2$.
			We'll say arbitrarily that $x_1 < x_2$.
			So, the area enclosed is
			\begin{equation*}
				A(x_1, x_2) = (x_2 - x_1)\frac{f(x_1) + f(x_2)}{2}.
			\end{equation*}
		\item However, the problem specifically restricts us to a rectangle, not a trapezoid.
			So, the heights of each side must be equal.
			\begin{align*}
				f(x_1) &= f(x_2) \\
				8 - x_1^2 &= 8 - x_2^2 \\
				x_1^2 &= x_2^2 \\
				\pm x_1 &\ \pm x_2 \\
				-x_1 &= x_2 \text{ because $x_1 \neq x_2$}.
			\end{align*}
			Substituting back into our primary equation,
			\begin{equation*}
				A(x_2) = (x_2 - (-x_2))\frac{f(x_2) + f(-x_2)}{2} = 2x_2f(x_2) = 16x_2 - 2x_2^3.
			\end{equation*}
		\item We'll do a second derivative test for fin extreme values.
			\begin{align*}
				A^\prime(x_2) &= 16 - 6x_2^2 \\
				A^\prime(x_2) &= 0 \text{ at } x = \pm\frac{4}{\sqrt{6}} \\
				A^{\prime\prime}(x_2) &= -12x_2 \\
				A^{\prime\prime}\left(-\frac{4}{\sqrt{6}}\right) &= 8\sqrt{6} > 0 \\
				A^{\prime\prime}\left(\frac{4}{\sqrt{6}}\right) &= -8\sqrt{6} < 0.
			\end{align*}
		\item So, $x_2 = 4/\sqrt{6}$ is a local maximum, and $x_2 = -4/\sqrt{6}$ is a local minimum.
			The problem asks for a maximum, so we select $x_2 = 4/\sqrt{6}$.
			The problem asks for a maximum area, so
			\begin{equation*}
				A_{max} = 16\left(\frac{4}{\sqrt{6}}\right) - 2\left(\frac{4}{\sqrt{6}}\right)^3 = \frac{128}{3\sqrt{6}} \approx 17.419.
			\end{equation*}
	\end{enumerate}
\end{answer}
\section{Related Rates}
In related rates problems, we generally have two related functions and want to know and answer questions about the rate of change of one function given that we know the rate of change of the other. The same problem solving steps as in modeling and optimization apply, but we'll usually be taking derivatives using implicit differentiation with respect to some variable like time.

\begin{example}
	Let $A$ be the area of a square with side length $x$.
	Assume that $x$ varies with time.
	How are $\dd{A}{t}$ and $\dd{x}{t}$ related?
	At a certain instant, the sides are 3 feet and growing at a rate of 2 feet per minute.
	How quickly is the area changing at this instant.
\end{example}
\begin{answer}
	Starting with the area of the square and implicitly differentiating with respect to $t$,
	\begin{align*}
		A &= x^2 \\
		\dd{A}{t} = 2x\dd{x}{t}.
	\end{align*}
	
	When $x=3\text{ft}$ and $\dd{x}{t}=3\text{ft/min}$,
	\begin{equation*}
		\dd{A}{t} = 2(3\text{ft})(3\text{ft/min}) = 12\text{ft$^2$/min}.
	\end{equation*}
\end{answer}

\begin{example}
	The top of a 13 foot ladder propped against a vertical wall begins falling towards the ground at 12ft/s.
	When the top of the ladder is 5 feet off the ground, how quickly is the bottom of the ladder moving away from the wall?
	How how is the angle between the ladder and the ground changing?
\end{example}
\begin{answer}
	Let $h$ be the height of the top of the ladder of the ground.
	Then $\dd{h}{t} = -12\text{ft/s}$.
	Let $b$ the the distance from the base of the ladder to the wall.
	We can relate $b$ and $h$ using the Pythagorean Theorem, where the 13-foot long ladder is the hypotenuse.
	\begin{equation*}
		b^2 + h^2 = 13^2.
	\end{equation*}
	
	We can also use this relationship to see that when $h=5\text{ft}$, $b=12\text{ft}$.
	Implicitly differentiating,
	\begin{equation*}
		2b\dd{b}{t} + 2h\dd{h}{t} = 0.
	\end{equation*}
	
	Plugging in what we know and solving for $\dd{b}{t}$,
	\begin{align*}
		2(12\text{ft})\dd{b}{t} + 2(5\text{ft})(-12\text{ft/s}) &= 0 \\
		24\text{ft}\dd{b}{t} &= 120\text{ft$^2$/s} \\
		\dd{b}{t} &= 5\text{ft/s}.
	\end{align*}
	
	So, the base of the ladder is moving away from the wall at a rate of 5ft/s.
	Let $\theta$ be the angle between the ladder and the ground.
	We can use $\sin$ to relate $\theta$ to $b$.
	\begin{equation*}
		13\sin{\theta} = b.
	\end{equation*}
	
	Implicitly differentiating,
	\begin{equation*}
		13\text{ft}\cos{(\theta)}\dd{\theta}{t} = \dd{b}{t}.
	\end{equation*}
	
	When $\cos$ is adjacent divided by hypotenuse, so $\cos{\theta} = 12/13$.
	Plugging in what we know and solving for $\dd{\theta}{t}$,
	\begin{align*}
		13\text{ft}(12/13)\dd{\theta}{t} &= -12\text{ft/s} \\
		\dd{\theta}{t} &= -1\text{/s}.
	\end{align*}
	
	So, the angle between the ladder and ground is decreasing at at rate of 1 rad/s.
\end{answer}

\begin{example}
	Grain is is poured at a rate of 10ft$^3$/min and falls into a cone-shaped pile whose bottom radius is half its altitude.
	How fast will the circumference of the base be increasing when the pile is 8 ft tall?
\end{example}
\begin{answer}
	Let $h$ the the altitude of the cone.
	At the instant we care about $h=8\text{ft}$.
	Let $r$ be the bottom radius of the cone.
	At the instant we care about, $r=h/2=4\text{ft}$.
	Let $V$ be the volume of the cone.
	We know that $\dd{V}{t}=10\text{ft$^3$/s}$.
	We can relate these three quantites using the formula for the volume of a cone.
	\begin{equation*}
		V = \frac{1}{3}\pi r^2 h.
	\end{equation*}
	
	Since we know that $2r = h$, we can simplify to get rid of $h$.
	\begin{equation*}
		V = \frac{2}{3}\pi r^3
	\end{equation*} 
	
	Implicitly differentiating,
	\begin{equation*}
		\dd{V}{t} = 2\pi r^2 \dd{r}{t}.
	\end{equation*}
	
	We know the formula for the circumference $C$ of the circular base.
	\begin{equation*}
		C = 2\pi r.
	\end{equation*}
	
	Implicitly differentiating,
	\begin{equation*}
		\dd{C}{t} = 2\pi \dd{r}{t}.
	\end{equation*}
	
	We can substitute into our equation involving $\dd{V}{t}$.
	\begin{equation*}
		\dd{V}{t} = r^2\dd{C}{t}.
	\end{equation*}
	
	Plugging in what we know,
	\begin{align*}
		10\text{ft}^3\text{/s} &= (4\text{ft})^2\dd{C}{t} \\
		\dd{C}{t} &= \frac{5}{8}\text{ft/s}.
	\end{align*}
\end{answer}

\section{Linearization \& Newton's Method}

\subsection{Linearization}
As we've seen, tangent lines intersect their function at most once: at the point of tangency.
However, we know that differentiable functions are locally linear, so we'd expect the tangent line to be a decent approximation of the function near the point of tangency.

\begin{definition}
	If $f$ is differentiable at $a$, then the approximating function
	\begin{equation*}
		L(x) = f^\prime(a)(x-a) + f(a)
	\end{equation*}
	is the linearization of $f$ at $a$.
\end{definition}

\begin{example}
	Find the linearization of $f(x) = \ln{(x+1)}$ at $x=0$.
	How accurate is this approximation at $x=0.1$?
\end{example}
Following the definition,
\begin{align*}
	f(0) = \ln{(0+1)} = 0 \\
	f^\prime(x) &= \frac{1}{x+1} \\
	f^\prime(0) &= \frac{1}{0+1} = 1 \\
	L(x) &= 1(x-0) + 0 = x.
\end{align*}
\indent
Calculating the error at $x=0.1$,
\begin{align*} 
	L(0.1) &= 0.1 \\
	f(0.1) &\approx .0953 \\
	\text{\% error} &= \frac{\abs{L(0.1)-f(0.1)}}{L(0.1)}100\text{\%} \approx 4.7\text{\%}.
\end{align*}
\indent
So, we can see the linear approximation is pretty good.

\noindent
We call the difference in $x$ between the point of tangency and the point we're trying to approximate the differential.
\begin{definition}
	Let $y=f(x)$ be a differentiable function.
	The differential $\mathrm{d}x$ is an independent variable.
	The differential $\mathrm{d}y$ is $\mathrm{d}y = f^\prime(x)\mathrm{d}x$.
\end{definition}
\noindent
$\mathrm{d}y$ is the approximated change in $y$ expected by the linearization for some given change in $x$, $\mathrm{d}x$.

\begin{example}
	Find $\mathrm{d}y$ for $y=\frac{2x}{1+x^2}$, $x=-2$, and $\mathrm{d}x = 0.1$.
\end{example}
\begin{align*}
	y^\prime &= \frac{-4x^2}{\left(1+x^2\right)^2} + \frac{2}{1+x^2} \\
	y^\prime(-2) &= -6/25 \\
	\mathrm{d}y &= (-6/25)(0.1) = -0.024.
\end{align*}

\subsection{Newton's Method}
We can use the fact that the tangent line approximates the function to find the zeroes of functions.
Starting with an initial guess $x_0$ for the $x$ value of the zero, we look at the the tangent line at $x_0$ and find where it intersects the $x-axis$.
\begin{align*}
	L_0(x) &= f^\prime(x_0)(x - x_0) + f(x_0) \\
	0 &= f^\prime(x_0)(x - x_0) + f(x_0) \\
	-f^\prime(x_0)(x - x_0) &= f(x_0) \\
	x - x_0 &= -\frac{f(x_0)}{f^\prime(x_0)} \\
	x &= x_0 - \frac{f(x_0)}{f^\prime(x_0)}.
\end{align*}
This $x$ value serves as our next guess for the zero.
We repeat this process, until we find the zero or are satisfied with our error\footnote{For most well-behaved functions, Newton's Method can get within a small margin of error or a zero relatively quickly. There is also a generalized, sometimes faster version of Newton's Method that approximates the function with higher-order polynomials than just lines.}.
This yields a recursive formula
\begin{equation*}
	x_{n+1} = x_n - \frac{f(x_n)}{f^\prime(x_n)}.
\end{equation*}
		\chapter{Integrals}

\section{Estimating with Sums}
If we know how something changes, we can use sums to estimate, or sometimes even know exactly the net change.

\begin{example}
	A train moves at 80 miles per hour for 3 hours.
	How far does it travel?
\end{example}
\begin{answer}
	This is the kind of simple problem you might see in a physics class.
	\begin{equation*}
		\frac{80\text{mi}}{\text{hr}} \hspace{3pt} 3\text{hr} = 80\cdot 3\text{mi} = 240 \text{mi}.
	\end{equation*}
	However, if we take a look at a graph of this situation, with speed in miles per hour on the $y$ axis and time in hours on the $x$ axis, we see that the area underneath the curve from $x=0\text{hr}$ to $x=3\text{hr}$ is exactly our answer of 240 miles.
	This is not a coincidence, as what we effectively did mathematically is find the area of this rectangle.
	
	\begin{figure}[H]
		\label{constant_graph}
		\centering
		\includegraphics[width = 0.33\textwidth]{./integrals/constant_graph.png}
		\caption{\hyperref{}{}{}{Our answer is the area under the curve.}}
	\end{figure}
	\end{answer}

The same idea of finding the area under the curve works not just for constant speeds.
\begin{example}
	A particle moves at velocity $v(t) = 3t + 3$ meters per second for time $t \geq 0$.
	What is the position of the particle at $t = 2$ seconds?
\end{example}
\begin{answer}
	\begin{figure}[H]
		\label{linear_graph}
		\centering
		\includegraphics[width = 0.5\textwidth]{./integrals/linear_graph.png}
		\caption{\hyperref{}{}{}{Our answer is still the area under the curve.}}
	\end{figure}
	\begin{align*}
		A &= \frac{1}{2}h(b_1 + b_2) \\
		&= \frac{1}{2}(2\text{s}-0\text{s})(v(0) + v(2)) \\
		&= \frac{1}{2}(2\text{s})(3\text{m/s} + 9\text{m/s}) \\
		&= 12\text{m}.
	\end{align*}
\end{answer}

\subsubsection{Left Endpoint Approximation}
In fact, the area under the curve even works for more complicated, non-linear curves like $v(t) = t^2 + 1$.
We just need a way to find the area underneath these curves.
One idea that was used to find areas as far back as Archimedes was to estimate the complex shape using easier shapes like rectangles.
The narrower the width of each rectangle, the better the estimate becomes.
 
\begin{figure}[H]
	\label{cos_blocks}
	\centering
	\includegraphics[width = 0.3\textwidth]{./integrals/cos_blocks1.png}
	\includegraphics[width = 0.3\textwidth]{./integrals/cos_blocks2.png}
	\includegraphics[width = 0.3\textwidth]{./integrals/cos_blocks3.png}
	\caption{\hyperref{}{}{}{Left Endpoint: Block widths of 1/3, 1/10, and 1/100.}}
\end{figure}


Although above we've used a left endpoint approximation, we could have also used the right endpoint or midpoint, which might give better approximations for certain types of curves.
No matter the approximation type, the estimate tends to get closer to the true area as the rectangle width is decreased.
The formulas for estimating the area of $f$ from $a$ to $b$ with $n$ rectangles are
\begin{align*}
	A_\text{left} &= \sum_{k=0}^{n-1}{f\left(a+k\Delta x\right)\Delta x} \\
	A_\text{right} &= \sum_{k=0}^{n-1}{f\left(a+(k+1)\Delta x\right)\Delta x} \\
	A_\text{mid} &= \sum_{k=0}^{n-1}{f\left(a+\frac{2k+1}{2}\Delta x\right)\Delta x} \\
	\text{where }\Delta x &= \frac{b-a}{n}.
\end{align*}

\subsubsection{Trapezoidal Rule}
Another common shape to use rather than rectangles is trapezoids.
Trapezoids allow us to find the exact area of functions made of straight lines\footnote{There are higher-order, more accurate estimations than the trapezoidal rule. The most common is Simpson's Rule. $A = \frac{2M+T}{3}$ where $M$ is the midpoint formula area and $T$ is the trapezoidal rule area. It can exactly give the area of quadratics because it corresponds to estimating areas using parabolas that intersect the curve at the left, middle, and right of each "strip".} and like the left endpoint approximation, get better the narrower the width of each trapezoid.
Starting from the formula, we can make some simplifications.
\begin{equation*}
	A = \frac{1}{2}(b_1 + b_2)h
\end{equation*}
Now summing each of these trapezoids' areas to approximate our function $f$ from $a$ to $b$,
\begin{align*}
	A_\text{trap} &= \sum_{k=0}^{n-1}{\frac{1}{2}\left(f(a+i\Delta a) + f(a + (i+1)\Delta x)\right)\Delta x} \\
	&= \frac{\Delta x}{2}\sum_{k=0}^{n-1}{f(a + i\Delta x) + f(a + (i+1)\Delta x)} \\
	&= \frac{\Delta x}{2}\left(\left(f(a)+f(a+\Delta x)\right)+\left(f(a+\Delta x)+f(a+2\Delta x)\right)+\ldots+\left(f(a+(n-1)\Delta x)+f(a+n\Delta x)\right)\right) \\
	&= \frac{\Delta x}{2}\left(f(a) + 2f(a+\Delta x) + 2f(a + 2\Delta x) + \ldots + 2f(a+(n-1)\Delta x) + f(a+n\Delta x)\right) \\
	&= \frac{A_\text{left} + A_\text{right}}{2}.
\end{align*}

The trapezoidal rule overestimates areas when the curve is concave up and underestimates when the curve is concave down.
\section{Definite Integrals \& Antiderivatives}

\subsection{Definition of a Definite Integral}
We saw several approximations for the areas under any type of curve.
We also saw how these approximations get better the narrower our "strips".
In the limit, we get exactly the area under the curve, which defines the definite integral.

\begin{definition}
	Let $f$ be continuous on the closed interval $[a,b]$.
	Let this interval be partitioned into $n$ equal\footnote{Technically, the intervals don't need to be of equal size, as long as the width of all intervals goes to 0 in the limit.} sub-intervals, each of length $\Delta x = \frac{b-a}{n}$.
	The definite integral of $f$ over $[a,b]$ is given by
	\begin{equation*}
		\int_{a}^{b}{f(x)\mathrm{d}x} = \lim_{n\to \infty}{\sum_{i=0}^{n-1}{f(c_i)\Delta x}}
	\end{equation*}
	where $c_k$ is any arbitrary value in the $k$th sub-interval.
	This particular type of limit of a sum is called a Riemann Sum.
\end{definition}
\noindent
Note that the left, right, and midpoint approximations are all different ways of choosing $c_i$, all of which in the limit give the area under the curve and definite integral from $a$ to $b$.

\begin{example}
	Find the value of the following definite integral using Riemann sums.
	\begin{equation*}
		\int_{0}^{1}{x^2\d{x}}.
	\end{equation*}
\end{example}
We'll use a Riemann sum with a left-endpoint approximation.
\begin{align*}
	\int_{0}^{1}{x^2\d{x}} &= \lim_{n\to\infty}{\sum_{i=0}^{n-1}{f\left(0+\frac{i}{n}\right)\frac{1-0}{n}}} \\
	&= \lim_{n\to\infty}{\sum_{i=0}^{n-1}{\left(\frac{i}{n}\right)^2\frac{1}{n}}} \\
	&= \lim_{n\to\infty}{\sum_{i=0}^{n-1}{\frac{i^2}{n^3}}} \\
	&= \lim_{n \to \infty}{\frac{1}{n^3}\left(\frac{n(n-1)(2n-1)}{6}\right)} \\
	&= \lim_{n\to\infty}{\frac{1}{6}\left(2-\frac{3}{n}+\frac{1}{n^2}\right)} \\
	&= \frac{1}{3}.
\end{align*}

\noindent
Note that area below the $x$-axis is counted as negative area.
In general
\begin{equation*}
	\int_{a}^{b}{f(x)\d{x}} = \text{ area above $x$-axis } - \text{ area below $y$-axis}.
\end{equation*}

\subsection{Basic Properties of Definite Integrals}
The following are properties of definite integrals.
Many should look familiar from properties of limits and derivatives.
Let $f$ and $g$ be continuous functions of $x$.
Let $a$, $b$, and $c$ be real constants where $a \neq b$.
Let $f_{[a,b]}$ be the values of $f$ on the interval $[a,b]$.
\begin{align*}
	\textbf{Order of Integration Rule: }& \int_{a}^{b}{f(x)\d{x}} = -\int_{b}^{a}{f(x)\d{x}} \\
	\textbf{Sum and Difference Rule: }& \int_{a}^{b}{(f(x) \pm g(x))\d{x}} = \int_{a}^{b}{f(x)\d{x}} \pm \int_{a}^{b}{g(x)\d{x}} \\
	\textbf{Zero Rule: }& \int_{a}^{a}{f(x)\d{x}} = 0 \\
	\textbf{Constant Multiple Rule: }& \int_{a}^{b}{cf(x)\d{x}} = c\int_{a}^{b}{f(x)\d{x}} \\
	\textbf{Additivity Rule: }& \int_{a}^{b}{f(x)\d{x}} + \int_{b}^{c}{f(x)\d{x}} = \int_{a}^{c}{f(x)\d{x}} \\
	\textbf{Max-Min Rule: }& (b-a)\min{f_{[a,b]}} \leq \int_{a}^{b}{f(x)\d{x}} \leq (b-a)\max{f_{[a,b]}} \\
	\textbf{Domination Rule: }& \min{f_{[a,b]}} \geq \max{g_{[a,b]}} \implies \int_{a}^{b}{f(x)\d{x}} \geq \int_{a}^{b}{g(x)\d{x}}.
\end{align*}

\subsubsection{Mean Value Theorem for Definite Integrals}
The mean value of $f$ over the interval $[a,b]$ is given by
\begin{equation*}
	\frac{1}{b-a}\int_{a}^{b}{f(x)\d{x}}.
\end{equation*}
You can effectively think of this as the definition of the average: adding up all the values and dividing by the number of values.
This is exactly what happens if you work through the Riemann sums.
Much like the Mean Value Theorem for Derivatives, since the function is continuous, it will take on its average value somewhere in the interval.
\begin{theorem}[Mean Value Theorem for Definite Integrals]
	If $f$ is continuous on the interval $[a,b]$, then there is some point $c$ in the interval such that
	\begin{equation*}
		f(c) = \frac{1}{b-a}\int_{a}^{b}{f(x)\d{x}}.
	\end{equation*}
\end{theorem}

\begin{figure}[H]
	\label{mvt}
	\centering
	\includegraphics[width = 0.33\textwidth]{./integrals/mvt.png}
	\caption{\hyperref{}{}{}{A continuous function always takes on its mean value}}
\end{figure}

\begin{example}
	A plane's airspeed is given by $v(t) = t/8\text{, }t\geq 0$.
	What what time $t$ does the Mean Value Theorem guarantee that the plane's speed was 1?
	When is this time?
\end{example}
Rather than go through the trouble of evaluating a limit to find the definite integral, we can notice that the graph of the plane's airspeed is a triangle, which we know how to find the area of geometrically.
\begin{equation*}
	\text{Area}(t) = \frac{1}{2}tv(t).
\end{equation*}
\indent
Dividing by the length of the interval to get the average airspeed,
\begin{equation*}
	\text{Avg}(t) = \frac{1}{t-0}\hspace{3pt}\frac{1}{2}tv(t) = \frac{1}{2}v(t).
\end{equation*}
\indent
Solving for $t$ when $\text{Avg}(t)=1$,
\begin{equation*}
	\frac{1}{2}\hspace{3pt}\frac{t}{8} = 1 \implies t = 16.
\end{equation*}
\indent
So, at time $t=16$, the Mean Value Theorem tells us that the plane's speed was 1.
Looking at the graph, we can see that at time $t=8$ the plane's speed was indeed 1.

\subsection{Fundamental Theorem of Calculus}
Although it's nice to be able to evaluate definite integrals for numerical bounds, it'd be more convenient if we only had to do the work of integrating once and could then have a function that would tell us the area like so:
\begin{equation*}
	F(x) = \int_{a}^{x}{f(t)\d{t}}.
\end{equation*}
Let's try taking the derivative of $F$ using the limit definition.
\begin{align*}
	F^\prime(x) &= \lim_{\Delta x \to 0}{\frac{F(x+\Delta x)-F(x)}{\Delta x}} \\
	&= \lim_{\Delta x\to 0}{\frac{\int_{a}^{x+\Delta x}f(t)\d{t} - \int_{a}^{x}{f(t)\d{t}}}{\Delta x}} \\
	&= \lim_{\Delta x\to 0}{\frac{\int_{x}^{x+\Delta x}{f(t)\d{t}}}{\Delta x}}\text{ (by Integral Rules)}\\
	&= \lim_{\Delta x\to 0}{\frac{f(k)\Delta x}{\Delta x}}\text{, }x\leq k \leq x + \Delta x\text{ (by Mean Value Theorem)} \\
	&= \lim_{\Delta x\to 0}{f(k)} \\
	&= f(x) \text{ (by Sandwich Theorem)}.
\end{align*}
Taking the derivative seems to undo the integration.
The same fact applies if we take the integral of a derivative.
Starting with what we've just shown,
\begin{equation*}
	\dd{}{x}\int_{0}^{x}{f(t)\d{t}} = f(x).
\end{equation*}
Let $g(x) = \dd{}{x}f(x)$.
\begin{equation*}
	\dd{}{x}\int_{0}^{x}{g(t)\d{t}} = g = \dd{}{x}f(x).
\end{equation*}
Since the derivatives are equal, we know the functions must differ by at most a constant $C$.
\begin{align*}
	\int_{0}^{x}{g(t)\d{t}} &= f(x) + C \\
	\int_{0}^{x}{\left[\dd{}{x}f(t)\right]\d{t}} &= f(x) + C.
\end{align*}
So, the integral of the derivative gets us a function that differs from the original by at most a constant.

\noindent
This idea that the integral and derivative undo each other is captured by the Fundamental Theorem of Calculus.
\begin{theorem}[Fundamental Theorem of Calculus]
	Let $f$ be a continuous function on the interval $[a,b]$.
	Then
	\begin{equation*}
		F(x) = \int_{a}^{x}{f(t)\d{t}}
	\end{equation*}
	has a derivative at every point in $[a,b]$, and
	\begin{equation*}
		\dd{F}{x} = \dd{}{x}\int_{a}^{x}{f(t)\d{t}} = f(x).
	\end{equation*}
	Further, if $F$ is the antiderivative of $f$ on $[a,b]$, then
	\begin{equation*}
		\int_{a}^{b}{f(x)\d{x}} = F(b) - F(a).
	\end{equation*}
\end{theorem}
\noindent
This last equation is especially useful for calculating definite integrals.
Rather than evaluating a limit of a Riemann sum, if we know the antiderivative, we can just evaluate at two points.

\begin{example}
	Evaluate the following definite integral using antiderivatives.
	\begin{equation*}
		\int_{0}^{1}{\frac{\d{x}}{1+x^2}}.
	\end{equation*}
\end{example}

We previously derived that the derivative of $\arctan{x}$ is $\frac{1}{1+x^2}$.
So, $\arctan{x}$ is the antiderivative of $\frac{1}{1+x^2}$.
Applying the Fundamental Theorem of Calculus (FTC)
\begin{align*}
	\int_{0}^{1}{\frac{\d{x}}{1+x^2}} &= \arctan{1} - \arctan{0} \\
	&= \frac{\pi}{4} - 0 \\
	&= \frac{\pi}{4}.
\end{align*} 
		
	\appendix
	
	\backmatter
\end{document}

