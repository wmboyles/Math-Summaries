\documentclass[oneside, 12pt]{book}

% ------------------------------------------------------------------------------
% Setup for table of contents
\setcounter{tocdepth}{3} 	% TOC should label down to subsubsections
\setcounter{secnumdepth}{2}	% TOC should not number further than a subsection number
% ------------------------------------------------------------------------------


% ------------------------------------------------------------------------------
% General Setup
\usepackage[english]{babel}
\usepackage{amsfonts, amsmath, amsthm, amssymb}		% Formatting symbols, theorems, lemmas, definitions, and examples

\usepackage{float}							% For making sure tables and figures stay in place
\usepackage{fullpage}						% Create ~1" margins
\usepackage{changepage}						% For setting up indents after examples
\usepackage[bottom,flushmargin]{footmisc}	% Put footnotes at bottom of page; don't intent footnotes
\usepackage[titletoc]{appendix}

\setcounter{chapter}{-1}				% Start with chapter 0

\usepackage{enumitem}					% For custom labels on enumerations
\usepackage{hyperref}					% For inserting links
\usepackage{graphicx} 					% For inserting images

% Tikz stuff
\usepackage{tikz}
\usetikzlibrary{patterns}
\usetikzlibrary{calc,patterns,decorations.pathmorphing,decorations.markings}
% ------------------------------------------------------------------------------

% ------------------------------------------------------------------------------
% Theorems, corollaries, definitions, lemmas, and examples should not be numbered
\newtheorem*{theorem}{Theorem}
\newtheorem*{corollary}{Corollary}
\newtheorem*{definition}{Definition}
\newtheorem*{lemma}{Lemma}
% ------------------------------------------------------------------------------

% ------------------------------------------------------------------------------
% Examples are not numbered. However, the answers that follow examples are indented and have an extra line break.
\newtheorem*{example}{Example}
\newenvironment{answer}{\begin{adjustwidth}{15pt}{}}{\end{adjustwidth}}
% ------------------------------------------------------------------------------

% ------------------------------------------------------------------------------
% Don't indent paragraphs by default
\setlength{\parindent}{0pt}
% ------------------------------------------------------------------------------

% ------------------------------------------------------------------------------
% Shortcuts
\newcommand{\dd}[2]{\frac{\mathrm{d} #1}{\mathrm{d} #2}}							% d[] / d[]
\renewcommand{\d}[1]{\mathrm{d} #1}
\renewcommand{\qedsymbol}{$\blacksquare$}
\newcommand{\abs}[1]{\big\lvert #1 \big\rvert}												% absolute value

\DeclareMathOperator{\arcsec}{arcsec}												% arc-secant
\DeclareMathOperator{\arccot}{arccot}												% arc-cotangent
\DeclareMathOperator{\arccsc}{arccsc}												% arc-cosecant

\newcommand{\R}{\mathbb{R}}															% Real numbers
% ------------------------------------------------------------------------------

\begin{document}
	% Title page setup
	\title{Single Variable Calculus: A Summary}
	\author{William Boyles}
	\date{}
	
	\frontmatter
		\maketitle
		\tableofcontents
		
	\mainmatter
		\chapter{Background \& Review}
\noindent
Everything mentioned in this chapter should already be familiar to you from other math classes. These topics span three main areas: algebra/pre-calculus, single variable calculus, and matrices. These topics will be used either implicitly or with only a passing reference.\\

\noindent
If you are unfamiliar with anything mentioned, you can use many of the great online resources, like Khan Academy, to familiarize yourself before moving forward.

\section{Algebra and Pre-Calculus}
\noindent

\subsection{Complex Numbers}
\noindent
$i$ is called the imaginary unit. It's defined by $i^2 = -1$. It and $-i$ are the solutions to the equation $x^2+1=0$.\\

\noindent
Complex numbers ($\mathbb{C}$) have the form $z = \alpha + \beta i$, where $\alpha$ and $\beta$ are real numbers. The $\alpha$ part of $z$ is called the real part, so $\Re(z) = \alpha$. The $\beta$ part of $z$ is called the imaginary part, so $\Im(z) = \beta i$.\\

\noindent
Often, complex numbers are visualized as points or vectors in a 2D plane, called the complex plane, where $\alpha$ is the x-component, and $\beta$ is the y-component. Thinking of complex numbers like points helps us define the magnitude of complex numbers and compare them. Since a point $(x,y)$ has a distance $\sqrt{x^2+y^2}$ from the origin, we can say the magnitude of $z$, $\lvert z \rvert$ is $\sqrt{\alpha^2 + \beta^2}$. Thinking of complex numbers like vectors helps us understand adding two complex numbers, since you just add the components like vectors.\\

\noindent
A common operation on complex numbers is the complex conjugate. The complex conjugate of $z = \alpha + \beta i$ is $\overline{z} = \alpha - \beta i$. $z$ and $\overline{z}$ are called a conjugate pair.\\

\noindent
Conjugate pairs have the following properties.
\\Let $z$, $w \in \mathbb{C}$.
\begin{enumerate}[label=]
	\item $\overline{z \pm w} = \overline{z} \pm \overline{w}$
	\item $\overline{zw}=\overline{z}\overline{w}$
	\item $\overline{z}=z \Leftrightarrow z \in \mathbb{R}$
	\item $z\overline{z} = \lvert z \rvert^2 = \lvert \overline{z} \rvert^2$
	\item $\overline{\overline{z}} = z$
	\item $\overline{z}^n = \overline{z^n}$
	\item $z^{-1} = \frac{\overline{z}}{\lvert z \rvert^2}$
\end{enumerate} 			% Complex Numbers
\subsection{Factoring Polynomials}
\noindent
We want to break up a polynomial like $f(x) = a_0 + a_1x^1 + \ldots a_nx^n$ into linear factors so that $f(x) = c(x-b_1)\cdot \ldots \cdot(x - b_n)$. This form makes it simple to see that the roots of $f$, solutions to $f(x) = 0$, are $x = b_1 \ldots b_n$.\\

\noindent
For quadratics, $f(x) = ax^2 + bx + c$, there exists a simple formula that will give us both roots, the quadratic formula.
\begin{equation*}
	x = \frac{-b \pm \sqrt{b^2-4ac}}{2a}
\end{equation*}

\noindent
We can see that when $b^2 - 4ac < 0$, like for $f(x) = x^2 + 5x + 1$, we will get complex roots $\alpha \pm \beta i$. For any polynomial, these roots come in pairs, so if $\alpha + \beta i$ is a root, then so is $\alpha - \beta i$. This means that every conjugate pair $\alpha \pm \beta i$ has a quadratic equation with those roots. Sometimes we will not factor quadratics with complex roots into linear terms.\\

\noindent
Although there do exist explicit formulas for finding roots for cubic (degree 3) and quartic (degree 4) equations, they are too long and not useful enough to memorize. When working by hand, we instead use other tricks to find roots.\\

\noindent
There are a few useful tricks that can help. If the polynomial doesn't have a constant term, then 0 is a root. If all the coefficients sum to 0, then 1 is a root. For certain polynomials with an even number of terms, like all cubics of the form $ax^3 + bx^2 + cax + cb$ we can factor out a term from the first two and last two terms to get $x^2(ax+b)+c(ax+b) = (ax+b)(x^2+c)$. For other polynomials, we might just try guessing and checking values. However, we need a more efficient way that works in general.\\

\noindent
Since we are looking to find linear factors $f(x) = (x-b_1)\cdot \ldots \cdot(x-b_n)$, we can see that the constant term in the polynomial is the product of the roots $b_1 \ldots b_n$. In fact, since the coefficients of polynomials are completely determined by the roots and the leading coefficient, all the coefficients are sums and products of roots. You might remember when factoring quadratics that the coefficient of $x$ term is the sum of the two roots. These rules are called Vieta's formulas.\\

\noindent
So, if we have the constant term, we can check all of its integer factors to see if any are roots. For each root, we can divide, using a technique like synthetic division, to continue finding the rest of the roots. This method is especially useful on tests because the roots tend to be integers.

%\begin{example}
%Factor the polynomial $x^5 + x^4 -2x^3 + 4x^2 -24x$.	
%\end{example}
%\noindent
%We can immediately see that there is no constant term, so $x=0$ is a root. Now we need to work on factoring $x^4 + x^3 -2x^2 + 4x - 24$.\\
%The factors of -24 are: -24, -12, -8, -6, -4, -3, -2, -1, 1, 2, 3, 4, 6, 8, 12, and 24. Starting from roots close to 0 and working outwards, we find that $x=2$ is a root. So, we synthetic divide like so
%\begin{table}[H]
%	\centering
%	\begin{tabular}{llllll}
%		$x=2 \mid$ & 1            & 1 & -2 & 4  & -24 \\
%		& $\downarrow$ & 2 & 6  & 8  & 24  \\ \hline
%		& 1            & 3 & 4  & 12 & $\mid 0$  
%	\end{tabular}
%\end{table}
%\noindent
%to see that now we need to work on factoring $x^3+3x^2+4x+12$.
%$x^3+3x^2+4x+12 = x^2(x+3)+4(x+3) = (x+3)(x^2+4)$, so $x=-3$ is a root, and we need to work on factoring $x^2+4$.
%$x^2+4$ has two complex roots $\pm 2i$, so we'll leave it as a quadratic.
%\begin{equation*}	
%	x^5 + x^4 -2x^3 + 4x^2 -24x = x(x-2)(x-3)(x^2+4)
%\end{equation*} 		% Factoring Polynomials
\subsection{Trig Functions \& The Unit Circle}
\noindent
Imagine aa circle of radius 1 centered at the origin that we'll call the unit circle. The x and y coordinates of a point on the unit circle are completely determined by the angle $\theta$ in radians between the x-axis and a line from the origin to the point.\\

\noindent
The function $\cos{\theta}$ tells us x-coordinate of the point, while $\sin{\theta}$ tells us the y-coordinate of the point. The function $\tan{\theta} = \frac{\sin{\theta}}{\cos{\theta}}$ tells us the slope of the line from the origin to the point. Most of the trig functions have geometric interpretations as shown below. The most used ones are $\sin$, $\cos$, $\tan=\frac{\sin}{\cos}$, $\cot = \frac{\cos}{\sin}$, $\csc=\frac{1}{\sin}$, and $\sec=\frac{1}{\cos}$.

\begin{figure}[H]
	\label{unitCircle}
	\centering
	\includegraphics[width = 0.75\textwidth]{./backgroundReview/algebraPreCalc/unitCircle2.png}
	\caption{\hyperref{https://en.wikipedia.org/wiki/Unit_circle}{}{}{Wikipedia - Unit circle}}
\end{figure}

\noindent
We can also think about the inverses of these trig functions. These are either notated with a -1 exponent on the function, or the prefix arc in front of the function name. Many of these functions are only defined on a part of the domain $\left[0, 2\pi\right]$. Below is a table of the inverse trig functions and their domains.

\begin{table}[H]
	\centering
	\begin{tabular}{l|l}
		Function  & Domain                                                 \\ \hline
		$\arcsin$ & $\left[-1, 1\right]$           						   \\
		$\arccos$ & $\left[-1, 1\right]$                                   \\
		$\arctan$ & $\left(-\infty, \infty\right)$                         \\
		$\arccot$ & $\left(-\infty, \infty\right)$                         \\
		$\arccsc$ & $\left(-\infty, -1\right] \cup \left[1, \infty\right)$ \\
		$\arcsec$ & $\left(-\infty, -1\right] \cup \left[1, \infty\right)$
	\end{tabular}
\end{table}
 	% Trig Functions / Unit Circle
\subsection{Trig Identities}
\noindent
As we could see in Figure \ref{unitCircle}, $\sin$ and $\cos$ form a right triangle with hypotenuse 1. So, using the Pythagorean Theorem
\begin{equation*}
	\sin^2{\theta} + \cos^2{\theta} = 1
\end{equation*}
By dividing by $\sin^2$ or $\cos^2$, we can also get
\begin{equation*}
	1 + \cot^2{\theta} = \csc^2{\theta} \text{ and } \tan^2{\theta} + 1 = \sec^2{\theta}
\end{equation*}
Together, these 3 identities are called the Pythagorean Identities.\\

\noindent
We can also relate functions and co-functions.
\begin{equation*}
	\text{xxx}(\theta) = \text{coxxx}\left(\frac{\pi}{2} - \theta\right)
\end{equation*}

Some of the most useful and used identities are the sum and difference identities.
\begin{equation*}
	\sin{\left(\alpha \pm \beta\right)} = \sin{\alpha}\cos{\beta} \pm \cos{\alpha}\sin{\beta}
\end{equation*} \begin{equation*}
	\cos{\left(\alpha \pm \beta\right)} = \cos{\alpha}\cos{\beta} \mp \sin{\alpha}\sin{\beta}
\end{equation*} \begin{equation*}
	\tan{\left(\alpha \pm \beta\right)} = \frac{\tan{\alpha} \pm \tan{\beta}}{1 \mp \tan{\alpha}\tan{\beta}}
\end{equation*} \begin{equation*}
	\sin{\alpha} \pm \sin{\beta} = 2\sin{\left(\frac{\alpha \pm \beta}{2}\right)}\cos{\left(\frac{\alpha \mp \beta}{2}\right)}
\end{equation*} \begin{equation*}
	\cos{\alpha} + \cos{\beta} = 2\cos{\left(\frac{\alpha + \beta}{2}\right)}\cos{\left(\frac{\alpha - \beta}{2}\right)}
\end{equation*} \begin{equation*}
	\cos{\alpha} - \cos{\beta} = -2\sin{\left(\frac{\alpha + \beta}{2}\right)}\sin{\left(\frac{\alpha - \beta}{2}\right)}
\end{equation*} 			% Trig Identites
\subsection{Exponentials \& Logarithms}
\begin{definition}
	e is the base of the natural logarithm. It's definied by the limit
	\begin{equation*}
		e = \lim\limits_{n\rightarrow\infty}{\left(1+\frac{1}{n}\right)^n}.
	\end{equation*}
\end{definition}


$\exp{x} = e^x$ and $\ln{x}$ are inverse functions of each other such that
\begin{equation*}
	e^{\ln{x}} = x \text{, } \ln{e^x} = x.
\end{equation*}


Just like other exponents, the normal rules for adding, subtracting, and multiplying powers apply.
\begin{equation*}
	e^xe^y = e^{x+y}\text{, }\frac{e^x}{e^y}=e^{x-y}\text{, and }\left(e^x\right)^k=e^{xk}.
\end{equation*}


Similar rules apply for logarithms.
\begin{equation*}
	\ln{x}+\ln{y} = \ln{xy}\text{, }\ln{x}-\ln{y} = \ln{\left(\frac{x}{y}\right)}\text{, and }\ln{\left(a^b\right)}=b\ln{a}.
\end{equation*}


You can also change a logarithm of any base to a natural logarithm.
\begin{equation*}
	\log_{b}{a} = \frac{\ln{a}}{\ln{b}}.
\end{equation*}


$e$ is also unique in that it is the only real number $a$ satisfying the equation
\begin{equation*}
	\frac{\mathrm{d}}{\mathrm{d}x}a^x = a^x.
\end{equation*}
meaning $e^x$ is its own derivative.\footnote{Don't worry if you don't know what a derivative is yet. It's one of the first topics we'll cover in calculus.} 		% Exponential and logarithms
\subsection{Partial Fractions}
\noindent
If we have a function of two polynomials $f(x) = \frac{P(x)}{Q(x)}$, it's often easier to break this quotient into a sum of parts where the denominator is a linear or quadratic factor and the numerator is always a smaller degree than the denominator.

\begin{example}
	\begin{equation*}
		\frac{2x-1}{x^3-6x^2+11x-6} = \frac{1/2}{x-1}+\frac{-3}{x-2}+\frac{5/2}{x-3}.
	\end{equation*}
\end{example}

\noindent
One natural way to find these small denominators comes from the linear factors of the denominator where we keep quadratics with complex roots.
This way, when making a common denominator, we get back the original big denominator.
However, there are a few special cases we have to take care of.

\input{./backgroundReview/algebraPreCalc/linearFactors.tex}
\input{./backgroundReview/algebraPreCalc/repeatedLinearFactors.tex}
\input{./backgroundReview/algebraPreCalc/quadraticFactors.tex}
\input{./backgroundReview/algebraPreCalc/repeatedQuadraticFactors.tex}
\input{./backgroundReview/algebraPreCalc/improperFractions.tex}
 			% Partial Fractions % Algebra and Pre-Calc
\section{Single Variable Calculus}
\noindent

% Derivatives and Integrals
% u-subtitution
% Integration by parts % Single Variable Calculus
\chapter{Matrices}

We'll introduce matrices, how they can represent linear maps and systems of linear equations, and useful operations we can perform on them.

\section{Definition}

\begin{definition}
	An $m \times n$ matrix is an array of objects (usually field elements) arranged in $m$ rows and $n$ columns.
\end{definition}

Matrices are usually written inside square brackets.
We tend to use uppercase letters like $M$ to represent matrices as variables.
The notation $M_{a,b}$ represents the element in row $a$ and column $b$.

\begin{example}
	Matrix $M$ is $3 \times 4$.
	\begin{equation*}
		M = \begin{bmatrix}
			1 & 4  & 0 \\
			8 & -1 & -2 \\
			3 & 7  & 4
		\end{bmatrix}
	\end{equation*}
	We see that $M_{2,2} = -1$ and $M_{3,1} = 3$.
\end{example}


\section{Basic Operations}

\subsection{Vector Space Operations}
Two matrices are considered equal if they have the same number of rows and columns and all entries are equal.
Scalar multiplication works by multiplying each element by the scalar.
Addition works element by element.

\begin{example}
	\begin{align*}
		A &= \begin{bmatrix}
			1 & 2 \\
			-1 & 3
		\end{bmatrix} \text{, } B = \begin{bmatrix}
			4 & 0 \\
			-2 & 5
		\end{bmatrix}. \\
		A + B &= \begin{bmatrix}
			1 + 4 & 2 + 0 \\
			-1 + -2 & 3 + 5
		\end{bmatrix} = \begin{bmatrix}
			5 & 2 \\
			-3 & 8
		\end{bmatrix} \\
		3A &= \begin{bmatrix}
			3\cdot1 & 3\cdot 2 \\
			3\cdot -1 & 3\cdot 3
		\end{bmatrix} = \begin{bmatrix}
			3 & 6 \\
			-3 & 9
		\end{bmatrix}.
	\end{align*}
\end{example}

Notice addition is commutative, associative, has an additive identity (the all 0's matrix), and has an additive inverse (scalar multiply by -1).
Further, scalar multiplication has a multiplicative identity (1), is distributive over both addition and field multiplication.
Thus, the set of all matrices of the same size form a vector space.
We denote the vector space of all $m \times n$ matrices with real entries as $\mathcal{M}_{m \times n}$.

\subsection{Multiplication}
We can also define an operation for multiplying two matrices of compatible sizes that outputs another matrix.

\begin{definition}
	Let $A$ be an $m \times n$ matrix, and let $B$ be and $n \times k$ matrix.
	Then $C = AB$ is an $m \times k$ matrix where
	\begin{equation*}
		C_{i,j} = \sum_{d=1}^{n}{A_{i,d} B_{d,j}}.
	\end{equation*}
\end{definition}
If you're familiar with the concept of dot products, then $C_{i,j}$ is the dot product of the $i$th row of $A$ with the $j$th column of $B$.

\begin{example}
	\begin{align*}
		A &= \begin{bmatrix}
			1 & 2 & 3 \\
			0 & -1 & 2
		\end{bmatrix} \text{, } B = \begin{bmatrix}
			1 & -1 & 0 & 2 \\
			2 & 0 & 3 & -1 \\
			0 & 1 & 1 & 5 
		\end{bmatrix}. \\
		AB &= \begin{bmatrix}
			5 & 2 & 9 & 15 \\
			-2 & 2 & -1 & 11
		\end{bmatrix}.
	\end{align*}
\end{example}

Similar to scalar multiplication, there exists a multiplicative identity matrix.
However, this matrix only behaves like an identity when the matrix it's being multiplied is $n \times n$ (i.e. a square matrix).

\begin{definition}
	The $n \times n$ matrix $I_n$ is called the identity matrix and is defined by
	\begin{equation*}
		I_{i,j} = \begin{cases}
			1 & i=j \\
			0 & \text{otherwise}
		\end{cases}.
	\end{equation*}
\end{definition}

\begin{theorem}
	Let $A$ be an $n \times n$ matrix.
	Then $AI_n = I_n A = A$.
\end{theorem}
\begin{proof}
	Let $C = AI_n$.
	Notice,
	\begin{align*}
		C_{i,j} &= \sum_{d=1}^{n}{A_{i,d}(I_{n})_{d,j}} \\
			&= \sum_{d=1}^{n}{A_{i,d} \begin{cases}
					1 & d=j \\
					0 & \text{otherwise}
			\end{cases}} \\
			&= \sum_{d=1}^{n}{\begin{cases}
				A_{i,j} & d=j \\
				0 & \text{otherwise}
			\end{cases}} \\
			&= A_{i,j}.
	\end{align*}
	This same line of reasoning works to also show that $I_nA = A$.
	Since all entries of $A$ and $C$ are equal, $C = AI_n = A$, as desired.
\end{proof}

Also similar to scalar multiplication, matrix multiplication is associative and distributive.
\begin{theorem}
	Let $A$, $B$, and $C$ be matrices.
	Let $k$ be a scalar.
	Then the following properties hold (assuming the matrices have the correct dimensions):
	\begin{itemize}
		\item \textbf{Associative}: $A(BC) = (AB)C$.
		\item \textbf{Distributive Over Matrix Multiplication}: $k(AB) = (kA)B = A(kB)$.
		\item \textbf{Left Distributive Over Addition}: $A(B + C) = AB + AC$.
		\item \textbf{Right Distributive Over Addition}: $(A + B)C = AC + BC$.
	\end{itemize}
\end{theorem}

Unlike scalar multiplication, matrix multiplication is not commutative.
For one, if $AB$ is defined, $BA$ won't also be defined unless $A$ and $B$ are both square matrices
Even if this is the case, $AB \neq BA$ in general.

\begin{theorem}
	An $n \times n$ matrix $A$ commutes only and all matrices in the vector space $\linspan(\{I_n, A\})$.
\end{theorem}
\section{As Linear Maps}

\input{./matrices/matricesAsLinearMapsDefinition.tex}
\input{./matrices/matricesAsLinearMapsProperties.tex}



\section{As Systems of Linear Equations}

\input{./matrices/matricesAsLinearMapsInvertability.tex}
\input{./matrices/matricesAsLinearMapsColumnSpaceNullSpace.tex}
\subsection{Determinants}
\noindent
The determinant of a matrix is a signed number that tells by how much the transformation represented by a matrix scales volumes in a space. The number is negative if the space was "flipped" during a transformation. The number is negative if the dimension of the output space is less than that of the input space.\\

\noindent
The determinant is only defined for square matrices. It's easiest to understand the definition of a determinant recursively.
\begin{equation*}
	\det{\left[ a \right]} = \lvert a \rvert = a
\end{equation*}
\begin{equation*}
	\det{\left[
		\begin{array}{cc}
			a & b \\
			c & d
		\end{array}
	\right]} = \begin{array}{|cc|}
		a & b \\
		c & d
	\end{array} = ad - bc
\end{equation*}
We can define $a_{ij}$ as the entry in the ith row and jth column of matrix $A$ and $A_{ij}$ as the adjudicate matrix, which is the matrix $A$ if row $i$ and column $j$ were removed. This allows us to write a general formula for the determinant.
\begin{definition}
	\begin{equation*}
		\det{A} = \sum_{j=1}^{n}{\left(-1\right)^{i+j}a_{ij}A_{ij}} \text{ (for fixed i)} = \sum_{i=1}^{n}{\left(-1\right)^{i+j}a_{ij}A_{ij}} \text{ (for fixed j)}
	\end{equation*}
\end{definition}
\noindent
This formula allows us to use any row or column to calculate the determinant, which is especially useful if a certain row contains lots of 0's.\\

\noindent
Below are some properties of the determinant for some $n \times n$ matrix $A$ and scalar $\lambda$.
\begin{enumerate}[label=]
	\item \begin{equation*}
		\det{I_n} = 1
	\end{equation*}
	\item \begin{equation*}
		\det{(A^T)} = \det{A}
	\end{equation*}
	\item If $A$ is invertable,
		\begin{equation*}
			\det{(A^{-1})} = \frac{1}{\det{A}}
		\end{equation*}
	\item \begin{equation*}
		\det{(\lambda A)} = \lambda^n\det{A}
	\end{equation*}
	\item \begin{equation*}
		\det{(AB)} = \det{A}\det{B}
	\end{equation*}
	\item If $A$ is triangular,
		\begin{equation*}
			\det{A} = \prod_{i=1}^{n}{a_{ii}}
		\end{equation*}
\end{enumerate}

\ifodd\includeBackgroundReviewExamples\input{./backgroundReview/matrices/determinants_example}\fi % Matrix operations
		\chapter{Limits \& Continuity}
\noindent
Limits are a way of describing what happens to a function $f(x)$ as $x$ gets arbitrarily close to a value from some direction (positive or negative).
This allows us not only to deal with "holes" in some functions but describe some of the building blocks of calculus, namely the derivative.

\section{Limit Definition}
\begin{definition}
	Let $f : D \subseteq \R \to \R$.
	Let $c \in R$ be a limit point (ie $c \in D$ or $c$ is on the boundary of $D$).
	$f$ has a limit $L$ as $x$ approaches $c$ if for any given positive real number $\epsilon$, there is a positive real number $\delta$ such that for all $x \in D$,
	\begin{equation}
		0 < \abs{x-c} < \delta \implies \abs{f(x) - L} < \epsilon.
	\end{equation}
	We write this as
	\begin{equation*}
		\lim_{x \to c}{f(x)} = L.	
	\end{equation*}
\end{definition}

\begin{figure}[H]
	\label{epsilon_delta}
	\centering
	\includegraphics[width = 0.5\textwidth]{./limits_continuity/limit_epsilon_delta.png}
	\caption{\hyperref{https://en.wikipedia.org/wiki/(\%CE\%B5,\_\%CE\%B4)-definition\_of\_limit}{}{}{Wikipedia - $(\epsilon, \delta)\text{-definition of limit}$}}
\end{figure}
\noindent
Visually, what this means is that for any "error bound" of $y$ values $\epsilon$, I can give you a corresponding error bound of $x$ values $\delta$ such that all values of $f(z)$ for $z \in (c -\delta, c+ \delta)$ bound are between $L - \epsilon$ and $L + \epsilon$.

\noindent
We don't use this definition of the limit very often because it's a bit cumbersome.
However, it's important to know that when we use the limit, this is the formal definition making things work.

\begin{example}
	Use the $(\epsilon, \delta)$ definition of the limit to show that
	\begin{equation*}
		\lim_{x\to 0}{x\sin{\frac{1}{x}}} = 0.
	\end{equation*}
\end{example}
Letting $\epsilon > 0$, we need to find corresponding $\delta > 0$ that satisfies the definition for $L = 0$.
Knowing that $\sin$ is bounded between -1 and 1,
\begin{equation*}
	\abs{x\sin{\frac{1}{x}} - 0} = \abs{x\sin{\frac{1}{x}}} = \abs{x}\abs{\sin{\frac{1}{x}}} \leq \abs{x}.
\end{equation*}
\indent
Letting $\delta = \epsilon$, if $0 < \abs{x - 0} < delta$, then $\abs{x\sin{\frac{1}{x}} - 0} \leq \abs{x} < \epsilon$, as required by the definition.
\section{Limit Properties}
Limit have many nice properties all allow us to make useful simplifications when evaluating a limit.
Let
\begin{equation*}
	\lim_{x \to c}{f(x)} = L \text{ and } \lim_{x \to c}{g(x)} = M.
\end{equation*}
\begin{align*}
	\textbf{Sum and Difference Rule: }& \lim_{x\to c}{\left(f(x) \pm g(x)\right)} = L \pm M \\
	\textbf{Product Rule: }& \lim_{x\to c}{\left(f(x)g(x) \right)} = LM \\
	\textbf{Constant Multiple Rule: }& \lim_{x \to c}{k\cdot f(x)} = k \lim_{x \to c}{f(x)} = kL \\
	\textbf{Quotient Rule: }& \lim_{x \to c}{\frac{f(x)}{g(x)}} = \frac{\lim_{x \to x}{f(x)}}{\lim_{x \to c}{g(x)}} = \frac{L}{M} \text{, if} M \neq 0 \\
	\textbf{Power Rule: }& \text{If } n \neq  \in \R \text{, } \lim_{x\to c}{\left(f(x)\right)^n} = \left(\lim_{x \to c}{f(x)}\right)^n = L^n
\end{align*}

\subsection{``Substitution Rule''}
Although it may seem obvious from our idea that limits describe behavior at a point that if $f(x)$ is defined at $x=c$, then $\lim_{x\to c}{f(x)} = f(c)$.
However, this is \textit{not} always the case.
Remember that our definition of a limit required these $\epsilon$ and $\delta$ neighborhoods around the limit point.
If $f(x)$ is defined at $x=c$, but $(c, f(c))$ is not a point in these neighborhoods for any $\epsilon > 0$, then the limit will not evaluate to $f(c)$.

\begin{example}
	Find the limit of $f(x)$ as $x$ approaches $2$ for the following function.
	\begin{equation*}
		f(x) = \begin{cases}
			x^2 & x \neq 2 \\
			0 & x = 2
		\end{cases}.
	\end{equation*}
\end{example}
We can clearly see that $f(2) =  0$, but for $\epsilon = 0.1$ for example, there is no $\delta$ that can satisfy our definition, as points like $(2 - \delta, 4 - 2\delta + \delta^2)$ would outside the neighborhood around $(2,0)$.
In fact, the correct limit value is $4$, the same as if $f(x) = x^2$ for all $x$.
There are some more nuances we'll need to describe before we can say when it's OK to substitute to evaluate a limit.
\section{Left \& Right Hand Limits}
Our definition of the limit requires that the function get arbitrarily close to the limit value when approaching from both the left and right hand sides.
However, we can evaluate limits by specifying that we only approach from one side.
We usually notate this with a superscript $+$ or $-$ next to the $x$ limit value.
So,
\begin{equation*}
	\lim_{x \to 0^+}{f(x)}
\end{equation*}
would mean ``the limit of $f(x)$ as $x$ approaches $0$ from the right'', while
\begin{equation*}
	\lim_{x \to 0^-}{f(x)}
\end{equation*}
would mean ``the limit of $f(x)$ as $x$ approaches $0$ from the left.''

Our definition of the limit from both sides requires the left and right sides to be the same.
If they are different, the the limit does not exist.
\begin{align*}
	\lim_{x \to c^+}{f(x)} = \lim_{x \to c^-}{f(x)} &\implies \lim_{x \to c^+}{f(x)} = \lim_{x \to c^-}{f(x)} = \lim_{x \to c}{f(x)} \\
	\lim_{x \to c^+}{f(x)} \neq \lim_{x \to c^-}{f(x)} &\implies \lim_{x \to c}{f(x)} \text{ does not exist (DNE)}.
\end{align*}
\section{Sandwich Theorem}
We can use the Sandwich Theorem to indirectly find limits by ``sandwiching'' the function in question between two functions we do know the limit of.
If these two sandwiching functions go to the same value in the limit, then so to must the function in question.
\begin{theorem}[The Sandwich Theorem]
	If $g(x) \leq f(x) \leq h(x)$ and $\lim_{x \to c}{g(x)} = \lim_{x\to c}{h(x)} = L$, then $\lim_{x \to c}{f(x)} = L$.
\end{theorem}

\begin{example}
	Evaluate the following limit
	\begin{equation*}
		\lim_{\theta \to 0}{\frac{\sin{\theta}}{\theta}}
	\end{equation*}
\end{example}
\begin{answer}
	We'll need to use some geometric ideas to solve this limit.
	Consider the following on a unit circle.
	
	\begin{figure}[H]
		\label{sin_limit_proof}
		\centering
		\includegraphics[width = 0.5\textwidth]{./limits_continuity/sin_limit_proof.png}
		\caption{\hyperref{}{}{}{Triangle with internal angle $\theta$ inside a unit circle.}}
	\end{figure}
	
	We can see that the area of the swept arc is between the two triangle with base of length 1 and heights of $\sin{\theta}$ and $\tan{\theta}$.
	So, we can write the following inequality.
	\begin{align*}
		\frac{1}{2}\sin{\theta} &\leq \frac{1}{2}\theta \leq \frac{1}{2}\frac{\sin{\theta}}{\cos{\theta}} \\
		\sin{\theta} \leq \theta &\leq \frac{\sin{\theta}}{\cos{\theta}}
	\end{align*}
	
	Taking the reciprocal of each part and multiplying by $\sin{\theta}$,
	\begin{equation*}
		1 \geq \frac{\sin{\theta}}{\theta} \geq \cos{\theta}.
	\end{equation*}
	
	Taking the limit of as $\theta$ approaches 0 of each term,
	\begin{align*}
		1 \geq & \lim_{\theta \to 0}{\frac{\sin{\theta}}{\theta}} \geq \lim_{\theta\to 0}{\cos{\theta}} \\
		1 \geq & \lim_{\theta \to 0}{\frac{\sin{\theta}}{\theta}} \geq 1.
	\end{align*}
	
	So, by the Sandwich Theorem,
	\begin{equation*}
		\lim_{\theta \to 0}{\frac{\sin{\theta}}{\theta}} = 1.
	\end{equation*}
\end{answer}
\section{Infinite Limits}
Although our limit definition works for finite values of $c$, it's also useful to think about what happens as $c$ goes to $\pm\infty$.
We'll need to add to our limit definition to incorporate infinite values, since it doesn't make sense to talk about neighborhoods at infinity.
\begin{definition}
	Let $f$ be a real-valued function defined on some subset $D \subseteq \R$ that contains arbitrarily large values.
	\begin{equation*}
		\lim_{x \to \infty}{f(x)} = L
	\end{equation*}
	if for every real $\epsilon > 0$, there is a real number $N > 0$ such that for all $x \in D$,
	\begin{equation}
		x > N \implies \abs{f(x) - L} < \epsilon.
	\end{equation}
\end{definition}

All the same properties that we described for finite limits, like the Sum and Difference Rule, still hold for infinite limits.

\subsection{End Behavior Model}
When x is numerically large, we can often model the behavior of a complicated function with a simplier one that behaves roughly the same for numerically large input values and is the same in the limit.
There are a few rules that these follow.
\begin{enumerate}
	\item For a polynomial, the end-behavior is highest-degree term.
	\item For a rational function, like a ratio of polynomials, the end behavior is the ratio of the highest degree terms.
	\item For more complicated functions, we may need to use some reasoning about the graph of the function and limit properties to determine end-behavior.
\end{enumerate}

\subsection{Horizontal Asymptotes}
Horizontal Asymptotes are a special type of end-behavior model.
\begin{definition}
	The line $y=b$ is a horizontal asymptote of $y = f(x)$ if $\lim_{x\to \infty}{f(x)} = b$ or $\lim_{x \to -\infty}{f(x)} = b$.
\end{definition}

We can determine horizontal asymptotes for rational functions (usually quotient of polynomials).
There are a few cases to consider
\begin{enumerate}
	\item If the numerator is a higher degree than the denominator, there is no horizontal asymptote, so we'll need a different method to calculate what happens at $\pm\infty$.
	\item If the denominator is a higher degree than the numerator, then there is a horizontal asymptote at $y = 0$.
	\item If the numerator and denominator have the same degree, there is a horizontal asymptote at $y = k$ where k is the ratio of the highest degree terms.
\end{enumerate}

\begin{example}
	Find the following limits, if they exist.\\
	\begin{table}[H]
	\begin{center}
	\begin{tabular}{ l l l}
		1. $\begin{aligned}[t]
			\lim_{x \to \infty}{\frac{x^3 - 6x + 1}{x^2 + 2x - 3}}
		\end{aligned}$ & 
		2. $\begin{aligned}[t]
			\lim_{x\to -\infty}{\frac{x-9}{2x-x^2}}
		\end{aligned}$ &
		3. $\begin{aligned}[t]
			\lim_{x\to \infty}{\frac{6x^2-4x^5+7x-1}{12x^5-3x^2+2}}
		\end{aligned}$ \\
		\hspace{1pt} & \hspace{1pt}\\
		4. $\begin{aligned}[t]
			\lim_{x\to \infty}{\frac{3x+1}{\abs{x}+2}}
		\end{aligned}$ &
		5. $\begin{aligned}[t]
			\lim_{x \to \infty}{x + e^{-x}}
		\end{aligned}$ &
		6. $\begin{aligned}[t]
			\lim_{x \to -\infty}{x + e^{-x}}
		\end{aligned}$
	\end{tabular}
	\end{center}
	\end{table}
\end{example}
\begin{answer}
	\begin{enumerate}
		\item Since the numerator degree is bigger than the denominator degree, we'll need to use the end behavior model.
			The end behavior model tells us that the numerator term dominates and has positive values, so the limit evaluates to $\infty$.
		\item Since the denominator has higher degree than the numerator, there is a HA at $y=0$, so the limit evaluates to $0$.
		\item Since the numerator and denominator have the same degree, the limit is the ratio of the highest-degree coefficients, $\frac{-1}{3}$.
		\item The numerator and denominator have the same degree. For $x > 0$, $\abs{x}+2 = x+2$, so the limit is the ratio of highest-degree coefficients, $3$.
		\item Looking at the two terms, we can see that as $x$ gets large, $e^{-x}$ gets very small, contributing less and less to the overall value.
			So, we can say that this function as a right end behavior model of $x$, so the limit is $\infty$.
		\item Looking at the two terms, we that that as $x$ gets very large and negative, $e^{-x}$ changes much faster than $x$.
			That is, $e^{-x}$ contributes more and more to the overall value of the function compared to $x$.
			So, we can say that this function has a left end behavior model of $e^{-x}$, so the limit is $\infty$.
	\end{enumerate}
\end{answer}
\section{Continuity Definition}
When we were looking at limits, we noticed that we can't always substitute to find the limit, even if the function is defined there.
In the example given to show that substitution and the limit can give different results, we saw a special type of function that seemed to have a "hole" at the point we were interested in finding the limit of.
This function is said to be discontinuous at this point, and in this section we'll define when a function is or isn't continuous at a point based on this idea of the limit and substitution giving different values.

\begin{definition}
	Let $f(x)$ be a real-valued function defined over $D \subseteq \R$.
	$f(x)$ is continuous at some point $x = c$ if all of the following hold.
	\begin{enumerate}
		\item $\lim_{x \to c}{f(x)}$ exists
		\item $f(c)$ is defined
		\item $\lim_{x \to c}{f(x)} = f(c)$ (substitution works)
	\end{enumerate}
	Otherwise, $f(x)$ is discontinuous at $c$.\footnote{Note that it's not necessary for $c \in D$.}
\end{definition}


We say that a function is continuous on an interval if it's continuous on every point in that interval.\footnote{If the interval is closed on one or both sides, we check continuity on the open interval. Then, we check the closed endpoints by looking at the limit from only one side.}

\begin{example}
	Find the points of continuity and discontinuity of the following functions
	\begin{table}[H]
	\begin{center}
	\begin{tabular}{ l l }
		1. $\begin{aligned}
			f(x) = \frac{1}{x^2+1}
		\end{aligned}$ &
		2. $\begin{aligned}
			g(x) = e^{1/x}
		\end{aligned}$
	\end{tabular}
	\end{center}
	\end{table}
\end{example}
\begin{answer}
	\begin{enumerate}
		\item There are no points where $f(x)$ or its limit are undefined.
			Further, there are no points where $f$ and its limit at that point are different.
			So, $f$ is continuous on $(-\infty, \infty)$ and discontinuous on $\emptyset$.
		\item Since $1/x$ is undefined at $x = 0$, $g(x)$ is also undefined at $x=0$.
			At every other point, $g$ and its limit are defined and are equal.
			So, $g$ is continuous on $(\infty, 0) \cup (0, \infty)$ and discontinuous on $[0]$.
	\end{enumerate}
\end{answer}
\section{Discontinuity Types}
There are four major types of discontinuity.
\begin{enumerate}[label=]
	\item \textbf{Removable: } If $f$ is discontinuous at $c$ but we can remove the discontinuity by setting $f$ equal to its limit at $c$, then $f$ has a removable discontinuity at $c$.
	\item \textbf{Jump: } If $f$ is discontinuous at $c$, and both of the one-sided limits exist but are different, then $f$ has a jump discontinuity at $c$.
	\item \textbf{Infinite: } If $f$ has a vertical asymptote at $c$, meaning one or both sides go to $\pm\infty$, then $f$ has an infinite discontinuity at $c$.
	\item \textbf{Oscillating: } If $f$ oscillates without limit at $c$, then $f$ has an oscillating discontinuity at $c$. An example of such a function would be $\sin{\frac{1}{x}}$ at $x=0$.
\end{enumerate}


It might seem strange that $\sin{\frac{1}{x}}$ has an oscillating discontinuity at $x=0$ because we were able to find the limit as $x$ approaches of 0 of $x\sin{\frac{1}{x}}$, a very similar function.
However, remembering how we applied the Sandwich Theorem to find this limit, we see that the $x$ term bounds the amplitude of the oscillations, allowing the limit to be $0$.

\begin{example}
	For the following function state the following: its domain, any discontinuities and their types, what values should redefine the function to remove any removable discontinuities (give the extended function).
	\begin{equation*}
		f(x) = \frac{x^3-7x-6}{x^2-9}
	\end{equation*}
\end{example}
\begin{answer}
	Polynomials are continuous on their entire domain of all real numbers.
	So, rational functions like $f$ can only be discontinuous when the denominator is equal to $0$.
	This happens in two places: $x=3$ and $x=-3$.
	We'll check the limits from each side at each of these points to determine the type of discontinuity.
	For $x=3$,
	\begin{equation*}
		\lim_{x\to 3^+}{f(x)} = \lim_{x\to 3^-}{f(x)} = \lim_{x\to 3}{f(x)} = \lim_{x\to 3}{\frac{(x+2)(x+1)(x-3)}{(x+3)(x-3)}} = \lim_{x\to 3}{\frac{(x+2)(x+1)}{(x+3)}} = \frac{20}{6} = \frac{10}{3}.
	\end{equation*}
	
	So, $f$ has a removable discontinuity at $x=3$ because the left and right limits are the same.
	For $x=-3$,
	\begin{equation*}
		\lim_{x\to -3^+}{f(x)} = -\infty \text{ and } \lim_{x\to -3^+}{f(x)} = \infty.
	\end{equation*}
	
	So, $f$ has an infinite discontinuity at $x=-3$ because both of the left and right limits go to $\pm\infty$.
	The value we got from the limits at $x=3$ gives us the value we need to redefine $f$ as to remove the discontinuity.
	The extended function is therefore
	\begin{equation*}
		f_{e}(x) = \begin{cases}
			f(x) & x \neq 3 \\
			\frac{10}{3} & x = 3
		\end{cases}
	\end{equation*}
\end{answer}
\section{Continuity Properties}
These properties should look very similar to the properties of limits.
Let $f$ and $g$ be continuous functions at $c$.
\begin{align*}
	\textbf{Sum and Difference Rule: }& f \pm g \text{ is continuous at } c. \\
	\textbf{Product Rule: }& f \cdot g \text{ is continuous at } c. \\
	\textbf{Constant Multiple Rule: }& kf \text{ is continous at } c \text{ for all real } k. \\
	\textbf{Quotient Rule: }& \frac{f}{g} \text{ is continuous at } c \text{ as long as the value of the extended function of } \\
		g \text{ at } c \text{ is not } 0. \\
	\textbf{Composition Rule: }& f \circ g \text{ is continuous at } c \text{ if } f \text{ is continuous at } g(c). \\
	\textbf{Absolute Value Rule: }& \abs{f} \text{ is continuous at } c.
\end{align*}

The following types of functions are continuous on their domains
\begin{itemize}
	\item Polynomials
	\item Rational functions, except where the denominator is 0
	\item Trigonometric functions where defined
\end{itemize}

\begin{example}
	Show that the following function is continuous.
	\begin{equation*}
		f(x) = \tan{\left(\frac{x^2}{x^2+4}\right)}.
	\end{equation*}
\end{example}
We can write $f$ as the composition of $\tan{x}$ and $\frac{x^2}{x^2+4}$.
$\tan{x}$ is continuous on its domain because it is a trigonometric function.
The only points not in its domain are $(2n+1)\frac{\pi}{2}$, where $n$ is an integer.
$\frac{x^2}{x^2+4}$ is a rational function, but it's denominator is never $0$, so it is continuous over all real numbers.
Now, we just need to check that all points in the range of $\frac{x^2}{x^2+4}$ are in the domain of $\tan{x}$.
The range of $\frac{x^2}{x^2+4}$ is $[0,1)$.
None of the points in this interval are not in the domain of $\tan{x}$, so the composition is continuous over all real numbers.
\section{Intermediate Value Theorem}
\begin{theorem}[Intermediate Value Theorem (IVT)]
	If $f$ is continuous on the closed interval $[a,b]$, then for all $c \in [f(a), f(b)]$, there exists $x \in [a,b]$ such that $f(x) = c$.
\end{theorem}

That is, if $f$ is continuous on $[a,b]$, then $f$ must take on every value between $f(a)$ and $f(b)$.
This encapsulates the idea that if a function is continuous on some interval, then it is "connected" on that interval.

\begin{example}
	Use the IVT to show that $e^{-x} = x$ has at least one solution.
\end{example}
\begin{answer}
	Let $f(x) = e^{-x} - x$.
	We are looking for $x$ where $f(x) = 0$.
	$f(0) = 1$ and $f(1) = \frac{1}{e} - 1$.
	Since $f$ is continuous on the closed interval $[0,1]$, it must take on every value between $1$ and $\frac{1}{e} - 1$.
	Since $1$ is positive and $\frac{1}{e} - 1$ is negative, 0 is between these two values.
	Thus, by the IVT, there must exist a solution between $x=0$ and $x=1$.
\end{answer}
		\subsubsection{Derivatives}
The derivative of a function $y = f(x)$, notated $f^\prime(x)$, gives the slope of the tangent line to $f$ at $x$. It's defined as
\begin{equation*}
f^\prime(x) = \lim\limits_{h \to 0}{\frac{f(x+h) - f(x)}{h}}
\end{equation*}

\noindent
Below are some properties of the derivative. Let $f$ and $g$ be functions of $x$ and $p$ a scalar.
\begin{enumerate}[label=]
	\item Linearity
	\begin{equation*}
	\left( pf \pm g \right)^\prime = pf^\prime \pm g^\prime
	\end{equation*}
	\item Product Rule
	\begin{equation*}
	\left( fg \right)^\prime = f^\prime g + f g^\prime
	\end{equation*}
	\item Quotient Rule
	\begin{equation*}
	\left( \frac{f}{g} \right)^\prime = \frac{f^\prime g - f g^\prime}{g^2}
	\end{equation*}
	\item Chain Rule
	\begin{equation*}
	\left( f \circ g \right)^\prime = \left( f^\prime \circ g \right) \cdot g^\prime
	\end{equation*}
	\item Power Rule
	\begin{equation*}
	\dd{x} p^x = px^{p-1} \text{, } p \neq 0
	\end{equation*}
	\item Exponent Rule
	\begin{equation*}
	\dd{x} p^x = p^x \ln{p} \text{, } p > 0
	\end{equation*}
	\item The Power Rule and Exponent Rule are two cases of the same rule
	\begin{equation*}
	\dd{x} f^g = gf^{g-1}f^\prime + f^g\ln{f}g^\prime
	\end{equation*}
\end{enumerate}
Using the definition of the derivative and these rules, we can find the derivatives to some common functions
\begin{enumerate}[label=]
	\item \begin{equation*}
	\dd{x} p = 0
	\end{equation*}
	\item \begin{equation*}
	\dd{x} e^x = e^x
	\end{equation*}
	\item \begin{equation*}
	\dd{x} \ln{x} = \frac{1}{x}
	\end{equation*}
	\item \begin{equation*}
	\dd{x} \sin{x} = \cos{x}
	\end{equation*}
	\item \begin{equation*}
	\dd{x} \cos{x} = -\sin{x}
	\end{equation*}
	\item \begin{equation*}
	\dd{x} \tan{x} = \sec^2{x}
	\end{equation*}
\end{enumerate}
		\chapter{Applications of the Derivative}

\section{Extreme Values}
Extreme values of a function are the minimum and maximum values it attains on an interval.
The absolute extreme values would the the extreme values accross the function's entire domain.

\begin{example}
	Find the extreme values of $x^2$ over the following intervals.
	\begin{enumerate}
		\item $(-\infty, \infty)$
		\item $[0,2]$
		\item $(0,2]$
		\item $(0,2)$
	\end{enumerate}
\end{example}
\begin{enumerate}
	\item For the max, there is no maximum value because we can keep increasing $x$ to get a larger output.
		The min is 0 when $x=0$.
	\item The max is 4 when $x=2$.
		The min is 0 when $x=0$.
	\item The max is 4 when $x=2$.
		Although the function approaches 0 in the limit as $x$ approaches 0, there is no min because 0 itself is not a value $x^2$ can take on the interval.
	\item Although the function approaches 4 as $x$ approaches 2, there is no max because 4 itself is not a value $x^2$ can take on the interval.
		Like in the previous question, there is no min.
\end{enumerate}
We see that a function can fail to have a max or min value, but this cannot happen on a finite, closed interval.

\begin{theorem}[Extreme Value Theorem]
	If $f$ is continuous on some finite, closed interval $[a,b]$, then $f$ must have both a minimum and maximum value on the interval.
\end{theorem}

\noindent
We can find these extreme values by following these steps.
\begin{enumerate}
	\item Find any relative/local minima and maxima.
	\item Find the function values for these local minima and maxima.
	\item Find the function values at the endpoints of the interval, $a$ and $b$.
	\item The smallest such function value will be the absolute minima, while the largest such function value will be the absolute maxima.
\end{enumerate}

\begin{theorem}
	If a function $f$ has a local extrema at a point $c$ interior to its domain and $f^\prime$ exists at $c$, then
	\begin{equation*}
		f^\prime(c) = 0.
	\end{equation*}
\end{theorem}
\noindent
So, the only candidate values we need to check are the endpoints and where the derivative is 0.
Such points are called critical points.

\begin{example}
	Find the absolute extrema of $y = x^3 + x^2 - 8x + 5$ on the interval $[-3,2]$.
\end{example}
\begin{enumerate}
	\item We'll take the derivative to find all the critical points.
			$y^\prime = 3x^2 + 2x - 8$, which is 0 when $x=-2$ and $x=4/3$.
	\item $(-2)^3 + (-2)^2 - 8(-2) + 5 = 17$ and $(4/3)^3 + (4/3)^2 - 8(4/3) + 5 = -41/27$.
	\item $(-3)^3 + (-3)^2 - 8(-3) + 5 = 11$ and $(2)^3 + (2)^2 - 8(2) + 5 = 1$.
	\item Of these values, $(-2,17)$ is the absolute maxima and $(4/3, -41/27)$ is the absolute minima.
\end{enumerate}
\section{Mean Value Theorem for Derivatives}
\begin{theorem}[Mean Value Theorem for Derivatives]
	If $f$ is continuous on the interval $[a,b]$ and differentiable on the interval $(a,b)$, then there exists at least one point in $(a,b)$ such that
	\begin{equation*}
		f^\prime(c) = \frac{f(b)-f(a)}{b-a}.
	\end{equation*}
\end{theorem}

That is, there's at least one point where the instantaneous rate of change and average rate of change are equal.
Another way of visualizing this is that there's at least one point where the tangent and secant lines are parallel.

\begin{figure}[H]
	\label{mvt_derivatives}
	\centering
	\includegraphics[width = 0.5\textwidth]{./applications_derivative/mvt.png}
	\caption{\hyperref{https://en.wikipedia.org/wiki/Mean\_value\_theorem}{}{}{Wikipedia - Mean Value Theorem}}
\end{figure}

\begin{example}
	A trucker drives 150 miles of a route in 2 hours.
	The speed limit along the route is 65 miles per hour.
	Show that at at least one point, the trucker must have been speeding.
\end{example}
\begin{answer}
	We can model the trucker's position along the route $s$ as a function of time $t$ where $s(0)=0$ and $s(2)=150$.
	We know that velocity is the derivative of position, so $v(t) = s^\prime(t)$.
	It's reasonable to assume that $s$ is differentiable on the interval $[0,2]$.
	So, by the Mean Value Theorem, there must exist a point $c$ where
	\begin{equation*}
		v(t) = \frac{s(2)-s(0)}{2-0} = \frac{150-0}{2} = 75\text{mph}.
	\end{equation*}
	At this point, the trucker was exceeding the speed limit of 65 miles per hour.
\end{answer}

\begin{definition}
	Let $f$ be defined on an interval $I$.
	Let $a$ and $b$ be any two different points in $I$.
	\begin{align*}
		\text{$f$ increases on $I$ if } a < b &\implies f(a) < f(b). \\
		\text{$f$ decreases on $I$ if } a < b &\implies f(a) > f(b).
	\end{align*}
\end{definition}

\begin{corollary}
	Let $f$ be continuous of $[a,b]$ and differentiable on $(a,b)$.
	\begin{align*}
		\text{If $f^\prime > 0$ at every point on $(a,b)$ then $f$ increases on $[a,b]$}. \\
		\text{If $f^\prime < 0$ at every point on $(a,b)$ then $f$ decreases on $[a,b]$}.
	\end{align*}
\end{corollary}

This should make sense given our theorem about local extrema.
If $f$ could still increase/decrease while its derivative was negative/positive, then we couldn't be sure that $f$ is at a local maxima/minima when $f^\prime=0$.

\begin{corollary}
	If $f^\prime(x) = 0$ at all points in an interval $I$, then there is some constant $C$ such that $f(x) = C$ for all points in $I$.
\end{corollary}

This follows from the Mean Value Theorem.
Since $f^\prime = 0$, the numerator in the Mean Value Theorem, $f(b) - f(a)$, must also be 0, meaning $f(b) = f(a) = C$.

\begin{corollary}
	If $f^\prime(x) = g^\prime(x)$ at every point in some interval $I$, then there is come constant $C$ such that $f(x) = g(x) + C$.
\end{corollary}

That is, functions with the same derivative differ by a constant.
This should make sense given our constant and sum and difference derivative rules.
If we let $h^\prime(x) = f^\prime(x) - g^\prime(x) = 0$ and apply the previous corollary, we get $C$.

\begin{definition}
	A function $F(x)$ is the antiderivative	of $f(x)$ if $F^\prime(x) = f(x)$ for all points in the domain of $f$.
\end{definition}

As we saw in the previous corollary, a function will have infinitely many antiderivatives that differ by a constant.
\section{First and Second Derivative Tests}
\subsection{First Derivative Test}
As we saw in previous sections, we can use the first derivative to find critical values, which allow us to find extrema.
\begin{theorem}[First Derivative Test]
	Let $f(x)$ be a continuous function.
	At a critical point $c$,
	\begin{enumerate}
		\item If $f^\prime$ changes sign from positive to negative ($f^\prime(x) > 0$ for $x < c$ and $f^\prime(x) < 0$ for $x > c$), then $f$ has a local maximum at $c$.
		\item If $f^\prime$ changes sign from negative to positive ($f^\prime(x) < 0$ for $x < c$ and $f^\prime(x) > 0$ for $x > c$), then $f$ has a local minimum at $c$.
		\item If $f^\prime$ does not change sign at $c$, then $f$ does not have a local extrema at $c$.
		\item At a left endpoint $a$, if ($f^\prime < 0$ / $f^\prime > 0$), then $f$ has a local (maximum / minimum) at $a$.
		\item At a right endpoint $b$, if ($f^\prime < 0$ / $f^\prime > 0$), then $f$ has a local (minimum / maximum) at $b$.
	\end{enumerate}
\end{theorem}

\begin{example}
	Find the local extrema of $f(x) = x^3 - 12x - 5$.
	Identify any absolute extrema.
\end{example}
Taking the derivative,
\begin{equation*}
	f^\prime(x) = 3x^2 - 12 = 3(x+2)(x-2).
\end{equation*}
\indent
So, the critical values are $x=-2$ and $x=2$.
$f^\prime$ is negative between these values and positive outside of them, so $x=-2$ is a local maximum, while $x=2$ is a local minimum.
\begin{equation*}
	f(-2) = 11 \text{ and } f(2) = -21.
\end{equation*}
\indent
As $x$ approaches $\infty$ (the right endpoint), $f$ also approaches $\infty$, so there is no absolute maximum.
Similarly, as $x$ approaches $-\infty$ (the left endpoint), $f$ also approaches $-\infty$, so there is no absolute minimum.

\subsection{Second Derivative Test}
The second derivative tells us how the derivative is changing.
Whether the derivative is increasing or decreasing describes the concavity.
\begin{definition}
	On some open interval $I$, the graph of a twice differentiable function $f(x)$ is
	\begin{enumerate}
		\item Concave up if $f^{\prime\prime} > 0$ on $I$.
		\item Concave down if $f^{\prime\prime} < 0$ on $I$.
	\end{enumerate}
\end{definition}

\noindent
Concave up portions of a graph tend to look like valleys, while concave down portions tend to look like hills.
Unlike the first derivative, we can tell if a function is increasing or decreasing using concavity.
We can however say if the function is increasing or decreasing more or less rapidly.

\begin{figure}[H]
	\label{mvt}
	\centering
	\includegraphics[width = 0.5\textwidth]{./applications_derivative/concavity.png}
	\caption{\hyperref{https://tutorial.math.lamar.edu/classes/calci/shapeofgraphptii.aspx}{}{}{Paul's Online Notes - The Shape of a Graph, Part II}}
\end{figure}

\noindent
The points where a function changes concavity are called "inflection points".
At these points, the function is increasing or decreasing most rapidly, depending on the sign of the first derivative.

\noindent
Rather than looking at the sign of $f^\prime$ around critical values, we can look at the concavity at the critical point.
\begin{theorem}[Second Derivative Test]
	Let $c$ be a critical value of $f$.
	\begin{enumerate}
		\item If $f^{\prime\prime}(c) < 0$, then $c$ is a local maximum.
		\item If $f^{\prime\prime}(c) > 0$, then $c$ is a local minimum.
		\item If $f^{\prime\prime}(c) = 0$, then the test is inclusive.
	\end{enumerate}
\end{theorem}
\noindent
In other words, if we're at a critical point that's turning into a valley, then the critical point must be the top of a hill.
If we're at a critical point that's turning into a hill, then the critical point must be the bottom of a valley.
This test is particularly useful because you only need to know $f^{\prime\prime}$ at $c$ rather than an entire interval\footnote{It also extends to higher dimensions much better than the first derivative test.}.

\begin{example}
	Use the second derivative test to find the local extreme values of $f(x) = x^3 - 12x - 5$.
\end{example}
Finding critical values,
\begin{align*}
	f^\prime(x) &= 3x^2 - 12 = 3(x+2)(x-2). \\
	f^\prime(x) &= 0\text{ at } x=-2, x=2.
\end{align*}
\indent
Taking the second derivative,
\begin{equation*}
	f^{\prime\prime}(x) = 6x.
\end{equation*}
\indent
Evaluating the second derivative at the critical values,
\begin{equation*}
	f^{\prime\prime}(-2) = -12 \text{ and } f^{\prime\prime}(2) = 12.
\end{equation*}
\indent
So, $x=-2$ is a local maximum and $x=2$ is a local minimum.
\section{Modeling \& Optimization}
Now that we know some ways to find extreme values, we can apply them to answer optimization problems.
The general steps needed to solve an optimization problem are:
\begin{enumerate}
	\item Write an equation that represents what you're trying to maximize/minimize. This is called your primary equation.
	\item Use additional information to eliminate excess variables.
	\item Find extreme values.
	\item Select the extreme values that fit the problem's constraints. Make sure your answer is what the problem is asking for.
\end{enumerate}

\begin{example}
	A farmer has 1000 linear feet of fence and wants to create a rectangular pasture.
	The pasture borders a river, which doesn't need a fence.
	What is the maximum area he can enclose?
\end{example}
\begin{answer}
	\begin{enumerate}
		\item Since the pasture is rectangular, we know the two lengths of fence perpendicular to the river will be the same length, which we'll call $x$.
			We'll say the remaining side parallel to the river has length $y$.
			So, the area enclosed is
			\begin{equation*}
				A(x,y) = xy.
			\end{equation*}
		\item There is no reason for the farmer not to use all 1000 linear feet of fence, so we'd expect the sum of the lengths of the 3 fenced sides to be equal to 1000 feet: $2x + y = 1000$, or $y = 1000 - 2x$.
			We can then substitute back into our primary equation to get it in terms of just $x$.
			\begin{equation*}
				A(y) = x(1000-2x) = 1000x - 2x^2.
			\end{equation*}
		\item We'll do a first derivative test to find extreme values.
		\begin{align*}
			A^\prime(y) &= 1000 - 4x \\
			A^\prime(y) &= 0 \text{ at } x = 250 \\
			A^\prime(200) &= 200 > 0 \\
			A^\prime(300) &= -200 < 0.
		\end{align*}
		\item So, $x=250 \implies y = 500$ is a local maximum.
			This corresponds to an area of $A = 250\cdot 500 = 125000\text{ft}^2$.
	\end{enumerate}
\end{answer}

\begin{example}
	What is the maximum area of a rectangle that has two vertices on the $x-axis$ and two vertices on the portion of the graph $y=8-x^2$ where $y > 0$?
\end{example}
\begin{answer}
	\begin{enumerate}
		\item We can define any such trapeziod (which include all such rectangles) by the $x$ values of the vertices on the $x$ axis.
			We'll call these two values $x_1$ and $x_2$.
			We'll say arbitrarily that $x_1 < x_2$.
			So, the area enclosed is
			\begin{equation*}
				A(x_1, x_2) = (x_2 - x_1)\frac{f(x_1) + f(x_2)}{2}.
			\end{equation*}
		\item However, the problem specifically restricts us to a rectangle, not a trapezoid.
			So, the heights of each side must be equal.
			\begin{align*}
				f(x_1) &= f(x_2) \\
				8 - x_1^2 &= 8 - x_2^2 \\
				x_1^2 &= x_2^2 \\
				\pm x_1 &\ \pm x_2 \\
				-x_1 &= x_2 \text{ because $x_1 \neq x_2$}.
			\end{align*}
			Substituting back into our primary equation,
			\begin{equation*}
				A(x_2) = (x_2 - (-x_2))\frac{f(x_2) + f(-x_2)}{2} = 2x_2f(x_2) = 16x_2 - 2x_2^3.
			\end{equation*}
		\item We'll do a second derivative test for fin extreme values.
			\begin{align*}
				A^\prime(x_2) &= 16 - 6x_2^2 \\
				A^\prime(x_2) &= 0 \text{ at } x = \pm\frac{4}{\sqrt{6}} \\
				A^{\prime\prime}(x_2) &= -12x_2 \\
				A^{\prime\prime}\left(-\frac{4}{\sqrt{6}}\right) &= 8\sqrt{6} > 0 \\
				A^{\prime\prime}\left(\frac{4}{\sqrt{6}}\right) &= -8\sqrt{6} < 0.
			\end{align*}
		\item So, $x_2 = 4/\sqrt{6}$ is a local maximum, and $x_2 = -4/\sqrt{6}$ is a local minimum.
			The problem asks for a maximum, so we select $x_2 = 4/\sqrt{6}$.
			The problem asks for a maximum area, so
			\begin{equation*}
				A_{max} = 16\left(\frac{4}{\sqrt{6}}\right) - 2\left(\frac{4}{\sqrt{6}}\right)^3 = \frac{128}{3\sqrt{6}} \approx 17.419.
			\end{equation*}
	\end{enumerate}
\end{answer}
\section{Related Rates}
In related rates problems, we generally have two related functions and want to know and answer questions about the rate of change of one function given that we know the rate of change of the other. The same problem solving steps as in modeling and optimization apply, but we'll usually be taking derivatives using implicit differentiation with respect to some variable like time.

\begin{example}
	Let $A$ be the area of a square with side length $x$.
	Assume that $x$ varies with time.
	How are $\dd{A}{t}$ and $\dd{x}{t}$ related?
	At a certain instant, the sides are 3 feet and growing at a rate of 2 feet per minute.
	How quickly is the area changing at this instant.
\end{example}
\begin{answer}
	Starting with the area of the square and implicitly differentiating with respect to $t$,
	\begin{align*}
		A &= x^2 \\
		\dd{A}{t} = 2x\dd{x}{t}.
	\end{align*}
	
	When $x=3\text{ft}$ and $\dd{x}{t}=3\text{ft/min}$,
	\begin{equation*}
		\dd{A}{t} = 2(3\text{ft})(3\text{ft/min}) = 12\text{ft$^2$/min}.
	\end{equation*}
\end{answer}

\begin{example}
	The top of a 13 foot ladder propped against a vertical wall begins falling towards the ground at 12ft/s.
	When the top of the ladder is 5 feet off the ground, how quickly is the bottom of the ladder moving away from the wall?
	How how is the angle between the ladder and the ground changing?
\end{example}
\begin{answer}
	Let $h$ be the height of the top of the ladder of the ground.
	Then $\dd{h}{t} = -12\text{ft/s}$.
	Let $b$ the the distance from the base of the ladder to the wall.
	We can relate $b$ and $h$ using the Pythagorean Theorem, where the 13-foot long ladder is the hypotenuse.
	\begin{equation*}
		b^2 + h^2 = 13^2.
	\end{equation*}
	
	We can also use this relationship to see that when $h=5\text{ft}$, $b=12\text{ft}$.
	Implicitly differentiating,
	\begin{equation*}
		2b\dd{b}{t} + 2h\dd{h}{t} = 0.
	\end{equation*}
	
	Plugging in what we know and solving for $\dd{b}{t}$,
	\begin{align*}
		2(12\text{ft})\dd{b}{t} + 2(5\text{ft})(-12\text{ft/s}) &= 0 \\
		24\text{ft}\dd{b}{t} &= 120\text{ft$^2$/s} \\
		\dd{b}{t} &= 5\text{ft/s}.
	\end{align*}
	
	So, the base of the ladder is moving away from the wall at a rate of 5ft/s.
	Let $\theta$ be the angle between the ladder and the ground.
	We can use $\sin$ to relate $\theta$ to $b$.
	\begin{equation*}
		13\sin{\theta} = b.
	\end{equation*}
	
	Implicitly differentiating,
	\begin{equation*}
		13\text{ft}\cos{(\theta)}\dd{\theta}{t} = \dd{b}{t}.
	\end{equation*}
	
	When $\cos$ is adjacent divided by hypotenuse, so $\cos{\theta} = 12/13$.
	Plugging in what we know and solving for $\dd{\theta}{t}$,
	\begin{align*}
		13\text{ft}(12/13)\dd{\theta}{t} &= -12\text{ft/s} \\
		\dd{\theta}{t} &= -1\text{/s}.
	\end{align*}
	
	So, the angle between the ladder and ground is decreasing at at rate of 1 rad/s.
\end{answer}

\begin{example}
	Grain is is poured at a rate of 10ft$^3$/min and falls into a cone-shaped pile whose bottom radius is half its altitude.
	How fast will the circumference of the base be increasing when the pile is 8 ft tall?
\end{example}
\begin{answer}
	Let $h$ the the altitude of the cone.
	At the instant we care about $h=8\text{ft}$.
	Let $r$ be the bottom radius of the cone.
	At the instant we care about, $r=h/2=4\text{ft}$.
	Let $V$ be the volume of the cone.
	We know that $\dd{V}{t}=10\text{ft$^3$/s}$.
	We can relate these three quantites using the formula for the volume of a cone.
	\begin{equation*}
		V = \frac{1}{3}\pi r^2 h.
	\end{equation*}
	
	Since we know that $2r = h$, we can simplify to get rid of $h$.
	\begin{equation*}
		V = \frac{2}{3}\pi r^3
	\end{equation*} 
	
	Implicitly differentiating,
	\begin{equation*}
		\dd{V}{t} = 2\pi r^2 \dd{r}{t}.
	\end{equation*}
	
	We know the formula for the circumference $C$ of the circular base.
	\begin{equation*}
		C = 2\pi r.
	\end{equation*}
	
	Implicitly differentiating,
	\begin{equation*}
		\dd{C}{t} = 2\pi \dd{r}{t}.
	\end{equation*}
	
	We can substitute into our equation involving $\dd{V}{t}$.
	\begin{equation*}
		\dd{V}{t} = r^2\dd{C}{t}.
	\end{equation*}
	
	Plugging in what we know,
	\begin{align*}
		10\text{ft}^3\text{/s} &= (4\text{ft})^2\dd{C}{t} \\
		\dd{C}{t} &= \frac{5}{8}\text{ft/s}.
	\end{align*}
\end{answer}

\section{Linearization \& Newton's Method}

\subsection{Linearization}
As we've seen, tangent lines intersect their function at most once: at the point of tangency.
However, we know that differentiable functions are locally linear, so we'd expect the tangent line to be a decent approximation of the function near the point of tangency.

\begin{definition}
	If $f$ is differentiable at $a$, then the approximating function
	\begin{equation*}
		L(x) = f^\prime(a)(x-a) + f(a)
	\end{equation*}
	is the linearization of $f$ at $a$.
\end{definition}

\begin{example}
	Find the linearization of $f(x) = \ln{(x+1)}$ at $x=0$.
	How accurate is this approximation at $x=0.1$?
\end{example}
Following the definition,
\begin{align*}
	f(0) = \ln{(0+1)} = 0 \\
	f^\prime(x) &= \frac{1}{x+1} \\
	f^\prime(0) &= \frac{1}{0+1} = 1 \\
	L(x) &= 1(x-0) + 0 = x.
\end{align*}
\indent
Calculating the error at $x=0.1$,
\begin{align*} 
	L(0.1) &= 0.1 \\
	f(0.1) &\approx .0953 \\
	\text{\% error} &= \frac{\abs{L(0.1)-f(0.1)}}{L(0.1)}100\text{\%} \approx 4.7\text{\%}.
\end{align*}
\indent
So, we can see the linear approximation is pretty good.

\noindent
We call the difference in $x$ between the point of tangency and the point we're trying to approximate the differential.
\begin{definition}
	Let $y=f(x)$ be a differentiable function.
	The differential $\mathrm{d}x$ is an independent variable.
	The differential $\mathrm{d}y$ is $\mathrm{d}y = f^\prime(x)\mathrm{d}x$.
\end{definition}
\noindent
$\mathrm{d}y$ is the approximated change in $y$ expected by the linearization for some given change in $x$, $\mathrm{d}x$.

\begin{example}
	Find $\mathrm{d}y$ for $y=\frac{2x}{1+x^2}$, $x=-2$, and $\mathrm{d}x = 0.1$.
\end{example}
\begin{align*}
	y^\prime &= \frac{-4x^2}{\left(1+x^2\right)^2} + \frac{2}{1+x^2} \\
	y^\prime(-2) &= -6/25 \\
	\mathrm{d}y &= (-6/25)(0.1) = -0.024.
\end{align*}

\subsection{Newton's Method}
We can use the fact that the tangent line approximates the function to find the zeroes of functions.
Starting with an initial guess $x_0$ for the $x$ value of the zero, we look at the the tangent line at $x_0$ and find where it intersects the $x-axis$.
\begin{align*}
	L_0(x) &= f^\prime(x_0)(x - x_0) + f(x_0) \\
	0 &= f^\prime(x_0)(x - x_0) + f(x_0) \\
	-f^\prime(x_0)(x - x_0) &= f(x_0) \\
	x - x_0 &= -\frac{f(x_0)}{f^\prime(x_0)} \\
	x &= x_0 - \frac{f(x_0)}{f^\prime(x_0)}.
\end{align*}
This $x$ value serves as our next guess for the zero.
We repeat this process, until we find the zero or are satisfied with our error\footnote{For most well-behaved functions, Newton's Method can get within a small margin of error or a zero relatively quickly. There is also a generalized, sometimes faster version of Newton's Method that approximates the function with higher-order polynomials than just lines.}.
This yields a recursive formula
\begin{equation*}
	x_{n+1} = x_n - \frac{f(x_n)}{f^\prime(x_n)}.
\end{equation*}
		\chapter{Integrals}

\section{Estimating with Sums}
If we know how something changes, we can use sums to estimate, or sometimes even know exactly the net change.

\begin{example}
	A train moves at 80 miles per hour for 3 hours.
	How far does it travel?
\end{example}
\begin{answer}
	This is the kind of simple problem you might see in a physics class.
	\begin{equation*}
		\frac{80\text{mi}}{\text{hr}} \hspace{3pt} 3\text{hr} = 80\cdot 3\text{mi} = 240 \text{mi}.
	\end{equation*}
	However, if we take a look at a graph of this situation, with speed in miles per hour on the $y$ axis and time in hours on the $x$ axis, we see that the area underneath the curve from $x=0\text{hr}$ to $x=3\text{hr}$ is exactly our answer of 240 miles.
	This is not a coincidence, as what we effectively did mathematically is find the area of this rectangle.
	
	\begin{figure}[H]
		\label{constant_graph}
		\centering
		\includegraphics[width = 0.33\textwidth]{./integrals/constant_graph.png}
		\caption{\hyperref{}{}{}{Our answer is the area under the curve.}}
	\end{figure}
	\end{answer}

The same idea of finding the area under the curve works not just for constant speeds.
\begin{example}
	A particle moves at velocity $v(t) = 3t + 3$ meters per second for time $t \geq 0$.
	What is the position of the particle at $t = 2$ seconds?
\end{example}
\begin{answer}
	\begin{figure}[H]
		\label{linear_graph}
		\centering
		\includegraphics[width = 0.5\textwidth]{./integrals/linear_graph.png}
		\caption{\hyperref{}{}{}{Our answer is still the area under the curve.}}
	\end{figure}
	\begin{align*}
		A &= \frac{1}{2}h(b_1 + b_2) \\
		&= \frac{1}{2}(2\text{s}-0\text{s})(v(0) + v(2)) \\
		&= \frac{1}{2}(2\text{s})(3\text{m/s} + 9\text{m/s}) \\
		&= 12\text{m}.
	\end{align*}
\end{answer}

\subsubsection{Left Endpoint Approximation}
In fact, the area under the curve even works for more complicated, non-linear curves like $v(t) = t^2 + 1$.
We just need a way to find the area underneath these curves.
One idea that was used to find areas as far back as Archimedes was to estimate the complex shape using easier shapes like rectangles.
The narrower the width of each rectangle, the better the estimate becomes.
 
\begin{figure}[H]
	\label{cos_blocks}
	\centering
	\includegraphics[width = 0.3\textwidth]{./integrals/cos_blocks1.png}
	\includegraphics[width = 0.3\textwidth]{./integrals/cos_blocks2.png}
	\includegraphics[width = 0.3\textwidth]{./integrals/cos_blocks3.png}
	\caption{\hyperref{}{}{}{Left Endpoint: Block widths of 1/3, 1/10, and 1/100.}}
\end{figure}


Although above we've used a left endpoint approximation, we could have also used the right endpoint or midpoint, which might give better approximations for certain types of curves.
No matter the approximation type, the estimate tends to get closer to the true area as the rectangle width is decreased.
The formulas for estimating the area of $f$ from $a$ to $b$ with $n$ rectangles are
\begin{align*}
	A_\text{left} &= \sum_{k=0}^{n-1}{f\left(a+k\Delta x\right)\Delta x} \\
	A_\text{right} &= \sum_{k=0}^{n-1}{f\left(a+(k+1)\Delta x\right)\Delta x} \\
	A_\text{mid} &= \sum_{k=0}^{n-1}{f\left(a+\frac{2k+1}{2}\Delta x\right)\Delta x} \\
	\text{where }\Delta x &= \frac{b-a}{n}.
\end{align*}

\subsubsection{Trapezoidal Rule}
Another common shape to use rather than rectangles is trapezoids.
Trapezoids allow us to find the exact area of functions made of straight lines\footnote{There are higher-order, more accurate estimations than the trapezoidal rule. The most common is Simpson's Rule. $A = \frac{2M+T}{3}$ where $M$ is the midpoint formula area and $T$ is the trapezoidal rule area. It can exactly give the area of quadratics because it corresponds to estimating areas using parabolas that intersect the curve at the left, middle, and right of each "strip".} and like the left endpoint approximation, get better the narrower the width of each trapezoid.
Starting from the formula, we can make some simplifications.
\begin{equation*}
	A = \frac{1}{2}(b_1 + b_2)h
\end{equation*}
Now summing each of these trapezoids' areas to approximate our function $f$ from $a$ to $b$,
\begin{align*}
	A_\text{trap} &= \sum_{k=0}^{n-1}{\frac{1}{2}\left(f(a+i\Delta a) + f(a + (i+1)\Delta x)\right)\Delta x} \\
	&= \frac{\Delta x}{2}\sum_{k=0}^{n-1}{f(a + i\Delta x) + f(a + (i+1)\Delta x)} \\
	&= \frac{\Delta x}{2}\left(\left(f(a)+f(a+\Delta x)\right)+\left(f(a+\Delta x)+f(a+2\Delta x)\right)+\ldots+\left(f(a+(n-1)\Delta x)+f(a+n\Delta x)\right)\right) \\
	&= \frac{\Delta x}{2}\left(f(a) + 2f(a+\Delta x) + 2f(a + 2\Delta x) + \ldots + 2f(a+(n-1)\Delta x) + f(a+n\Delta x)\right) \\
	&= \frac{A_\text{left} + A_\text{right}}{2}.
\end{align*}

The trapezoidal rule overestimates areas when the curve is concave up and underestimates when the curve is concave down.
\section{Definite Integrals \& Antiderivatives}

\subsection{Definition of a Definite Integral}
We saw several approximations for the areas under any type of curve.
We also saw how these approximations get better the narrower our "strips".
In the limit, we get exactly the area under the curve, which defines the definite integral.

\begin{definition}
	Let $f$ be continuous on the closed interval $[a,b]$.
	Let this interval be partitioned into $n$ equal\footnote{Technically, the intervals don't need to be of equal size, as long as the width of all intervals goes to 0 in the limit.} sub-intervals, each of length $\Delta x = \frac{b-a}{n}$.
	The definite integral of $f$ over $[a,b]$ is given by
	\begin{equation*}
		\int_{a}^{b}{f(x)\mathrm{d}x} = \lim_{n\to \infty}{\sum_{i=0}^{n-1}{f(c_i)\Delta x}}
	\end{equation*}
	where $c_k$ is any arbitrary value in the $k$th sub-interval.
	This particular type of limit of a sum is called a Riemann Sum.
\end{definition}
\noindent
Note that the left, right, and midpoint approximations are all different ways of choosing $c_i$, all of which in the limit give the area under the curve and definite integral from $a$ to $b$.

\begin{example}
	Find the value of the following definite integral using Riemann sums.
	\begin{equation*}
		\int_{0}^{1}{x^2\d{x}}.
	\end{equation*}
\end{example}
We'll use a Riemann sum with a left-endpoint approximation.
\begin{align*}
	\int_{0}^{1}{x^2\d{x}} &= \lim_{n\to\infty}{\sum_{i=0}^{n-1}{f\left(0+\frac{i}{n}\right)\frac{1-0}{n}}} \\
	&= \lim_{n\to\infty}{\sum_{i=0}^{n-1}{\left(\frac{i}{n}\right)^2\frac{1}{n}}} \\
	&= \lim_{n\to\infty}{\sum_{i=0}^{n-1}{\frac{i^2}{n^3}}} \\
	&= \lim_{n \to \infty}{\frac{1}{n^3}\left(\frac{n(n-1)(2n-1)}{6}\right)} \\
	&= \lim_{n\to\infty}{\frac{1}{6}\left(2-\frac{3}{n}+\frac{1}{n^2}\right)} \\
	&= \frac{1}{3}.
\end{align*}

\noindent
Note that area below the $x$-axis is counted as negative area.
In general
\begin{equation*}
	\int_{a}^{b}{f(x)\d{x}} = \text{ area above $x$-axis } - \text{ area below $y$-axis}.
\end{equation*}

\subsection{Basic Properties of Definite Integrals}
The following are properties of definite integrals.
Many should look familiar from properties of limits and derivatives.
Let $f$ and $g$ be continuous functions of $x$.
Let $a$, $b$, and $c$ be real constants where $a \neq b$.
Let $f_{[a,b]}$ be the values of $f$ on the interval $[a,b]$.
\begin{align*}
	\textbf{Order of Integration Rule: }& \int_{a}^{b}{f(x)\d{x}} = -\int_{b}^{a}{f(x)\d{x}} \\
	\textbf{Sum and Difference Rule: }& \int_{a}^{b}{(f(x) \pm g(x))\d{x}} = \int_{a}^{b}{f(x)\d{x}} \pm \int_{a}^{b}{g(x)\d{x}} \\
	\textbf{Zero Rule: }& \int_{a}^{a}{f(x)\d{x}} = 0 \\
	\textbf{Constant Multiple Rule: }& \int_{a}^{b}{cf(x)\d{x}} = c\int_{a}^{b}{f(x)\d{x}} \\
	\textbf{Additivity Rule: }& \int_{a}^{b}{f(x)\d{x}} + \int_{b}^{c}{f(x)\d{x}} = \int_{a}^{c}{f(x)\d{x}} \\
	\textbf{Max-Min Rule: }& (b-a)\min{f_{[a,b]}} \leq \int_{a}^{b}{f(x)\d{x}} \leq (b-a)\max{f_{[a,b]}} \\
	\textbf{Domination Rule: }& \min{f_{[a,b]}} \geq \max{g_{[a,b]}} \implies \int_{a}^{b}{f(x)\d{x}} \geq \int_{a}^{b}{g(x)\d{x}}.
\end{align*}

\subsubsection{Mean Value Theorem for Definite Integrals}
The mean value of $f$ over the interval $[a,b]$ is given by
\begin{equation*}
	\frac{1}{b-a}\int_{a}^{b}{f(x)\d{x}}.
\end{equation*}
You can effectively think of this as the definition of the average: adding up all the values and dividing by the number of values.
This is exactly what happens if you work through the Riemann sums.
Much like the Mean Value Theorem for Derivatives, since the function is continuous, it will take on its average value somewhere in the interval.
\begin{theorem}[Mean Value Theorem for Definite Integrals]
	If $f$ is continuous on the interval $[a,b]$, then there is some point $c$ in the interval such that
	\begin{equation*}
		f(c) = \frac{1}{b-a}\int_{a}^{b}{f(x)\d{x}}.
	\end{equation*}
\end{theorem}

\begin{figure}[H]
	\label{mvt}
	\centering
	\includegraphics[width = 0.33\textwidth]{./integrals/mvt.png}
	\caption{\hyperref{}{}{}{A continuous function always takes on its mean value}}
\end{figure}

\begin{example}
	A plane's airspeed is given by $v(t) = t/8\text{, }t\geq 0$.
	What what time $t$ does the Mean Value Theorem guarantee that the plane's speed was 1?
	When is this time?
\end{example}
Rather than go through the trouble of evaluating a limit to find the definite integral, we can notice that the graph of the plane's airspeed is a triangle, which we know how to find the area of geometrically.
\begin{equation*}
	\text{Area}(t) = \frac{1}{2}tv(t).
\end{equation*}
\indent
Dividing by the length of the interval to get the average airspeed,
\begin{equation*}
	\text{Avg}(t) = \frac{1}{t-0}\hspace{3pt}\frac{1}{2}tv(t) = \frac{1}{2}v(t).
\end{equation*}
\indent
Solving for $t$ when $\text{Avg}(t)=1$,
\begin{equation*}
	\frac{1}{2}\hspace{3pt}\frac{t}{8} = 1 \implies t = 16.
\end{equation*}
\indent
So, at time $t=16$, the Mean Value Theorem tells us that the plane's speed was 1.
Looking at the graph, we can see that at time $t=8$ the plane's speed was indeed 1.

\subsection{Fundamental Theorem of Calculus}
Although it's nice to be able to evaluate definite integrals for numerical bounds, it'd be more convenient if we only had to do the work of integrating once and could then have a function that would tell us the area like so:
\begin{equation*}
	F(x) = \int_{a}^{x}{f(t)\d{t}}.
\end{equation*}
Let's try taking the derivative of $F$ using the limit definition.
\begin{align*}
	F^\prime(x) &= \lim_{\Delta x \to 0}{\frac{F(x+\Delta x)-F(x)}{\Delta x}} \\
	&= \lim_{\Delta x\to 0}{\frac{\int_{a}^{x+\Delta x}f(t)\d{t} - \int_{a}^{x}{f(t)\d{t}}}{\Delta x}} \\
	&= \lim_{\Delta x\to 0}{\frac{\int_{x}^{x+\Delta x}{f(t)\d{t}}}{\Delta x}}\text{ (by Integral Rules)}\\
	&= \lim_{\Delta x\to 0}{\frac{f(k)\Delta x}{\Delta x}}\text{, }x\leq k \leq x + \Delta x\text{ (by Mean Value Theorem)} \\
	&= \lim_{\Delta x\to 0}{f(k)} \\
	&= f(x) \text{ (by Sandwich Theorem)}.
\end{align*}
Taking the derivative seems to undo the integration.
The same fact applies if we take the integral of a derivative.
Starting with what we've just shown,
\begin{equation*}
	\dd{}{x}\int_{0}^{x}{f(t)\d{t}} = f(x).
\end{equation*}
Let $g(x) = \dd{}{x}f(x)$.
\begin{equation*}
	\dd{}{x}\int_{0}^{x}{g(t)\d{t}} = g = \dd{}{x}f(x).
\end{equation*}
Since the derivatives are equal, we know the functions must differ by at most a constant $C$.
\begin{align*}
	\int_{0}^{x}{g(t)\d{t}} &= f(x) + C \\
	\int_{0}^{x}{\left[\dd{}{x}f(t)\right]\d{t}} &= f(x) + C.
\end{align*}
So, the integral of the derivative gets us a function that differs from the original by at most a constant.

\noindent
This idea that the integral and derivative undo each other is captured by the Fundamental Theorem of Calculus.
\begin{theorem}[Fundamental Theorem of Calculus]
	Let $f$ be a continuous function on the interval $[a,b]$.
	Then
	\begin{equation*}
		F(x) = \int_{a}^{x}{f(t)\d{t}}
	\end{equation*}
	has a derivative at every point in $[a,b]$, and
	\begin{equation*}
		\dd{F}{x} = \dd{}{x}\int_{a}^{x}{f(t)\d{t}} = f(x).
	\end{equation*}
	Further, if $F$ is the antiderivative of $f$ on $[a,b]$, then
	\begin{equation*}
		\int_{a}^{b}{f(x)\d{x}} = F(b) - F(a).
	\end{equation*}
\end{theorem}
\noindent
This last equation is especially useful for calculating definite integrals.
Rather than evaluating a limit of a Riemann sum, if we know the antiderivative, we can just evaluate at two points.

\begin{example}
	Evaluate the following definite integral using antiderivatives.
	\begin{equation*}
		\int_{0}^{1}{\frac{\d{x}}{1+x^2}}.
	\end{equation*}
\end{example}

We previously derived that the derivative of $\arctan{x}$ is $\frac{1}{1+x^2}$.
So, $\arctan{x}$ is the antiderivative of $\frac{1}{1+x^2}$.
Applying the Fundamental Theorem of Calculus (FTC)
\begin{align*}
	\int_{0}^{1}{\frac{\d{x}}{1+x^2}} &= \arctan{1} - \arctan{0} \\
	&= \frac{\pi}{4} - 0 \\
	&= \frac{\pi}{4}.
\end{align*} 
		\chapter{Applications of Integrals}

\section{Physics}
\subsection{Position, Velocity, \& Acceleration}
Since we know by the FTC that integration is the opposite of differentiation, we can also interpret integrals in a similar physical sense as derivatives.
\begin{table}[H]
	\begin{center}
		\begin{tabular}{ l l }
			$\begin{aligned}\int{v(t)\d{t}}=x(t) + C\end{aligned}$ & $\begin{aligned}\int{a(t)\d{t}}=v(t)\end{aligned}$
		\end{tabular}
	\end{center}
\end{table}

\begin{example}
	A particle starts at $t=0$ with an initial velocity of 5m/s and accelerates for 8 seconds.
	Its acceleration is given by $a(t)=2.4t$ m/s$^2$.
	What is the particle's velocity after the 8 seconds pass?
	What is the particle's displacement after the 8 seconds pass?
\end{example}
\begin{answer}
	We can integrate acceleration to get velocity.
	\begin{equation*}
		v(t) = \int{2.4t\d{t}} = 1.2t^2 + C.
	\end{equation*}
	
	We know that $v(0)=5$m/s so we can solve\footnote{When we solve for $C$ like this, we're solving what's called an ``inital value problem,'' which we'll do more of when talking about differential equations.} for $C$.
	\begin{align*}
		1.2(0) + C &= 5 \\
		C &= 5 \\
		v(t) = 1.2t^2 + 5.
	\end{align*}
	
	So, at $t=8$, $v(8) = 1.2(8)^2 + 5 = 81.8$m/s.
	We can now integrate velocity to get displacement
	\begin{equation*}
		x(t) = \int{1.2t^2 + 5 \d{t}} = 0.4t^3 + 5t + C.
	\end{equation*}
	
	We know that $x(0)=0$m, so $C=0$.
	\begin{equation*}
		x(t) = 0.4t^3 + 5t.
	\end{equation*}
	
	So, at $t=8$, $x(t) = 0.4(8)^3 + 5(8) = 244.8$m.
	We could have also tackled this problem with definite integrals.
	\begin{align*}
		\Delta x = \int_{0}^{8}{v(t)\d{t}} &= 244.8\text{m} \implies \text{Net Displacement} = x_0 + \Delta x = 244.8\text{m} \\
		\Delta v = \int_{0}^{8}{a(t)\d{t}} &= 76.8\text{m/s} \implies \text{Net Velocity} = v_0 + \Delta v = 81.8\text{m/s}
	\end{align*}
\end{answer}

\subsection{Work}
Work is defined as
\begin{equation*}
	W = Fd
\end{equation*}
where $F$ is force and $d$ is displacement.
The force applied by stretching or compressing a spring beyond its natural length is given by Hooke's Law
\begin{equation*}
	F = kx
\end{equation*}
where $k$ is some spring constant and $x$ is the displacement beyond the spring's natural length.
We can apply ideas as integrals representing net change to find the work needed to compress or stretch a spring.

\begin{example}
	It takes 10N of force to stretch a spring 2m beyond its natural length.
	How much work is done stretching the spring 4m beyond its natural length?
\end{example}
\begin{answer}
	We can use the first bit of information to get the spring constant.
	\begin{align*}
		F &= kx \\
		10\text{N} &= k(2\text{m}) \\
		k &= 5\text{N/m}.
	\end{align*}
	
	So, $F(x)=5x\text{N}$.
	If we stretch the spring by $\Delta x$, the work done over this interval is approximately $\Delta W = F(x)\Delta x= 5x\Delta x$.
	In the limit, $\d{W} = 5x\d{x}$.
	Integrating both sides from $x=0$ to $x=4$,
	\begin{equation*}
		W = \int_{0}^{4}{\d{W}} = \int_{0}^{4}{5x\d{x}} = 40\text{Nm}.
	\end{equation*}
\end{answer}


We can even bring in other concepts to these problems, like related rates.
\begin{example}
	An inverted conical tank with a height of 10ft and a base radius of 5ft is filled to within 2ft of the top with a liquid with a density of 57lbs/ft$^3$.
	How much work does it take to fill the remaining 2ft of the tank with liquid, assuming you only have to pump the liquid to the current liquid level in the tank?
\end{example}
\begin{answer}
	Let $V$ be the volume of the tank and $x$ the height of the liquid.
	Imagine we pump in some liquid that changes the height of the liquid in the tank by $\Delta x$.
	Then
	\begin{align*}
		\Delta V &= \pi r^2 \Delta x \\
		\d{V} &= \pi r^2 \d{x}.
	\end{align*}
	
	Since the height of the tank is 10ft and the base radius 5ft, the radius of the liquid level will always be half the liquid depth.
	\begin{align*}
		r &= x/2 \\
		\d{V} &= \pi (x/2)^2 \d{x}.
	\end{align*}
	
	Since weight in pounds is already a unit of force, we can multiply $\d{V}$ by the density of the liquid to get $F$.
	\begin{equation*}
		F = 57\pi(x/2)^2 \d{x}.
	\end{equation*}
	
	The displacement of the liquid is the current height of the liquid $x$, getting us $\d{W}$.
	\begin{align*}
		\d{W} &= 57\pi x(x/2)^2 \d{x} \\
		W &= \int_{8}^{10}{57\pi x(x/2)^2 \d{x}} \\
		&= 21033\pi \text{ft lbs}.
	\end{align*}
\end{answer}


\section{Lengths of Curves}
If we're at some point $(x,f(x))$ and move $\d{x}$ units over, we'll be at a new point $(x+\d{x},f(x+\d{x}))$.
This point is $\d{x}$ units horizontally and $\d{y}=f(x+\d{x})-f(x)$ vertically away from $(x,f(x))$.
So, by the Pythagorean Theorem, the new point is $\d{s} = \sqrt{(\d{x})^2+(\d{y})^2}$ units away.

\begin{figure}[H]
	\label{arclength}
	\centering
	\includegraphics[width=0.66\textwidth]{./applications_integrals/arclength.png}
	\caption{\hyperref{}{}{}{Secant length approaches arc length}}
\end{figure}
\noindent
As $\d{x}$ approaches 0, $\d{s}$, the length of the secant line, approaches the length of the curve.
If we summed all of these $\d{s}$'s in some interval we'd have the length of the curve on that interval.
\begin{align*}
	s &= \int_{a}^{b}{\d{s}} \\
	&= \int_{a}^{b}{\sqrt{(\d{x})^2+(\d{y})^2}} \\
	&= \int_{a}^{b}{\sqrt{(\d{x})^2\left(1+\left(\frac{\d{y}}{\d{x}}\right)^2\right)}} \\
	&= \int_{a}^{b}{\sqrt{1+\left(\frac{\d{y}}{\d{x}}\right)^2}\d{x}}.
\end{align*}

\begin{example}
	Show that the circumference of a circle with radius $r$ is $C=2\pi r$.
\end{example}
Starting with the equation of a circle of radius $r$ and implicitly differentiating,
\begin{align*}
	x^2 + y^2 &= r^2 \\
	2x + 2y\dd{y}{x} &= 0 \\
	\dd{y}{x} &= \frac{-x}{y} \\
	\left(\dd{y}{x}\right)^2 &= \frac{x^2}{y^2} = \frac{x^2}{r^2-x^2} \\
	C &= 2\int_{-r}^{r}{\sqrt{1+\frac{x^2}{r^2-x^2}}\d{x}} \\ 
	&\text{ (2 b/c we need to count upper and lower half)} \\
	&= 2\int_{-r}^{r}{\sqrt{\frac{r^2}{r^2-x^2}}\d{x}} \\
	&= 2\int_{-r}^{r}{r\sqrt{\frac{1}{r^2-x^2}}\d{x}} \\
	&= 2r\arcsin{\left(\frac{x}{r}\right)}\biggr\rvert_{-r}^{r} \\
	&= 2\pi r.
\end{align*}

\noindent
Note that when deriving the arc length formula, we could have just as easily have divided by $(\d{y})^2$.
This would give us an equivalent arc length formula that's applicable when $x$ is a function of $y$.

\begin{example}
	Find the length of the curve $y=x^{1/3}$ from $(-8,2)$ to $(8,2)$.
\end{example}
Rather than  tediously integrate the square root of a cube root if we set our bounds in terms of $x$, we can rewrite our equation and set out bounds in terms of $y$.
\begin{align*}
	x &= y^3 \\
	\dd{x}{y} &= 3y^2 \\
	\left(\dd{x}{y}\right)^2 &= 9y^4 \\
	s &= \int_{-2}^{2}{\sqrt{1+9y^4}\d{y}} \\
	&\footnotemark\approx 17.261.
\end{align*}
\footnotetext{You shouldn't be expected to evaluate this integral analytically. Using a calculator is fine.}
\section{Areas in the Plane}
\begin{lemma}
	If $f(x) \geq g(x)$ on $[a,b]$ then the area between $f(x)$ and $g(x)$ on $[a,b]$ is given by
	\begin{equation*}
		A = \int_{a}^{b}{\left(f(x)-g(x)\right)\d{x}}.
	\end{equation*}
\end{lemma}
\noindent
You can easily visualize why this is true.
We can break the integral into two parts: adding the area under $f$ and the other subtracting the area under $g$.
The first integral will over count the area between $f$ and $g$, counting all area between $f$ and the $x$-axis.
The second integral will subtract exactly the amount of area that is over counted: the area that is also between $g$ and the $x$-axis.

\begin{figure}[H]
	\label{area_between_curves}
	\centering
	\includegraphics[width = 0.33\textwidth]{./applications_integrals/curve1.png}
	\includegraphics[width = 0.33\textwidth]{./applications_integrals/curve2.png}
	\caption{\hyperref{}{}{}{Subtracting two areas to get the area between them}}
\end{figure}
\noindent
Even if both of the curves has more area below the $x$-axis, the same idea applies.
$f$ will give a smaller-magnitude negative area, while $g$ will give a larger magnitude negative area.
Subtracting a larger-magnitude negative number from a smaller magnitude negative number will give a positive area.

\begin{example}
	Find the area enclosed between the two curves $y=2-x^2$ and $y=-x$.
\end{example}
These two curves intersect at $x=-1$ and $x=2$.
Applying the formula,
\begin{align*}
	A &= \int_{-1}^{2}{((2-x^2)-(-x))\d{x}} \\
	&= 2x-\frac{x^3}{3} + \frac{x^2}{2}\biggr\rvert_{-1}^{2} \\
	&= \frac{9}{2}.
\end{align*}

\subsection{Subregions}
Sometimes it might be useful to break the enclosed regions into subregions and find the area of each separately.
\begin{example}
	Find the area above the $x$-axis, below $y=\sqrt{x}$, and above $y=x-2$.
\end{example}
If we simply took the area between the two curves, we'd count some area we don't want beneath the $x$-axis\footnote{Although this extra region is just a triangle and not too hard to find the area of, you'd effectively be applying the same strategy of two subregions, they'd just overlap. However, using geometry to your advantage is certainly a valid approach.}.
Instead, we can break the area into two subregions: one between the $x$-axis and $\sqrt{x}$ and the other between $x-2$ and $\sqrt{x}$.
\begin{figure}[H]
	\label{subregions}
	\centering
	\includegraphics[width = 0.33\textwidth]{./applications_integrals/two_regions.png}
	\caption{\hyperref{}{}{}{Break complicated areas into simpler subregions}}
\end{figure}
\indent
Finding the area of the blue region,
\begin{equation*}
	A_1 = \int_{0}^{2}{\sqrt{x}\d{x}} = \frac{2x\sqrt{x}}{3}\biggr\rvert_0^2 = \frac{4\sqrt{2}}{3}.
\end{equation*}
\indent
Finding the area of the green region,
\begin{equation*}
	A_2 = \int_{2}^{4}{(\sqrt{x}-(x-2))\d{x}} = \frac{2x\sqrt{x}}{3} - \frac{x^2}{2} + 2x \biggr\rvert_2^4 = \frac{10}{3} - \frac{4\sqrt{2}}{3}.
\end{equation*}
\indent
Adding the areas of the two regions,
\begin{equation*}
	A = A_1 + A_2 = \frac{10}{3}.
\end{equation*}

\subsection{Integrating with Respect to $y$}
Another strategy when finding the area of more complex regions is to see if they become easier to deal with if we were to instead integrate with respect to $y$.
In the above example, it is indeed easier to work with respect to $y$ because both bounding curves are between the same $y$ values.
\begin{example}
	Find the area above the $x$-axis, below $y=\sqrt{x}$, and above $y=x-2$.
\end{example}
We need to rearrange our equations to be of the form $x=\ldots$ instead of $y=\ldots$ by simply solving for $x$.
\begin{equation*}
	x = y^2, y\geq 0 \text{ and } x = y+2.
\end{equation*}
\indent
Since $x=y+2$ is further from the $x$-axis, it becomes our top curve.
\begin{align*}
	A &= \int_{0}^{2}{((y+2)-y^2)\d{y}} \\
	&= \frac{y^2}{2} + 2y - \frac{y^3}{3} \biggr\rvert_0^2 \\
	&= \frac{10}{3}.
\end{align*}
\section{Volumes}
\subsection{Volumes from Cross Sections}
Imagine we want to find the volume of some complex object.
One way we could approximate it is by slicing it into narrow cross sections.
The volume of each cross section would roughly be the area of the cross sectional face, times the width of the cross section.

\begin{figure}[H]
	\label{volumes}
	\centering
	\includegraphics[width=0.5\textwidth]{./applications_integrals/volumes.png}
	\caption{\hyperref{https://tutorial.math.lamar.edu/classes/calci/Area\_Volume\_Formulas.aspx}{}{}{Paul's Online Notes - Area and Volume Formulas}}
\end{figure}

Adding the volumes of these cross sections up, we'd get our approximation, which would get better and better the narrower the width of each cross section.
In the limit, this is exactly the definition of an integral.

\begin{definition}
	The volume of a solid with integrable corss sectional area $A(x)$ from $x=a$ to $x=b$ is given by
	\begin{equation*}
		V = \int_{a}^{b}{A(x)\d{x}}.
	\end{equation*}
\end{definition}


Let's start by finding the volume of a solid we already know.
\begin{example}
	Find the volume of a cube with sidelength $a$.
\end{example}
\begin{answer}
	For any slice of a cube, the cross section is a square with sidelength $a$.
	\begin{align*}
		A(x) &= a^2 \\
		V &= \int_{0}^{a}{a^2\d{x}} \\
		&= a^2x\biggr\rvert_0^a \\
		&= a^3.
	\end{align*}
\end{answer}


Now let's find the volume of a slightly more complicated shape.
\begin{example}
	Find the volume of a square pyramid with base sidelength $a$ and height $h$.
\end{example}
\begin{answer}
	Like the cube, any slice of the square pyramid is a square.
	However, the sidelength of the square depends on the height of your slice.
	A slice at the very tip of the pyramid would have sidelength 0, while a slice at the very bottom of the pyramid would have sidelength $a$.
	The sidelength grows linearly from $x=0$ to $x=h$, so it must be $\frac{a}{h}x$.
	\begin{align*}
		A(x) &= \left(\frac{a}{h}x\right)^2 \\
		V &= \int_{0}^{h}{\frac{a^2}{h^2}x^2\d{x}} \\
		&= \frac{a^2}{3h^2}x^3 \biggr\rvert_0^h \\
		&= \frac{a^2h}{3} \\
		&= \frac{1}{3}a^2 h.
	\end{align*}
\end{answer}

\subsection{Solids of Revolution}
You can think of solids of revolution as a special case of volumes from cross sections.
We'll tend to be on the lookout for a function that defines the radius, and we can then use the formula for the area of a circle to get the cross sectional area.
\begin{equation*}
	A(x) = \pi r^2(x).
\end{equation*}

\begin{example}
	Find the volume of the cone formed by rotating the line $y=x/3$ about the $x$-axis for $0 \leq x \leq 6$.
\end{example}
\begin{answer}
	The radius is simply the distance from the $x$-axis, which is just another name for $y$.
	So,
	\begin{equation*}
		A(x) = \pi \left(\frac{x}{3}\right)^2.
	\end{equation*}
	
	Integrating,
	\begin{align*}
		V &= \int_{0}^{6}{\pi\left(\frac{x}{3}\right)^2\d{x}} \\
		&= \frac{\pi}{9}\int_{0}^{6}{x^2\d{x}} \\
		&= \frac{\pi}{9}\left(\frac{x^3}{3}\biggr\rvert_0^6\right) \\
		&= 8\pi.
	\end{align*}
	
	This is the volume of a cone with radius 2 and height 6: $\frac{1}{3}\pi(2)^2(6)=8\pi$.
\end{answer}


Let's try a more complicated solid.
\begin{example}
	Find the volume of the solid of revolution bounded by $y=2+x\cos{x}$ from $-\frac{\pi}{2} \leq x \leq \frac{\pi}{2}$.
\end{example}
\begin{answer}
	Again, the radius is simply the distance from the $x$-axis, which is $y$.
	\begin{align*}
		A(x) &= \pi \left(2+x\cos{x}\right)^2 \\
		&= \pi \left(4 + 4x\cos{x} + x^2\cos^2{x}\right) \\
		V &= \int_{-\pi/2}^{\pi/2}{\pi \left(4 + 4x\cos{x} + x^2\cos^2{x}\right)\d{x}} \\
		&= \pi\left(\int_{-\pi/2}^{\pi/2}{4\d{x}}+\int_{-\pi/2}^{\pi/2}{4x\cos{x}\d{x}}+\int_{-\pi/2}^{\pi/2}{x^2\cos^2{x}\d{x}}\right) \\
		&= \pi\left(4\pi + 0 + \frac{1}{24}\pi\left(\pi^2-6\right)\right) \text{ (use integration by parts)}\\
		&= \frac{\pi^4}{24} + \frac{15\pi^2}{4}.
	\end{align*}
\end{answer}

\subsubsection{Washer Method}
Imagine now we want to find the volume of a solid of revolution with a ``hole.''
If we were to take a cross section of such a solid, it'd look like a washer with an outer radius $R(x)$ and inner radius $r(x)$.

\begin{figure}[H]
	\label{washers}
	\centering
	\includegraphics[width=0.33\textwidth]{./applications_integrals/Washer-1.png}
	\caption{\hyperref{https://www.shelovesmath.com/calculus/integral-calculus/applications-integration-area-volume/}{}{}{She Loves Math - Applications of Integration: Area and Volume}}
\end{figure}


We can think of this in a similar way as the area between curves in 2D.
We'll find the volume swept by the outer radius and the subtract the volume swept and removed by the inner radius.
\begin{equation*}
	V = \pi\int_{a}^{b}{(R^2(x)-r^2(x))\d{x}}.
\end{equation*}

\begin{example}
	Find the volume of shape formed by revolving the area enclosed by the $y$-axis, $y=\cos{x}$, and $y=\sin{x}$ around the $x$-axis.
\end{example}
\begin{answer}
	In the first quadrant, $\cos{x} \geq \sin{x}$ for $x \leq \frac{\pi}{4}$, so our bounds are $0 \leq x \leq \frac{\pi}{4}$, $R(x)=\cos{x}$, and $r(x)=\sin{x}$.
	\begin{align*}
		V &= \pi\int_{0}^{\pi/4}{(\cos^2{x}-\sin^2{x})\d{x}} \\
		&= \pi\int_{0}^{\pi/4}{\cos{(2x)}\d{x}} \\
		&= \pi\left(\frac{\sin{2x}}{2}\right)\biggr\rvert_0^{\pi/4} \\
		&= \frac{\pi}{2}.
	\end{align*}
\end{answer}

\subsubsection{Cylindrical Shells Method}
All the previous methods for finding the volumes of solids of rotation have relied on summing the volumes of thin cross sectional slices that are perpendicular to the axis of rotation.
However, we can instead sum the volume of thin cylindrical shells that grow outwards from and parallel to the axis of revolution.

\begin{figure}[H]
	\label{shells}
	\centering
	\includegraphics[width=0.5\textwidth]{./applications_integrals/shells.png}
	\caption{\hyperref{https://en.wikipedia.org/wiki/Shell\_integration}{}{}{Wikipedia - Shell Integration}}
\end{figure}


Each cylindrical shell will have a radius $r(x)$, a height $h(x)$, and a thickness $\d{x}$, meaning a volume of $2\pi r(x)h(x)\d{x}$.
\begin{equation*}
	V = 2\pi\int_{a}^{b}{r(x)h(x)\d{x}}.
\end{equation*}

\begin{example}
	Find the volume of the area bounded by the $y$-axis, $y=4-x^2$, and $y=x$ revolved around the $y$-axis.
\end{example}
\begin{answer}
	Each shell's height is parallel to the axis of rotation.
	In this case that's the distance between the two curves.
	\begin{equation*}
		h(x) = 4-x^2-x \text{ and } r(x)=x.
	\end{equation*}
	
	The two curves intersect at $x=\frac{-1+\sqrt{17}}{2}$.
	Finding the volume,
	\begin{align*}
		V &= 2\pi\int_{0}^{\frac{-1+\sqrt{17}}{2}}{x(4-x^2-x)\d{x}} \\
		&= 2\pi\left(-\frac{x^4}{4}-\frac{x^3}{3}+2x^2\right)\biggr\rvert_{0}^{\frac{-1+\sqrt{17}}{2}} \\
		&= \frac{\pi}{12}\left(121-17\sqrt{17}\right).
	\end{align*}
\end{answer}

\begin{example}
	Find the volume of the area bounded by the $x$-axis and the curve $y=(x-1)^2(x-2)^2$ rotated about the $y$-axis.
\end{example}
\begin{answer}
	The height of the shell is parallel to the $y$-axis.
	In this case it's exactly equal to the value of the bounding curve.
	\begin{equation*}
		h(x)=(x-1)^2(x-2)^2 \text{ and } r(x)=x.
	\end{equation*}
	
	Finding the volume,
	\begin{align*}
		V &= 2\pi\int_{1}^{2}{x(x-1)^2(x-2)^2\d{x}} \\
		&= 2\pi\left(\frac{x^6}{6}-\frac{6x^5}{5}+\frac{13x^4}{4}-4x^3+2x^2\right)\biggr\rvert_{1}^{2} \\
		&= \frac{\pi}{10}.
	\end{align*}
\end{answer}

\subsubsection{Other Axes of Rotation}
Although rotating about the $x$ and $y$ axes are the most common, the methods we have here apply to rotating about any axis parallel to the $x$ or $y$ axis.
The includes any lines of the form $y=k$ or $x=k$ where $k$ is some constant. \\


One valid approach is simply to rewrite the equations for your bounding curves by shifting them such that your axis of rotation is the $x$ or $y$ axis.
However, it's often more convenient to not rewrite the equations and simply apply the methods.

\begin{example}
	Find the volume of the solid generated by rotating the region in the first quadrant bounded by $y=x^2$ and $y=2x$ about the line $x=-2$.
\end{example}
\begin{answer}
	We can use the washer method (although shells also works).
	Since we're rotating about a line parallel to the $y$-axis, we'll have to rewrite our equations in the form $x=\ldots$.
	\begin{equation*}
		x = \sqrt{y} \text{ and } x = \frac{y}{2}.
	\end{equation*}
	
	The inner radius is given by the distance from $x=-2$ to $x=y/2$, and the outer radius is given by the distance from $x=-2$ to $x=\sqrt{y}$.
	The curves intersect at $y=0$ and $y=4$.
	\begin{align*}
		r(x) = 2 + \frac{y}{2} &\text{ and } R(x) = 2 + \sqrt{y} \\
		V &= \pi\int_{0}^{4}{(2+\sqrt{y})^2-\left(2+\frac{y}{2}\right)^2\d{y}} \\
		&= \pi\int_{0}^{4}{\left(4\sqrt{y}-y-\frac{y^2}{4}\right)\d{y}} \\
		&= 8\pi.
	\end{align*}
\end{answer}

\begin{example}
	Find the volume of the solid generated by rotating the region bounded by $y=x^2$ and $y=x+2$ about the line $x=3$.
\end{example} 
\begin{answer}
	We can use the shells method.
	Our height is simply the difference in $y$ values between the two curves.
	The radius is the distance between the lines $x=3$ and $x+2$.
	\begin{equation*}
		h(x) = x+2-x^2 \text{ and } r(x) = 3-x.
	\end{equation*}
	
	The curves intersect at $x=-1$ and $x=2$.
	Finding the volume,
	\begin{align*}
		V &= 2\pi\int_{-1}^{2}{(3-x)(x+2-x^2)\d{x}} \\
		&= 2\pi\left(\frac{x^4}{4}-\frac{4x^3}{3}+\frac{x^2}{2}+6x\right)\biggr\rvert_{-1}^{2} \\
		&= \frac{45\pi}{2}.
	\end{align*}
\end{answer}

\subsubsection{Surface Area}
Can apply the idea behind shell integration to derive a formula for surface area of a solid of rotation.
A cylindrical shell would have height $\d{s}$, which is given to us by the arc length formula, and radius $x$ or $y$, if the axis of rotation is the $y$ or $x$ axis respectively.
\begin{align*}
	S &= \begin{cases}
		2\pi\int{y\d{s}} & \text{Rotation about $x$-axis} \\
		2\pi\int{x\d{x}} & \text{Rotation about the $y$-axis}
	\end{cases} \\
	& \text{where } \\
	\d{s} &= \begin{cases}
		\sqrt{1+\left(\dd{y}{x}\right)^2}\d{x} & y=f(x) \\
		\sqrt{1+\left(\dd{x}{y}\right)^2}\d{y} & x=g(y) \\
	\end{cases}.
\end{align*}

\begin{example}
	Find the surface area of a sphere with radius $r$.
\end{example}
\begin{answer}
	We can obtain a sphere by rotating $y=\sqrt{r^2-x^2}, -r\leq x\leq r$ about the $x$-axis.
	\begin{align*}
		\dd{y}{x} &= \frac{-x}{\sqrt{r^2-x^2}} \\
		\left(\dd{y}{x}\right)^2 &= \frac{x^2}{r^2-x^2} \\
		S &= 2\pi\int_{-r}^{r}{y\sqrt{1+\frac{x^2}{r^2-x^2}}\d{x}} \\
		&= 2\pi\int_{-r}^{r}{\sqrt{r^2-x^2}\frac{r}{\sqrt{r^2-x^2}}\d{x}} \\
		&= 2\pi\int_{-r}^{r}{r\d{x}} \\
		&= 4\pi r^2.
	\end{align*}
\end{answer}
\section{Differential Equations}
Differential equations are equations, or sometimes systems of equations where we are given how a function relates to its derivatives and would like to find satisfying functions or families of functions.
Differential equations land themselves well to modeling real-world phenomena and are still an area of active mathematical study.
We'll see some very basic differential equations and how what we've learned about integrals can be used to solve them.

\subsection{Separable Differential Equations}
If we can break up a first-order (just involving a first derivative) ordinary differential equation into the following form,
\begin{equation*}
	\dd{y}{x} = f(x)g(y)
\end{equation*}
then we say the differential equation is separable and may be able to be solved using a technique called separation of variables.
The steps to solve a separable differential equation are:
\begin{enumerate}
	\item Write the equation in differential form (i.e using $\dd{y}{x}$)
	\item Separate the variables ($\frac{\d{y}}{g(y)}=f(x)\d{x}$)
	\item Integrate both sides
	\item Solve for $y$ in terms of $x$, if possible
	\item Find the general solution
	\item Find a particular solution, given any initial conditions
\end{enumerate}

\begin{example}
	Solve the following differential equation.
	\begin{equation*}
		\dd{y}{x} = (xy)^2, y(1)=1.
	\end{equation*}
\end{example}
\begin{answer}
	Separating and integrating,
	\begin{align*}
		\dd{y}{x} &= (xy)^2 = x^2y^2 \\
		\frac{\d{y}}{y^2} &= x^2\d{x} \\
		\frac{1}{y} &= \frac{x^3}{3} + C. \\
	\end{align*}
	
	Solving for $C$,
	\begin{align*}
		\frac{-1}{1} &= \frac{1^3}{3} + C \\
		C &= \frac{-4}{3}.
	\end{align*}
	
	Solving for $y$,
	\begin{align*}
		\frac{-1}{y} &= \frac{x^3}{3} - \frac{4}{3} \\
		y &= \frac{-3}{x^3 - 4}.
	\end{align*}
\end{answer}

\subsubsection{Exponential Growth \& Decay}
We can use differential equations to model growth and decay.

\begin{example}
	Imagine we have some money in a bank account earning interest.
	The more money in the bank, the more the more the account will receive in interest.
	So, the rate of growth of the account value is proportional to the current account value.
	In the language of differential equations,
	\begin{equation*}
		\dd{y}{t} = ky
	\end{equation*}
	where $k$ is some constant of proportionality.
\end{example}
\begin{answer}
	This equation is separable, so we'll try to solve it using separation of variables.
	\begin{align*}
		\frac{\d{y}}{t} &= k\d{x} \\
		\ln{\abs{y}} &= kt + C \\
		\abs{y} &= Ce^{kt} \\
		y &= Ce^{kt} \\
		y(0) &= Ce^{k\cdot 0} = C \\
		y &= y_0e^{kt}.
	\end{align*}
\end{answer}

This is exactly the equation for continually compounding interest: $y_0$ is the principle and $k$ is an interest rate.


Note that $k$ could theoretically be negative, meaning the amount would decrease proportionally to the remaining amount.
\begin{example}
	The half-life of Pu-239 is 24360 years.
	Suppose that 10g of Pu-239 were released in a nuclear accident, how long would it take to decay to 1g?
\end{example}
\begin{answer}
	Our ``principle'' is 10g.
	We know that half-life follows the exponential decay differential equation, so it's modeled by $y = 10e^{kt}$.
	We also know that $y(24360)=5$, which should give us enough information to solve for $k$.
	\begin{align*}
		5 &= 10e^{k\cdot 24360} \\
		\frac{1}{2} &= e^{k\cdot 24360} \text{ (see the ``half`` in half-life?)} \\
''		\ln{\frac{1}{2}} &= 24360k \\
		-\ln{2} &= 24360k \\
		k &= \frac{-\ln{2}}{24360}.
	\end{align*}
	
	We now can plug $k$ back into our equation to get the full model.
	\begin{align*}
		y &= 10e^{\frac{-\ln{2}}{24360}t} \\
		&= 10\left(e^{\ln{2}}\right)^{\frac{-t}{24360}} \\
		&= 10\cdot2^{-\frac{-t}{24360}}.
	\end{align*}
	
	We can now plug in 1 for $y$ and solve for $t$
	\begin{align*}
		1 &= 10\cdot2^{-\frac{t}{24360}} \\
		\frac{1}{10} &= 2^{-\frac{t}{24360}} \\
		\log_{2}{\frac{1}{10}} &= -\frac{t}{24360} \\
		\log_{2}{10} &= \frac{t}{24360} \\
		t &= 24360\log_{2}{10} \approx 80922\text{yr}.
	\end{align*}
	
	If you're familiar with half-life equations, this is exactly $t=t_{1/2}\log_{2}{\frac{N_0}{N_f}}$.
\end{answer}

\subsubsection{Logistic Growth \& Decay}
Although our differential equations assuming that growth is proportional to amount work well for things like bank accounts, bacteria or radioactive particles that can grow and decay without limits, that model is a little too simplistic to model populations that are limited by resources.

\begin{example}
	Imagine that we have a population of animals in a forest.
	If the forest has lots of resources to support to animal population, then they can grow basically like normal.
	However, as the population grows, resources become more scarce, so population growth would slow down, or some of the population would starve.
	We can model this with the following differential equation.
	\begin{equation*}
		\dd{P}{t} = kP(M-P)
	\end{equation*}
	where $k$ is some constant of proportionality and $M$ is some maximum population before growth starts to decline.
\end{example}
\begin{answer}
	This differential equation is also separable, so let's try to solve it.
	\begin{align*}
		\dd{P}{t} &= kP(M-P) \\
		\frac{\d{P}}{P(M-P)} &= k\d{t} \\
		\d{P}\left(\frac{1/M}{P}+\frac{1/M}{M-P}\right) &= k\d{t} \text{ (using partial fractions)} \\
		\frac{1}{M}\left(\ln{\abs{P}} - \ln{\abs{M-P}}\right) &= kt + C \\
		\ln{\abs{\frac{P}{M-P}}} &= Mkt + C \\
		\frac{P}{M-P} &= Ce^{Mkt} \text{ (b/c $P \geq 0$)} \\
		P &= CMe^{Mkt} - CPe^{Mkt} \\
		P\left(1+Ce^{Mkt}\right) &= CMe^{Mkt} \\
		P &= \frac{CMe^{Mkt}}{1+Ce^{Mkt}} \\
		&= \frac{M}{1+Ce^{-Mkt}}.
	\end{align*}
\end{answer}

\begin{figure}[H]
	\label{logistic}
	\centering
	\includegraphics[width=0.5\textwidth]{./applications_integrals/logistic_growth.png}
	\caption{\hyperref{}{}{}{Logistic Growth}}
\end{figure}

Looking at a graph of this function, we can see that it starts growing like an exponential curve but begins to flatten, obtaining a maximum value of $M$, which is called the carrying capacity.
The population is growing the fastest when $P=M/2$, which you can verify by finding the global maxima of $\dd{P}{t}$ using a first derivative test.

\subsection{Slope Fields \& Euler's Method}
Unfortunately, not all differential equations are as easy to solve as separable differential equations.
In fact, some are impossible to get nice, closed-form solutions.
\subsubsection{Slope Fields}
We may still be able to visualize what the graph of a solution might look like by drawing lines that have the same slope as a solution.
Any solution that passes through the points where we draw the sloped lines must be tangent to these lines, meaning a solution will follow the ``flow`` of these lines.
''
\begin{figure}[H]
	\label{slope_field}
	\centering
	\includegraphics[width=0.75\textwidth]{./applications_integrals/slope_field.png}
	\caption{\hyperref{}{}{}{Slope field of $\dd{y}{x}=-x/y$ with a possible solution}}
\end{figure}

\subsubsection{Euler's Method}
If your differential equation also has an initial condition, you can start at the initial point and follow the slope field to find an approximate solution.
This is what Euler's Method tries to accomplish.
It can approximate the value of a solution at some $x$ value by starting at some point, usually given by the initial condition, and iteratively taking small steps of size $\Delta x$ in the direction determined by the slope field.
\begin{align*}
	x_{n+1} &= x_n + \Delta x \\
	y_{n+1} &= y_n + \Delta x\dd{y}{x}_{(x_n,y_n)}.
\end{align*}

The smaller the steps, the more accurate the approximation.
\begin{figure}[H]
	\label{eulers_method}
	\centering
	\includegraphics[width=0.55\textwidth]{./applications_integrals/Eulers-Approximation.png}
	\includegraphics[width=0.35\textwidth]{./applications_integrals/Eulers-Approximation2.png}
	\caption{\hyperref{https://calcworkshop.com/first-order-differential-equations/eulers-method-table/}{}{}{Calc Workshop - Euler's Method};\hspace{5pt}\hyperref{https://en.wikipedia.org/wiki/Euler\_method}{}{}{Wikipedia - Euler Method}}
\end{figure}

\begin{example}
	Given that $\dd{y}{x}=3-x$ and $y(4)=2$, approximate the value of $y(5)$ using Euler's method with increments of $\Delta x = 0.25$.
\end{example}
\begin{answer}
	\begin{table}[H]
		\begin{center}
			\begin{tabular}{|c|c|c|c|c|}
				\hline
				$(x,y)$ & $\dd{y}{x}$ & $\Delta x$ & $\Delta y = \Delta x\dd{y}{x}$ & $(x+\Delta x, y+\Delta y)$ \\
				\hline
				$(4,2)$ & $-1$ & $0.25$ & $-0.25$ & $(4.25,1.75)$ \\
				\hline
				$(4.25,1.75)$ & $-1.25$ & $0.25$ & $-0.3125$ & $(4.5,1.4375)$ \\
				\hline
				$(4.5,1.4375)$ & $-1.5$ & $0.25$ & $-0.375$ & $(4.75,1.0625)$ \\
				\hline
				$(4.75,1.0625)$ & $-1.75$ & $0.25$ & $-0.4375$ & $(5,0.625)$ \\
				\hline
			\end{tabular}
		\end{center}
	\end{table}
	
	So, Euler's Method yields an approximate\footnote{The actual value of the solution is $(5,0.5)$, so not too far off.} value of $(5,0.625)$.
\end{answer}
		\chapter{Parametric, Vector, \& Polar Functions}

\section{Parametric \& Vector Functions}
Up to this point, almost all the graphs we have worked with have been of the form $y=f(x)$, defining the $y$ coordinate in terms of the $x$ coordinate.
These sorts of functions are limited in the types of graphs they can draw.
If we instead let both the $x$ and $y$ coordinates be defined in terms of another variable $t$, like $(x(t),y(t))$, then we can draw much more interesting graphs.
For example, a unit circle, which can't be defined with a single function $y=f(x)$, would be $(\cos{t}, \sin{t})$. \\


We are always able to translate a function of the form $y=f(x)$ into a parametric function as $(t, f(t))$.
Sometimes, but not always, we are also able to translate parametric functions into $y$ as a function of $x$.

\begin{example}
	Given the following parametric function, find $y$ as a function of $x$.
	\begin{equation*}
		(\sqrt{t}, t-2).
	\end{equation*}
\end{example}
\begin{answer}
	Squaring both sides of the $x$ equation,
	\begin{equation*}
		x^2 = t.
	\end{equation*}
	
	Substituting our expressing for $t$ in terms of $x$ into the $y$ equation,
	\begin{equation*}
		y = x^2 - 2.
	\end{equation*}
\end{answer}

\subsection{Vector Functions}
Vector and parametric functions are essentially the same thing.
In fact, in multivariable calculus, we drop the idea of parametric functions almost completely and exclusively talk about vector-valued functions.
Both can graph the exact same functions.
Visually, you might imagine an arrow rooted at the origin tracing out the graph of a vector function. 
You're more likely to see vector functions written in the following form
\begin{equation*}
	\vec{r}(t) = \langle x(t), y(t) \rangle.
\end{equation*}

All the normal vector operations, like addition and subtraction, scalar multiplication, and dot products work exactly the same.
If we think of $\vec{r}(t)$ as a position function,
\begin{align*}
	\textbf{Velocity: }& \vec{v}(t) = \vec{r^\prime}(t) = \langle x^\prime(t), y^\prime(t) \rangle \\
	\textbf{Speed: }& \abs{\vec{v}(t)} = \sqrt{\left(x^\prime(t)\right)^2 + \left(y^\prime(t)\right)^2} \\
	\textbf{Acceleration: }& \vec{a}(t) = \vec{v^\prime}(t) = \langle x^{\prime\prime}(t), y^{\prime\prime}(t) \rangle \\
	\textbf{Direction: }& \frac{\vec{t}(t)}{\abs{\vec{v}(t)}} \\
\end{align*}

\subsection{Slope \& Concavity}
Just like with functions like $y=f(x)$, we can find the slope and concavity of parametric functions using first and second derivatives respectively.
We just apply the chain rule.
\begin{align*}
	\dd{y}{x} &= \frac{\dd{y}{t}}{\dd{x}{t}} \\
	\dd{^2y}{x^2} &= \dd{y^\prime}{x} = \frac{\d{y^\prime}/\d{t}}{\d{x}/\d{t}}.
\end{align*}

\begin{example}
	Consider the following parametric function:
	\begin{equation*}
		(t^2-5, 2\sin{t}), 0\leq t\leq\pi.
	\end{equation*}
	Find the first and second derivatives of $y$ with respect to $x$.
\end{example}
\begin{answer}
	Differentiating both $x$ and $y$ with respect to $t$,
	\begin{align*}
		x^\prime(t) &= 2t \\
		y^\prime(t) &= 2\cos{t} \\
		\dd{y}{x} &= \frac{2\cos{t}}{2t} = \frac{\cos{t}}{t}.
	\end{align*}
	Finding the derivative of $y^\prime$ with respect to $t$,
	\begin{align*}
		\dd{}{t}y^\prime &= \dd{}{t}\frac{\cos{t}}{t} \\
		&= \frac{-t\sin{t}-\cos{t}}{t^2} \\
		\dd{^2y}{x^2} &= \frac{\d{y^\prime}/\d{t}}{\d{x}/\d{t}} \\
		&= \frac{\frac{-t\sin{t}-\cos{t}}{t^2}}{2t} \\
		&= -\frac{t\sin{t}+\cos{t}}{2t^3}.
	\end{align*}
\end{answer}

\subsection{Arc Length}
Remember that we had the following formula for $\d{s}$ when deriving arc length.
\begin{equation*}
	\d{s} = \sqrt{\left(\d{x}\right)^2 + \left(\d{y}\right)^2}.
\end{equation*}
Since we now have $x$ and $y$ as functions of $t$, we can rewrite this formula to get a formula for arc length of a parametric function.
\begin{align*}
	\d{s} &= \sqrt{\left(\dd{x}{t}\right)^2 + \left(\dd{y}{t}\right)^2}\d{t} \\
	s &= \int_{a}^{b}{\sqrt{\left(\dd{x}{t}\right)^2 + \left(\dd{y}{t}\right)^2}\d{t}}.
\end{align*}

When talking about vector-valued functions or working in a more physics-based context, you might hear the term ``distance traveled" instead of arc length and see the following formula.
They are equivalent ideas.
\begin{equation*}
	s = \int_{a}^{b}{\abs{\vec{v}(t)}\d{t}}.
\end{equation*}

\begin{example}
	A circle of radius $r$ is defined parametrically as
	\begin{equation*}
		(r\cos{t}, r\sin{t}), 0 \leq t \leq 2\pi.
	\end{equation*}
	Use this definition to find its circumference.
\end{example}
\begin{answer}
	\begin{align*}
		\dd{x}{t} &= -r\sin{t} \\
		\left(\dd{x}{t}\right)^2 &= r^2\sin^2{t} \\
		\dd{y}{t} &= r\cos{t} \\
		\left(\dd{y}{t}\right)^2 &= r^2\cos^2{t} \\
		C &= \int_{0}^{2\pi}{\sqrt{r^2\sin^2{t}+r^2\cos^2{t}}\d{t}} \\
		&= \int_{0}^{2\pi}{r\sqrt{\sin^2{t}+\cos^2{t}}\d{t}} \\
		&= \int{0}^{2\pi}{r\d{t}} \\
		&= 2\pi r.
	\end{align*}
\end{answer}
\section{Polar Functions}
\subsection{Polar Coordinates}
Up to this point, we've mostly described points in the plane by listing two numbers: the distance along the $x$-axis and the distance along the $y$-axis.
This description, called ``rectangular coordinates'' is pretty simple and has the advantage that every pair of coordinates describes a unique point on the plane. \\

However, we could instead describe points in the plane with two numbers $(r,\theta)$, where $r$ is the point's distance from the origin and $\theta$ is the point's angle of inclination.
This system is more suited to describing points related to trig functions.
For example the point at $x=\cos{\frac{\pi}{4}}$ and $y=\sin{\frac{\pi}{4}}$ is $\left(\frac{\sqrt{2}}{2},\frac{\sqrt{2}}{2}\right)$ in rectangular coordinates but $\left(1,\frac{\pi}{4}\right)$ in polar coordinates. \\


Note that unlike rectangular coordinates, multiple pairs of numbers can describe the same point.
For example $\left(1,0\right)$ is the same point as $\left(1,2\pi\right)$ is the same point as $\left(1,-2\pi\right)$ is the same point as $\left(1,4\pi\right)$. \\


We can easily convert between polar and rectangular coordinates.
\begin{align*}
	\left(r,\theta\right) \text{polar} &= \left(r\cos{\theta}, r\sin{\theta}\right) \text{rectangular} \\
	\left(x,y\right) \text{rectangular} &= \left(\sqrt{x^2+y^2}, \arctan{\frac{y}{x}}\right) \text{polar}.
\end{align*}

\subsection{Polar Functions}
Polar functions are written in the form $r = f(\theta)$.
Using our polar coordinate conversion formulas, we can convert any polar function to a parametric function.
\begin{align*}
	x(\theta) &= r\cos{\theta} = f(\theta)\cos{\theta} \\
	y(\theta) &= r\sin{\theta} = f(\theta)\sin{\theta}.
\end{align*}

Now we can use our parametric function formulas to get $\dd{y}{x}$.
\begin{equation*}
	\dd{y}{x} = \frac{\d{y}/\d{\theta}}{\d{x}/\d{\theta}}.
\end{equation*}

\begin{example}
	A cardioid is defined by $r=1-\cos{\theta}, 0 \leq \theta \leq 2\pi$.
	Find $\dd{y}{x}$.
\end{example}
\begin{answer}
	\begin{align*}
		x(\theta) &= \left(1-\cos{\theta}\right)\cos{\theta} \\
		\dd{x}{\theta} &= \sin{(2\theta)} - \sin{\theta} \\
		y(\theta) &= \left(1-\cos{\theta}\right)\sin{\theta} \\
		\dd{y}{\theta} &= \cos{\theta} - \cos{(2\theta)} \\
		\dd{y}{x} &= \frac{\cos{\theta} - \cos{(2\theta)}}{\sin{(2\theta)} - \sin{\theta}} \\
		&= \tan{\frac{3\theta}{2}}.
	\end{align*}
\end{answer}

\subsubsection{Area Enclosed}
When a polar function is changed by $\d{\theta}$, it sweeps out an circular sector.

\begin{figure}[H]
	\label{polar_area}
	\centering
	\includegraphics[width=0.66\textwidth]{./parametric_vector_polar/polar_area.png}
	\caption{\hyperref{}{}{}{Polar Area}}
\end{figure}

This circular sector has area $\d{A} = \frac{\d{\theta}}{2}r^2$.
Integrating $\d{A}$ for $\alpha \leq \theta \leq \beta$,
\begin{equation*}
	A = \int_{\alpha}^{\beta}{\frac{1}{2}r^2\d{\theta}} = \int_{\alpha}^{\beta}{\frac{1}{2}f^2(\theta)\d{\theta}}.
\end{equation*}

\begin{example}
	Find the area inside the smaller loop of the lima\c{c}on $r=2\cos{\theta}+1$.
\end{example}
\begin{answer}
	First, we need to find our bounds on $\theta$.
	We know that the small loop begins and ends when $r=0$.
	\begin{align*}
		0 &= 2\cos{\theta}+1 \\
		-\frac{1}{2} &= \cos{\theta} \\
		\theta &= \frac{\pi}{3}, \frac{4\pi}{3}.
	\end{align*}
	
	Now that we have our bounds, we can integrate.
	\begin{align*}
		A &= \int_{2\pi/3}^{4\pi/3}{\frac{1}{2}\left(2\cos{\theta}+1\right)^2\d{\theta}} \\
		&= \frac{1}{2}\left(\int_{\frac{2\pi}{3}}^{\frac{4\pi}{3}}4\cos^{2}\left(\theta\right)\d{\theta}+4\int_{\frac{2\pi}{3}}^{\frac{4\pi}{3}}\cos\left(\theta\right)\d{\theta}+\int_{\frac{2\pi}{3}}^{\frac{4\pi}{3}}\d{\theta}\right) \\
		&= \int_{\frac{2\pi}{3}}^{\frac{4\pi}{3}}\left(1+\cos\left(2\theta\right)\right)\d{\theta}+2\int_{\frac{2\pi}{3}}^{\frac{4\pi}{3}}\cos\left(\theta\right)\d{\theta}+\frac{1}{2}\int_{\frac{2\pi}{3}}^{\frac{4\pi}{3}}\d{\theta} \\
		&= \int_{\frac{2\pi}{3}}^{\frac{4\pi}{3}}\cos\left(2\theta\right)\d{\theta}+2\int_{\frac{2\pi}{3}}^{\frac{4\pi}{3}}\cos\left(\theta\right)\d{\theta}+\frac{3}{2}\int_{\frac{2\pi}{3}}^{\frac{4\pi}{3}}\d{\theta} \\
		&= \frac{1}{2}\sin\left(2\theta\right)+2\sin\left(\theta\right)+\frac{3}{2}\theta \biggr\rvert_{2\pi/3}^{4\pi/3} \\
		&= \left(\frac{1}{2}\sin\left(\frac{8\pi}{3}\right)+2\sin\left(\frac{4\pi}{3}\right)+\frac{3}{2}\frac{4\pi}{3}\right)-\left(\frac{1}{2}\sin\left(\frac{4\pi}{3}\right)+2\sin\left(\frac{2\pi}{3}\right)+\frac{3}{2}\frac{2\pi}{3}\right) \\
		&= \left(\frac{\sqrt{3}}{4}-\sqrt{3}+2\pi\right)-\left(-\frac{\sqrt{3}}{4}+\sqrt{3}+\pi\right) \\
		&= \frac{\sqrt{3}}{2}-2\sqrt{3}+\pi \\
		&= \pi-\frac{3\sqrt{3}}{2}.
	\end{align*}
\end{answer}

\subsubsection{Area Between Curves}
The area between $r_1(\theta)$ and $r_2(\theta)$ is simply the difference between the areas.
\begin{equation*}
	A = \frac{1}{2}\int_{\alpha}^{\beta}{r_1^2(\theta)\d{\theta}} - \frac{1}{2}\int_{\alpha}^{\beta}{r_2^2(\theta)\d{\theta}} = \frac{1}{2}\int_{\alpha}^{\beta}{(r_1^2(\theta)-r_2^2(\theta))\d{\theta}}.
\end{equation*}

\begin{example}
	Find the area that lies inside the circle $r=1$ and outside the cardioid $r=1-\cos{\theta}$.
\end{example}
\begin{answer}
	To find the bounds, we need to find where these curves intersect.
	\begin{align*}
		1 &= 1-\cos{\theta} \\
		\cos{\theta} &= 0 \\
		\theta &= \frac{\pi}{2}, \frac{-\pi}{2}.
	\end{align*}
	
	Since we want the area inside of the circle and outside of the cardioid, our bounds are $\frac{-\pi}{2} \leq \theta \leq \frac{\pi}{2}$.
	We'll also have $r_1(\theta)$ be the circle and $r_2(\theta)$ be the cardioid, since we are effectively finding the area inside the circle and subtracting away the area that is also in the cardioid.
	\begin{align*}
		A &= \frac{1}{2}\int_{-\pi/2}^{\pi/2}{\left(1^2 - \left(1-\cos{\theta}\right)^2\right)\d{\theta}} \\
		&= \frac{1}{2}\int_{\frac{-\pi}{2}}^{\frac{\pi}{2}}\left(2\cos\left(\theta\right)-\cos^{2}\theta\right)\d{\theta} \\
		&= \frac{1}{2}\int_{\frac{-\pi}{2}}^{\frac{\pi}{2}}\left(2\cos\left(\theta\right)-\frac{1+\cos\left(2\theta\right)}{2}\right)\d{\theta} \\
		&= \frac{1}{2}\left(2\sin\left(\theta\right)-\frac{\theta}{2}-\frac{\sin\left(2\theta\right)}{4}\right) \biggr\rvert_{\frac{-\pi}{2}}^{\frac{\pi}{2}} \\
		&= \frac{1}{2}\left(\left(2\sin\left(\frac{\pi}{2}\right)-\frac{\pi}{4}-\frac{\sin\left(\pi\right)}{4}\right)-\left(2\sin\left(\frac{-\pi}{2}\right)+\frac{\pi}{4}-\frac{\sin\left(-\pi\right)}{4}\right)\right) \\
		&= \left(2\sin\left(\frac{\pi}{2}\right)-\frac{\pi}{4}-\frac{\sin\left(\pi\right)}{4}\right) \\
		&= 2-\frac{\pi}{4}.
	\end{align*}
\end{answer}

\begin{example}
	Find the area that lies outside the circle $r=1$ and inside the cardioid $r=1-\cos{\theta}$.
\end{example}
\begin{answer}
	We need to make sure our bounds are sweeping out the correct area.
	If we did $\frac{-\pi}{2} \leq \theta \leq \frac{\pi}{2}$, we'd get the area inside the circle and outside the cardioid, which isn't what we want here.
	We know that for polar coordinates, $\frac{-\pi}{2} \equiv \frac{3\pi}{2}$.
	So, out bounds are $\frac{\pi}{2} \leq \theta \leq \frac{3\pi}{2}$.
	Since we want the area inside the cardioid and outside the circle, effectively taking the cardioid and subtracting away the intersection, so $r_1(\theta)$ is the cardioid, and $r_2(\theta)$ is the circle.
	\begin{align*}
		A &= \frac{1}{2}\int_{\pi/2}^{3\pi/2}{\left(\left(1-\cos{\theta}\right)^2-1^2\right)\d{\theta}} \\
		&= \frac{1}{2}\int_{\frac{\pi}{2}}^{\frac{3\pi}{2}}\left(\cos^{2}\left(\theta\right)-2\cos\left(\theta\right)\right)\d{\theta} \\
		&= \frac{1}{2}\int_{\frac{\pi}{2}}^{\frac{3\pi}{2}}\left(\frac{1+\cos\left(2\theta\right)}{2}-2\cos\left(\theta\right)\right)\d{\theta} \\
		&= \frac{1}{2}\left(\frac{\theta}{2}+\frac{\sin\left(2\theta\right)}{4}-2\sin\left(\theta\right)\right)\biggr\rvert_{\pi/2}^{3\pi/2} \\
		&= \frac{1}{2}\left(\left(\frac{3\pi}{4}+\frac{\sin\left(3\pi\right)}{4}-2\sin\left(\frac{3\pi}{2}\right)\right)-\left(\frac{\pi}{4}+\frac{\sin\left(\pi\right)}{4}-2\sin\left(\frac{\pi}{2}\right)\right)\right) \\
		&= \frac{\pi}{4}+2.
	\end{align*}
\end{answer}

\begin{example}
	Find the area inside both the circle $r=1$ and the cardioid $r=1-\cos{\theta}$.
\end{example}
\begin{answer}
	This area isn't between two polar curves like the previous examples in the sense that we can't define it as one region minus another.
	We'll prove it it two ways: logically and by using two regions.
	Logically, we know that the total area of the circle is $\pi$.
	We also know the area inside the circle but outside the cardioid is $2-\frac{\pi}{4}$.
	So,
	\begin{equation*}
		A_{\text{both}} = \pi - \left(2-\frac{\pi}{4}\right) = \frac{5\pi}{4} - 2.
	\end{equation*}
	
	We can break the region inside both curves into two parts: a half circle for $x\leq 0$ and the two cardioid bulges for $x\geq 0$.
	The area of the half-circle is $\pi/2$.
	We can find the area of the two cardioid bulges.
	\begin{align*}
		A_{\text{bulges}} &= 2A_{\text{bulge}} \\
		&= \int_{0}^{\pi/2}{\left(1-\cos{\theta}\right)^2\d{\theta}} \\
		&= \int_{0}^{\frac{\pi}{2}}\left(\cos^{2}\theta-2\cos\left(\theta\right)+1\right)\d{\theta} \\
		&= \int_{0}^{\frac{\pi}{2}}\left(\frac{1+\cos\left(2\theta\right)}{2}-2\cos\left(\theta\right)+1\right)\d{\theta} \\
		&= \frac{3\theta}{2}+\frac{\sin\left(2\theta\right)}{4}-2\sin\left(\theta\right)\biggr\rvert_{0}^{\frac{\pi}{2}} \\
		&= \frac{3\pi}{4}+\frac{\sin\left(\pi\right)}{4}-2\sin\left(\frac{\pi}{2}\right) \\
		&= \frac{3\pi}{4}-2
	\end{align*}
	
	Adding in the area of the half-circle,
	\begin{align*}
		A_{\text{both}} &= A_{\text{half}} + A_{\text{bulges}} \\
		&= \frac{\pi}{2} + \frac{3\pi}{4} - 2 \\
		&= \frac{5\pi}{4} - 2.
	\end{align*}
	
	We see that we get the same answer either way.
\end{answer}

\subsubsection{Arc Length}
Since we know how to convert polar functions to parametric, we can simply adapt the parametric arc length formula.
\begin{equation*}
	s = \int_{\alpha}^{\beta}{\sqrt{\left(\dd{x}{\theta}\right)^2 + \left(\dd{y}{\theta}\right)^2}\d{\theta}}.
\end{equation*}


However, there is an alternate form that works just for polar functions.
For some small change $\d{\theta}$, we see a corresponding small changes $\d{r}$ and $r\d{\theta}$.
\begin{figure}[H]
	\label{polar_arclength}
	\centering
	\includegraphics[width=0.66\textwidth]{./parametric_vector_polar/polar_length.png}
	\caption{\hyperref{}{}{}{Polar Arc Length}}
\end{figure}
We see that these changes for a right triangle with hypotenuse $\d{s}$.
\begin{align*}
	(\d{s})^2 &= (r\d{\theta})^2 + (\d{r})^2 \\
	&= \left(r^2 + \left(\dd{r}{\theta}\right)^2\right)\left(\d{\theta}\right)^2 \\
	\d{s} &= \sqrt{r^2 + \left(\dd{r}{\theta}\right)^2}\d{\theta} \\
	s &= \int_{\alpha}^{\beta}{\sqrt{r^2 + \left(\dd{r}{\theta}\right)^2}\d{\theta}}.
\end{align*}

Giving us an alternate formula for polar arc length.

\begin{example}
	Find the arc length of the cardioid $r=1-\cos{\theta}$.
\end{example}
\begin{answer}
	The bounds are $0 \leq \theta \leq 2\pi$.
	Using the polar arc length formula,
	\begin{align*}
		\dd{r}{\theta} &= \sin{\theta} \\
		\left(\dd{r}{\theta}\right)^2 &= \sin^2{\theta} \\
		s &= \int_{0}^{2\pi}{\sqrt{\left(1-\cos{\theta}\right)^2+\sin^2{\theta}}\d{\theta}} \\
		&= \int_{0}^{2\pi}{\sqrt{2+2\cos{\theta}}\d{\theta}} \\
		&= \int_{0}^{2\pi}{2\sin{\left(\frac{\theta}{2}\right)}\d{\theta}} \\
		&= -4\cos{\left(\frac{\theta}{2}\right)}\biggr\rvert_{0}^{2\pi} \\
		&= 8.
	\end{align*}
\end{answer}

		\chapter{Sequences, L'H\^{o}pital's Rule, \& Improper Integrals}

\section{Sequences}
\begin{definition}
	A sequence $\left\{a_n\right\} = \left\{a_1, a_2, \ldots, a_n\right\}$ is an ordered list of numbers.
	Each element of a sequences is called a term and is identified by its index in the sequence.
	Sequences can be finite or infinite.
\end{definition}

\begin{example}
	The nth term of a sequence is defined by the following formula:
	\begin{equation*}
		a_n = \frac{(-1)^n}{n^2+1}.
	\end{equation*}
	Find the 1st, 2nd, and 100th terms of the sequence.
\end{example}
\begin{align*}
	a_1 &= \frac{(-1)^1}{1^2 + 1} = \frac{-1}{2} \\
	a_2 &= \frac{(-1)^2}{2^2 + 1} = \frac{1}{5} \\
	a_{100} &= \frac{(-1)^{100}}{100^2 + 1} = \frac{1}{10001}.
\end{align*}

\noindent
The above sequence was defined explicitly, meaning that we have a formula for the nth term of the sequence only in terms of n.
However, sequences can also be defined recursively, meaning the formula for subsequent terms of the sequence contains previous terms.
For a recursive sequence to be properly defined, there need to be one or more base terms that aren't defined recursively.
For example, the Fibonacci sequence, one of the most famous recursive sequences, as $a_1$ and $a_2$ as base terms.
\begin{equation*}
	a_n = \begin{cases}
		1 & n = 1, 2 \\
		a_{n-1} + a_{n-2} & n \geq 3
	\end{cases}.
\end{equation*}

\subsection{Common Types of Sequences}
There are some common types of sequences that you should be familiar with.
You might recognize these types of sequences and some of the formulas surrounding them from previous math classes.

\subsubsection{Arithmetic Sequences}
\begin{definition}
	An arithmetic sequence is one where $a_{n+1} - a_{n} = d$, a common difference, for all terms.
\end{definition}
\noindent
That is, to get the next term, we simply add some number $d$ (which could be negative) to the previous term.
Arithmetic sequences can be defined either explicitly or recursively.
Let $a_0$ be the starting term of the sequence.
\begin{align*}
	a_n &= dn + a_0 \\
	&= \begin{cases}
		a_0 & n = 0 \\
		d + a_{n-1} & n \geq 1
	\end{cases}.
\end{align*}
As we can see from the explicit formula, if we graphed values of an arithmetic sequence on in the plane with $x$ coordinate $n$ and $y$ coordinate $a_n$, all points would lie on a line with slope $d$ and $y$ intercept $a_0$.

\begin{example}
	Write an explicit formula for the following arithmetic sequence.
	\begin{equation*}
		\left\{\ln{2}, \ln{6}, \ln{18}, \ldots\right\}.
	\end{equation*}
\end{example}
Since we are given that this sequence is arithmetic, we'll find the common difference.
\begin{equation*}
	d = \ln{6} - \ln{2} = \ln{\frac{6}{2}} = \ln{3}.
\end{equation*}
\indent
So, applying the explicit formula for an arithmetic sequence with starting term $\ln{2}$ and common difference $\ln{3}$,
\begin{equation*}
	a_n = \ln{(3)}n + \ln{2}, n\geq 0.
\end{equation*}
\indent
We might have also noticed that each term inside the $\ln$ is triple the previous one, meaning we can write an explicit formula and then simplify to the same answer as before.
\begin{align*}
	a_n &= \ln{(3^{n}\cdot 2)}, n\geq 0 \\
	&= \ln{3^n} + \ln{2}, n \geq 0 \\
	&= \ln{(3)}n + \ln{2}, n \geq 0.
\end{align*}

\subsubsection{Geometric Sequences}
\begin{definition}
	A geometric sequence is one where $\frac{a_{n+1}}{a_n} = r$, a common ratio, for all terms.
\end{definition}
\noindent
That is, to get the next term, we simply multiply some number $r$ (which could be negative) by the previous term.
Geometric sequences can also be defined explicitly or recursively.
Let $a_0$ be the starting term of the sequence.
\begin{align*}
	a_n &= a_0(r)^n \\
	&= \begin{cases}
		a_0 & n = 0 \\
		ra_{n-1} & n \geq 1
	\end{cases}.
\end{align*}
\noindent
As we can see with the explicit formula, if we graphed terms of a geometric sequence for positive $r$, the points would lie on an exponential curve with $y$ intercept $a_0$ and exponential base $r$.

\begin{example}
	Write an explicit formula for the following geometric sequence.
	\begin{equation*}
		\left\{2,-6,18,-54,\ldots\right\}.
	\end{equation*}
\end{example}
Since we are given that this sequence is geometric, we'll find the common ratio.
\begin{equation*}
	r = \frac{-6}{2} = 3.
\end{equation*}
\indent
So, applying the explicit formula for a geometric sequence with starting term 2 and common ratio -3,
\begin{equation*}
	a_n = 2(-3)^n.
\end{equation*}

\subsection{Limits of a Sequence}
Once we have a formula for a sequence, we might be interested to know if $a_n$ tends towards some value as $n$ gets large.
\begin{definition}
	Let $L$ be a real number, the sequence $\left\{a_n\right\}$ as limit $L$ as $n$ approaches infinity if given any real $\epsilon > 0$, there is some index $m$ such that for all $n > m$
	\begin{equation*}
		\abs{a_n - L} < \epsilon.
	\end{equation*}
	We notate this as 
	\begin{equation*}
		\lim_{n\to\infty}{a_n} = L
	\end{equation*}
	and say the sequence converges to $L$.
	If the sequence does not have a limit, then we say the sequences diverges.
\end{definition}

\noindent
The following rules we gave for limits of a function: Sum and Difference Rule, Product Rule, Constant Multiple Rule, and Quotient Rule, all still apply to limits of sequences.
The only rule that doesn't still hold is the Power Rule because of the following sort of problem:
\begin{align*}
	a_n &= (-1)^n \\
	\lim_{n\to\infty}{a_n^2} &= 1 \\
	\left(\lim_{n\to\infty}{a_n}\right)^2 &= \text{DNE} \\
	1 &\neq \text{DNE}.
\end{align*}

\subsubsection{The Sandwich Theorem for Sequences}
\begin{theorem}[Sandwich Theorem for Sequences]
	If $\lim_{n\to\infty}{a_n} = \lim_{n\to\infty}{c_n} = L$ and there is an integer $m$ such that $a_n \leq b_n \leq c_n$ for all $n > m$, then $\lim_{n\to\infty}{b_n} = L$.
\end{theorem}
\noindent
This is essentially the same as the Sandwich Theorem for limits of a function.
The only added caveat is that we have to find some index $m$ for which the sandwiching inequality always holds for terms after the $m$th.

\begin{example}
	Determine if the following sequence converges or diverges.
	If it converges, find its limit.
	\begin{equation*}
		a_n = (-1)^n\frac{n-1}{2}, n\geq 1.
	\end{equation*}
\end{example}
We see that as $n$ grows large $\abs{a_n}$ approaches 1.
However, $a_n$ bounces between 1 and -1 depending on whether $n$ is even or odd.
Thus, we could let $\epsilon = 1/2$, which would show that the limit diverges.

\begin{example}
	Determine if the following sequences converges or diverges.
	If it converges, find its limit.
	\begin{equation*}
		a_n = \frac{\cos{n}}{n}, n\geq 1.
	\end{equation*}
\end{example}
It might at first seem that this limit diverges because $\cos$ bounces between -1 and 1.
However, we can use the Sandwich Theorem to show that the limit converges.
\begin{align*}
	\frac{-1}{n} &\leq \frac{\cos{n}}{n} \leq \frac{1}{n}, n\geq 1 \\
	\lim_{n\to\infty}{\frac{-1}{n}} &\leq \lim_{n\to\infty}{\frac{\cos{n}}{n}} \leq \lim_{n\to\infty}{\frac{1}{n}} \\
	0 &\leq \lim_{n\to\infty}{\frac{\cos{n}}{n}} \leq 0 \\
	\lim_{n\to\infty}{\frac{\cos{n}}{n}} &= 0.
\end{align*}

\subsubsection{Absolute Value Theorem}
We can apply the Sandwich Theorem to show that sequences whose absolute value converges to 0 must also converge to 0.
\begin{theorem}[Absolute Value Theorem]
	If $\lim_{n\to\infty}{\abs{a_n}} = 0$, then $\lim_{n\to\infty}{a_n} = 0$.
\end{theorem}
\begin{proof}
	For all $n$,
	\begin{equation*}
		-\abs{a_n} \leq a_n \leq \abs{a_n}.
	\end{equation*}
	Applying the Sandwich Theorem and limit properties,
	\begin{align*}
		-\lim_{n\to\infty}{\abs{a_n}} &\leq \lim_{n\to\infty}{a_n} \leq \lim_{n\to\infty}{\abs{a_n}} \\
		-0 &\leq \lim_{n\to\infty}{a_n} \leq 0 \\
		\lim_{n\to\infty}{a_n} &= 0.
	\end{align*}
\end{proof}
\section{L'H\^{o}pital's Rule}
\subsection{Indeterminate Form 0/0}
If both $f(x)$ and $g(x)$ are 0 at $x=a$, then the limit
\begin{equation*}
	\lim_{x\to a}{\frac{f(x)}{g(x)}}
\end{equation*}
is an indeterminate form of 0/0, meaning we can't substitute $x=a$ to evaluate the limit.
However, L'\^{o}pital's Rule allows us to modify this limit to get another limit, which might not have this indeterminate form but is guaranteed to have the same limit.

\begin{theorem}[L'H\^{o}pital's Rule, Weaker Form]
	If $f(a) = g(a) = 0$; $f^\prime(a)$ and $g^\prime(a) \neq 0$ exist, then
	\begin{equation*}
		\lim_{x\to a}{\frac{f(x)}{g(x)}} = \frac{f^\prime(a)}{g^\prime(a)}.
	\end{equation*}
\end{theorem}
\begin{proof}
	\begin{align*}
		\lim_{x\to a}{\frac{f(x)}{g(x)}} &= \lim_{x\to a}{\frac{f(x)-0}{g(x)-0}} \\
		&= \lim_{x\to a}{\frac{f(x)-f(a)}{g(x)-g(a)}} \\
		&= \lim_{x\to a}{\frac{\frac{f(x)-f(a)}{x-a}}{\frac{g(x)-g(a)}{x-a}}} \\
		&= \frac{\lim_{x\to a}{\frac{f(x)-f(a)}{x-a}}}{\lim_{x\to a}{\frac{g(x)-g(a)}{x-a}}} \\
		&= \frac{f^\prime(a)}{g^\prime(a)}.
	\end{align*}
\end{proof}

\begin{example}
	Find $\lim_{x\to 2}{\frac{x^2-4}{x-2}}$ using L'H\^{o}pital's Rule.
\end{example}
\begin{answer}
	\begin{equation*}
		\lim_{x\to 2}{\frac{x^2-4}{x-2}} = \frac{2(2)}{1} = 4.
	\end{equation*}
	
	Note that we get the same answer if we didn't use L'H\^{o}pital's Rule and instead factored.
	\begin{equation*}
		\lim_{x\to 2}{\frac{x^2-4}{x-2}} = \lim_{x\to 2}{\frac{(x+2)(x-2)}{x-2}} \lim_{x\to 2}{x+2} = 4.
	\end{equation*}
\end{answer}


It's possible that $f^\prime(a) = g^\prime(a) = 0$, meaning we're still left with the indeterminate form 0/0.
However, we can use a stringer form of L'H\^{o}pital's Rule that allows us to not have to immediate substitute $x=a$ and allows us to apply the rule multiple times if needed.
\begin{theorem}[L'H\^{o}pital's Rule, Stronger Form]
	If $f(a)=g(a)=0$; $f$ and $g$ are differentiable on an open interval $I$ that contains $a$; $g^\prime(x)\neq 0$ if $x\neq a$, then
	\begin{equation*}
		\lim_{x\to a}{\frac{f(x)}{g(x)}} = \lim_{x\to a}{\frac{f^\prime(x)}{g^\prime(x)}}
	\end{equation*}
	if the right-hand limit exists.
\end{theorem}

\begin{example}
	Find the following limit or show that it doesn't exist.
	\begin{equation*}
		\lim_{x\to 0}{\frac{\cos{x}-1}{e^x - x - 1}}.
	\end{equation*}
\end{example}
\begin{answer}
	$\cos{0}-1 = 0 = e^0 - 0 - 1$; $e^x - x - 1 \neq 0$ if $x\neq 0$ on all real numbers.
	\begin{equation*}
		\lim_{x\to 0}{\frac{\cos{x}-1}{e^x - x - 1}} = \lim_{x\to 0}{\frac{-\sin{x}}{e^x - 1}}.
	\end{equation*}
	
	$-\sin{0} = 0 = e^0 - 1$; $e^x - 1 \neq 0$ if $x\neq 0$ on all real numbers.
	\begin{equation*}
		\lim_{x\to 0}{\frac{-\sin{x}}{e^x - 1}} = \lim_{x\to 0}{\frac{-\cos{x}}{e^x}} = \frac{-1}{1} = -1.
	\end{equation*}
\end{answer}

\subsection{Indeterminate Forms $\infty/\infty$, $\infty\cdot 0$, \& $\infty - \infty$}
\subsubsection{$\infty/\infty$}
L'H\^{o}pital's still applies as written for the indeterminate form $\infty/\infty$.

\begin{example}
	Find the following limit or show that it doesn't exist.
	\begin{equation*}
		\lim_{x\to\pi/2}\frac{\tan{x}}{1+\tan{x}}.
	\end{equation*}
\end{example}
\begin{answer}
	Applying L'H\^{o}pital's Rule,
	\begin{equation*}
		\lim_{x\to\pi/2}{\frac{\tan{x}}{1+\tan{x}}} = \lim_{x\to\pi/2}\frac{\sec^2{x}}{\sec^2{x}} = 1.
	\end{equation*}
\end{answer}

\subsubsection{$\infty\cdot 0$}
We need to rearrange the limit into a 0/0 or $\infty/\infty$ indeterminate form.
\begin{example}
	Find the following limit or show that it doesn't exist.
	\begin{equation*}
		\lim_{x\to\infty}{x\sin{\frac{1}{x}}}.
	\end{equation*}
\end{example}
\begin{answer}
	Rearranging,
	\begin{equation*}
		\lim_{x\to\infty}{x\sin{\frac{1}{x}}} = \lim_{x\to\infty}\frac{\sin{\frac{1}{x}}}{\frac{1}{x}}.
	\end{equation*}
	
	Since the limit now has indeterminate form 0/0, we can apply L'H\^{o}pital's Rule.
	\begin{equation*}
		\lim_{x\to\infty}\frac{\sin{\frac{1}{x}}}{\frac{1}{x}} = \lim_{x\to\infty}\frac{\frac{-1}{x^2}\cos{\frac{1}{x}}}{\frac{-1}{x^2}} = cos(0) = 1.
	\end{equation*}
\end{answer}

\subsubsection{$\infty - \infty$}
We need to rearrange the limit into a 0/0 or $\infty/\infty$ indeterminate form.
\begin{example}
	Find the following limit or show that it doesn't exist.
	\begin{equation*}
		\lim_{x\to 1}{\frac{1}{\ln{x}} - \frac{1}{x-1}}.
	\end{equation*}
\end{example}
\begin{answer}
	Rearranging,
	\begin{equation*}
		\lim_{x\to 1}{\frac{1}{\ln{x}} - \frac{1}{x-1}} = \lim_{x\to 1}{\frac{x-1-\ln{x}}{(x-1)\ln{x}}}.
	\end{equation*}
	Since the limit now has indeterminate form 0/0, we can apply L'H\^{o}pital's Rule.
	\begin{equation*}
		\lim_{x\to 1}{\frac{x-1-\ln{x}}{(x-1)\ln{x}}} = \lim_{x\to 1}{\frac{1-\frac{1}{x}}{(x-1)\frac{1}{x} + \ln{x}}} = \lim_{x\to 1}{\frac{\frac{1}{x^2}}{\frac{1}{x} - (x-1)\frac{1}{x^2} + \frac{1}{x}}} = \frac{1}{1-0+1} = \frac{1}{2}.
	\end{equation*}
\end{answer}

\subsection{Indeterminate Forms $1^\infty$, $0^0$, \& $\infty^0$}
For indeterminate forms with exponents, we should take the natural log of the limit, solve that limit, and then exponentiate.

\subsubsection{$1^\infty$}
\begin{example}
	Find the following limit or show that it doesn't exist.
	\begin{equation*}
		\lim_{x\to \infty}{\left(1+\frac{1}{x}\right)^x}.
	\end{equation*}
\end{example}
\begin{answer}
	Let $L$ be the value of the limit.
	\begin{align*}
		L &= \lim_{x\to \infty}{\left(1+\frac{1}{x}\right)^x} \\
		\ln{L} &= \lim_{x\to\infty}{\ln{\left(\left(1+\frac{1}{x}\right)^x\right)}} \\
		&= \lim_{x\to\infty}{x\ln{\left(1+\frac{1}{x}\right)}} \text{ (indeterminate form $\infty\cdot 0$)} \\
		&= \lim_{x\to\infty}{\frac{\ln{\left(1+\frac{1}{x}\right)}}{\frac{1}{x}}} \text{ (indeterminate form $0/0$)} \\
		&= \lim_{x\to\infty}{\frac{\frac{1}{1+\frac{1}{x}}\frac{-1}{x^2}}{\frac{-1}{x^2}}} \\
		&= \lim_{x\to\infty}{\frac{1}{1+\frac{1}{x}}} \\
		&= 1 \\
		e^{\ln{L}} &= e^1 \\
		L &= e.
	\end{align*}
\end{answer}

\subsubsection{$0^0$}
\begin{example}
	Find the following limit or show that it doesn't exist.
	\begin{equation*}
		\lim_{x\to 0^+}{x^x}.
	\end{equation*}
\end{example}
\begin{answer}
	Let $L$ be the value of the limit.
	\begin{align*}
		L &= \lim_{x\to 0^+}{x^x} \\
		\ln{L} &= \lim_{x\to 0^+}{x\ln{x}} \text{ (indeterminate form $0\cdot-\infty$)} \\
		&= \lim_{x\to 0^+}{\frac{\ln{x}}{\frac{1}{x}}} \text{ (indeterminate form $-\infty/\infty$)} \\
		&= \lim_{x\to 0^+}{\frac{\frac{1}{x}}{\frac{-1}{x^2}}} \\
		&= \lim_{x\to 0^+}{\frac{1}{\frac{-1}{x}}} \\
		&= \lim_{x\to 0^+}{-x} \\
		&= 0 \\
		e^{\ln{L}} &= e^0 \\
		L &= 1.
	\end{align*}
\end{answer}

\subsubsection{$\infty^0$}
\begin{example}
	Find the following limit or show that it doesn't exist.
	\begin{equation*}
		\lim_{x\to\infty}{x^{\frac{1}{x}}}.
	\end{equation*}
\end{example}
\begin{answer}
	Let $L$ be the value of the limit.
	\begin{align*}
		L &= \lim_{x\to\infty}{x^{\frac{1}{x}}} \\
		\ln{L} &= \lim_{x\to\infty}{\frac{1}{x}\ln{x}} \text{ (indeterminate form $0\cdot\infty$)} \\
		&= \lim_{x\to\infty}{\frac{\ln{x}}{x}} \text{ (indeterminate form $\infty/\infty$)} \\
		&= \lim_{x\to\infty}{\frac{\frac{1}{x}}{1}} \\
		&= 0 \\
		e^{\ln{L}} &= e^0 \\
		L &= 1.
	\end{align*}
\end{answer}
\section{Relative Growth Rates}
In many practical applications, like the run times of computer algorithms for example, we often want to know if a function grows slower, the same, or faster than another.
\begin{definition}
	If
	\begin{equation*}
		\lim_{x \to \infty}{\frac{f(x)}{g(x)}} = \infty \Leftrightarrow \lim_{x \to \infty}{\frac{g(x)}{f(x)}} = 0,
	\end{equation*}
	then $f$  grows faster than $g$.
	If
	\begin{equation*}
		\lim_{x \to \infty}{\frac{f(x)}{g(x)}} = c \Leftrightarrow \lim_{x \to \infty}{\frac{g(x)}{f(x)}} = \frac{1}{c}
	\end{equation*}
	for some non-zero constant $c$, then $f$ and $g$ grow at the same rate.
\end{definition}

\begin{example}
	Compare $e^x$ and $x^{100}$.
	Does one grow faster than the other, or do they grow at the same rate?
\end{example}
\begin{answer}
	\begin{align*}
		\lim_{x\to\infty}{\frac{e^x}{x^{100}}} &= \lim_{x\to\infty}{\frac{e^x}{100x^{99}}} \\
		&= \vdots \text{ (after many applications of L'H\^{o}pital's Rule)} \\
		&= \lim_{x\to\infty}{\frac{e^x}{100!}}
		&= \infty.
	\end{align*}
	
	So, $e^x$ grows faster than $x^{100}$.
	In fact, any exponential $b^x$ grows faster than any polynomial, as long as $b > 1$.
\end{answer}

\subsection{Transitive Grow Rates}
For sufficiently large $x$, growth rates are transitive.
That is, if $f$ grows the same/faster/slower/ than $g$, and $g$ grows the same/slower/faster than $h$, then $f$ also grows the same/faster/slower than $h$.

\begin{example}
	Show that $f(x)=\sqrt{x^2+5}$ and $g(x)=\left(2\sqrt{x}-1\right)^2$ grow at the same rate.
\end{example}
\begin{answer}
	We'll show that both $f$ and $g$ grow at the same rate as $h(x)=x$.
	Starting with $f$ and $h$,
	\begin{align*}
		\lim_{x\to\infty}{\frac{f(x)}{h(x)}} &= \lim_{x\to\infty}{\frac{\sqrt{x^2+5}}{x}} \\
		&= \lim_{x\to\infty}{\sqrt{\frac{x^2+5}{x^2}}} \\
		&= \sqrt{\lim_{x\to\infty}{\frac{x^2+5}{x^2}}} \text{ (by the Power Rule)} \\
		&= \sqrt{1} \\
		&= 1.
	\end{align*}
	
	So, $f$ and $h$ grow at the same rate.
	Moving on to $g$ and $h$,
	\begin{align*}
		\lim_{x\to\infty}{\frac{g(x)}{h(x)}} &= \lim_{x\to\infty}{\frac{\left(2\sqrt{x}-1\right)^2}{x}} \\
		&= \lim_{x\to\infty}{\left(\frac{2\sqrt{x}-1}{\sqrt{x}}\right)^2} \\
		&= \left(\lim_{x\to\infty}{\frac{2\sqrt{x}-1}{\sqrt{x}}}\right)^2 \text{ (by the Power Rule)} \\
		&= \left(2\right)^2 \\
		&= 4.
	\end{align*}
	
	So, $g$ and $h$ grow at the same rate.
	Since $f$ and $g$ both grow at the same rate as $h$, $f$ and $g$ must grow at the same rate as each other.
\end{answer}

\subsection{Growth Rate Hierarchy ($n^n$FEPL)}
For most of the common types of functions we see, we can establish families of functions and rank these families by their growth rates from fastest-growing to slowest-growing.
If two functions are in different families, we can be sure that one grows faster than the other.
If two functions are in the same family, we'll have to do more work to compare them.
These families are summarized by the acronym $n^n$FEPL\footnote{You might recognize these families as a sort of Big-O family from computer science.}.
\begin{itemize}[align=left, leftmargin=0.66in]
	\item[$\textbf{n}^\textbf{n}$] These are functions that have a variable both in the base and exponent.
	\item[\textbf{F}actorials] These are functions that have an $n!$ term.
	\item[\textbf{E}xponentials] These are functions that have a constant base and a variable exponent.
		Note that if the variable base if less than 1, the function actually gets smaller for larger $n$.
	\item[\textbf{P}olynomials] These are functions with a variable base and constant exponent.
		Certain polynomials can still grow faster than others.
		For example, $x^2$ grows faster than $x$, which grows faster than $\sqrt{x}$.
	\item[\textbf{L}ogarithms] These are functions that have a log of a polynomial.
\end{itemize}

Although these rules are indeed true, don't just apply them blindly.
You should try to simplify a function first before figuring out to which family it belongs.
For example, although $\ln{x^x}$ contains an $x^x$ and a $\ln$, it's neither in the $n^n$ family nor in the logarithms family.
In fact, although this function is not a polynomial, it grows faster than $x$ but slower than $x^2$. \\


A function belongs to the family of its fastest-growing positive term.
Negative terms can either be ignored or used to simplify other terms.
For example, although $x^3 + e^x$ contains a polynomial $x^3$ term, for very large $x$, the $e^x$ term dominates the growth, meaning this function is part of the exponentials family.
\section{Improper Integrals}
Now that we've developed the tools to deal with limits as they approach infinity and the possible indeterminate forms that may arise, we can apply these ideas to integrals, allowing us to have $-\infty$ and $\infty$ as limits of integration.
We call these integrals with $\pm\infty$ as limits of integration, and functions that become $\pm\infty$ somewhere within the interval we're integrating on improper integrals.

\subsection{Infinite Integration Limits}
\begin{definition}
	If $f$ is continuous on $[a,\infty)$, then
	\begin{equation*}
		\int_{a}^{\infty}{f(x)\d{x}} = \lim_{b\to\infty}{\int_{a}^{b}{f(x)\d{x}}}.
	\end{equation*}
	If $f$ is continuous on $(-\infty,b]$, then
	\begin{equation*}
		\int_{\infty}^{b}{f(x)\d{x}} = \lim_{a\to-\infty}{\int_{a}^{b}{f(x)\d{x}}}.
	\end{equation*}
	If $f$ is continuous on $(-\infty,\infty)$, then
	\begin{equation*}
		\int_{-\infty}^{\infty}{f(x)\d{x}} = \int_{-\infty}^{c}{f(x)\d{x}} + \int_{c}^{\infty}{f(x)\d{x}}
	\end{equation*}
	for any real constant $c$.
	If these limits exist, then the integral converges and has a value.
	Otherwise, the integral diverges and does not have a value.
\end{definition}

\begin{example}
	Evaluate the following integral or state that it diverges.
	\begin{equation*}
		\int_{2}^{\infty}{\frac{3}{x^2-x}\d{x}}.
	\end{equation*}
\end{example}
\begin{answer}
	Applying the definition,
	\begin{align*}
		\int_{2}^{\infty}{\frac{3}{x^2-x}\d{x}} &= \lim_{b\to\infty}{\int_{2}^{b}{\frac{3}{x^2-x}\d{x}}} \\
		&= \lim_{b\to\infty}{3\int_{2}^{b}{\left(\frac{1}{x-1}-\frac{1}{x}\right)\d{x}}} \\
		&= \lim_{b\to\infty}{3\ln{\bigg\lvert\frac{x-1}{x}\bigg\rvert}\Biggr\rvert_{2}^{b}} \\
		&= \lim_{b\to\infty}{3\ln{\bigg\lvert\frac{b-1}{b}\bigg\rvert}} - 3\ln{\bigg\lvert\frac{1}{2}\bigg\rvert} \\
		&= 0 + 3\ln{2} \\
		&= 3\ln{2}.
	\end{align*}
\end{answer}

\begin{example}
	Evaluate the following integral or state that it diverges.
	\begin{equation*}
		\int_{1}^{\infty}{\frac{\d{x}}{\sqrt[4]{x}}}.
	\end{equation*}
\end{example}
\begin{answer}
	Applying the definition,
	\begin{align*}
		\int_{1}^{\infty}{\frac{\d{x}}{\sqrt[4]{x}}} &= \lim_{b\to\infty}{\int_{1}^{b}{\frac{\d{x}}{\sqrt[4]{x}}}} \\
		&= \lim_{b\to\infty}{\frac{4}{3}\sqrt[4]{x^3}\biggr\rvert_{1}^{b}} \\
		&= \lim_{b\to\infty}{\frac{4}{3}\sqrt[4]{b^3}} - \frac{4}{3}\sqrt[4]{1^3} \\
		&= \infty - \frac{4}{3} \\
		&= \text{diverges}.
	\end{align*}
\end{answer}

\subsection{Infinite Discontinuities}
An infinite discontinuity occurs when a function takes on a value of $\pm\infty$ on the interval we're integrating on.
In this case, we'll need to split the integral into pieces, evaluating the limit as we approach this infinite discontinuity from both sides.
\begin{definition}
	If $f$ is continuous on $(a,b]$, then
	\begin{equation*}
		\int_{a}^{b}{f(x)\d{x}} = \lim_{c\to a^+}{\int_{c}^{b}{f(x)\d{x}}}.
	\end{equation*}
	If $f$ is continuous on $[a,b)$, then
	\begin{equation*}
		\int_{a}^{b}{f(x)\d{x}} = \lim_{c\to b^-}{\int_{a}^{c}{f(x)\d{x}}}.
	\end{equation*}
	If $f$ is continuous on $[a,c) \cup (c,b]$, then
	\begin{equation*}
		\int_{a}^{b}{f(x)\d{x}} = \int_{a}^{c}{f(x)\d{x}} + \int_{c}^{b}{f(x)\d{x}}.
	\end{equation*}
	If these limits exist, then the integral converges and has a value.
	Otherwise, the integral diverges and does not have a value.
\end{definition}

\begin{example}
	Evaluate the following integral or state that it diverges.
	\begin{equation*}
		\int_{0}^{1}{\frac{\d{x}}{x^2}}.
	\end{equation*}
\end{example}
\begin{answer}
	We see that we have an infinite discontinuity at $x=0$.
	Applying the definition,
	\begin{align*}
		\int_{0}^{1}{\frac{\d{x}}{x^2}} &= \lim_{c\to 0^+}{\int_{c}^{1}{\frac{\d{x}}{x^2}}} \\
		&= \lim_{c\to 0^+}{\frac{-1}{x}\biggr\rvert_{c}^{1}} \\
		&= \frac{-1}{1} + \lim_{c\to 0^+}{\frac{1}{c}} \\
		&= -1 + \infty \\
		&= \text{diverges}.
	\end{align*}
\end{answer}

\begin{example}
	Evaluate the following integral or state that it diverges.
	\begin{equation*}
		\int_{0}^{1}{\frac{\d{x}}{x^{1/2}}}.
	\end{equation*}
\end{example}
\begin{answer}
	We see that we have an infinite discontinuity at $x=0$.
	Applying the definition,
	\begin{align*}
		\int_{0}^{1}{\frac{\d{x}}{x^{1/2}}} &= \lim_{c\to 0^+}{\int_{c}^{1}{\frac{\d{x}}{x^{1/2}}}} \\
		&= \lim_{c\to 0^+}{2x^{1/2}\biggr\rvert_{c}^{1}} \\
		&= 2 - \lim_{c\to 0^+}{2c^{1/2}} \\
		&= 2 - 0 \\
		&= 2.
	\end{align*}
\end{answer}

\subsection{Convergence Tests}
\subsubsection{P-Test}
\begin{lemma}
	The following integral will converge when $p > 1$ and diverge if $0 < p \leq 1$.
	\begin{equation*}
		\int_{1}^{\infty}{\frac{\d{x}}{x^p}}.
	\end{equation*}
\end{lemma}

\begin{example}
	Evaluate the following integral or state that it diverges.
	\begin{equation*}
		\int_{1}^{\infty}{\frac{\d{x}}{x}}.
	\end{equation*}
\end{example}
\begin{answer}
	We have an integral where $p=1$.
	So, by the P-Test, the integral diverges.
\end{answer}

\begin{example}
	Evaluate the following integral or state that it diverges.
	\begin{equation*}
		\int_{1}^{\infty}{\frac{\d{x}}{x^{1.001}}}.
	\end{equation*}
\end{example}
\begin{answer}
	We have an integral where $p=1.001$.
	So, by the P-Test, the integral converges.
	\begin{align*}
		\int_{1}^{\infty}{\frac{\d{x}}{x^{1.001}}} &= \lim_{b\to\infty}{\int_{1}^{b}{\frac{\d{x}}{x^{1.001}}}} \\
		&= \lim_{b\to\infty}{-1000x^{-0.001}\biggr\rvert_{1}^{b}} \\
		&= \lim_{b\to\infty}{-1000b^{-0.001}} + 1000(1)^{-0.001} \\
		&= 0 + 1000 \\
		&= 1000.
	\end{align*}
\end{answer}

\subsubsection{Direct Comparison Test}
\begin{lemma}
	Let $f$ and $g$ be continuous on $[a,\infty)$ with $0 \leq f(x) \leq g(x)$ for all $x \geq a$.
	\begin{align*}
		\int_{a}^{\infty}{f(x)\d{x}} &\text{ converges if } \int_{a}^{\infty}{g(x)\d{x}} \text{ converges.} \\
		\int_{a}^{\infty}{g(x)\d{x}} &\text{ diverges if } \int_{a}^{\infty}{f(x)\d{x}} \text{ diverges.}
	\end{align*}
\end{lemma}

That is, if a larger function converges, then so will a smaller funtion; if a smaller function diverges, then so will a larger function. \\


The hardest part of the Direct Comparison Test is deciding what function you should compare to.
A general rule is to pick a function that is similar to, but simpler than then given function.

\begin{example}
	Evaluate the following integral or state that it diverges.
	\begin{equation*}
		\int_{1}^{\infty}{\frac{\d{x}}{x^2-0.1}}.
	\end{equation*}
\end{example}
\begin{answer}
	If the 0.1 wasn't inside the square root, the function would simplify to $1/x$.
	Since the 0.1 is subtracted, the denominator is smaller than $1/x$.
	So, $1/x$ is a function that is smaller on $[1,\infty)$, meaning if it diverges, then so will the original function.
	We know by the P-Test that the integral of $1/x$ from 1 to $\infty$ will diverge, so the original function also diverges.
\end{answer}

\begin{example}
	Evaluate the following integral or state that it diverges.
	\begin{equation*}
		\int_{1}^{\infty}{e^{-x^2}\d{x}}.
	\end{equation*}
\end{example}
\begin{answer}
	Since the exponent is negative, a smaller exponent would mean a larger value.
	So, $e^{-x}$ is a larger function on $[1,\infty)$.
	\begin{equation*}
		\int_{1}^{\infty}{e^{-x}\d{x}} = \frac{1}{e},
	\end{equation*}
	meaning it converges, so the original function also converges.
\end{answer}

\subsubsection{Limit Comparison Test}
\begin{lemma}
	If positive functions $f$ and $g$ are continuous on $[a,\infty)$ and
	\begin{equation*}
		\lim_{x\to\infty}{\frac{f(x)}{g(x)}}
	\end{equation*}
	converges to a positive real number, then
	\begin{equation*}
		\int_{a}^{\infty}{f(x)\d{x}} \text { and } \int_{a}^{\infty}{g(x)\d{x}}
	\end{equation*}
	both converge or both diverge.
\end{lemma}


Many functions to which you can apply the Limit Comparison Test you can also apply the Direct Comparison Test.
The practical use of the limit comparison test is to take an uglier function, that may be tedious to integrate and compare it to a function that is easy to determine whether it diverges using something like the P-Test.
A common strategy, especially for rational functions, is to look at their end behavior model.

\begin{example}
	Evaluate the following integral or state that it diverges.
	\begin{equation*}
		\int_{1}^{\infty}{\frac{\d{x}}{1+x^2}}.
	\end{equation*}
\end{example}
\begin{answer}
	Although you might recognize this as the derivative of $\arctan$, let's continue with the Direct Comparison Test.
	This function looks very similar to $1/x^2$, which we know by the P-Test will converge on $[1,\infty)$.
	\begin{equation*}
		\lim_{x\to\infty}{\frac{\frac{1}{x^2}}{\frac{1}{1+x^2}}} = \lim_{x\to\infty}{\frac{1+x^2}{x^2}} = 1.
	\end{equation*}
	Since 1 is a positive real constant and the integral of $1/x^2$ converges, then the original integral also converges by the Limit Comparison Test.
\end{answer}

\begin{example}
	Evaluate the following integral or state that it diverges.
	\begin{equation*}
		\int_{1}^{\infty}{\frac{3x+6}{1-5x+7x^2}\d{x}}.
	\end{equation*}
\end{example}
\begin{answer}
	Looking at this rational function, we see a degree 1 polynomial in the numerator and a degree 2 polynomial in the denominator.
	So, we'd expect this rational function to have the same end behavior model as $1/x$, which we know by the P-Test diverges.
	\begin{equation*}
		\lim_{x\to\infty}{\frac{\frac{3x+6}{1-5x+7x^2}}{\frac{1}{x}}} = \lim_{x\to\infty}{\frac{3x^2+6x}{7x^2-5x+1}} = \frac{3}{7}.
	\end{equation*}
	Since 3/7 is a positive real constant and the integral of $1/x$ diverges, the the original integral also diverges by the Limit Comparison Test.
\end{answer}
		\chapter{Infinite Series}
We previously discussed some special types of series and formulas for their nth term.
Now, we'll look at sums of infinitely many terms and eventually see how summing variables rather than just numbers allows us to approximate functions.

\section{Power Series}
\subsection{Geometric Series}
First, we need to define what we mean by an infinite series.
\begin{definition}
	An infinite series is of the form
	\begin{equation*}
		a_1 + a_2 + \ldots + a_n + \ldots \text{ or equivalently, } \sum_{k=1}^{\infty}{a_k}.
	\end{equation*}
	Just like with finite series, each $a_i$ is a term, and $a_n$ is the nth term.
\end{definition}

We can describe the behavior of an infinite series by looking at how its value behaves after summing a finite number of terms.
We can define what it means for an infinite sum to have a value by looking at the limit of the partial sums as $n$ grows large.
\begin{definition}
	The nth partial sum of an infinite series is
	\begin{equation*}
		s_n = \sum_{k=1}^{n}{a_k}.
	\end{equation*}
	The infinite series converges to value $L$ if
	\begin{equation*}
		\lim_{n\to\infty}{s_n} = L.
	\end{equation*}
	Otherwise, the series diverges and does not have a value.
\end{definition}

\begin{example}
	State if the following infinite series converges or diverges.
	\begin{equation*}
		\frac{3}{10} + \frac{3}{100} + \ldots + \frac{3}{10^n} + \ldots.
	\end{equation*}
\end{example}
\begin{answer}
	Looking at the partial sums,
	\begin{align*}
		s_1 &= 0.3 \\
		s_2 &= 0.33 \\
		&\vdots \\
		s_n &= 0.\underbrace{33333\ldots}_{\text{$n$ total 3's}}
	\end{align*}
	
	So, it seems the limit of the partial sums tends towards a decimal with an infinite number of 3's.
	This value corresponds to the decimal expansion of 1/3, which clearly is real and finite, so the series converges.
\end{answer}


The above series is a geometric series since each subsequent term is 10 times smaller than the previous one (i.e $r=1/10$).
\begin{lemma}
	The geometric series
	\begin{equation*}
		\sum_{k=0}^{\infty}{a_0(r)^k}
	\end{equation*}
	converges to a value of $a_0/(1-r)$ if $\abs{r} < 1$ and diverges otherwise.
\end{lemma}
\begin{proof}
	We'll first find a formula for the partial sums and then find the limit of the partial sums for $1 < r < 1$.
	\begin{align*}
		s_n &= \sum_{k=0}^{n}{a_0(r)^k} \\
		&= a_0 + a_0r + a_0r^2 + \ldots + a_0r^n \\
		rs_n &= a_0r + a_0r^2 + a_0r^3 + \ldots + a_0r^n + a_0r^{n+1} \\
		&= -a_0 + s_n + a_0r^{n+1} \\
		s_n(r-1) &= a_0\left(r^{n+1} - 1\right) \\
		s_n &= a_0\frac{r^{n+1}-1}{r-1}.
	\end{align*}
	This formula for partial sums holds for all values of $r$.
	Now we'll take the limit of $s_n$ and see for what values of $r$ the limit exists.
	\begin{align*}
		\sum_{k=0}^{\infty}{a_0(r)^n} &= \lim_{n\to\infty}{s_n} \\
		&= \lim_{n\to\infty}{a_0\frac{r^{n+1}-1}{r-1}} \\
		&= \frac{-a_0}{r-1}, \abs{r} < 1 \\
		&= \frac{a_0}{1-r}, \abs{r} < 1.
	\end{align*}
\end{proof}

\subsection{Functions from Geometric Series}
What happens if rather than letting $r$ be some fixed value we know beforehand, we let $r$ be some variable $x$?
Applying the formula,
\begin{equation*}
	a_0 + a_0x + a_0x^2 + \ldots = \frac{a_0}{1-x}, \abs{x} < 1.
\end{equation*}
This sort of sum of powers of $x$ is called a power series, and the condition that $\abs{x}<1$ is called the interval of convergence.
Right now, this power series is centered at $x=0$, but we can generalize it a bit to be centered at $x=h$.
\begin{equation*}
	a_0 + a_0(x-h) + a_0(x-h)^2 + \ldots = \frac{a_0}{1-(x-h)}, \abs{x-h} < 1.
\end{equation*}
Note that this formula allows us to find the power series of any function $a_0/(mx+b)$.
\begin{equation*}
	\frac{a_0}{mx + b} = \frac{a_0}{1-(1-mx-b)} = a_0 + a_0(1-mx+b) + a_0(1-mx+b)^2 + \ldots, \abs{1-mx-b} < 1.
\end{equation*}
So,
\begin{align*}
	\frac{1}{x} &= 1 + (1-x) + (1-x)^2 + \ldots, \abs{1-x} < 1 \\
	\frac{1}{1-x} &= 1 + x + x^2 + x^3 + \ldots, \abs{x} < 1 \\
	\frac{1}{1+x} &= 1 - x + x^2 - x^3 + \ldots, \abs{x} < 1.
\end{align*}

\subsubsection{Term-By-Term Differentiation}
\begin{theorem}
	If the power series
	\begin{equation*}
		f(x) = \sum_{k=0}^{\infty}{c_k(x-a)^k} = c_0 + c_1(x-a) + c_2(x-a)^2 + \ldots
	\end{equation*}
	converges for $\abs{x-a} < R$, including $R=\infty$, then the power series
	\begin{equation*}
		\sum_{k=1}^{\infty}{kc_k(x-a)^{k-1}} = c_1 + 2c_2(x-a) + 3c_3(x-a)^2 + \ldots
	\end{equation*}
	also converges for $\abs{x-a} < R$ and is equal to $f^\prime(x)$ on that interval.
\end{theorem}

Applying the theorem,
\begin{align*}
	\frac{-1}{x^2} &= -1 - 2(1-x) - 3(1-x)^2 - \ldots, \abs{1-x} < 1 \\
	\frac{1}{(1-x)^2} &= 1 + 2x + 3x^2 + \ldots, \abs{x} < 1 \\
	\frac{-1}{(1+x)^2} &= -1 + 2x - 3x^2 + \ldots, \abs{x} < 1.
\end{align*}

\subsubsection{Term-By-Term Integration}
\begin{theorem}
	If the power series
	\begin{equation*}
		f(x) = \sum_{k=0}^{\infty}{c_k(x-a)^k} = c_0 + c_1(x-a) + c_2(x-a)^2 + \ldots
	\end{equation*}
	converges for $\abs{x-a} < R$, including $R=\infty$, then the power series
	\begin{equation*}
		\sum_{k=0}^{\infty}{c_k\frac{(x-a)^{k+1}}{k+1}} = c_0(x-a) + c_1\frac{(x-a)^2}{2} + c_2\frac{(x-a)^3}{3} + \ldots
	\end{equation*}
	also converges for $\abs{x-a} < R$ and represents the antiderivative of $f$ on that interval.
\end{theorem}

Applying the theorem,
\begin{align*}
	\ln{\abs{x}} &= 1 + \frac{(1-x)^2}{2} + \frac{(1-x)^3}{3} + \ldots, \abs{1-x} < 1 \\
	-\ln{\abs{1-x}} &= x + \frac{x^2}{2} + \frac{x^3}{3} + \ldots, \abs{x} < 1 \\
	\ln{\abs{1+x}} &= x - \frac{x^2}{2} + \frac{x^3}{3} - \frac{x^4}{4} + \ldots, \abs{x} < 1.
\end{align*}

This is pretty impressive: we now have an formula for $\ln{x}$ in terms of polynomials.
This also includes some pretty surprising identities.
For example\footnote{Technically, we're substituting $x=1$ which isn't in the interval of convergence. However, since we are dealing with an alternating sum, the series will also converge for $x=-1$ and $x=1$.},
\begin{equation*}
	\ln{2} = 1 - \frac{1}{2} + \frac{1}{3} - \ldots + \frac{(-1)^{n+1}}{n} + \ldots.
\end{equation*}

\begin{example}
	Find a power series for $\arctan{x}$ and state the interval of convergence.
\end{example}
\begin{answer}
	We know that
	\begin{equation*}
		\dd{}{x}\arctan{x} = \frac{1}{1+x^2}.
	\end{equation*}
	
	So, if we can find a power series that represents $1/(1+x^2)$, we can integrate term-by-term to get a power series for $\arctan{x}$.
	We also already know the following power series.
	\begin{equation*}
		\frac{1}{1+u} = 1 - u + u^2 - u^3 + \ldots, \abs{u} < 1.
	\end{equation*}
	
	Letting $u=x^2$,
	\begin{align*}
		\frac{1}{1+x^2} &= 1 - x^2 + x^4 - x^6 + \ldots, \abs{x^2} < 1 \\
		&= 1 - x^2 + x^4 - x^6 + \ldots, \abs{x} < 1.
	\end{align*}
	
	Integrating term-by-term,
	\begin{align*}
		\arctan{x} = x - \frac{x^3}{3} + \frac{x^5}{5} + \ldots + (-1)^n\frac{x^{2n+1}}{2n+1} + \ldots, \abs{x} < 1.
	\end{align*}
	
	This power series also gives rise to a pretty interesting identity\footnote{See footnote 1.}.
	\begin{equation*}
		\arctan{1} = \frac{\pi}{4} = 1 - \frac{1}{3} + \frac{1}{5} - \frac{1}{7} + \ldots + (-1)^n\frac{x^{2n+1}}{2n+1} \ldots.
	\end{equation*}
\end{answer}
\section{Taylor Series}
\subsection{Construction}
Although we can approximate lots of functions using power series derived from the geometric series, it'd be nice if we had a more general way to approximate any function using a power series.
We already have an approximation using a tangent line: the 0th and 1st derivatives of the function and line are equal at the point of tangency.
We could extend this idea of derivatives being equal to higher-order derivatives and higher degree polynomials.

\begin{example}
	Construct  polynomial $P(x)=a_0+a_1x + a_2x^2 + a_3x^3 + a_4x^4$ with the following behavior at $x=0$:
	\begin{align*}
		P(0) &= 1 \\
		P^\prime(0) &= 2 \\
		P^{\prime\prime}(0) &= 3 \\
		P^{\prime\prime\prime}(0) &= 4 \\
		P^{(4)}(0) &= 5.
	\end{align*}
\end{example}
\begin{answer}
	Plugging in $x=0$ and solving for $a_0$,
	\begin{align*}
		a_0 + a_1(0) + a_2(0)^2 + a_3(0)^4 + a_4(0)^4 &= 1 \\
		a_0 &= 1. 
	\end{align*}
	
	Differentiating once, plugging in $x=0$, and solvinf for $a_1$,
	\begin{align*}
		P^\prime(x) &= a_1 + 2a_2x + 3a_3x^2 + 4a_4x^3 \\
		2 &= a_1 + 2a_2(0) + 3a_3(0)^2 + 4a_4(0)^3 \\
		a_1 &= 2.
	\end{align*}
	
	Continuing with differentiating and plugging in $x=0$, we get $a_2 = \frac{3}{2}$, $a_3 = \frac{2}{3}$, and $a_4 = \frac{5}{25}$.
	So,
	\begin{equation*}
		P(x) = 1 + 2x + \frac{3}{2}x^2 + \frac{2}{3}x^3 + \frac{5}{24}x^4.
	\end{equation*}
\end{answer}

\begin{example}
	Construct a degree 4 polynomial that approximates $\ln{(1+x)}$ at $x=0$.
\end{example}
\begin{answer}
	\begin{table}[H]
		\begin{center}
			\begin{tabular}{ccc}
				$f(x)$ & $P(x)$ & $a_n$ \\
				\hline
				$f(x)=\ln{(1+x)}$ & $P(x)=a_0 + a_1x + a_2x^2 + a_3x^3 + a_4x^4$ & \\
				$f(0)=\ln{(1+0)}=0$ & $P(0)=a_0$ & $a_0 = 0$ \\
				\hline
				$f^\prime(x)=\frac{1}{1+x}$ & $P^\prime(x) = a_1 + 2a_2x + 3a_3x^2 + 4a_4x^3$ & \\
				$f^\prime(0)=\frac{1}{1+0}=1$ & $P^\prime(0) = a_1$ & $a_1 = 1$ \\
				\hline
				$f^{\prime\prime}(x)=\frac{-1}{(1+x)^2}$ & $P^{\prime\prime}(x)=2a_2 + 6a_3x + 12a_4x^2$ & \\
				$f^{\prime\prime}(0)=\frac{-1}{(1+0)^2} = -1$ & $P^{\prime\prime}(x)=2a_2$ & $a_2 = \frac{-1}{2}$ \\
				\hline
				$f^{(3)}(x) = \frac{2}{(1+x)^3}$ & $P^{(3)}(x) = 6a_3 + 24a_4x$ & \\
				$f^{(3)}(0) = \frac{2}{(1+0)^3}=2$ & $P^{(3)}(0) = 6a_3$ & $a_3 = \frac{1}{3}$ \\
				\hline
				$f^{(4)}(x) = \frac{-6}{(1+x)^4}$ & $P^{(4)}(x) = 24a_4$ & \\
				$f^{(4)}(x) = \frac{-6}{(1+0)^4}=-6$ & $P^{(4)}(0) = 24a_4$ & $a_4 = \frac{-1}{4}$ \\
				\hline
			\end{tabular}
		\end{center}
	\end{table}
	
	So, our polynomial is
	\begin{equation*}
		P(x) = x - \frac{x^2}{2} + \frac{x^3}{3} - \frac{x^4}{4}.
	\end{equation*}
	
	Note that this polynomial exactly matches the power series we derived for $\ln{(1+x)}$. \\
	This polynomial is called the 4th order Taylor polynomial of $\ln{(1+x)}$ at $x=0$.
	The series created from all order Taylor polynomials is called the Taylor series of $\ln{(1+x)}$ at $x=0$.
\end{answer}

\subsection{Definition}
\begin{definition}
	Let $f$ be a $n$ times differentiable function where all derivatives exist at $x=a$.
	The Taylor series for $f$ at $x=a$ is
	\begin{equation*}
		\sum_{k=0}^{\infty}{\frac{f^{(k)}(x)}{k!}(x-a)^k} = f(a) + f^\prime(a)(x-a) + \frac{f^{\prime\prime}(a)}{2!}(x-a)^2 + \ldots + \frac{f^{(n)}(a)}{n!}(x-a)^n + \ldots.
	\end{equation*}
	The $n$th partial sum of the Taylor Series,
	\begin{equation*}
		P_n(x) = \sum_{k=0}^{n}{\frac{f^{(k)}(x)}{k!}(x-a)^k}
	\end{equation*}
	is the Taylor polynomial of order $n$ for $f$ at $x=a$.
\end{definition}

When $a=0$ you might also hear Taylor series referred to as Maclaurin series.
Like power series, Taylor series have intervals of convergence.

\begin{example}
	Find the Taylor series for $e^x$ at $x=0$.
	Verify using term-by-term differentiation that $e^x$ is its own derivative.
\end{example}
\begin{answer}
	We know that $e^x$ is its own derivative, so $f^{(k)}(0)=e^0 = 1$ for all $i$.
	Applying the definition,
	\begin{align*}
		e^x &= e^0 + e^0(x-0) + \frac{e^0}{2!}(x-0)^2 + \frac{e^0}{3!}(x-0)^3 + \ldots + \frac{e^0}{n!}(x-0)^n + \ldots \\
		&= 1 + x + \frac{x^2}{2!} + \frac{x^3}{3!} + \ldots + \frac{x^n}{n!} + \ldots.
	\end{align*}
	
	Differentiating term-by-term,
	\begin{equation*}
		\dd{}{x}e^x = 1 + x + \frac{x^2}{2!} + \ldots + \frac{x^{n-1}}{(n-1)!} + \ldots
	\end{equation*}
	we see that we get the same series, confirming that $e^x$ is its own derivative.
\end{answer}

\subsection{Common Maclaurin Series}
\begin{align*}
	\frac{1}{1-x} &= 1 + x + x^2 + \ldots = \sum_{k=0}^{\infty}{x^k}, \abs{x} < 1 \\
	\frac{1}{1+x} &= 1 - x + x^2 - \ldots = \sum_{k=0}^{\infty}{(-1)^kx^k}, \abs{x} < 1 \\
	e^x &= 1 + x + \frac{x^2}{2!} + \ldots = \sum_{k=0}^{\infty}{\frac{x^k}{k!}}, \text{ all real $x$} \\
	\sin{x} &= x - \frac{x^3}{3!} + \frac{x^5}{5!} - \ldots = \sum_{k=0}^{\infty}{(-1)^k\frac{x^{2k+1}}{(2k+1)!}}, \text{ all real $x$} \\
	\cos{x} &= 1 - \frac{x^2}{2!} + \frac{x^4}{4!} - \ldots = \sum_{k=0}^{\infty}{(-1)^k\frac{2^{2k}}{(2k)!}}, \text{ all real $x$} \\
	\ln{(1+x)} &= x - \frac{x^2}{2} + \frac{x^3}{3} - \ldots = \sum_{k=0}^{\infty}{(-1)^k\frac{x^{k+1}}{k+1}}, \abs{x} \leq 1 \\
	\arctan{x} &= x - \frac{x^3}{3} + \frac{x^5}{5} - \ldots = \sum_{k=0}^{\infty}{(-1)^k\frac{x^{2k+1}}{2k+1}}, \abs{x} \leq 1.
\end{align*}

\subsubsection{Euler's Identity}
You might have seen the identity $e^{i\pi} + 1 = 0$ or even $e^{ix} = \cos{x} + i\sin{x}$.
Using our common Taylor Series, we can derive this famous identity. \\


Starting with the Taylor series for $e^{ix}$,
\begin{align*}
	e^{ix} &= 1 + (ix) + \frac{(ix)^2}{2!} + \frac{(ix)^3}{3!} + \frac{(ix)^4}{4!} + \frac{(ix)^5}{5!} + \ldots = \sum_{k=0}^{\infty}{\frac{(ix)^k}{k!}}, \text{ all real $x$} \\
	&= 1 + ix + i^2\frac{x^2}{2!} + i^3\frac{x^3}{3!} + i^4\frac{x^4}{4!} + i^5\frac{x^5}{5!} + \ldots = \sum_{k=0}^{\infty}{i^k\frac{x^k}{k!}}, \text{ all real $x$} \\
	&= 1 + ix - \frac{x^2}{2!} - i\frac{x^3}{3!} + \frac{x^4}{4!} + i\frac{x^5}{5!} - \ldots, \text{ all real $x$} \\
	&= \left(1 - \frac{x^2}{2!} + \frac{x^4}{4!} + \ldots \right) + i\left(x - \frac{x^3}{3!} + \frac{x^5}{5!} + \ldots \right) \\
	&= \cos{x} + i\sin{x}.
\end{align*}

Plugging in $x=\pi$,
\begin{align*}
	e^{i\pi} &= \cos{\pi} + i\sin{\pi} \\
	&= -1 + 0 \\
	e^{i\pi} + 1 &= 0.
\end{align*}
\section{Approximation Error}
Although knowing these infinite Taylor series is nice, if we want to use them to find function values, we'll need to approximate.
This means only using some finite number of terms in the Taylor series for our approximation.
Because we're not using all the terms, we'll naturally have some truncation error.
We'd like to be able to say how big this truncation error is so we can be sure that our approximation is good enough.

\subsection{Alternating Series Estimation Theorem}
\begin{theorem}[Alternating Series Estimation Theorem]
	Let $s$ be a convergent, alternating (i.e the sign of each term alternates) series where the terms of $\abs{s}$ are strictly decreasing.
	Then the $n$th term truncation error is the same sign as and less than in absolute value the $(n+1)$th term.
\end{theorem}

\begin{example}
	Give a bound for the truncation error of using the first 10 terms of the Maclaurin series of $\ln{(1+x)}$ to approximate $\ln{2}$.
\end{example}
\begin{answer}
	The Maclaurin series for $\ln{(1+x)}$ is
	\begin{equation*}
		\ln{(1+x)} = x - \frac{x^2}{2} + \frac{x^3}{3} - \ldots = \sum_{k=0}^{\infty}{(-1)^k\frac{x^{k+1}}{k+1}}, \abs{x} \leq 1,
	\end{equation*}
	
	which is an alternating series.
	We see that to approximate $\ln{2}$, we'd use $x=1$, which is in the interval of convergence, so the series converges.
	The first missing term of the series is $1/11$.
	Let $s_{10}$ be the partial sum of the first 10 terms when $x=1$.
	By the Alternating Series Estimation Theorem,
	\begin{equation*}
		0 < \ln{2} - s_{10} < \frac{1}{11}.
	\end{equation*}
\end{answer}

\subsection{Taylor's Theorem}
Although the Alternating Series Estimation Theorem is useful for quickly bounding the error of alternating series, we'd like something that can apply more generally to all Taylor series.
\begin{theorem}[Taylor's Theorem]
	Let $f$ be a $k+1$ times differentiable function on an open interval $I$ containing $a$.
	Then for all $x$ in $I$,
	\begin{equation*}
		f(x) = f(a) + f^\prime(a)(x-a) + \frac{f^{\prime\prime}(a)}{2!}(x-a)^2 + \ldots + \frac{f^{(n)}(a)}{n!}(x-a)^n + R_n(x)
	\end{equation*}
	where
	\begin{equation*}
		\abs{R_n(x)} = \frac{\abs{f^{(n+1)}(c)}}{(n+1)!}\abs{x-a}^{n+1}
	\end{equation*}
	for some $c$ between $x$ and $a$.
\end{theorem}

We can use this theorem to find the maximum value of $\abs{R_n(x)}$ over some interval.
We can also be more precise about what it means for a Taylor series to converge to some function over some interval.
\begin{definition}
	Let $R_n(x)$ be the remainder of truncating after the degree $n$ term in the Taylor series for $f$ centered at $x=a$.
	If for all $x$ in some interval $I$ containing $a$,
	\begin{equation*}
		\lim_{n\to\infty}{R_n(x)} = 0,
	\end{equation*}
	then we say the Taylor series for $f$ at $x=a$ converges to $f$ on $I$.
\end{definition}

\begin{example}
	Show that the Maclaurin series for $\sin{x}$ converges to $\sin{x}$ for all real $x$.
\end{example}
\begin{answer}
	We need to find the remainder and show that in the limit it goes to 0 as $n$ grows large.
	By Taylor's Theorem,
	\begin{align*}
		\abs{R_n(x)} &= \frac{\abs{f^{(n+1)}(c)}}{(n+1)!}\abs{x-a}^{n+1} \\
		&\leq \frac{\abs{x}^{n+1}}{(n+1)!}.
	\end{align*}
	
	The numerator is an exponential function, while the denominator is a factorial function, so using $n^n$FEPL,
	\begin{align*}
		0 \leq \lim_{n\to\infty}{R_n(x)} &\leq \lim_{n\to\infty}{\frac{\abs{x}^{n+1}}{(n+1)!}} = 0. \\
		\lim_{n\to\infty}{R_n(x)} &= 0.
	\end{align*}
	
	So, the Maclaurin series converges to $\sin{x}$ for all real $x$.
\end{answer}

\subsubsection{Remainder Estimation Theorem}
Notice that we didn't have to actually find the value of $f^{(n+1)}(c)$.
We just had to find a suitable upper bound where the limit would still go to 0.
\begin{theorem}[Remainder Estimation Theorem]
	If there are positive constants $M$ and $r$ such that
	\begin{equation*}
		\abs{f^{(n+1)}(t)} \leq Mr^{(n+1)}
	\end{equation*}
	for all $t$ between $a$ and $x$, then $R_n(x)$ satisfies the inequality
	\begin{equation*}
		\abs{R_n(x)} \leq M\frac{r^{n+1}\abs{x-a}^{n+1}}{(n+1)!}.
	\end{equation*}
\end{theorem}

\begin{example}
	Give a maximum error bound for using $\ln{(1+x)} = x - x^2/2$ when $\abs{x} \leq 0.1$.
\end{example}
\begin{answer}
	We are using the second-order Taylor polynomial, so we need to find $R_2(x)$.
	By Taylor's Theorem,
	\begin{align*}
		\abs{R_2(x)} &= \frac{\abs{f^{(2+1)}(c)}}{(2+1)!}\abs{x-0}^{2+1}  = \frac{\abs{f^{(3)}(c)}}{3!}\abs{x}^3\\
		f^{(3)}(x) &= \dd{^3}{x^3}\ln{(1+x)} = \frac{2}{(1+x)^3}.
	\end{align*}
	
	On $-0.1 \leq x \leq 0.1$, $\abs{f^{(3)}(x)}$ is maximal at $(-0.1, \frac{2000}{729})$.
	\begin{equation*}
		\abs{R_2(x)} \leq \frac{\frac{2000}{729}}{6}\abs{x}^3.
	\end{equation*}
	
	On $-0.1 \leq x \leq 0.1$, $\abs{x}^3$ is maximal at $(-0.1, \frac{1}{1000})$.
	\begin{equation*}
		\abs{R_2(x)} \leq \frac{\frac{2000}{729}}{6}\frac{1}{1000} = \frac{1}{2187} \approx 4.572 \times 10^{-4}.
	\end{equation*}
	
	So, the error of approximating $\ln{(1+x)}$ with $x-x^2/2$ when $-0.1 \leq x \leq 0.1$ is at most $4.572 \times 10^{-4}$.
\end{answer}
\section{Convergence}
We have described what it means for a Taylor series to converge to a function over some interval.
What if we're given an infinite series that's not a function, just an infinite sum of numbers?
Can we still check if the series converges or diverges?
We'll develop several tests that we can apply to check for convergence or divergence.

\subsection{nth Term Test for Divergence}
\begin{lemma}
	Let $a_n$ be the nth term of a series $s$.
	If
	\begin{equation*}
		\lim_{n\to\infty}{a_n} \neq 0,
	\end{equation*}
	then the series diverges.
\end{lemma}
\begin{proof}
	Assume not.
	Let $a_n$ be the nth term and $s_n$ the nth partial sum of a convergent series $s$ whose terms do not tend to 0.
	Since $s$ converges to some value $L$, there exists some positive integer $m$ such that for all  $n > m$ and $\epsilon > 0$,
	\begin{equation*}
		\abs{s_n - L} < \epsilon.
	\end{equation*}
	We can also say the same for $s_{n+1}$.
	\begin{equation*}
		\abs{s_{n+1}-L} < \epsilon.
	\end{equation*}
	So, subtracting one inequality from the other,
	\begin{equation*}
		\abs{s_{n+1}-s_n} < 2\epsilon.
	\end{equation*}
	The difference between the two partial sums is just $a_{n+1}$.
	So,
	\begin{equation*}
		\abs{a_{n+1}} < 2\epsilon.
	\end{equation*}
	Since the terms of $s$ don't tend to 0, there exists some positive integer $h$ and real value $\delta > 0$ such that for all $n > h$,
	\begin{equation*}
		\abs{a_n} > \delta.
	\end{equation*}
	We can also say the same for $a_{n+1}$.
	\begin{equation*}
		\abs{a_{n+1}} > \delta.
	\end{equation*}
	Combining the two inequalities involving $a_{n+1}$, for all $n > \max{(m,h)}$,
	\begin{equation*}
		\delta < \abs{a_{n+1}} < 2\epsilon.
	\end{equation*}
	However, for $\epsilon \leq \delta/2$, the inequality creates a contradiction.
\end{proof}

Although the proof has to be a bit specific to cover the case of alternating series, the idea behind the test makes sense.
If you're adding on terms that don't get smaller in absolute value, then you can't "zero in" on a particular value and converge.

\begin{example}
	Show that the following series diverges:
	\begin{equation*}
		\sum_{i=0}^{\infty}{(-1)^n} = 1 - 1 + 1 - 1 + \ldots.
	\end{equation*}
\end{example}
\begin{answer}
	We see that
	\begin{equation*}
		\lim_{n\to\infty}{a_n} = \text{DNE} \neq 0.
	\end{equation*}
	
	So, by the nth Term Test for Divergence, the series diverges.
\end{answer}

\subsection{Geometric Series with $\abs{r} < 1$}
\begin{lemma}
	If $s$ is a geometric series with common ratio $r$, then $s$ converges if and only if $\abs{r} < 1$.
\end{lemma}

We already proved this when talking about power series and gave a formula to find its value.

\begin{example}
	Show that the following series diverges:
	\begin{equation*}
		\sum_{i=0}^{\infty}{\frac{2^n}{3}}.
	\end{equation*}	
\end{example}
\begin{answer}
	This is a geometric series with initial term $1/3$ and common ratio $r=2$.
	Since $\abs{r} > 1$, the series diverges.
\end{answer}

\subsection{P-Series Test}
\begin{lemma}
	The p-series
	\begin{equation*}
		\sum_{i=1}^{\infty}{\frac{1}{i^p}}
	\end{equation*}
	converges if and only if $p > 1$.
\end{lemma}

\begin{example}
	Show that the Harmonic Series diverges:
	\begin{equation*}
		\sum_{i=1}^{\infty}{\frac{1}{i}}.
	\end{equation*}
\end{example}
\begin{answer}
	
	This is a p-series with $p=1$.
	So, by the P-Series Test, the sum diverges.
\end{answer}

\subsection{Direct Comparison Test}
\begin{lemma}
	Let $s = \sum_{i=0}^{\infty}{a_i}$ be a series where all $a_i \geq 0$.
	$s$ converges if there exists some convergent series $c = \sum_{i=0}^{\infty}{c_i}$ and positive integer $m$ such that for all $n > m$,
	\begin{equation*}
		c_n \geq a_n.
	\end{equation*}
	$s$ diverges if there exists some divergent series $d = \sum_{i=0}^{\infty}{d_i}$ and positive integer $m$ such that for all $n > m$,
	\begin{equation*}
		a_n \geq d_n.
	\end{equation*}
\end{lemma}

\begin{example}
	Show that the following series converges:
	\begin{equation*}
		\sum_{i=0}^{\infty}{\frac{1}{2 + 3^i}}.
	\end{equation*}
\end{example}
\begin{answer}
We see that this series looks really similar to a geometric series with common ratio 1/3, just with an extra 2 in the denominator.
For all $n > 0$,
\begin{equation*}
	\frac{1}{2+3^n} \leq \frac{1}{3^n}.
\end{equation*}

So, by the Direct Comparison Test, the series converges.
\end{answer}

\begin{example}
	Show that the following series diverges:
	\begin{equation*}
		\sum_{i=0}^{\frac{1}{2+\sqrt{i}}}.
	\end{equation*}
\end{example}
\begin{answer}
	Although the sum looks like a p-series with $p=1/2$, we can use that series because our series has smaller terms.
	Instead, we can compare to a p-series where $p=1/2$.
	\begin{align*}
		0 \leq \frac{1}{n} &\geq \frac{1}{2+\sqrt{n}} \\
		0 \leq 2 + \sqrt{n} &\leq n \\
		n & \geq 4.
	\end{align*}
	
	So, for all $n \geq 4$, our series has larger values than the p-series with $p=1$.
	We know the p-series diverges by the P-Test, so by the Direct Comparison Test, our series also diverges.
\end{answer}

\subsection{Limit Comparison Test}
\begin{lemma}
	Let $a_n$ and $b_n$ be the nth terms of two series $a$ and $b$ that have all positive terms after some point.
	If
	\begin{equation*}
		\lim_{n\to\infty}{\frac{a_n}{b_n}}
	\end{equation*}
	converges to a finite value greater than 0, then $a$ and $b$ either both converge or both diverge.
	If the limit converges to 0 and $b$ converges, then $a$ also converges.
	If the limit goes to $\infty$ and $b$ diverges, then $a$ also diverges.
\end{lemma}

A common tactic is to select one of $a$ or $b$ to be a geometric or p-series.

\begin{example}
	Show that the following series diverges:
	\begin{equation*}
		\sum_{i=2}^{\infty}{\frac{2i}{i^2-i+a}}.
	\end{equation*}
\end{example}
\begin{answer}
	Looking at this rational function, we see a degree 1 term in the numerator and a degree 2 term in the denominator.
	So, we might expect that the series behaves similarly to $1/n$.
	\begin{equation*}
		\lim_{n\to\infty}{\frac{\frac{2n}{n^2-n+a}}{\frac{1}{n}}} = \lim_{n\to\infty}{\frac{2n^2}{n^2-n+a}} = 2.
	\end{equation*}
	
	Since the limit converges to a finite value greater than 0, and we know that $1/n$ diverges by the P-Test, then the series must also diverge by the Limit Comparison Test.
\end{answer}

\begin{example}
	Show that the following series converges:
	\begin{equation*}
		\sum_{i=1}^{\infty}{\frac{1}{2^i - 1}}.
	\end{equation*}
\end{example}
\begin{answer}
	This series looks similar to the geometric series $1/2^n$.
	\begin{equation*}
		\lim_{n\to\infty}{\frac{\frac{1}{2^n}}{\frac{1}{2^n-1}}} = \lim_{n\to\infty}{\frac{2^n - 1}{2^n}} = 1.
	\end{equation*}
	
	Since the limit converges to a finite value greater than 0, and we know that $1/2^n$ converges because it's a geometric series with $r=1/2$, then the series must also converge the the Limit Comparison Test.
\end{answer}

\subsection{Integral Test}
\begin{lemma}
	Let $a$ be a sequence of positive terms where $a_n = f(n)$.
	If $f$ is continuous, positive after some $m$, and decreasing, then the series
	\begin{equation*}
		\sum_{i=m}^{\infty}{a_i}
	\end{equation*}
	converges if and only if the integral
	\begin{equation*}
		\int_{m}^{\infty}{f(x)\d{x}}
	\end{equation*}
	converges.
\end{lemma}

\begin{example}
	Show that the following series diverges:
	\begin{equation*}
		\sum_{i=1}^{\infty}{\frac{2i}{i^2+1}}.
	\end{equation*}
\end{example}
\begin{answer}
	The function $f(x)=2x/(x^2+1)$ is continuous, positive for all $x \geq 1$, and decreasing.
	\begin{equation*}
		\int_{1}^{\infty}{\frac{2x}{x^2+1}\d{x}} = (u = x^2+1) \int_{2}^{\infty}{\frac{\d{u}}{u}} = \ln{u}\biggr\rvert_{2}^{\infty} = \text{diverges}.
	\end{equation*}
	
	So, by the Integral Test, the series also diverges.
\end{answer}

\begin{example}
	Show that the following series converges:
	\begin{equation*}
		\sum_{i=1}^{\infty}{\frac{1}{i^2+1}}.
	\end{equation*}
\end{example}
\begin{answer}
	The function $f(x)=1/(x^2+1)$ is continuous, positive for all $x \geq 1$, and decreasing.
	\begin{equation*}
		\int_{1}^{\infty}{\frac{\d{x}}{x^2+1}} = \arctan{x}\biggr\rvert_{1}^{\infty} = \frac{\pi}{2} - \frac{\pi}{4} = \frac{\pi}{4}.
	\end{equation*}
	
	So, by the Integral Test\footnote{We omitted evaluating the improper integral with limits, but the value is what you would get from doing that.} the series also converges.
\end{answer}

\subsection{Ratio Test}
\begin{lemma}
	Let $a$ be a series with only positive terms after some index.
	If the limit
	\begin{equation*}
		\lim_{n\to\infty}{\frac{a_{n+1}}{a_n}}
	\end{equation*}
	is less than 1, then the series converges.
	If the limit is greater than 1, then the series diverges.
	If the limit is equal to 1, then the test is inconclusive.
\end{lemma}

\begin{example}
	State whether the following series converges or diverges:
	\begin{equation*}
		\sum_{i=1}^{\infty}{\frac{i\ln{i}}{2^i}}.
	\end{equation*}
\end{example}
\begin{answer}
	Taking the limit of the ratio of subsequent terms,
	\begin{equation*}
		\lim_{n\to\infty}{\frac{\frac{(n+1)\ln{(n+1)}}{2^{n+1}}}{\frac{n\ln{n}}{2^n}}} = \lim_{n\to\infty}{\frac{(n+1)\ln{n+1}}{2n\ln{n}}} = \lim_{n\to\infty}{\frac{n+1}{2n}} = \frac{1}{2}.
	\end{equation*}
	
	Since the limit is less than 1, the series converges by the Ratio Test.
\end{answer}

\subsection{nth Root Test}
\begin{lemma}
	Let $a$ be a series with all positive terms after some index.
	If the limit
	\begin{equation*}
		\lim_{n\to\infty}{\sqrt[n]{a_n}}
	\end{equation*}
	is less than 1, then the series converges.
	If the limit is greater than 1, the series diverges.
	If the limit is equal to 1, the test is inconclusive.
\end{lemma}

This test is most useful for series that look like geometric series, but the common ratio is not a constant.

\begin{example}
	State whether the following series converges or diverges:
	\begin{equation*}
		\sum_{i=1}^{\infty}{\left(\frac{i}{2i-1}\right)^i}.
	\end{equation*}
\end{example}
\begin{answer}
	Taking the limit of the nth root,
	\begin{equation*}
		\lim_{n\to\infty}{\sqrt[n]{\left(\frac{n}{2n-1}\right)^n}} = \lim_{n\to\infty}{\frac{n}{2n-1}} = \frac{1}{2}.
	\end{equation*}
	
	Since the limit is less than 1, the series converges by the nth Root Test.
\end{answer}

\subsection{Alternating Series Test}
All of the previous 8 tests have tested whether a series converges absolutely.
That is, all terms in the series could be made positive and the series would still converge or diverge.
There are some series that do converge but don't converge absolutely.
We say that these series converge conditionally.
You've already seen  few, like the formula for $\pi/4$ using the Maclaurin series for $\arctan{x}$.
All series that converge absolutely also converge conditionally.
Series that converge conditionally do so when they pass the following test.

\begin{lemma}
	The series
	\begin{equation*}
		\sum_{i=0}^{\infty}{(-1)^iu_i}
	\end{equation*}
	converges if all of the following conditions are satisfied.
	\begin{enumerate}
		\item All $u_i$ are positive.
		\item There exists an integer $m$ such that for all $n > m$, $u_{n+1} \leq u_n$.
		\item The series of $u$'s pass the nth Term Test.
	\end{enumerate}
\end{lemma}

\begin{example}
	State whether the following series converges or diverges:
	\begin{equation*}
		\sum_{i=1}^{\infty}{(-1)^i\frac{1}{i}}.
	\end{equation*}
\end{example}
\begin{answer}
	\begin{enumerate}
		\item All the terms $1, 1/2, 1/3, \ldots$ are positive.
		\item All terms are less than the previous term.
		\item The terms tend to 0, passing the nth Term Test.
	\end{enumerate}
	
	So, by the Alternating Series Test, the series converges conditionally\footnote{You've already seen this series too. It converges to $\ln{2}$.}.
	Note that the series does not converge absolutely because it fails the P-Test.
\end{answer}

\subsubsection{Riemann Rearrangement Theorem}
\begin{theorem}[Riemann Rearrangement Theorem]
	If an infinite series is conditionally convergent, but not absolutely convergent, then its terms can be rearranged to form a divergent series or to converge to any constant.
\end{theorem}

This result may seem counter intuitive, given that we know addition is associative and commutative.
However, the result stems from the fact that we defined convergence of an infinite series at the limit of partial sums.
If we rearrange terms of the series, the partial sums and their limits can change, meaning the limit of partial sums can diverge or converge to some other value.
		
	\appendix
	
	\backmatter
\end{document}

