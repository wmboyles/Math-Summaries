\section{Separable Differential Equations}
\noindent
The most basic approach for solving a 1st-order differential equation is simply integrating both sides. You're probably already familiar with this technique from taking indefinite integrals. This approach only works when the independent and dependent variables can be arranged on different sides of the equation. We'll formalize this idea with separability.

\begin{definition}
	A 1st order ODE is separable if it can be written in the form
	\begin{equation*}
		\frac{\mathrm{d} y}{\mathrm{d} x} = f(x)g(y)
	\end{equation*}
\end{definition}

\noindent
Separable equations provide a special way of solving them that can be useful. If we treat the derivative like a fraction (which is not formally allowed but OK here),
\begin{equation*}
	\frac{\mathrm{d} y}{\mathrm{d} x} = f(x)g(y) \implies \frac{\mathrm{d} y}{g(y)} = f(x) \mathrm{d}x \implies \int{\frac{\mathrm{d} y}{g(y)}} = \int{f(x) \mathrm{d}x}
\end{equation*}
We then have a function in $y$ on the left and a function in $x$ on the right, meaning we only have to solve for $y$ to get the solution.

\input{./1stOrderLinearODE/separability/separability_example.tex}