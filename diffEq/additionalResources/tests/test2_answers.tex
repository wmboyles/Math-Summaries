\subsection{Test 2 Answers}
\begin{enumerate}[label=\arabic*.]
	\item
		For both (a) and (b), we'll need to compute the general solution to make sure any terms that we'd guess as part of the particular solution aren't already part of the general solution. The auxiliary equation and its roots are
		\begin{equation*}
			r^2 + 10r + 25 \implies r = -5 \text{ (double root)}
		\end{equation*}
		So, the general solution is
		\begin{equation*}
			C_1e^{-5t} + C_2te^{5t}
		\end{equation*}
		\begin{enumerate}[label=(\alph*)]
			\item
				We have a trig term and a linear term times a trig term, so the particular solution has the form
				\begin{equation*}
					y_p = A\cos{(-5t)} + B\sin{(-5t)} + Ct\cos{(-5t)} + Dt\sin{(-5t)}
				\end{equation*}
				So, without solving for $A$, $B$, $C$, and $D$, the general form of a particular solution is
				\begin{equation*}
					C_1e^{-5t} + C_2te^{5t} + A\cos{(-5t)} + B\sin{(-5t)} + Ct\cos{(-5t)} + Dt\sin{(-5t)}
				\end{equation*}
			\item
				We have a linear term time $e^{-5t}$. However, up to linear terms are already represented in the general solution, so we need to include another factor of $t$. So, the particular solution has the form
				\begin{equation*}
					y_p = At^2e^{-5t}
				\end{equation*}
				So, without solving for $A$, the general form of a particular solution is
				\begin{equation*}
					C_1e^{-5t} + C_2te^{5t} + At^2e^{-5t}
				\end{equation*}
		\end{enumerate}
	\item
		Finding $p(\lambda)$,
		\begin{equation*}
			p(\lambda) = \det\begin{bmatrix}
				1 & 0 & 6 \\
				3 & 1 & 3 \\
				-3 & 3 & -8
			\end{bmatrix} = -\lambda^3 - 6\lambda^2 + 6\lambda + 55
		\end{equation*}
		So, $p(\lambda) = 0$ when
		\begin{equation*}
			\lambda = -5, \frac{-1 \pm 3\sqrt{5}}{2}
		\end{equation*}
		We'll find the eigenvector for $\lambda = -5$.
		\begin{equation*}
			\left[
				\begin{array}{ccc|c}
					6 & 0 & 6 & 0\\
					3 & 6 & 3 & 0\\
					-3 & 3 & 3 & 0
				\end{array}
			\right] \to \left[
				\begin{array}{ccc|c}
					1 & 0 & 1 & 0 \\
					0 & 1 & 0 & 0 \\
					0 & 0 & 0 & 0
				\end{array}
			\right] \implies \vec{v} = t\begin{bmatrix}
				-1 \\
				0 \\
				1
			\end{bmatrix}
		\end{equation*}
	\item
		The instructions say only do one, but we'll do both for the answers.
		\begin{enumerate}[label=(\alph*)]
			\item
				The auxiliary equation and its roots are
				\begin{equation*}
					r^2 - 5r + 6 = 0 \implies r = 2, 3
				\end{equation*}
				So, the homogeneous solution is
				\begin{equation*}
					y_h = C_1e^{2t} + C_2e^{3t}
				\end{equation*}
				We use the method of undetermined coefficients to find $y_p$. We'll guess that $y_p$ has the form
				\begin{equation*}
					y_p = Ae^{4t} + Be^{t}
				\end{equation*}
				Solving for $A$ and $B$\footnote{The algebra of solving for $A$ and $B$ have been omitted for brevity.},
				\begin{equation*}
					y_p'' - 5y_p' + 6y_p = 6e^{4t} - 10e^t \implies A = 3, B = -5
				\end{equation*}
				So, we can write the particular solution
				\begin{equation*}
					y_p = 3e^{4t} - 5e^{t}
				\end{equation*}
				So, the general solution is
				\begin{equation*}
					y = C_1e^{2t} + C_2e^{3t} + 3e^{4t} - 5e^{t}
				\end{equation*}
			\item
				Although this is a second-order equation, we'll do the version of variation of parameters that also works for higher orders too. First, we need to find our fundamental solutions that are part of the homogeneous solution by finding the roots of the auxiliary equation.
				\begin{equation*}
					r^2 + 9 = 0 \implies r = \pm 3i
				\end{equation*}
				So, the homogeneous solution is
				\begin{equation*}
					y_h = C_1\cos{(3t)} + C_2\sin{(3t)}
				\end{equation*}
				and the fundamental solutions are
				\begin{align*}
					y_1 = \cos{(3t)} \\
					y_2 = \sin{(3t)}
				\end{align*}
				So, the Wronskian matrix and its determinant are
				\begin{equation*}
					\left[W\right] = \begin{bmatrix}
						\cos{(3t)} & \sin{(3t)} \\
						-3\sin{(3t)} & 3\cos{(3t)}
					\end{bmatrix} \text{, } W = 3\cos^2{(3t)} + 3\sin^2{(3t)} = 3
				\end{equation*}
				The sub-matrices are
				\begin{align*}
					W_1 &= \det\begin{bmatrix}
						0 & \sin{(3t)} \\
						1 & 3\cos{(3t)}
					\end{bmatrix} = -\sin{(3t)} \\
					 W_2 &= \det\begin{bmatrix}
						\cos{(3t)} & 0 \\
						-3\sin{(3t)} & 1
					\end{bmatrix} = \cos{(3t)}
				\end{align*}
				Solving for $v_1$ and $v_2$,
				\begin{align*}
					v_1 &= \int{\frac{-\csc{(3t)}\sin{(3t)}}{3} \mathrm{d}t} = \frac{-1}{3}t + C_1 \\
					v_2 &= \int{\frac{\csc{(3t)}\cos{(3t)}}{3} \mathrm{d}t} = \frac{1}{9}\ln{\abs{\sin{(3t)}}} + C_2
				\end{align*}
				Solving for $y$,
				\begin{align*}
					y &= \cos{(3t)}\left(\frac{-1}{3}t + C_1\right) + \sin{(3t)}\left(\frac{1}{9}\ln{\abs{\sin{(3t)}}} + C_2\right) \\
					&= C_1\cos{(3t)} + C_2\sin{(3t)} - \frac{1}{3}t\cos{(3t)} + \frac{1}{9}\sin{(3t)}\ln{\abs{\sin{(3t)}}}
				\end{align*}
		\end{enumerate}
		\item
			Below is a diagram that depicts the situation.
			
			\begin{center}
				\begin{tikzpicture}
					\tikzstyle{spring}=[thick,decorate,decoration={zigzag, pre length=0.5cm, post
						length=0.5cm, segment length=0.33cm, amplitude=0.25cm}]
					
					\tikzstyle{ground}=[fill, pattern=north east lines, draw=none, minimum
					width=1cm, minimum height=0.5cm]
					
					\fill[pattern = north east lines] (0, 0) rectangle (0.5cm, 2.5cm);
					\draw[thick] (0, 0) -- (8cm, 0);
					\node (m) at (5cm, 1cm) [draw, thick, minimum width=2cm, minimum height=2cm] {$m = 1$};
					\draw[spring] (0.5cm, 1cm) -- (m);
					\node (k) at (1.5cm, 1.75cm) {$k = 8$};
					\node (b) at (5cm, 2.5cm) {$b = 6$};
					\node (ini) at (10cm, 1cm) {$\begin{cases} y'(0) = 1 \\ F_\text{ext}(t) = 8\sin{(2t)} \end{cases}$};
					\draw[dashed] (5cm, -0.25cm) -- (5cm, 0.25cm);
					\node (y) at (5cm, -0.5cm) {$y = 0$};
				\end{tikzpicture}
			\end{center}
			
			\begin{enumerate}[label=(\alph*)]
				\item
					Our IVP to model this is
					\begin{equation*}
						\begin{cases}
							y'' + 6y' + 8y = 8\sin{(2t)} \\
							y'(0) = 1 \\
							y(0) = 0
						\end{cases}
					\end{equation*}
					Solving the auxiliary equation,
					\begin{equation*}
						r^2 + 6r + 8 = 0 \implies r = -4, -2
					\end{equation*}
					So, our homogeneous solution is
					\begin{equation*}
						y_h = C_1e^{-4t} + C_2e^{-2t}
					\end{equation*}
					We'll use the method of undetermined coefficients to find $y_p$. We'll guess that $y_p$ has the form
					\begin{equation*}
						y_p = A\cos{(2t)} + B\sin{(2t)}
					\end{equation*}
					Solving for $A$ and $B$\footnote{The algebra of solving for $A$ and $B$ has been omitted for brevity.},
					\begin{equation*}
						y_p'' + 6y_p' + 8y_p = 8\sin{(2t)} \implies A = \frac{-3}{5}, B = \frac{1}{5}
					\end{equation*}
					So, our particular solution is
					\begin{equation*}
						y_p = \frac{-3}{5}\cos{(2t)} + \frac{1}{5}\sin{(2t)}
					\end{equation*}
					and our general solution is
					\begin{equation*}
						y = C_1e^{-4t} + C_2e^{-2t} - \frac{3}{5}\cos{(2t)} + \frac{1}{5}\sin{(2t)}
					\end{equation*}
					Plugging in our initial condition to solve for $C_1$ and $C_2$\footnote{The algebra of solving for $C_1$ and $C_2$ has been omitted for brevity.},
					\begin{equation*}
						y'(0) = 1, y(0) = 0 \implies C_1 = \frac{-9}{10}, C_2 = \frac{3}{2}
					\end{equation*}
					So, our solution to the IVP is
					\begin{equation*}
						y(t) = \frac{-9}{10}e^{-4t} + \frac{3}{2}e^{-2t} - \frac{3}{5}\cos{(2t)} + \frac{1}{5}\sin{(2t)} \text{ m}
					\end{equation*}
				\item
					As $t$ grows large, the exponential terms will decrease to 0 and have minimal effect. So, the steady-state solution $y_{ss}$ is just the terms with $\cos$ and $\sin$.
					\begin{equation*}
						y_{ss}(t) = \frac{-3}{5}\cos{(2t)} + \frac{1}{5}\sin{(2t)} \text{ m}
					\end{equation*}
				\item
					\begin{equation*}
						y_{ss}(t) = \sqrt{\frac{2}{5}}\cos{\left(2x + \arctan{\left(\frac{1}{3}\right)} + \pi\right)} \text{ m}
					\end{equation*}
					The amplitude, frequency, period, and phase shift are
					\begin{align*}
						A &= \sqrt{\frac{2}{5}} \text{ m} \\
						f &= \frac{1}{\pi} \text{ Hz} \\
						T &= \pi \text{ secs} \\
						\phi &= \arctan{\left(\frac{1}{3}\right)} + \pi
					\end{align*}
			\end{enumerate}
		\item
			The instructions say only do one, but we'll do both for the answers.
			\begin{enumerate}[label=(\alph*)]
				\item
					\begin{enumerate}[label=(\roman*)]
						\item
							The system in matrix form is
							\begin{equation*}
								\vec{x}' = \begin{bmatrix}
									2 & -3 \\
									1 & -2
								\end{bmatrix}\vec{x} \text{, } \vec{x}(0) = \begin{bmatrix}
									2 \\
									3
								\end{bmatrix}
							\end{equation*}
						\item
							We'll check the first vector.
							\begin{equation*}
								\begin{bmatrix}
									3e^t \\
									e^t
								\end{bmatrix} = \begin{bmatrix}
									2 & -3 \\
									1 & -2
								\end{bmatrix} \begin{bmatrix}
									3e^t \\
									e^t
								\end{bmatrix} = \begin{bmatrix}
									3e^t \\
									e^t
								\end{bmatrix}
							\end{equation*}
							Next we'll check the second vector.
							\begin{equation*}
								\begin{bmatrix}
									-e^{-t} \\
									-e^{-t}
								\end{bmatrix} = \begin{bmatrix}
									2 & -3 \\
									1 & -2
								\end{bmatrix} \begin{bmatrix}
									e^{-t} \\
									e^{-t}
								\end{bmatrix} = \begin{bmatrix}
									-e^{-t} \\
									-e^{-t}
								\end{bmatrix} 
							\end{equation*}
							So, both vectors are solutions to the homogeneous equation.
						\item
							Since we know two linearly independent solutions from (ii), we can write the general solution as
							\begin{equation*}
								\vec{x} = C_1 \begin{bmatrix}
									3e^t \\
									e^t
								\end{bmatrix} + C_2 \begin{bmatrix}
									-e^{-t} \\
									-e^{-t}
								\end{bmatrix} 
							\end{equation*}
							Applying the initial conditions\footnote{The algebra of solving for $C_1$ and $C_2$ has been omitted for brevity},
							\begin{equation*}
								\vec{x}(0) = \begin{bmatrix}
									2 \\
									3
								\end{bmatrix} \implies C_1 = \frac{-1}{2}, C_2 = \frac{-7}{2}
							\end{equation*}
							So, the solution to the IVP is
							\begin{equation*}
								\vec{x} = \frac{-1}{2} \begin{bmatrix}
									3e^t \\
									e^t
								\end{bmatrix} - \frac{7}{2} \begin{bmatrix}
									-e^{-t} \\
									-e^{-t}
								\end{bmatrix} 
							\end{equation*}
					\end{enumerate}
				\item
					Since the eigenvalues are a complex conjugate pair, we only need to consider one eigenvalue to find both corresponding eigenvectors. We'll use $\lambda = -2 + i$.
					\begin{equation*}
						\left[
							\begin{array}{cc|c}
								1-i & 2 & 0 \\
								-1 & -1-i & 0
							\end{array}
						\right] \to \left[
							\begin{array}{cc|c}
								1 & 1-i & 0 \\
								0 & 0 & 0
							\end{array}
						\right] \to t\left(\begin{bmatrix}
							-1 \\
							1
						\end{bmatrix} + \begin{bmatrix}
							1 \\
							0
						\end{bmatrix}\right)
					\end{equation*}
					So, the two eigenvectors are
					\begin{equation*}
						\begin{bmatrix}
						-1 \\
						1
						\end{bmatrix}, \begin{bmatrix}
						1 \\
						0
						\end{bmatrix}
					\end{equation*}
					Remembering our solution form for complex eigenvalues, we get the solution
					\begin{equation*}
						\vec{x} = C_1e^{-2t}\left(\cos{(t)}\begin{bmatrix}
							-1 \\
							1
						\end{bmatrix} - \sin{(t)}\begin{bmatrix}
							1 \\
							0
						\end{bmatrix}\right) + C_2e^{-2t}\left(\cos{(t)}\begin{bmatrix}
							1 \\
							0
						\end{bmatrix} + \sin{(t)}\begin{bmatrix}
							-1 \\
							1
						\end{bmatrix}
						\right)
					\end{equation*}
			\end{enumerate}
		\item
			We'll use the method of undetermined coefficients for systems to find $\vec{x_p}$. We'll guess that $\vec{x_p}$ has the form
			\begin{equation*}
				\vec{x_p} = \vec{a}t + \vec{b}
			\end{equation*}
			Solving for $\vec{a}$ and $\vec{b}$\footnote{The algebra of solving for $\vec{a}$ and $vec{b}$ has been omitted for brevity.},
			\begin{equation*}
				\vec{x_p}' = \begin{bmatrix}
					2 & -1 \\
					3 & -2
				\end{bmatrix}\vec{x_p} \implies \vec{a} = \begin{bmatrix}
					-1 \\
					0
				\end{bmatrix}, \vec{b} = \begin{bmatrix}
					1 \\
					3
				\end{bmatrix}
			\end{equation*}
			So, we have our solution for $\vec{x_p}$,
			\begin{equation*}
				\vec{x_p} = t\begin{bmatrix}
					-1 \\
					0
				\end{bmatrix} + \begin{bmatrix}
					1 \\
					3
				\end{bmatrix}
			\end{equation*}
			The general solution is thus
			\begin{equation*}
				\vec{x} = C_1e^{t}\begin{bmatrix}
					1 \\
					1
				\end{bmatrix} + C_2e^{-t}\begin{bmatrix}
					1 \\
					3
				\end{bmatrix} + t\begin{bmatrix}
					-1 \\
					0
				\end{bmatrix} + \begin{bmatrix}
					1 \\
					3
				\end{bmatrix} 
			\end{equation*}
\end{enumerate}