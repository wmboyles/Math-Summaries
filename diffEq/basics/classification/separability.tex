\subsection{Separability}
\begin{definition}
	A separable differential is a differential equation that can be written in the form
	\begin{equation*}
		f^\prime(x) = g(x)h(f(x))
	\end{equation*}
\end{definition}

\noindent
None of the equations listed are separable. An example of a separable equation that models constrained population growth is
\begin{equation*}
	\frac{\mathrm{d} P}{\mathrm{d} t} = kP\left(1 - \frac{P}{K}\right)
\end{equation*}
\noindent
In this case the function $g$ is just $k$.\\

\noindent
Separable equations provide a special way of solving them that can be useful. We may also try assuming that the solution is separable when looking for solutions, like in the heat equation.\\