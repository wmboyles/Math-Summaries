\section{Higher Order Heterogeneous Equations}
\noindent
If we modify our equation for free vibrations to have a function as the net force, then out equations becomes
\begin{equation*}
	my'' + by' + ky = b(x)
\end{equation*}
Depending on the form of $b(x)$, like a $\sin$ or $\cos$ curve or an exponential, we might be able to guess the form of the solution. However, there is an important thing to keep in mind.

\begin{theorem}
	If $f(x)$ is a solution to the above equation, and $g(x)$ is a solution to the homogeneous form of the equation $b(x) = 0$, then $f(x) + g(x)$ is also a solution to the above equation.
\end{theorem}
\begin{proof}
	If $f(x) + g(x)$ is a solution, then
	\begin{equation*}
		m\left(f(x) + g(x)\right)'' + b\left(f(x) + g(x)\right)' + k\left(f(x) + g(x)\right) = b(x)
	\end{equation*}
	Rearranging,
	\begin{equation*}
		\left(mf''(x) + bf'(x) + kf(x)\right) + \left(mg''(x) + bg'(x) + kg(x)\right) = 0
	\end{equation*}
	Using the definitions of $f(x)$ and $g(x)$,
	\begin{equation*}
		b(x) + 0 = b(x)
	\end{equation*}
\end{proof}

\noindent
$f(x)$ and $g(x)$ actually have special meanings in terms of solving these higher-order equations.
\begin{definition}
	$f(x)$ is called the particular solution to the heterogeneous equation, and $g(x)$ is called a homogeneous solution.
\end{definition}

\subsection{Method of Undetermined Coefficients}
\noindent
The type of equation we're trying to solve is a heterogeneous linear ODE with constant coefficients.
These have the form
\begin{equation*}
	a_ny^{(n)} + a_{n-1}y^{(n-1)} + \ldots + a_1y' + a_0y = b(x).
\end{equation*}
We will assume that the solution has the form $y = y_h + y_p$, where $y_h$ is the general solution to the homogeneous equation ($b(x) = 0$), and $y_p$ is the particular solution.\\

\noindent
We know how to solve for $y_h$ exactly without guessing.
However, we will make a guess for the form of $y_p$ based on the form of $b(x)$ and the form of $y_h$ using the rules described in the below examples.
For each term in our guess for $y_p$, we will solve for a constant.\\

\noindent
When solving, we'll first solve the homogeneous equation to find the general homogeneous solution $y_h$. Then, we'll guess a form for the particular solution $y_p$ based on the form of $b(x)$ and solve. Finally, we'll add these two solutions together to get the full general solution. We'll see that the constants come from $y_h$ and not $y_p$.

\begin{example}
	Find the general solution to the following equation.
	\begin{equation*}
		y'' + 2y' + y = 27e^{2x}
	\end{equation*}
\end{example}
\noindent
First, we'll solve the homogeneous equation
\begin{equation*}
	y'' + 2y' + y = 0
\end{equation*}
Extracting the auxiliary equation and finding the roots,
\begin{equation*}
	r^2 + 2r + 1 = \left(r+1\right)^2 \implies r = -1 \left(\text{double root}\right)
\end{equation*}
So, our general solution is
\begin{equation*}
	y_h = e^{-x}\left(C_1 + C_2x\right)
\end{equation*}
Since $b(x)$ is an exponential with a power of $2x$, it's safe guess to say that the particular solution is also an exponential with a power of $2x$. So, we'll guess that $y_p = Ae^{2x}$ and solve for $A$.
\begin{equation*}
	\left(Ae^{2x}\right)'' + 2\left(Ae^{2x}\right)' + Ae^{2x} = 2te^{2x}
\end{equation*}
\begin{equation*}
	A\left(4e^{2x} + 4e^{2x} + e^{2x}\right) = 27e^{2x} \implies A = 3
\end{equation*}
So,
\begin{equation*}
	y_p = 3e^{2x}
\end{equation*}
Putting $y_h$ and $y_p$ together,
\begin{equation*}
	y = y_h + y_p = e^{-x}\left(C_1 + C_2x\right) + 3e^{2x}
\end{equation*}

\noindent
There are a couple of catches we need to think about with it comes to guessing the form of the particular solution.
\begin{example}
	Find the general solution to the following equation.
	\begin{equation*}
		y'' + 2y' + y = 2e^{-x}
	\end{equation*}
\end{example}
\noindent
We already know from the previous example what $y_h$ is.
\begin{equation*}
	y_h = C_1e^{-x} + C_2xe^{-x}
\end{equation*}
However, even though $b(x)$ is an exponential with power $-x$, guessing that $y_p$ is of the form $Ae^{-x}$ won't work, since that is already covered in $y_h$. So, we instead include factors of $x$ until we hit a factor not already covered by $y_h$. In this case, we need up to $x^2$, so we guess that $y_p = Ax^2e^{-x}$.
\begin{equation*}
	y_p'' + 2y_p'+ y_p = 2Ae^{-x} = 2e^{-x} \implies A = 1
\end{equation*}
So, the general solution is
\begin{equation*}
	y = y_h + y_p = C_1e^{-x} + C_2xe^{-x} + x^2e^{-x}
\end{equation*}

\noindent
For certain forms of $b(x)$, like $\sin{x}$ or $\cos{x}$, our guess for $y_p$ will have multiple terms. We also need to make sure that these terms aren't already in $y_h$ and include factors of $x$.
\begin{example}
	Find the general solution to the following equation
	\begin{equation*}
		y'' + 4y = 8\cos{(2t)}
	\end{equation*}
	given that $y_h = C_1\cos{(2t)} + C_2\sin{(2t)}$.
\end{example}
\noindent
When $b(x)$ has a $\sin$ or $\cos$ term in it, we need our guess for $y_p$ to include both a $\sin$ and $\cos$ part in it, meaning there are two unknowns we'll have to solve for. However, since $y_h$ already has these $\sin$ and $\cos$ terms, we need to include an extra factor of $x$. So, our guess is that $y_p = Ax\cos{(2x)} + Bx\sin{(2x)}$.
\begin{equation*}
	y_p'' + 4y_p = -4A\sin{(2x)} + 4B\cos{(2x)} = 8\cos{(2t)} \implies A = 0 \text{, } B = 2
\end{equation*}
Note how the $\sin$ and $\cos$ terms that have a factor of $x$ cancel each other out. This is expected since $b(x)$ does not have any terms with a factor of $x$.\\

\noindent
So, the general solution is
\begin{equation*}
	y = y_h + y_p = C_1\cos{(2t)} + C_2\sin{(2t)} + 2x\sin{(2x)}
\end{equation*}

\noindent
If $b(x)$ has multiple terms, we need to include each term fully in our guess. For times when $b(x)$ has a factor that is a polynomial of degree $n$, our guess will also have a factor that is a polynomial of degree $n$ and $n$ coefficients to solve for.
\begin{example}
	Find the general solution to the following equation
	\begin{equation*}
		y'' - 3y' - 4y = 4x^2 - 1
	\end{equation*}
	given that $y_h = C_1e^{-x} + C_2e^{4x}$.
\end{example}
\noindent
Since $b(x)$ is a degree 2 polynomial, we'll guess that $y_p = Ax^2 + Bx + C$.
\begin{equation*}
	y_p'' - 3y_p' - 4y_p = x^2(-4A) + x(-6A-4B) + (2A-3B-4C) = 4x^2 - 1
\end{equation*}
So, we have a system of linear equations,
\begin{equation*}
	\begin{cases}
		-4A = 4 \\
		-6A - 4B = 0 \\
		2A - 3B - 4C = -1 \\
	\end{cases} \implies \begin{cases}
		A = -1 \\
		B = 3/2 \\
		C = -11/8
	\end{cases}
\end{equation*}
So, our general solution is
\begin{equation*}
	y = C_1e^{-x} + C_2e^{4x} - x^2 + \frac{3}{2}x - \frac{11}{8}
\end{equation*}
\subsection{Variation of Parameters}
\noindent
Although the method of undetermined coefficients is useful and relatively quick because it is algebra-based, it cannot solve many equations, even simple-looking second order equations, like
\begin{equation*}
	y'' + y = \csc{x}.
\end{equation*}
The method also requires guessing, meaning for very complicated forms of $b(x)$, things can get very messy.\\

\noindent
Instead, we'll look at a more rigorous, calculus-based, approach developed by Lagrange called ``variation of parameters.''
We'll first see how to apply the method to 2nd order linear ODEs with constant coefficients, like forced vibrations, and then we'll extend the method to order $n$.

\subsubsection{Second Order Variation of Parameters}
\noindent
We'll modify our second order equation of have a 1 as the coefficient of the $y^{\prime\prime}$ term by dividing to get an equation of the form
\begin{equation*}
	y^{\prime\prime} + py^\prime + qy = g(x)
\end{equation*}
Just like for undetermined coefficients, we'll find homogeneous and particular solutions $y = y_h + y_p$. Since the equation is second-order, the solution to the homogeneous equation will yield two fundamental solutions $y_1$ and $y_2$ where $y_h = C_1y_1 + C_2y_2$.\\

\noindent
So, we can write $y$ as
\begin{equation*}
	y(x) = A(x)y_1 + B(x)y_2
\end{equation*}
where
\begin{equation*}
	\begin{cases}
		A^\prime y_1 + B^\prime y_2 = 0 \\
		A^\prime y_1^\prime + B^\prime y_2^\prime = g(x)
	\end{cases}
\end{equation*}
We will then solve this system to solve for $A^\prime$ and $B^\prime$ and integrate.

\ifodd\includeHigherOrderExamples\begin{example}
	Find the general solution to the following equation
	\begin{equation*}
		y^{\prime\prime} + y = \csc{x}
	\end{equation*}
	given that $y_h = C_1\cos{x} + C_2\sin{x}$.
\end{example}
\noindent
$y_h$ gives us our two fundamental solutions
\begin{equation*}
	\begin{cases}
		y_1 = \cos{x} \\
		y_2 = \sin{x}
	\end{cases}
\end{equation*}
So, our system of equations is
\begin{equation*}
	\begin{cases}
		A^\prime\cos{x} + B^\prime\sin{x} = 0 \\
		-A^\prime\sin{x} + B^\prime\cos{x} = \csc{x}
	\end{cases} \to \begin{cases}
		A^\prime\cos{x} + B^\prime\sin{x} = 0 \\
		-A^\prime\cos{x} + B^\prime\frac{\cos^{2}{x}}{\sin{x}} = \frac{\cos{x}}{\sin^{2}{x}}
	\end{cases}
\end{equation*}
So, 
\begin{equation*}
	B^\prime\frac{1}{\sin{x}} = \frac{\cos{x}}{\sin^{2}{x}} \implies B^\prime = \frac{\cos{x}}{\sin{x}} = \cot{x} \implies B = \ln{\abs{\sin{x}}} + C_2
\end{equation*}
and
\begin{equation*}
	A^\prime\cos{x} + \cos{x} = 0 \implies A^\prime = -1 \implies A = -x + C_1
\end{equation*}
So, our general solution is
\begin{equation*}
	y = \left(C_1 - x\right)\cos{x} + \left(\ln{\abs{\sin{x}}} + C_2\right)\sin{x} = C_1\cos{x} + C_2\sin{x} - x\cos{x} + \sin{x}\ln{\abs{\sin{x}}}
\end{equation*}
Note how $y_h$ and $y_p$ appear together.\fi
\subsubsection{Higher Order Variation of Parameters}
\noindent
We are trying to find a solution to the nth order linear ODE
\begin{equation*}
	a_n(x)y^{(n)} + \ldots + a_0(x)y = g(x)
\end{equation*}
assuming that we already know the fundamental solutions for the corresponding homogeneous equation
\begin{equation*}
	y_h = C_1y_1 + \ldots + C_ny_n.
\end{equation*}

\noindent
For this method, we'll assume that $y$ can be written as
\begin{equation*}
	y = v_1(x)y_1 + \ldots + v_n(x)y_n
\end{equation*}
and we'll try to find $v_1, \ldots, v_n$.\\

\noindent
Since there are $n$ unknown functions, we'll need $n$ equations to find them all.
We can generate these by differentiating $y_p$.
\begin{equation*}
	y' = \left(v_1y_1' + \ldots + v_ny_n'\right) + \left(v_1'y_1 + \ldots + v_n'y_n\right)
\end{equation*}
to avoid second derivatives of $v_1, \ldots, v_n$ from entering the formula for $y''$, we also have the condition
\begin{equation*}
	v_1'y_1 + \ldots +v_n'y_n = 0.
\end{equation*}
We can now continue differentiating to get $n-2$ more equations involving $v_1', \ldots, v_n'$.
We also impose a final $n^{\text{th}}$ condition that
\begin{equation*}
	v_1'y^{(n-1)} + \ldots + v_n'y_n^{(n-1)} = g.
\end{equation*}
This gets us a system of $n$ equations,
\begin{equation*}
	\begin{cases}
		v_1'y_1 + \ldots + v_n' y_n & = 0 \\
		\vdots & \vdots \\
		v_1'y_1^{(n-2)} + \ldots + v_n'y_n^{(n-2)} & = 0 \\
		v_1'y_1^{(n-1)} + \ldots + v_n'y_n^{(n-1)} & = g
	\end{cases} .
\end{equation*}
Hopefully this system looks familiar from second order equations.\\

\noindent
We can rewrite this system in terms of matrices and vectors.
\begin{equation*}
	\begin{bmatrix}
		y_1 & \ldots & y_n \\
		\vdots & & \vdots\\
		y_1^{(n-2)} & \ldots & y_n^{(n-2)} \\
		y_1^{(n-1)} & \ldots & y_n^{(n-1)}
	\end{bmatrix} \begin{bmatrix}
		v_1' \\
		\vdots \\
		v_{n-1}' \\
		v_n'
	\end{bmatrix} = \begin{bmatrix}
		0 \\
		\vdots \\
		0 \\
		g
	\end{bmatrix}.
\end{equation*}
It's sufficient to show that a solution to this system exists if the determinant of the square matrix on the left is non-zero. The determinant of this matrix actually has a special name.

\begin{definition}
	The Wronskian of $n$ $n-1$ times differentiable functions $\left\{f_1, \ldots, f_n\right\}$ on an interval $I$ is
	\begin{equation*}
		W[f_1, \ldots, f_n](x) = \begin{vmatrix}
			f_1(x) & \ldots & f_n(x) \\
			f_1'(x) & \ldots & f_n'(x) \\
			\vdots &        & \vdots \\
			f_1^{(n-1)}(x) & \ldots & f_n^{(n-1)}(x)
		\end{vmatrix} \text{, } x \in I.
	\end{equation*}
\end{definition} 

\noindent
Using the Wronskian, we can solve the system using Cramer's Rule.
\begin{equation*}
	v'_i(x) = \frac{g(x)W_i(x)}{W[y_1, \ldots, y_n](x)} \text{, } i = 1, \ldots, n,
\end{equation*}
where $W_i(x)$ is the determinant of the matrix obtained from the Wronskian $W(x)$ by replacing the $i^{\text{th}}$ column with $\text{col}[0, \ldots, 1]$. Using the cofactor expansion along this column, we can write $W_i(x)$ as
\begin{equation*}
	W_i(x) = (-1)^{n-i}W[y_1, \ldots, y_{i-1}, y_{i+1}, \ldots, y_n](x) \text{, } i = 1, \ldots, n.
\end{equation*}
Now with a solution for $v_i'$, we can integrate to get $v_i$.
\begin{equation*}
	v_i = \int{\frac{g(x)W_i(x)}{W[y_1, \ldots, y_n](x)} \mathrm{d}x}.
\end{equation*}
Now with a solution for $v_i$, we can substitute back to find $y(x)$.
\begin{equation*}
	y(x) = \sum_{i=1}^{n}{y_i(x)\int{\frac{g(x)W_i(x)}{W[y_1, \ldots, y_n](x)} \mathrm{d}x}}
\end{equation*}

\begin{example}
	Find the general solution to the equation
	\begin{equation*}
		x^3y''' + x^2y'' - 2xy' = x^3\sin{x} \text{, } x > 0
	\end{equation*}
	given that $\left\{x, x^{-1}, x^2\right\}$ is the set of fundamental solutions.
\end{example}
\noindent
First we divide by $x^3$ to get a leading coefficient of 1.
\begin{equation*}
	y''' + x^{-1}y''' - 2x^{-2}y' = \sin{x} \text{, } x > 0.
\end{equation*}
Next we calculate the $W(x)$ and each $W_i(x)$ for the fundamental solution set.
\begin{align*}
	W[x,x^{-1}, x^2](x) &= \begin{vmatrix}
		x & x^{-1} & x^2 \\
		1 & -x^{-2} & 2x \\
		0 & 2x^{-3} & 2
	\end{vmatrix} = 6x^{-1} \\
	W_1(x) &= (-1)^{3-1}W[x^{-1}, x^2](x) = (-1)^{2}\begin{vmatrix}
		x^{-1} & x^2 \\
		-x^{-2} & 2x
	\end{vmatrix} = 3 \\
	W_2(x) &= (-1)^{3-2}\begin{vmatrix}
		x & x^{2} \\
		1 & 2x
	\end{vmatrix} = -x^2 \\
	W_2(x) &= (-1)^{3-2}\begin{vmatrix}
		x & x^{-1} \\
		1 & -x^{-2}
	\end{vmatrix} = -2x^{-1}.
\end{align*}
Now we can calculate $y$.
\begin{align*}
	y(x) &= x\int{\frac{(\sin{x})^3}{-6x^{-1}} \mathrm{d}x} + x^{-1}\int{\frac{(\sin{x})(-x^2)}{-6x^{-1}} \mathrm{d}x} + x^2\int{\frac{(\sin{x})(-2x^{-1})}{-6x^{-1}} \mathrm{d}x} \\
	&= x\int{\left(\frac{-1}{2}x\sin{x}\right)\mathrm{d}x} + x^{-1}\int{\left(\frac{1}{6}x^3\sin{x}\right)\mathrm{d}x} + x^2\int{\left(\frac{1}{3}\sin{x}\right)\mathrm{d}x} \\
	&= C_1x + c_2x^{-1} + C_3x^2 + \cos{x} - x^{-1}\sin{x}
\end{align*}