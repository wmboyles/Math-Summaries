\subsubsection{Defective Matrix}
If $A$ has an eigenvalue $\lambda$ with multiplicity $k$ that does not generate $k$ corresponding linearly independent eigenvectors, the the matrix $A$ is defective. To generate enough vectors, we need to extend eigenvectors.
\begin{definition}
	Let $A$ be a square matrix. A nonzero vector $\vec{v}$ satisfying
	\begin{equation*}
		(A-\lambda I)^n \vec{v} = \vec{0}
	\end{equation*}
	for some eigenvalue $\lambda$ and some positive integer $n$ is a generalized eigenvector or rank $n$.
\end{definition}
These generalized eigenvectors also make up solutions.
\begin{theorem}
	Let $\vec{v}$ be an generalized eigenvector of a square matrix $A$ corresponding to an eigenvalue $\lambda$ with multiplicity $k$. Then
	\begin{equation*}
		e^{At}\vec{v} = e^{\lambda t}\left(\vec{v} + t(A-\lambda I)\vec{v} + \frac{t^2}{2!}(A-\lambda I)^2\vec{v} + \ldots\right)
	\end{equation*}
	is the corresponding fundamental solution. Further, the above sequence will terminate after $k$ or fewer terms (all subsequent terms are 0).
\end{theorem}

\begin{example}
	Find the general solution to the system
	\begin{equation*}
		\vec{x}' = \begin{bmatrix}
			1 & 0 & 0 \\
			1 & 3 & 0 \\
			0 & 1 & 1
		\end{bmatrix} \vec{x}
	\end{equation*}
\end{example}
\noindent
Since $A$ is diagonal, the eigenvalues are simply the diagonal entries
\begin{equation*}
	\lambda = 3, 1 \text{(double root)}
\end{equation*}
Finding the eigenvector for $\lambda  = 3$,
\begin{equation*}
	(A - 3I)\vec{v} = \vec{0} \implies \vec{v} = C_1 \begin{bmatrix}
		0 \\
		2 \\
		1
	\end{bmatrix}
\end{equation*}
Finding the eigenvector for $\lambda = 1$
\begin{equation*}
	(A - I)\vec{v} = \vec{0} \implies \vec{v} = C_2 \begin{bmatrix}
		0 \\
		0 \\
		1
	\end{bmatrix}
\end{equation*}
Since $\lambda  =1$ had multiplicity 2, we need to find a generalized eigenvector.
\begin{equation*}
	(A - I)^2\vec{v} = \vec{0} \implies \vec{v} = C_2 \begin{bmatrix}
		0 \\
		0 \\
		1
	\end{bmatrix} + C_3 \begin{bmatrix}
		-2 \\
		1 \\
		0
	\end{bmatrix}
\end{equation*}
Note how then eigenvector showed up again as a generalized eigenvector of rank 2. This is because The eigenvector makes $(A - I)\vec{v} = 0$, so $(A - I)^2\vec{v} = (A - I)\vec{0} = \vec{0}$.
So our solution is
\begin{equation*}
	\vec{x} = C_1e^{3t}\begin{bmatrix}
		0 \\
		2 \\
		1
	\end{bmatrix} + C_2e^{t}\begin{bmatrix}
		0 \\
		0 \\
		1
	\end{bmatrix} + C_3e^{t}\left(\begin{bmatrix}
		-2 \\
		1 \\
		0
	\end{bmatrix} + (A - I)t\begin{bmatrix}
		-2 \\
		1 \\
		0
	\end{bmatrix}\right)
\end{equation*}
\begin{equation*}
	= C_1e^{3t}\begin{bmatrix}
		0 \\
		2 \\
		1
	\end{bmatrix} + C_2e^{t}\begin{bmatrix}
		0 \\
		0 \\
		1
	\end{bmatrix} + C_3e^{t}\begin{bmatrix}
		-2 \\
		1 \\
		t
	\end{bmatrix}
\end{equation*}
We can rewrite this a bit more cleanly as
\begin{equation*}
	\vec{x} = \begin{bmatrix}
		0 & 0 & -2e^t \\
		2e^{2t} & 0 & e^t \\
		e^{3t} & e^t & te^t
	\end{bmatrix} \begin{bmatrix}
		C_1 \\
		C_2 \\
		C_3
	\end{bmatrix}
\end{equation*}
This square matrix is called the fundamental matrix of the system.