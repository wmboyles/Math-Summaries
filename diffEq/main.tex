\documentclass[a4paper, oneside, 12pt]{book}

% ------------------------------------------------------------------------------
% Setup for table of contents
\setcounter{tocdepth}{3} 	% TOC should label down to subsubsections
\setcounter{secnumdepth}{2}	% TOC should not number further than a subsection number
% ------------------------------------------------------------------------------


% ------------------------------------------------------------------------------
% General Setup
\usepackage[english]{babel}
\usepackage{amsfonts, amsmath, amsthm}		%Formatting symbols, theorems, lemmas, definitions, and examples
\usepackage{eucal}							% Makes the Laplace transform L look a little better

\usepackage{float}							% For making sure tables and figures stay in place
\usepackage{fullpage}						% Create ~1" margins
\usepackage[bottom,flushmargin]{footmisc}	% Put footnotes at bottom of page; don't intent footnotes

\setcounter{chapter}{-1}				% Start with chapter 0

\usepackage{enumitem}					% For custom labels on enumerations
\usepackage{hyperref}					% For inserting links
\usepackage{graphicx} 					% For inserting images
% ------------------------------------------------------------------------------


\usepackage{tikz}
\usetikzlibrary{patterns}
\usetikzlibrary{calc,patterns,decorations.pathmorphing,decorations.markings}

% ------------------------------------------------------------------------------
% Theorems, lemmas, definitions and examples should not be numbered
\newtheorem*{theorem}{Theorem}
\newtheorem*{definition}{Definition}
\newtheorem*{lemma}{Lemma}
\newtheorem*{example}{Example}
% ------------------------------------------------------------------------------

% ------------------------------------------------------------------------------
% Shortcuts
\newcommand{\dd}[2]{\frac{\mathrm{d} #1}{\mathrm{d} #2}}							% d[] / d[]
\newcommand{\pp}[2]{\frac{\partial #1}{\partial #2}}								% partial[] / partial[]

\newcommand{\abs}[1]{\lvert #1 \rvert}												% absolute value

\DeclareMathOperator{\arcsec}{arcsec}												% arc-secant
\DeclareMathOperator{\arccot}{arccot}												% arc-cotangent
\DeclareMathOperator{\arccsc}{arccsc}												% arc-cosecant

\newcommand{\R}{\mathbb{R}}															% Real numbers
\newcommand{\Laplace}[1]{\mathcal{L}\left\{ #1 \right\}\left(s\right)}				% Laplace Transform
\newcommand{\inverseLaplace}[1]{\mathcal{L}^{-1}\left\{ #1 \right\}\left(t\right)}	% Inverse Laplace Transform
% ------------------------------------------------------------------------------


% ------------------------------------------------------------------------------
% Macros for including certain parts of the document. 1 for true, 0 for false.
\def\includeBackgroundReview{1}					% Should the background/review chapter be included?
	\def\includeBackgroundReviewExamples{1}		% Should examples be included in the background/review chapter?
	
\def\includeBasics{1}							% Should the introduction/basics chapter be included?
	\def\includeBasicsExamples{1}				% Should examples be included in the basics chapter?
	
\def\includeFirstOrderLinearODE{1}				% Should the 1st order linear ODE chapter be included?
	\def\includeFirstOrderLinearODEExamples{1}	% Should examples be included in the 1st order linear ODEs chapter?
	
\def\includeHigherOrder{1}						% Should the higher order ODE chapter be included?
	\def\includeHigherOrderExamples{1}			% Should examples be included in the higher order ODE chapter?
	
\def\includeLinearSystems{1}					% Should the linear systems of DiffEq's chapter be included?
	\def\includeLinearSystemsExamples{1}		% Should examples be included in the linear systems chapter?
	
\def\includeLaplaceTransforms{1}				% Should the Laplace transforms chapter be included?
	\def\includeLaplaceTransformExamples{1}		% Should examples be included in the Laplace transforms chapter?

\def\includeAdditionalResources{1}				% Should the additional resources chapter be included?	

% ------------------------------------------------------------------------------


\begin{document}
	% Title page setup
	\title{Intro to Differential Equations: A Summary}
	\author{William Boyles}
	\date{}
		
	\frontmatter
		\maketitle
		\tableofcontents
	
	\mainmatter
		\ifodd\includeBackgroundReview\chapter{Background \& Review}
\noindent
Everything mentioned in this chapter should already be familiar to you from other math classes. These topics span three main areas: algebra/pre-calculus, single variable calculus, and matrices. These topics will be used either implicitly or with only a passing reference.\\

\noindent
If you are unfamiliar with anything mentioned, you can use many of the great online resources, like Khan Academy, to familiarize yourself before moving forward.

\section{Algebra and Pre-Calculus}
\noindent

\subsection{Complex Numbers}
\noindent
$i$ is called the imaginary unit. It's defined by $i^2 = -1$. It and $-i$ are the solutions to the equation $x^2+1=0$.\\

\noindent
Complex numbers ($\mathbb{C}$) have the form $z = \alpha + \beta i$, where $\alpha$ and $\beta$ are real numbers. The $\alpha$ part of $z$ is called the real part, so $\Re(z) = \alpha$. The $\beta$ part of $z$ is called the imaginary part, so $\Im(z) = \beta i$.\\

\noindent
Often, complex numbers are visualized as points or vectors in a 2D plane, called the complex plane, where $\alpha$ is the x-component, and $\beta$ is the y-component. Thinking of complex numbers like points helps us define the magnitude of complex numbers and compare them. Since a point $(x,y)$ has a distance $\sqrt{x^2+y^2}$ from the origin, we can say the magnitude of $z$, $\lvert z \rvert$ is $\sqrt{\alpha^2 + \beta^2}$. Thinking of complex numbers like vectors helps us understand adding two complex numbers, since you just add the components like vectors.\\

\noindent
A common operation on complex numbers is the complex conjugate. The complex conjugate of $z = \alpha + \beta i$ is $\overline{z} = \alpha - \beta i$. $z$ and $\overline{z}$ are called a conjugate pair.\\

\noindent
Conjugate pairs have the following properties.
\\Let $z$, $w \in \mathbb{C}$.
\begin{enumerate}[label=]
	\item $\overline{z \pm w} = \overline{z} \pm \overline{w}$
	\item $\overline{zw}=\overline{z}\overline{w}$
	\item $\overline{z}=z \Leftrightarrow z \in \mathbb{R}$
	\item $z\overline{z} = \lvert z \rvert^2 = \lvert \overline{z} \rvert^2$
	\item $\overline{\overline{z}} = z$
	\item $\overline{z}^n = \overline{z^n}$
	\item $z^{-1} = \frac{\overline{z}}{\lvert z \rvert^2}$
\end{enumerate} 			% Complex Numbers
\subsection{Factoring Polynomials}
\noindent
We want to break up a polynomial like $f(x) = a_0 + a_1x^1 + \ldots a_nx^n$ into linear factors so that $f(x) = c(x-b_1)\cdot \ldots \cdot(x - b_n)$. This form makes it simple to see that the roots of $f$, solutions to $f(x) = 0$, are $x = b_1 \ldots b_n$.\\

\noindent
For quadratics, $f(x) = ax^2 + bx + c$, there exists a simple formula that will give us both roots, the quadratic formula.
\begin{equation*}
	x = \frac{-b \pm \sqrt{b^2-4ac}}{2a}
\end{equation*}

\noindent
We can see that when $b^2 - 4ac < 0$, like for $f(x) = x^2 + 5x + 1$, we will get complex roots $\alpha \pm \beta i$. For any polynomial, these roots come in pairs, so if $\alpha + \beta i$ is a root, then so is $\alpha - \beta i$. This means that every conjugate pair $\alpha \pm \beta i$ has a quadratic equation with those roots. Sometimes we will not factor quadratics with complex roots into linear terms.\\

\noindent
Although there do exist explicit formulas for finding roots for cubic (degree 3) and quartic (degree 4) equations, they are too long and not useful enough to memorize. When working by hand, we instead use other tricks to find roots.\\

\noindent
There are a few useful tricks that can help. If the polynomial doesn't have a constant term, then 0 is a root. If all the coefficients sum to 0, then 1 is a root. For certain polynomials with an even number of terms, like all cubics of the form $ax^3 + bx^2 + cax + cb$ we can factor out a term from the first two and last two terms to get $x^2(ax+b)+c(ax+b) = (ax+b)(x^2+c)$. For other polynomials, we might just try guessing and checking values. However, we need a more efficient way that works in general.\\

\noindent
Since we are looking to find linear factors $f(x) = (x-b_1)\cdot \ldots \cdot(x-b_n)$, we can see that the constant term in the polynomial is the product of the roots $b_1 \ldots b_n$. In fact, since the coefficients of polynomials are completely determined by the roots and the leading coefficient, all the coefficients are sums and products of roots. You might remember when factoring quadratics that the coefficient of $x$ term is the sum of the two roots. These rules are called Vieta's formulas.\\

\noindent
So, if we have the constant term, we can check all of its integer factors to see if any are roots. For each root, we can divide, using a technique like synthetic division, to continue finding the rest of the roots. This method is especially useful on tests because the roots tend to be integers.

%\begin{example}
%Factor the polynomial $x^5 + x^4 -2x^3 + 4x^2 -24x$.	
%\end{example}
%\noindent
%We can immediately see that there is no constant term, so $x=0$ is a root. Now we need to work on factoring $x^4 + x^3 -2x^2 + 4x - 24$.\\
%The factors of -24 are: -24, -12, -8, -6, -4, -3, -2, -1, 1, 2, 3, 4, 6, 8, 12, and 24. Starting from roots close to 0 and working outwards, we find that $x=2$ is a root. So, we synthetic divide like so
%\begin{table}[H]
%	\centering
%	\begin{tabular}{llllll}
%		$x=2 \mid$ & 1            & 1 & -2 & 4  & -24 \\
%		& $\downarrow$ & 2 & 6  & 8  & 24  \\ \hline
%		& 1            & 3 & 4  & 12 & $\mid 0$  
%	\end{tabular}
%\end{table}
%\noindent
%to see that now we need to work on factoring $x^3+3x^2+4x+12$.
%$x^3+3x^2+4x+12 = x^2(x+3)+4(x+3) = (x+3)(x^2+4)$, so $x=-3$ is a root, and we need to work on factoring $x^2+4$.
%$x^2+4$ has two complex roots $\pm 2i$, so we'll leave it as a quadratic.
%\begin{equation*}	
%	x^5 + x^4 -2x^3 + 4x^2 -24x = x(x-2)(x-3)(x^2+4)
%\end{equation*} 		% Factoring Polynomials
\subsection{Trig Functions \& The Unit Circle}
\noindent
Imagine aa circle of radius 1 centered at the origin that we'll call the unit circle. The x and y coordinates of a point on the unit circle are completely determined by the angle $\theta$ in radians between the x-axis and a line from the origin to the point.\\

\noindent
The function $\cos{\theta}$ tells us x-coordinate of the point, while $\sin{\theta}$ tells us the y-coordinate of the point. The function $\tan{\theta} = \frac{\sin{\theta}}{\cos{\theta}}$ tells us the slope of the line from the origin to the point. Most of the trig functions have geometric interpretations as shown below. The most used ones are $\sin$, $\cos$, $\tan=\frac{\sin}{\cos}$, $\cot = \frac{\cos}{\sin}$, $\csc=\frac{1}{\sin}$, and $\sec=\frac{1}{\cos}$.

\begin{figure}[H]
	\label{unitCircle}
	\centering
	\includegraphics[width = 0.75\textwidth]{./backgroundReview/algebraPreCalc/unitCircle2.png}
	\caption{\hyperref{https://en.wikipedia.org/wiki/Unit_circle}{}{}{Wikipedia - Unit circle}}
\end{figure}

\noindent
We can also think about the inverses of these trig functions. These are either notated with a -1 exponent on the function, or the prefix arc in front of the function name. Many of these functions are only defined on a part of the domain $\left[0, 2\pi\right]$. Below is a table of the inverse trig functions and their domains.

\begin{table}[H]
	\centering
	\begin{tabular}{l|l}
		Function  & Domain                                                 \\ \hline
		$\arcsin$ & $\left[-1, 1\right]$           						   \\
		$\arccos$ & $\left[-1, 1\right]$                                   \\
		$\arctan$ & $\left(-\infty, \infty\right)$                         \\
		$\arccot$ & $\left(-\infty, \infty\right)$                         \\
		$\arccsc$ & $\left(-\infty, -1\right] \cup \left[1, \infty\right)$ \\
		$\arcsec$ & $\left(-\infty, -1\right] \cup \left[1, \infty\right)$
	\end{tabular}
\end{table}
 	% Trig Functions / Unit Circle
\subsection{Trig Identities}
\noindent
As we could see in Figure \ref{unitCircle}, $\sin$ and $\cos$ form a right triangle with hypotenuse 1. So, using the Pythagorean Theorem
\begin{equation*}
	\sin^2{\theta} + \cos^2{\theta} = 1
\end{equation*}
By dividing by $\sin^2$ or $\cos^2$, we can also get
\begin{equation*}
	1 + \cot^2{\theta} = \csc^2{\theta} \text{ and } \tan^2{\theta} + 1 = \sec^2{\theta}
\end{equation*}
Together, these 3 identities are called the Pythagorean Identities.\\

\noindent
We can also relate functions and co-functions.
\begin{equation*}
	\text{xxx}(\theta) = \text{coxxx}\left(\frac{\pi}{2} - \theta\right)
\end{equation*}

Some of the most useful and used identities are the sum and difference identities.
\begin{equation*}
	\sin{\left(\alpha \pm \beta\right)} = \sin{\alpha}\cos{\beta} \pm \cos{\alpha}\sin{\beta}
\end{equation*} \begin{equation*}
	\cos{\left(\alpha \pm \beta\right)} = \cos{\alpha}\cos{\beta} \mp \sin{\alpha}\sin{\beta}
\end{equation*} \begin{equation*}
	\tan{\left(\alpha \pm \beta\right)} = \frac{\tan{\alpha} \pm \tan{\beta}}{1 \mp \tan{\alpha}\tan{\beta}}
\end{equation*} \begin{equation*}
	\sin{\alpha} \pm \sin{\beta} = 2\sin{\left(\frac{\alpha \pm \beta}{2}\right)}\cos{\left(\frac{\alpha \mp \beta}{2}\right)}
\end{equation*} \begin{equation*}
	\cos{\alpha} + \cos{\beta} = 2\cos{\left(\frac{\alpha + \beta}{2}\right)}\cos{\left(\frac{\alpha - \beta}{2}\right)}
\end{equation*} \begin{equation*}
	\cos{\alpha} - \cos{\beta} = -2\sin{\left(\frac{\alpha + \beta}{2}\right)}\sin{\left(\frac{\alpha - \beta}{2}\right)}
\end{equation*} 			% Trig Identites
\subsection{Exponentials \& Logarithms}
\begin{definition}
	e is the base of the natural logarithm. It's definied by the limit
	\begin{equation*}
		e = \lim\limits_{n\rightarrow\infty}{\left(1+\frac{1}{n}\right)^n}.
	\end{equation*}
\end{definition}


$\exp{x} = e^x$ and $\ln{x}$ are inverse functions of each other such that
\begin{equation*}
	e^{\ln{x}} = x \text{, } \ln{e^x} = x.
\end{equation*}


Just like other exponents, the normal rules for adding, subtracting, and multiplying powers apply.
\begin{equation*}
	e^xe^y = e^{x+y}\text{, }\frac{e^x}{e^y}=e^{x-y}\text{, and }\left(e^x\right)^k=e^{xk}.
\end{equation*}


Similar rules apply for logarithms.
\begin{equation*}
	\ln{x}+\ln{y} = \ln{xy}\text{, }\ln{x}-\ln{y} = \ln{\left(\frac{x}{y}\right)}\text{, and }\ln{\left(a^b\right)}=b\ln{a}.
\end{equation*}


You can also change a logarithm of any base to a natural logarithm.
\begin{equation*}
	\log_{b}{a} = \frac{\ln{a}}{\ln{b}}.
\end{equation*}


$e$ is also unique in that it is the only real number $a$ satisfying the equation
\begin{equation*}
	\frac{\mathrm{d}}{\mathrm{d}x}a^x = a^x.
\end{equation*}
meaning $e^x$ is its own derivative.\footnote{Don't worry if you don't know what a derivative is yet. It's one of the first topics we'll cover in calculus.} 		% Exponential and logarithms
\subsection{Partial Fractions}
\noindent
If we have a function of two polynomials $f(x) = \frac{P(x)}{Q(x)}$, it's often easier to break this quotient into a sum of parts where the denominator is a linear or quadratic factor and the numerator is always a smaller degree than the denominator.

\begin{example}
	\begin{equation*}
		\frac{2x-1}{x^3-6x^2+11x-6} = \frac{1/2}{x-1}+\frac{-3}{x-2}+\frac{5/2}{x-3}.
	\end{equation*}
\end{example}

\noindent
One natural way to find these small denominators comes from the linear factors of the denominator where we keep quadratics with complex roots.
This way, when making a common denominator, we get back the original big denominator.
However, there are a few special cases we have to take care of.

\input{./backgroundReview/algebraPreCalc/linearFactors.tex}
\input{./backgroundReview/algebraPreCalc/repeatedLinearFactors.tex}
\input{./backgroundReview/algebraPreCalc/quadraticFactors.tex}
\input{./backgroundReview/algebraPreCalc/repeatedQuadraticFactors.tex}
\input{./backgroundReview/algebraPreCalc/improperFractions.tex}
 			% Partial Fractions % Algebra and Pre-Calc
\section{Single Variable Calculus}
\noindent

% Derivatives and Integrals
% u-subtitution
% Integration by parts % Single Variable Calculus
\chapter{Matrices}

We'll introduce matrices, how they can represent linear maps and systems of linear equations, and useful operations we can perform on them.

\section{Definition}

\begin{definition}
	An $m \times n$ matrix is an array of objects (usually field elements) arranged in $m$ rows and $n$ columns.
\end{definition}

Matrices are usually written inside square brackets.
We tend to use uppercase letters like $M$ to represent matrices as variables.
The notation $M_{a,b}$ represents the element in row $a$ and column $b$.

\begin{example}
	Matrix $M$ is $3 \times 4$.
	\begin{equation*}
		M = \begin{bmatrix}
			1 & 4  & 0 \\
			8 & -1 & -2 \\
			3 & 7  & 4
		\end{bmatrix}
	\end{equation*}
	We see that $M_{2,2} = -1$ and $M_{3,1} = 3$.
\end{example}


\section{Basic Operations}

\subsection{Vector Space Operations}
Two matrices are considered equal if they have the same number of rows and columns and all entries are equal.
Scalar multiplication works by multiplying each element by the scalar.
Addition works element by element.

\begin{example}
	\begin{align*}
		A &= \begin{bmatrix}
			1 & 2 \\
			-1 & 3
		\end{bmatrix} \text{, } B = \begin{bmatrix}
			4 & 0 \\
			-2 & 5
		\end{bmatrix}. \\
		A + B &= \begin{bmatrix}
			1 + 4 & 2 + 0 \\
			-1 + -2 & 3 + 5
		\end{bmatrix} = \begin{bmatrix}
			5 & 2 \\
			-3 & 8
		\end{bmatrix} \\
		3A &= \begin{bmatrix}
			3\cdot1 & 3\cdot 2 \\
			3\cdot -1 & 3\cdot 3
		\end{bmatrix} = \begin{bmatrix}
			3 & 6 \\
			-3 & 9
		\end{bmatrix}.
	\end{align*}
\end{example}

Notice addition is commutative, associative, has an additive identity (the all 0's matrix), and has an additive inverse (scalar multiply by -1).
Further, scalar multiplication has a multiplicative identity (1), is distributive over both addition and field multiplication.
Thus, the set of all matrices of the same size form a vector space.
We denote the vector space of all $m \times n$ matrices with real entries as $\mathcal{M}_{m \times n}$.

\subsection{Multiplication}
We can also define an operation for multiplying two matrices of compatible sizes that outputs another matrix.

\begin{definition}
	Let $A$ be an $m \times n$ matrix, and let $B$ be and $n \times k$ matrix.
	Then $C = AB$ is an $m \times k$ matrix where
	\begin{equation*}
		C_{i,j} = \sum_{d=1}^{n}{A_{i,d} B_{d,j}}.
	\end{equation*}
\end{definition}
If you're familiar with the concept of dot products, then $C_{i,j}$ is the dot product of the $i$th row of $A$ with the $j$th column of $B$.

\begin{example}
	\begin{align*}
		A &= \begin{bmatrix}
			1 & 2 & 3 \\
			0 & -1 & 2
		\end{bmatrix} \text{, } B = \begin{bmatrix}
			1 & -1 & 0 & 2 \\
			2 & 0 & 3 & -1 \\
			0 & 1 & 1 & 5 
		\end{bmatrix}. \\
		AB &= \begin{bmatrix}
			5 & 2 & 9 & 15 \\
			-2 & 2 & -1 & 11
		\end{bmatrix}.
	\end{align*}
\end{example}

Similar to scalar multiplication, there exists a multiplicative identity matrix.
However, this matrix only behaves like an identity when the matrix it's being multiplied is $n \times n$ (i.e. a square matrix).

\begin{definition}
	The $n \times n$ matrix $I_n$ is called the identity matrix and is defined by
	\begin{equation*}
		I_{i,j} = \begin{cases}
			1 & i=j \\
			0 & \text{otherwise}
		\end{cases}.
	\end{equation*}
\end{definition}

\begin{theorem}
	Let $A$ be an $n \times n$ matrix.
	Then $AI_n = I_n A = A$.
\end{theorem}
\begin{proof}
	Let $C = AI_n$.
	Notice,
	\begin{align*}
		C_{i,j} &= \sum_{d=1}^{n}{A_{i,d}(I_{n})_{d,j}} \\
			&= \sum_{d=1}^{n}{A_{i,d} \begin{cases}
					1 & d=j \\
					0 & \text{otherwise}
			\end{cases}} \\
			&= \sum_{d=1}^{n}{\begin{cases}
				A_{i,j} & d=j \\
				0 & \text{otherwise}
			\end{cases}} \\
			&= A_{i,j}.
	\end{align*}
	This same line of reasoning works to also show that $I_nA = A$.
	Since all entries of $A$ and $C$ are equal, $C = AI_n = A$, as desired.
\end{proof}

Also similar to scalar multiplication, matrix multiplication is associative and distributive.
\begin{theorem}
	Let $A$, $B$, and $C$ be matrices.
	Let $k$ be a scalar.
	Then the following properties hold (assuming the matrices have the correct dimensions):
	\begin{itemize}
		\item \textbf{Associative}: $A(BC) = (AB)C$.
		\item \textbf{Distributive Over Matrix Multiplication}: $k(AB) = (kA)B = A(kB)$.
		\item \textbf{Left Distributive Over Addition}: $A(B + C) = AB + AC$.
		\item \textbf{Right Distributive Over Addition}: $(A + B)C = AC + BC$.
	\end{itemize}
\end{theorem}

Unlike scalar multiplication, matrix multiplication is not commutative.
For one, if $AB$ is defined, $BA$ won't also be defined unless $A$ and $B$ are both square matrices
Even if this is the case, $AB \neq BA$ in general.

\begin{theorem}
	An $n \times n$ matrix $A$ commutes only and all matrices in the vector space $\linspan(\{I_n, A\})$.
\end{theorem}
\section{As Linear Maps}

\input{./matrices/matricesAsLinearMapsDefinition.tex}
\input{./matrices/matricesAsLinearMapsProperties.tex}



\section{As Systems of Linear Equations}

\input{./matrices/matricesAsLinearMapsInvertability.tex}
\input{./matrices/matricesAsLinearMapsColumnSpaceNullSpace.tex}
\subsection{Determinants}
\noindent
The determinant of a matrix is a signed number that tells by how much the transformation represented by a matrix scales volumes in a space. The number is negative if the space was "flipped" during a transformation. The number is negative if the dimension of the output space is less than that of the input space.\\

\noindent
The determinant is only defined for square matrices. It's easiest to understand the definition of a determinant recursively.
\begin{equation*}
	\det{\left[ a \right]} = \lvert a \rvert = a
\end{equation*}
\begin{equation*}
	\det{\left[
		\begin{array}{cc}
			a & b \\
			c & d
		\end{array}
	\right]} = \begin{array}{|cc|}
		a & b \\
		c & d
	\end{array} = ad - bc
\end{equation*}
We can define $a_{ij}$ as the entry in the ith row and jth column of matrix $A$ and $A_{ij}$ as the adjudicate matrix, which is the matrix $A$ if row $i$ and column $j$ were removed. This allows us to write a general formula for the determinant.
\begin{definition}
	\begin{equation*}
		\det{A} = \sum_{j=1}^{n}{\left(-1\right)^{i+j}a_{ij}A_{ij}} \text{ (for fixed i)} = \sum_{i=1}^{n}{\left(-1\right)^{i+j}a_{ij}A_{ij}} \text{ (for fixed j)}
	\end{equation*}
\end{definition}
\noindent
This formula allows us to use any row or column to calculate the determinant, which is especially useful if a certain row contains lots of 0's.\\

\noindent
Below are some properties of the determinant for some $n \times n$ matrix $A$ and scalar $\lambda$.
\begin{enumerate}[label=]
	\item \begin{equation*}
		\det{I_n} = 1
	\end{equation*}
	\item \begin{equation*}
		\det{(A^T)} = \det{A}
	\end{equation*}
	\item If $A$ is invertable,
		\begin{equation*}
			\det{(A^{-1})} = \frac{1}{\det{A}}
		\end{equation*}
	\item \begin{equation*}
		\det{(\lambda A)} = \lambda^n\det{A}
	\end{equation*}
	\item \begin{equation*}
		\det{(AB)} = \det{A}\det{B}
	\end{equation*}
	\item If $A$ is triangular,
		\begin{equation*}
			\det{A} = \prod_{i=1}^{n}{a_{ii}}
		\end{equation*}
\end{enumerate}

\ifodd\includeBackgroundReviewExamples\input{./backgroundReview/matrices/determinants_example}\fi % Matrix operations\fi 			% Background and Review
		\ifodd\includeBasics\chapter{The Basics of Differential Equations}
% Classifying Differential Equations
	% Order
	% Linearity
	% Homogenity
	% Ordinary / Partial
	% Seperability
% Solutions to DiffEq's
% Initial Value Programs 											% Introduction / Basics	
		\ifodd\includeFirstOrderLinearODE\chapter{1st Order Linear ODE's}
% Seperable Differential Equations
% Method for solving (Integrating Factor)
% Applications
	% Newton's Law of Cooling
	% Logistic Equation (Viral Spread)
	% Compound Interest
	% RC Circuit\fi 		% 1st order linear ODE's
		\ifodd\includeHigherOrder\chapter{Higher Order Linear ODE's}
% Constant Coefficients
\section{Constant Coefficients}
The general form of an nth order linear equation is
\begin{equation*}
	a_n(x)y^{(n)} + a_{n-1}y^{(n-1)} + \ldots + a_1(x)y' + a_0y = b(x)
\end{equation*}
If each $a_i(x)$ is a constant, then the equation has constant coefficients.\\

\noindent
We already know how to solve linear first order differential equations using an integrating factor, but let's see if we can develop a method that can solve any order linear, homogeneous differential equation with constant coefficients.

\input{./higherOrder/constCoeffs/constCoeffs_example.tex}

% Auxillary Equation
\subsection{The Auxiliary Equation}
\noindent
It's not a coincidence that the coefficients of the polynomial that we had to find the 0's of in the above example matched the coefficients of the differential equation. We call this polynomial the auxiliary equation, and it can help us solve linear, homogeneous differential equations with constant coefficients of any order.
\begin{definition}
	A nth order, linear, homogeneous differential equation with constant coefficients has the form
	\begin{equation*}
		a_ny^{(n)} + a_{n-1}y^{(n-1)} + \ldots + a_1y^\prime + a_0y = 0
	\end{equation*}
	The corresponding auxiliary equation is
	\begin{equation*}
		a_nr^n + a_{n-1}r^{n-1} + \ldots + a_1r + a_0 = 0
	\end{equation*}
\end{definition}

\noindent
We now have a method for solving these equations with the roots of the auxiliary equation are all unique. 
\begin{theorem}
	Let $\left\{r_1, \ldots, r_n\right\}$ be the set of unique roots to an auxiliary equation corresponding to a nth order, linear, homogeneous differential equation with constant coefficients. The set of fundamental solutions are $\left\{C_1e^{r_1x}, \ldots, C_ne^{r_nx}\right\}$, and the general solution is
	\begin{equation*}
		y = C_1e^{r_1x} + \ldots + C_ne^{r_nx}
	\end{equation*}
\end{theorem}

\noindent
We can easily extend this method to deal with roots of higher multiplicities.
\begin{theorem}
	Let $\alpha$ be a root with multiplicity $k$ to an auxiliary equation corresponding to a nth order, linear, homogeneous differential equation with constant coefficients. Then $e^{\alpha x}, xe^{\alpha x}, \ldots, x^{k-1}e^{\alpha x}$ are fundamental solutions.
\end{theorem}

\input{./higherOrder/constCoeffs/complexRoots.tex}

% Application: Mechanical Vibrations (Homogenous, Linear)
\section{Free Vibrations}
\noindent
Free damped vibrations, like in a massed spring system, are a common application of second order linear ODEs.
In a massed spring system, there are three main forces acting on the mass that make up external forces.
\begin{enumerate}[label=\arabic*)]
	\item Acceleration of The Mass -- Since acceleration is the 2nd derivative of position $y(t)$, and Newton's Second Law tells us that $F = ma$, the force from the acceleration of the mass is $my''$.
	\item Dampening -- We'll assume that this term is proportional to the velocity, $y'$, and a term $b$. So, the force from dampening is $by'$.
	\item Spring Stretch -- Hooke's Law tells us that the force from a spring is $ky$, where $k$ is some term that gives the spring's "stiffness"
\end{enumerate}
Since we assume that the net force is 0 (that's what free means), our equations is
\begin{equation*}
	my'' + by' + ky = 0.
\end{equation*}

\noindent
Extracting the coefficients and solving the auxiliary equation,
\begin{equation*}
	mr^2 + br + k = 0 \implies r = \frac{-b \pm \sqrt{b^2 - 4mk}}{2m}.
\end{equation*}
We will consider two cases.
One in which there is no damping ($b = 0$), and one in which there is damping ($b > 0$).

\subsection{Free Undamped Vibrations ($b = 0$)}
\noindent
In this case, our equation simplifies to
\begin{equation*}
	my'' + ky = 0.
\end{equation*}
The two roots of ou auxiliary equation are
\begin{equation*}
	r = \pm i \sqrt{\frac{k}{m}} = \pm i\omega.
\end{equation*}
So, our solution becomes
\begin{equation*}
	y = C_1\cos{(\omega t)} + C_2\sin{(\omega t)}.
\end{equation*}
This is the same $\omega$ from physics that means angular frequency, so the same physics formulas apply, like $T = \frac{2\pi}{\omega}$ for the period of the oscillation.\\

\noindent
We can simplify this a bit further. If we think of the $\cos$ and $\sin$ components as being sides of a right triangle like so,
\begin{center}
	\includegraphics[width=0.5\textwidth]{./higherOrder/freeVibrs/triangle.png}
\end{center}
then we rewrite our equation as
\begin{equation*}
	y = A\left( \frac{C_1}{\sqrt{C_1^2 + C_2^2}}\cos{(\omega t)} + \frac{C_2}{\sqrt{C_1^2 + C_2^2}}\sin{(\omega t)} \right).
\end{equation*}
Note that since $\left(\frac{C_1}{A}\right)^2 + \left(\frac{C_2}{A}\right)^2 = 1$, we can rewrite these coefficients as $\cos{\phi}$ and $\sin{\phi}$ respectively where
\begin{equation*}
	\phi = \begin{cases}
		\arctan{(\frac{C_2}{C_1})} & C_1 > 0 \\
		\arctan{(\frac{C_2}{C_1})} + \pi & C_1 \leq 0
	\end{cases}.
\end{equation*} 
So, our equation becomes
\begin{equation*}
	y = A\left(\cos{(\omega t)}\cos{\phi} + \sin{(\omega t)}\sin{\phi}\right).
\end{equation*}
Using the $\cos$ angle addition formula,
\begin{equation*}
	y = A\cos{(\omega t - \phi)}.
\end{equation*}
\begin{center}
	\includegraphics[width=0.5\textwidth]{./higherOrder/freeVibrs/undampedfree.png}
\end{center}
As we can see, an undamped free vibration will simply oscillate back and forth without decay.

\begin{example}
	A 2kg mass in an undamped system is attached to a spring with $k = 50 \text{N/m}$. The initial position of the masss is $y_0 = -0.25\text{m}$. The initial velocity is $v_0 = -1\text{m/s}$. Find an expression for $y(t)$, the position of the mass at time $t$. Write your answer in terms of a $\cos$ and a phase shift. Find the period and frequency in proper units.
\end{example}
\noindent
The IVP describing this problem is
\begin{equation*}
	\begin{cases}
		2y'' + 50y = 0 \\
		y'(0) = -1 \\
		y(0) = -0.25
	\end{cases}.
\end{equation*}
Extracting the auxiliary equation and finding the roots,
\begin{equation*}
	2r^2 + 50 = 0 \implies r = \pm 5i.
\end{equation*}
So, our general solution is
\begin{equation*}
	y = C_1\cos{(5t)} + C_2\sin{(5t)}.
\end{equation*}
Solving for $C_1$ and $C_2$,
\begin{align*}
	y(0) &= -0.25 = C_1 \implies C_1 = -0.25 \\
	y' &= -5C_1\sin{(5t)} + 5C_2\cos{(5t)} \\
	y'(0) &= -1 = 5C_2 \implies C_2 = -0.2.
\end{align*}
Solving for $\phi$, keeping in mind that $C_1 < 0$,
\begin{equation*}
	\phi = \arctan{\frac{C_2}{C_1}} + \pi = \arctan{\frac{4}{5}} + \pi.
\end{equation*}
So, our answer is (in units of meters)
\begin{equation*}
	y = \sqrt{(-0.25)^2 + (-0.2)^2}\cos{\left(5t - \arctan{\left(\frac{4}{5}\right)} - \pi\right)} \approx 0.32\cos{\left(5t - 3.82\right)}.
\end{equation*}
Solving for the period,
\begin{equation*}
	T = \frac{2\pi}{\omega} = \frac{2\pi}{5} \text{s}.
\end{equation*}
Solving for the frequency,
\begin{equation*}
	f = \frac{1}{T} = \frac{5}{2\pi} \text{Hz}.
\end{equation*}
\subsection{Free Damped Vibrations ($b > 0$)}
We need to consider three cases where the discriminant $\Delta = b^2 - 4mk$ is positive, zero, and negative. 

\input{./higherOrder/freeVibrs/overdamped}
\input{./higherOrder/freeVibrs/critically_damped}
\input{./higherOrder/freeVibrs/underdamped}

\ifodd\includeHigherOrderExamples\input{./higherOrder/freeVibrs/freeDamped_example}\fi
	% Simple: mx'' + kx = 0
	% Free Damped: mx'' + by' + ky = 0

% Non-Homogenous Linear DiffEq's
	% Method of Undetermined Coefficients
	% Variation of Parameters
% Forced Vibrations
	% Undamped Forced
		% 1.1 omega != gamma
		% 1.2 omega = gamma
	% Damped Forced\fi 							% Higher (mostly 2nd) Order linear ODE's
		\ifodd\includeLinearSystems\chapter{Linear Systems of Differential Equations}
% Motivating Example: Water Tanks
% Homogenous
	% Eigenvalue Method
		% n distinct eigenvalues
		% repeated eigenvales with enough eigenvectors
		% Defective matrices
		% Complex Roots
% Non-Homogenous
	% Undetermined coefficients
	% Variation of Parameters
% Higher-order linear diffeq's as systems of 1st order linear diffeq's\fi 					% Linear Systems of DiffEq's
		\ifodd\includeLaplaceTransforms\chapter{Laplace Transforms}
% Algebra and Pre-Calc
	% Factoring Polynomials
	% Trig Functions / Unit Circle
	% Trig Identites
	% Exponential and logarithms
	% Partial Fractions
	% Complex Numbers
% Single Variable Calculus
	% Derivatives and Integrals
	% u-subtitution
	% Integration by parts
% Matrix operations
	% Types of matrices
	% Determinants
	% Row Reductions
	% Eigenvalues/vectors\fi 		% Laplace Transforms
		\ifodd\includeAdditionalResources\chapter{Additional Resources}

% Contributors
\section{Contributors}
Special thanks to everyone who made contributions to this project on \href{https://github.com/wmboyles/Math-Summaries}{Github}.
They are listed in order of number of commits as \texttt{name (GitHub username)}.

\begin{center}
    \begin{tabular}{ c c c }
    	William Boyles (\href{https://github.com/wmboyles}{wmboyles}) & Nate A (\href{https://github.com/aneziac}{aneziac}) & Ashwin Murali (\href{https://github.com/Suzukazole}{Suzukazole}) \\
		github-actions[bot] (\href{https://github.com/apps/github-actions}{github-actions[bot]}) & आदित्य देव (\href{https://github.com/dev-aditya}{dev-aditya}) & Calvin McPhail-Snyder (\href{https://github.com/esselltwo}{esselltwo}) \\
		Jared Geller (\href{https://github.com/jaredgeller}{jaredgeller}) & Robert Washbourne (\href{https://github.com/rawsh}{rawsh})\end{tabular}
\end{center}
\fi	% Additional Resources
		
		% Other possible sections
			% "Real World" Applications
				% Viral spread / logistic function
				% Circuits
			% Numerical methods
				% Euler's method
				% Picard's method
				% Taylor series
				% Ranga Kutta
			% Chaos
				% Period doubling
				% Strange attractor
		% Notes / TODOS
			% I don't put periods after equations that are in the middle of a sentence. Apparently this is a mortal sin. I'm pretty sure I obeyed capitalization rules, so you can see when a new sentence starts after an equation because the new sentence will begin with a capital letter.
	\backmatter
\end{document}