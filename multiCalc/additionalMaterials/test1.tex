\subsection{Test 1}
\begin{enumerate}
	\item Consider two intersecting lines $\vec{r_1}(t) = \langle 2, 3, 4t \rangle$ and $\vec{r_2}(t) = \langle 2+t, 3+2t, 0 \rangle$. Give the direction vector of each line. Find the equation of the plane which contains both lines. Draw a diagram of the lines, the plane, and the relevant vectors.\\
	\indent
	The direction vector of a line is the derivative of the position vector.
	\begin{itemize}
		\item Direction 1: $\langle 0, 0, 4 \rangle$
		\item Direction 2: $\langle 1, 2, 0 \rangle$
	\end{itemize}
	A the normal vector of the plane is the cross product of the direction vectors.
	\begin{itemize}
		\item $\vec{n} = \langle 0, 0, 4 \rangle \times \langle 1, 2, 0 \rangle = \langle -8, 4, 0 \rangle$
	\end{itemize}
	The lines intersect when $t = 0$ at $(2,3,0)$.
	So, the plane equation is: $\langle -8, 4, 0 \rangle \cdot \langle x-2, y-3, z \rangle = 0$
	
	\begin{figure}[h]
		\centering
		\includegraphics[scale=.5]{Images/additionalMaterials/test1_plane}
	\end{figure}
	
	\item Given the VVF $\vec{r}(t)\langle 10t, 7\cos{t}, 7\sin{t} \rangle$...\\
	\begin{enumerate}[a.]
		\item Compute the unit tangent vector $\hat{T}(t)$ and the unit normal vector $\hat{N}(t)$.\\
		\indent
		$\hat{T} = \frac{\vec{r^\prime}(t)}{\norm{\vec{r^\prime}(t)}}$\\
		$\vec{r^\prime}(t) = \langle 10, -7\sin{t}, 7\cos{t} \rangle$\\
		$\norm{\vec{r^\prime}(t)} = \sqrt{10^2 + (-7\sin{t})^2 + (7\cos{t})^2} = \sqrt{149}$\\
		$\hat{T}(t) = \frac{1}{\sqrt{149}}\langle 10, -7\sin{t}, 7\cos{t} \rangle$\\
		$\hat{N}(t) = \frac{\mathrm{d}\hat{T}/\mathrm{d}t}{\norm{\mathrm{d}\hat{T}/\mathrm{d}t}}$\\
		$\frac{\mathrm{d}\hat{T}}{\mathrm{d}t} = \frac{1}{\sqrt{149}} \langle 0, -7\cos{t}, -7\sin{t} \rangle$\\
		$\norm{\frac{\mathrm{d}\hat{T}}{\mathrm{d}t}} = \frac{1}{\sqrt{149}}\sqrt{(-7\cos{t})^2 + (-7\sin{t})^2} = \frac{7}{\sqrt{149}}$\\
		$\hat{N}(t) = \langle 0, -\cos{t},-\sin{t}\rangle$
		
		\item Show that $\hat{T}\perp\hat{N}$ for all $t$.
		\indent
		If $\hat{T}\perp\hat{N}$, then $\hat{T}\cdot\hat{N} = 0$ for all $t$.\\
		$\hat{T} \cdot \hat{N} = \frac{1}{\sqrt{149}}\langle 10, -7\sin{t}, 7\cos{t} \rangle \cdot \langle 0, -\cos{t}, -\sin{t}\rangle$\\
		$= \frac{1}{\sqrt{149}}(0 + 7\sin{t}\cos{t} - 7\sin{t}\cos{t}) = 0$\\
		$\implies \hat{T}\perp\hat{N}$\\
	\end{enumerate}
	
	\item A cannon fires cannonballs with a speed of $20 \text{ m} / \text{s}$. Take acceleration due to gravity to be $g = 10 \text{m} / \text{s}^2$.
	\begin{enumerate}[a.]
		\item Starting with a constant acceleration function $\vec{a} = \langle 0, -g \rangle$, find the velocity and position functions ($\vec{r^\prime}(t) \text{ and } \vec{r}(t)$ respectively) of the cannonball if the cannon is fired from an angle $\theta$ with respect to the horizontal. Assume the cannonball is initially positioned at the origin.\\
		\indent
		We know that velocity is the integral of acceleration.\\
		$\vec{v}(t) = \vec{r^\prime}(t) = \langle c_1, c_2-gt \rangle$\\
		We are given that the initial speed is $20 \text{ m} / \text{s}$ at an angle $\theta$.\\
		$\vec{v_0} = 20\langle \cos{\theta}, \sin{\theta} \rangle$\\
		$\vec{v}(t) = \langle 20\cos{\theta}, 20\sin{\theta}-gt \rangle$\\
		We know that position is the integral of velocity.\\
		$\vec{r}(t) = \langle 20t\cos{\theta} + c_1, 20t\sin{\theta} - \frac{1}{2}gt^2 + c_2 \rangle$\\
		We are given that the cannonball starts at the origin.\\
		$\vec{r}(t) = \langle 20t\cos{\theta}, 20t\sin{\theta} - \frac{1}{2}gt^2 \rangle$\\
		Taking $g = 10 \text{m} / \text{s}^2$,\\
		$\vec{r}(t) = \langle 20t\cos{\theta}, 20t\sin{\theta}-5t^2 \rangle$\\
			
		\item What angle $\theta$ should the cannon be fired to hit a target on the ground at a distance $40\text{ m}$ away?\\
		\indent
		We want to find a point on the trajectory where $y = 0$ and $x = 40$.
		$y = 0$ when $t = 0, 4\sin{\theta}$. We can reasonably eliminate $t = 0$ because this is when the cannon first fires and $x = 0$.\\
		Plugging in $t = 4\sin{\theta}$ to the x-component of position when $x = 40$,\\
		$20\cos{\theta} \cdot 4\sin{\theta} = 40$\\
		$2\sin{\theta}\cos{\theta} = 1$\\
		$\sin{(2\theta)}=1, 2\theta = \pi/2$\\
		$\implies \theta = \pi/4$\\
	\end{enumerate}
	
	\item Consider the following particle trajectory: $\vec{r}(t) = \langle R\cos{e^t}, R\sin{e^t}, \frac{h}{2\pi}e^t \rangle$ for $t \geq 0$. The shape of the trajectory is a helix with radius $R$ and vertical spacing $h$. Find the arc length function $s(t)$ of the trajectory starting with $s(0) = 0$. Give the arc length reparameterization of the helix.\\
	\indent
	$s(t) = \int_{0}^{t}{\norm{\vec{r^\prime}(\tau)}\mathrm{d}\tau}$\\
	$\vec{r^\prime}(t) = \langle -Re^{t}\sin{e^t}, Re^{t}\cos{e^t}, \frac{h}{2\pi}e^{t} \rangle$\\
	$\norm{\vec{r^\prime}(t)} = \sqrt{(-Re^{t}\sin{e^t})^2 + (Re^{t}\cos{e^t})^2 + (\frac{h}{2\pi}e^t)^2}$\\
	$= e^{t}\sqrt{R^2 + \frac{h^2}{4\pi^2}}$\\
	$s(t) = \int_{0}^{t}{e^{\tau}\sqrt{R^2 + \frac{h^2}{4\pi^2}}\mathrm{d}\tau} = \sqrt{R^2 + \frac{h^2}{4\pi^2}(e^{t} - 1)}$\\
	Solving for $t$,\\
	$t = \ln{\left(\frac{s}{\sqrt{R^2 + \frac{h^2}{4\pi^2}}} + 1\right)}$\\
	$\vec{r}(s) = \left< R\cos{\left(\frac{s}{\sqrt{R^2 + \frac{h^2}{4\pi^2}}} + 1\right)}, R\sin{\left(\frac{s}{\sqrt{R^2 + \frac{h^2}{4\pi^2}}} + 1\right)}, \frac{h}{2\pi}\left(\frac{s}{\sqrt{R^2 +\frac{h^2}{4\pi^2}}} + 1\right) \right>$\\
	
	\item Let $\vec{r}(t)$ be the position function of a particle trapped on the surface of a sphere centered at the origin. Show that $\vec{r}(t)\perp\frac{\mathrm{d}}{\mathrm{d}t}\vec{r}(t)$ for all $t$.\\
	\indent
	Since $\vec{r}(t)$ is on a sphere, $\norm{\vec{r}(t)} = R$ and $\vec{r}(t) \cdot \vec{r}(t) = R^2$.\\
	$\frac{\mathrm{d}}{\mathrm{d}t}(\vec{r}(t) \cdot \vec{r}(t)) = 2\vec{r}(t) \cdot \vec{r^\prime}(t)$\\
	$\frac{\mathrm{d}}{\mathrm{d}t}(\vec{r}(t) \cdot \vec{r}(t)) = \frac{\mathrm{d}}{\mathrm{d}t}R^2 = 0$\\
	So, $2\vec{r}(t) \cdot \vec{r^\prime}(t) = 0$ and $\vec{r}(t) \cdot \vec{r^\prime}(t) = 0$\\
	$\implies \vec{r}(t)\perp\vec{r^\prime}(t)$\\
\end{enumerate}