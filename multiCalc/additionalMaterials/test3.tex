\subsection{Test 3}
\begin{enumerate}
	\item Evaluate each of the following integrals as they appear or by changing coordinate systems. Sketch and/or describe the region geometrically to help in choosing an appropriate coordinate system.
	\begin{enumerate}[label=\alph*.]
		\item \begin{equation*}
			\int_{0}^{3}{\int_{0}^{2}{\int_{0}^{1}{ze^{x+y+z^2}\mathrm{d}x}\mathrm{d}y}\mathrm{d}x}.
		\end{equation*}
		We can use a simple u-substitution.
		\begin{align*}
			u &= z^2 + x + y, \mathrm{d}u = 2z\mathrm{d}z \\
			I &= \int_{0}^{3}{\int_{0}^{2}{\int_{x+y}^{x+y+1}{\frac{1}{2}e^{u}\mathrm{d}u}\mathrm{d}y}\mathrm{d}x} \\
			&= \frac{1}{2}\int_{0}^{3}{\int_{0}^{2}{e^{x+y+1} - e^{x+y}\mathrm{d}y}\mathrm{d}x} \\
			&= \frac{1}{2}\int_{0}^{3}{\left((e^{3+x}-e^{2+x}) - (e^{1+x}-e^{x})\right)\mathrm{d}x} \\
			&= \frac{1}{2}(e^6 - e^5 - e^4 + e^2 + e - 1)
		\end{align*}
		
		\item \begin{equation*}
			\int_{0}^{\sqrt{2}}{\int_{-\sqrt{2-x^2}}^{\sqrt{2-x^2}}{\int_{0}^{5}{z\mathrm{d}z}\mathrm{d}y}\mathrm{d}x}.
		\end{equation*}
		The region we are integrating is a half-cylinder, so we will use cylindrical coordinates.
		\begin{align*}
			I &= \int_{-\pi/2}^{\pi/2}{\int_{0}^{\sqrt{2}}{\int_{0}^{5}{rz\mathrm{d}z}\mathrm{d}r}\mathrm{d}\theta} \\
			&= \pi\int_{0}^{\sqrt{2}}{r\mathrm{d}r}\cdot\int_{0}^{5}{z\mathrm{d}z} \\
			&= \pi\left(\frac{\sqrt{2}^{2}}{2} - \frac{0^2}{2}\right) \cdot \left(\frac{5^2}{2} - \frac{0^2}{2}\right) \\	
			&= \frac{25\pi}{2}
		\end{align*}
		
		\item \begin{equation*}
			\int_{-3}^{3}{\int_{0}^{\sqrt{9 - x^2}}{\int_{0}^{\sqrt{9 - x^2 - y^2}}{z\mathrm{d}z}\mathrm{d}y}\mathrm{d}x}.
		\end{equation*}
		The region we are integrating is a quarter sphere, so we will use spherical coordinates.
		\begin{align*}
			I &= \int_{0}^{3}{\int_{0}^{\pi}{\int_{0}^{\pi/2}{\rho^2\sin{\phi}\cos{\phi}\mathrm{d}\phi}\mathrm{d}\theta}\mathrm{d}\rho} \\
			&= \int_{0}^{3}{\rho^3\mathrm{d}\rho}\cdot\int_{0}^{\pi}{\mathrm{d}\theta} \cdot \int_{0}^{\pi/2}{\sin{\phi}\cos{\phi}\mathrm{d}\phi} \\
			&= \frac{3^4}{4} \cdot \pi \cdot \frac{1}{2} = \frac{81\pi}{8}
		\end{align*}
	\end{enumerate}
	
	\item Let $\Omega \subset \mathbb{R}^3$ be a spherical ball of radius $R$ centered at the origin. Set up and evaluate $\iiint\limits_{\Omega}{\mathrm{d}V}$.\\
	Since we are finding the volume of a ball, we'll use spherical coordinates.
	\begin{align*}
		\mathrm{d}V &= \rho^2\sin{\phi}\mathrm{d}\rho\mathrm{d}\theta\mathrm{d}\phi \\
		I &= \int_{0}^{R}{\int_{0}^{2\pi}{\int_{0}^{\pi}{\rho^2\sin{\phi}\mathrm{d}\phi}\mathrm{d}\theta}\mathrm{d}\rho} \\
		&= \int_{0}^{R}{\rho^2\mathrm{d}\rho} \cdot \int_{0}^{2\pi}{\mathrm{d}\theta} \cdot \int_{0}^{\pi}{\sin{\phi}\mathrm{d}\phi} \\
		&= \frac{R^3}{3} \cdot 2\pi \cdot 2 = \frac{4\pi}{3}R^3
	\end{align*}
	
	\item A plane lamina with density $\sigma(x,y) = \sqrt{x^2+y^2}$ occupies the region $D$, the region bounded by the Archimedian spiral $r = \theta$ and the half line $\theta = \alpha, r \geq 0$, where $\alpha$ is an unknown angle in radians. Find $\alpha$ such that the average density $\bar{\sigma}$ of the lamina is $\pi / 2$.
	\begin{equation*}
			\bar{\sigma} = \frac{\iint\limits_{D}{\sigma\mathrm{d}A}}{\iint\limits_{D}{\mathrm{d}A}} = \frac{\pi}{2}
	\end{equation*}
	\begin{equation*}
		D = \left\{(r,\theta) \mid 0 \leq \theta \leq \alpha, 0 \leq r \leq \theta \right\}, \sigma(r,\theta) = r
	\end{equation*}
	\begin{align*}
		\bar{\sigma} &= \frac{\int_{0}^{\alpha}{\int_{0}^{\theta}{r^2\mathrm{d}r}\mathrm{d}\theta}}{\int_{0}^{\alpha}{\int_{0}^{\theta}{r\mathrm{d}r}\mathrm{d}\theta}} \\
		&= \frac{\int_{0}^{\alpha}{\frac{\theta^3}{3}\mathrm{d}\theta}}{\int_{0}^{\alpha}{\frac{\theta^2}{2}\mathrm{d}\theta}} \\
		&= \frac{\alpha^4/12}{\alpha^3/6} = \frac{\alpha}{2} \\
		\frac{\alpha}{2} &= \frac{\pi}{2} \\
		&\implies \alpha = \pi
	\end{align*}
	
	\item Consider the 2D Gaussian function $f(x,y) = e^{-(x^2+y^2)}$. Evaluate $\iint\limits_{D}{f(x,y)\mathrm{d}A}$, where $D$ is a disk of radius $a$ centered at the origin. Use the results to evaluate $\iint\limits_{\mathbb{R}^2}{f(x,y)\mathrm{d}A}$.
	\begin{equation*}
		D = \left\{(r,\theta) \mid 0 \leq r \leq a, 0 \leq \theta \leq 2\pi \right\}
	\end{equation*}
	\begin{equation*}
		f(x,y) = \exp{(-x^2-y^2)} = \exp{(-r^2)}
	\end{equation*}
	\begin{align*}
		\iint\limits_{D}{f(x,y)\mathrm{d}A} &= \int_{0}^{2\pi}{\int_{0}^{a}{e^{-r^2}r\mathrm{d}r}\mathrm{d}\theta} \\
		&= \int_{0}^{2\pi}{\mathrm{d}\theta} \cdot \frac{-1}{2}\int_{0}^{-a^2}{e^{u}\mathrm{d}u} \\
		&= 2\pi \cdot \frac{-1}{2} \cdot \left(e^{-a^2} - 1\right) \\
		&= \pi\left(1 - e^{-a^2}\right)
	\end{align*}
	\begin{equation*}
		\iint\limits_{\mathbb{R}^2}{f(x,y)\mathrm{d}A} = \lim_{a \to \infty}{\pi\left(1 - e^{-a^2}\right)} = \pi
	\end{equation*}
\end{enumerate}