\subsection{Linear Approximations}
\noindent
Since $\partial z = f_x\partial x + f_y\partial y$, we can approximate $\Delta z$ (the change in any function) as $\Delta z \approx f_x\Delta x + f_y\Delta y$ because values of $f$ and the tangent plane are close.
We can rewrite this approximation as a dot product:
\begin{equation*}
	\Delta z \approx \langle f_x, f_y\rangle \cdot \langle \Delta x, \Delta y \rangle.
\end{equation*}

For example, say a cylindrical can has a radius $r=1$ and a height $h=5$. If the radius is increased by 0.1 and the height is increased by 1, what is the approximate $\Delta V$?
\begin{align*}
	V(r,h) &= \pi r^2 h \\
	V_r &= 2\pi rh \text{ and } V_h = \pi r^2 \\
	V_{r}(1,5) &= 10\pi  \text{ and } V_{h}(1,5) = \pi \\
	\Delta V &\approx 10\pi(0.1) + \pi(1) = 2\pi.
\end{align*}
Comparing this to the actual answer of $2.26\pi$, we see our approximation is decent.